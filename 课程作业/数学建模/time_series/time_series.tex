\documentclass[a4paper]{article}
\usepackage{xcolor}%颜色支持
\definecolor{vscode_backgroundcolor}{rgb}{0.15, 0.2, 0.22}
\definecolor{vscode_localvariablecolor}{rgb}{0.93,1,0.89}
\definecolor{vscode_keywordcolor}{rgb}{0.57, 0.52, 1}
\definecolor{vscode_commentcolor}{rgb}{0.2, 0.34, 0.45}
\definecolor{vscode_stringcolor}{rgb}{0.76, 0.91, 0.55}
\definecolor{vscode_semicolomncolor}{rgb}{1,1,1}
\definecolor{vscode_headerfilecolor}{rgb}{0.93,1,0.89}
\definecolor{vscode_linenumbercolor}{rgb}{0.2, 0.34, 0.45}
\definecolor{vscode_numbercolor}{rgb}{0.96, 0.54, 0.42}
\definecolor{vscode_parametercolor}{rgb}{0.96, 0.54, 0.42}
\definecolor{vscode_operatorcolor}{rgb}{0.53, 0.85, 0.99}
\definecolor{vscode_callablecolor}{rgb}{0.36, 0.62, 0.96}
\definecolor{vscode_rulecolor}{rgb}{0.22, 0.28, 0.31}
\definecolor{vscode_classcolor}{rgb}{1, 0.8, 0.42}
\definecolor{vscode_selfcolor}{rgb}{0.97, 0.31, 0.43}

\usepackage{listings}%高亮代码块支持
\lstloadlanguages{Python}
\lstdefinestyle{MyPython}{
    extendedchars=false,
    numbers=left,
    firstnumber=auto,
    frame=leftline,
    backgroundcolor=\color{vscode_backgroundcolor},
    framerule=0.5ex,
    columns=fixed,
    language=Python,
    basicstyle=\ttfamily\color{vscode_localvariablecolor},
    % commentstyle=\color{vscode_commentcolor},
    keywordstyle=\color{vscode_keywordcolor},
    stringstyle=\color{vscode_stringcolor},
    morecomment=[l][\color{vscode_commentcolor}]{\#},
    morecomment=[s][\color{vscode_headerfilecolor}]{"""}{"""},
    numberstyle=\code\color{vscode_linenumbercolor},
    morekeywords={
        None,
        True,
        yield,
        False,
    },
    alsoletter={
        =+-*<>^&
        % ;0123456789
        % 0,1,2,3,4,5,6,7,8,9
    },%把;设置为可以识别的letter,也可以使用otherkeywords,但是无法将其和其他keywords区分开来
    emph={=,+,-,*,>,<,<<,>>,^,+=,&},emphstyle=\color{vscode_operatorcolor},
    % alsoletter={!,\%,&,*,-,+,=,/,<,>},
    emph={[2]%一些全局的函数和可调用对象
    DataFrame,
    arange,
    interp2d,
    f,
    subplots,
    contour,
    show,
    clabel,
    getWalls,
    AnyFoodSearchProblem,
    print,
    range,
    scatter,
    spline,
    UnivariateSpline,
    diff,
    len,
    append,
    array,
    sqrt,
    sum,
    mean,
    as,
    figure,
    scatter
    },emphstyle={[2]\color{vscode_callablecolor}},
    emph={[5]
    self},emphstyle={[5]\color{vscode_selfcolor}},
    % emphstyle={[3]\color{vscode_parametercolor}}
     % emph={[3]0,1,2,3,4,5,6,7,8,9},emphstyle={[3]\color{yellow}},
    tabsize=4,
    rulecolor=\color{vscode_rulecolor},
    breaklines=true,
}
\lstset{
    style=MyPython,
}
\usepackage[
    fontset=none,%设置中文支持,并自定义字体
    zihao=5,%默认字号为五号
    heading=true,%允许后续自定义标题样式
    scheme=chinese,%自动将文档样式中文化,例如图标标题
    punct=quanjiao,%全角式标点符号
    space=auto,%中文后接换行不会添加空格,但是英文会添加空格,需要用%手动取消
    linespread=1.3,%行距倍数是1.3
    autoindent=true,%自动缩进两个中文宽度
    ]{ctex}
\ctexset{
    % tday=small,%小写样式的日期
    contentsname={目录},
    listfigurename={插图},
    listtablename={表格},
    figurename={图},
    tablename={表},
    abstractname={摘要},
    indexname={索引},
    appendixname={附录},
    bibname={参考文献},
    proofname={证明},
    % refname={参考文献},%只适用于beamer
    % algorithmname={算法},
    % continuation={(续)},%beamer续页的标识
    section={
        format+ = \Large\heiti\raggedright,
        name = {,\num\textbf{.}\hspace{1ex}},
        number={\num\thesection},
        nameformat={},
        numberformat={},
        aftername={},
        titleformat={},
        aftertitle={},
        runin=false,%对section级以下有用,标题是否和正文在同一段上
        beforeskip={3.5ex plus 1ex minus .2ex},%标题前垂直间距
        afterskip={2.3ex plus .2ex}%标题后垂直间距
    },
    subsection={
        format+ = \large\heiti\raggedright,
        name = {,\num\textbf{.}\hspace{1ex}},
        number={\num\thesubsection},
        nameformat={},
        numberformat={},
        aftername={},
        titleformat={},
        aftertitle={},
        runin=false,%对section级以下有用,标题是否和正文在同一段上
        beforeskip={3.5ex plus 1ex minus .2ex},%标题前垂直间距
        afterskip={2.3ex plus .2ex}%标题后垂直间距
    },
    subsubsection={
        format+ = \normalsize\heiti\raggedright,
        name = {,\num\textbf{.}\hspace{1ex}},
        number={\num\thesubsubsection},
        nameformat={},
        numberformat={},
        aftername={},
        titleformat={},
        aftertitle={},
        runin=false,%对section级以下有用,标题是否和正文在同一段上
        beforeskip={3.5ex plus 1ex minus .2ex},%标题前垂直间距
        afterskip={2.3ex plus .2ex}%标题后垂直间距
    },
    }

% 中文默认字体: 思源宋体,粗体为思源宋体半粗体,斜体为方正楷体_GBK
\setCJKmainfont{Source Han Serif SC}[BoldFont={Source Han Serif SC Heavy}, ItalicFont=FZKai-Z03S]
% 中文无衬线字体:思源黑体,粗体为思源黑体粗体
\setCJKsansfont{Source Han Sans CN}[BoldFont={Source Han Sans CN Heavy}]
% 中文等宽字体:微软雅黑light
\setCJKmonofont{Microsoft YaHei}[ItalicFont={Microsoft YaHei Light}]

\newCJKfontfamily\songti{Source Han Serif SC}[BoldFont={Source Han Serif SC Heavy}]
\newCJKfontfamily\xbsong{Source Han Serif SC SemiBold} % 小标宋
\newCJKfontfamily\dbsong{Source Han Serif SC Bold} % 大标宋
\newCJKfontfamily\cusong{Source Han Serif SC Heavy} % 粗宋
\newCJKfontfamily\heiti{Source Han Sans CN}[BoldFont={Source Han Sans CN Heavy}]
\newCJKfontfamily\dahei{Source Han Sans CN Medium} % 大黑
\newCJKfontfamily\cuhei{Source Han Sans CN Heavy} % 粗黑
\newCJKfontfamily\fangsong{FZFangSong-Z02S}
\newCJKfontfamily\kaiti{FZKai-Z03S}[ItalicFont={Microsoft YaHei Light}]%这个斜体只是用于lstlisting环境中的中文注释
% \newCJKfontfamily\kaiti{FZKai-Z03S}[ItalicFont={FZZJ-LZXTFSJW}]%这个斜体只是用于lstlisting环境中的中文注释
\setsansfont{Arial}
\setmonofont{Consolas}%设置西文等宽字体
\newfontfamily\code{Consolas}
\newfontfamily\num{Arial}

\usepackage{geometry}%设置整体页面布局
\geometry{a4paper}
\geometry{left=2cm,right=2cm,top=2.54cm,bottom=2.54cm}%word常规页边距
% \geometry{left=1.27cm,right=1.27cm,top=1.27cm,bottom=1.27cm}%word窄页边距
\setlength{\headheight}{13pt}%避免warning


\usepackage{fancyhdr}%必须在geometry包之后使用
\fancyhf{}
\makeatletter
\lhead{{\dahei \@title}}%可以使用thepage,CTEXthechapter,CTEXthesection
\makeatother
\rhead{\textbf{\num- \thepage{} -}}
\renewcommand\headrulewidth{1.5pt}%设置眉头宽度
\pagestyle{fancy}

\usepackage[ruled,algosection,lined,longend,fillcomment,linesnumbered,resetcount,titlenotnumbered]{algorithm2e}
%参数解释:带框,按section编码,有竖线,end前带if等关键词,注释占满整行,代码部分编号(不包括输入输出、注释),每个代码块重新编号,可以调用TitleOfAlgo来打印算法标题但不作为单独的算法编码
%附带algorithm,function,procedure环境,其中function,procedure环境下,设置caption时,必须带有(),
%()之前的字符会被视为宏,可以在接下来的部分用\名字()来调用,所以推荐辅助函数用function,其中的某些展开部分用procedure,描述算法整体使用algorithm
\DontPrintSemicolon
\SetAlCapSkip{2ex}
\SetSideCommentRight
\SetFillComment
\newcommand{\forcond}{$i=0$ \KwTo $n$}
\SetKw{downto}{downto}%自定义关键词
\SetKwFunction{funcmacro}{text}%自定义函数名,实际上function环境是在定义宏的同时说明了其内容
\SetKwProg{procedmacro}{text}{begin text}{end text}%自定义步骤,和function类似,但是后面两个参数可以设置开始和结尾的标志,和if等环境一样
\SetKwData{datamacro}{text}%可以用于突出特殊的变量,例如数据结构
\SetKwFunction{FRecurs}{FnRecursive}
\SetKwProg{Fn}{Function}{begin}{end}

\usepackage[strict]{changepage}
\usepackage{framed}%色块支持
\definecolor{formalshade}{rgb}{0.95,0.95,1} % 文本框颜色
% ------------------******-------------------
% 注意行末需要把空格注释掉,不然画出来的方框会有空白竖线
\newenvironment{formal}{%
\def\FrameCommand{%
\hspace{1pt}%
{\color{DarkBlue}\vrule width 2pt}%
{\color{formalshade}\vrule width 4pt}%
\colorbox{formalshade}%
}%
\MakeFramed{\advance\hsize-\width\FrameRestore}%
\noindent\hspace{-4.55pt}% disable indenting first paragraph
\begin{adjustwidth}{}{7pt}%
\vspace{2pt}\vspace{2pt}%
}
{%
\vspace{2pt}\end{adjustwidth}\endMakeFramed%
}

% 自定义标题格式
\makeatletter
\renewcommand{\maketitle}{
  \begin{center}
    \thispagestyle{fancy}
    {\quad}\\
    \vspace{0.1\textheight}
    {\huge\sffamily\bfseries\@title}\\ % 标题字体大小、粗体、颜色
    \vspace{2em} % 标题与作者名之间的垂直空间
    {\large\sffamily\@author} \\
  \end{center}
}
\makeatother

\usepackage{graphicx}
\usepackage{amsmath}

\title{时间序列}
\author{姓名:范潇{\quad}学号:2254298{\quad}日期:\today}
\date{}
\begin{document}
\maketitle
\section{(8.1)}
我利用Python完成了本题的求解。

采用趋势移动平均法所得到的模型的误差为19.35.

当$\alpha=0.3$时,直线指数平滑预测模型的误差为4.43;当$\alpha=0.6$时,直线指数平滑预测模型的误差为7.10。

综上,采用$\alpha=0.3$的直线指数平滑预测模型较优。当使用该模型时,对于1982年和1985年产量的预测分别为143.83亿和163.12亿。

本题所用的代码如下:
\begin{lstlisting}
import numpy as np
from math import *
data = [
    [1974, 1975, 1976, 1977, 1978, 1979, 1980, 1981],
    [80.8, 94.0, 88.4, 101.5, 110.3, 121.5, 134.7, 142.7]
]
val = [80.8, 94.0, 88.4, 101.5, 110.3, 121.5, 134.7, 142.7]
predict = val[:3]
for i in range(3,len(val)):
    predict.append(np.mean(np.array(val[i-3:i])))
err = sqrt(sum((np.array(val[3:])-np.array(predict[3:]))**2)/(len(val)-3))
print("================  Q1 ====================")
print(f"true value:{val}")
print(f"prediction:{predict}")
print(f"err:{err}")



s1,s2 = 87.7,87.7
S1=[s1]
S2=[s2]
alpha = 0.3
for i in range(2,len(val)):
    S1.append(alpha*val[i]+(1-alpha)*S1[-1])
    S2.append(alpha*S1[-1]+(1-alpha)*S2[-1])
exp_prediction = []
exp_prediction.append(s1)
for i in range(1,len(val)):
    exp_prediction.append((1+1/(1-alpha))*S1[i-1]-S2[i-1]/(1-alpha))

err = sqrt(sum((np.array(val[:])-np.array(exp_prediction[:]))**2)/(len(val)))
print("================  Q2 ====================")
print("alpha = 0.3")
print(f"true value:{val}")
print(f"prediction:{exp_prediction}")
print(f"err:{err}")

S1=[s1]
S2=[s2]
alpha = 0.6
for i in range(2,len(val)):
    S1.append(alpha*val[i]+(1-alpha)*S1[-1])
    S2.append(alpha*S1[-1]+(1-alpha)*S2[-1])
exp_prediction = []
exp_prediction.append(s1)
for i in range(1,len(val)):
    exp_prediction.append((1+1/(1-alpha))*S1[i-1]-S2[i-1]/(1-alpha))

err = sqrt(sum((np.array(val[:])-np.array(exp_prediction[:]))**2)/(len(val)))
print("alpha = 0.6")
print(f"true value:{val}")
print(f"prediction:{exp_prediction}")
print(f"err:{err}")


s1,s2 = 87.7,87.7
S1=[s1]
S2=[s2]
alpha = 0.3
for i in range(2,len(val)):
    S1.append(alpha*val[i]+(1-alpha)*S1[-1])
    S2.append(alpha*S1[-1]+(1-alpha)*S2[-1])
exp_prediction = []
exp_prediction.append(s1)
for i in range(1,len(val)):
    exp_prediction.append((1+1/(1-alpha))*S1[i-1]-S2[i-1]/(1-alpha))

err = sqrt(sum((np.array(val[:])-np.array(exp_prediction[:]))**2)/(len(val)))
print("================  Q3 ====================")
print(S1)
print(S2)
a = 2*S1[-1]-S2[-1]
b = alpha*(S1[-1]-S2[-1])/(1-alpha)
print(a,b)
print("1982预测产量为:",a+b)
print("1985预测产量为:",a+4*b)
\end{lstlisting}

输出为图\ref*{q1}
\begin{figure}[!h]
    \centering
    \includegraphics*[width = \textwidth]{q1.png}
    \caption{第一题程序输出}\label{q1}
\end{figure}
\newpage
\section{(8.2)}
我利用Python完成了本题的求解。

首先我分别测试了$\alpha$值为$0.1,0.2,\cdots,0.9$时三次指数平滑法得到的模型拟合误差。
经过比较,当$\alpha=0.5$时,模型拟合误差最小,所以最后选取$\alpha=0.5$。这样得到的
三次指数平滑法对于1983年和1985年的全国社会商品零售额分别为2858.3亿元和3465.8亿元。

本题所用的代码如下:
\begin{lstlisting}[emph={[3]alpha},emphstyle={[3]\color{vscode_parametercolor}},emph={[4]RegressionModel,GameState,MinimaxAgent,AlphaBetaAgent},emphstyle={[4]\color{vscode_classcolor}}]
import numpy as np
from math import *
data = [[1960, 1961, 1962, 1963, 1964, 1965, 1966, 1967, 1968, 1969, 1970, 1971, 1972, 1973, 1974, 1975, 1976, 1977, 1978, 1979, 1980, 1981, 1982],
 [696.9, 607.7, 604.0, 604.5, 638.2, 670.3, 732.8, 770.5, 737.3, 801.5, 858.0, 929.2, 1023.3, 1106.7, 1163.6, 1271.1, 1339.4, 1432.8, 1558.6, 1800.0, 2140.0, 2350.0, 2570.0]]
val = [696.9, 607.7, 604.0, 604.5, 638.2, 670.3, 732.8, 770.5, 737.3, 801.5, 858.0, 929.2, 1023.3, 1106.7, 1163.6, 1271.1, 1339.4, 1432.8, 1558.6, 1800.0, 2140.0, 2350.0, 2570.0]

def third_exp_predict(alpha):
    init = np.mean(val[:3])
    S1,S2,S3 = [init],[init],[init]
    for i in range(1,len(val)):
        S1.append(alpha*val[i]+(1-alpha)*S1[i-1])
        S2.append(alpha*S1[i]+(1-alpha)*S2[i-1])
        S3.append(alpha*S2[i]+(1-alpha)*S3[i-1])
    prediction = [init]
    for i in range(1,len(val)):
        prediction.append((3-3*alpha+alpha*alpha)*S1[i-1]/(1-alpha)**2-(3-alpha)*S2[i-1]/(1-alpha)**2 + S3[i-1]/(1-alpha)**2)
    err = sqrt(sum((np.array(prediction)-np.array(val))**2)/len(val))

    print(alpha)
    print(err)
    print(S1)
    print(S2)
    print(S3)
# 用于选取最佳alpha值
# alphas = np.linspace(0.1,0.9,9)
# for i in range(len(alphas)):
#     third_exp_predict(alphas[i])

# alpha取0.5时err最小
best = 0.5
s1 = 2343.3
s2 = 2122.8
s3 = 1920.8
a = 3*s1-3*s2+s3
b = ((6-5*best)*s1-2*(5-4*best)*s2+(4-3*best)*s3)*best/(2*(1-best)*(1-best))
c = (s1-2*s2+s3)*best*best/(2*(1-best)*(1-best))
# print("1982年预测值:",a)
print("1983年预测值:",a+b+c)
print("1985年预测值:",a+3*b+9*c)
\end{lstlisting}
\newpage
\section{(8.3)}
我利用Python完成了本题的求解。

首先我绘制了所给数据的折线图,如图\ref{q2}。
\begin{figure}[!h]
    \centering
    \includegraphics*[width = \textwidth]{q2.png}
    \caption{第二题数据}\label{q2}
\end{figure}

可以看到,数据大致成线性,因此采用直线指数平滑预测模型和一阶差分指数平滑模型。

通过分别设定$\alpha=0.1,0.2,\cdots,0.9$,得到对于直线指数平滑模型,最佳参数为$\alpha=0.3$,对应的
拟合误差为0.48;对于一阶差分指数平滑模型,最佳参数为$\alpha=0.4$,对应的拟合误差为0.5.

选取最佳参数后,直线指数平滑模型预测出的1985年和1990年的粮食产量分别为13.7,16.4;
一阶差分平滑模型预测出的1985年和1990年的粮食产量分别为13.1,16.3。

本题所用的代码如下:
\begin{lstlisting}[emph={[3]alpha},emphstyle={[3]\color{vscode_parametercolor}},emph={[4]RegressionModel,GameState,MinimaxAgent,AlphaBetaAgent},emphstyle={[4]\color{vscode_classcolor}}]
import matplotlib.pyplot as plt
import numpy as np
from math import *
val = [3.78,4.19,4.83,5.46,6.71,7.99,8.60,9.24,9.67,9.87,10.49,10.92,10.93,
       12.39,12.59,]
plt.figure()
plt.scatter(range(len(val)),  val)
plt.show()

def second_exp_predict(alpha):
    init = np.mean(val[:2])
    S1,S2 = [init],[init]
    for i in range(2,len(val)):
        S1.append(alpha*val[i]+(1-alpha)*S1[-1])
        S2.append(alpha*S1[-1]+(1-alpha)*S2[-1])
    prediction = [init]
    for i in range(1,len(val)):
        prediction.append((1+1/(1-alpha))*S1[i-1]-S2[i-1]/(1-alpha))
    err = sqrt(sum((np.array(prediction)-np.array(val))**2)/len(val))

    print(alpha)
    print(err)
    print(S1)
    print(S2)

def first_diff_predict(alpha):
    init = np.mean(val[:2])
    predict_diff =[0]
    prediction = [val[0]]
    for i in range(1,len(val)):
        predict_diff.append(alpha*diff[i-1]+(1-alpha)*predict_diff[i-1])
        prediction.append(predict_diff[i]+val[i-1])

    err = sqrt(sum((np.array(prediction)-np.array(val))**2)/len(val))
    print(alpha)
    print(predict_diff)
    print(err)

# 用于选取最佳alpha值
# alphas = np.linspace(0.1,0.9,9)
# for i in range(len(alphas)):
#     print()
#     second_exp_predict(alphas[i])

best = 0.3
s1 = 11.33
s2 = 10.05
a = 2*s1-s2
b = best*(s1-s2)/(1-best)
print("二次指数平滑1985年预测值:",a+2*b)
print("二次指数平滑1990年预测值:",a+7*b)

diff = [0] + [val[i]-val[i-1] for i in range(1,len(val))]

# 用于选取最佳alpha值
# alphas = np.linspace(0.1,0.9,9)
# for i in range(len(alphas)):
#     print()
#     first_diff_predict(alphas[i])
best = 0.4
diff_predict = best*diff[-1]+(1-best)*0.761
print("一次差分平滑1985年预测值:",12.59+diff_predict)
print("一次差分平滑1990年预测值:",12.59+7*diff_predict)
\end{lstlisting}
\newpage
\section{(8.4)}
我在Matlab的实时编辑器中完成了本题的求解。

得到的模型为
\[y_t = 1.2276y_{t-1}-0.68478y_{t-2}+\varepsilon_t-0.5022\varepsilon_{t-1}\]
其中$y_t=x_{t+1}-x_t$为差分数组。

得到的10步预测值分别为
6470,
6879,
7393,
8027,
8743,
9483,
10202,
10882,
11536,
12190.

实时编辑器页面如下一页所示。
\newpage
\section{(8.5)}
我在Matlab的实时编辑器中完成了本题的求解。

首先我绘制了所给数据的折线图,如图\ref{q5}。

\begin{figure}[!h]
    \centering
    \includegraphics*[width =0.5 \textwidth]{q51.jpg}
    \caption{第五题数据}\label{q5}
\end{figure}

可以看到,数据成周期性,周期大致为10。所以我先对原始数据进行差分处理,转化为差分序列$y_t = x_{t+10}-x_t$。

\begin{figure}[!h]
    \centering
    \includegraphics*[width = 0.5\textwidth]{q52.jpg}
    \caption{差分序列}\label{q52}
\end{figure}

从图\ref*{q52}可以看到,经过差分后,在大部分时间段内,序列较平稳。

接着,我尝试建立ARMA模型,根据AIC值选取最佳参数。经比较,当p=3,q=10时,AIC值最小。
同时,在该参数下,模型能够通过LBQ检验。

通过选取出的模型得到的后两年的预测值分别为2737和2077。

实时编辑器页面如后所示。
\newpage
\section{(8.6)}
我在Matlab的实时编辑器中完成了本题的求解。

首先我绘制了所给数据的折线图,如图\ref{q6}。


\begin{figure}[!h]
    \centering
    \includegraphics*[width = 0.5\textwidth]{q6.jpg}
    \caption{第六题数据}\label{q6}
\end{figure}

可以看到,各个季度的变化趋势基本一致,且相对大小较为稳定。因此,我先以年为单位进行
时间序列模型。然后我计算根据所给数据计算出各季度的数据对于一整年数据的贡献平均占比。
综合两者,得到对于未来各季度的估计。

对于时间序列,我尝试建立ARMA模型,根据AIC值选取最佳参数。经比较,当p=2,q=3时,AIC值最小。
同时,在该参数下,模型能够通过LBQ检验。

通过选取出的模型得到的后两年的预测值分别为242和237。乘以比例系数后,得到后8各季度的预测数据分别为
58.3,61.0,61.0,61.9,57.3,59.9,59.9,60.8。

实时编辑器页面如后所示。
\newpage
\end{document}