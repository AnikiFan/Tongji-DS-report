\documentclass[12pt, a4paper, oneside]{article}
\usepackage{amsmath, amsthm, amssymb, bm, graphicx, hyperref, mathrsfs}
\makeatletter
\newcommand{\mytitle}{\@title}
\makeatother

\usepackage[
    fontset=none,%设置中文支持,并自定义字体
    zihao=5,%默认字号为五号
    heading=true,%允许后续自定义标题样式
    scheme=chinese,%自动将文档样式中文化,例如图标标题
    punct=quanjiao,%全角式标点符号
    space=auto,%中文后接换行不会添加空格,但是英文会添加空格,需要用%手动取消
    linespread=1.3,%行距倍数是1.3
    autoindent=true,%自动缩进两个中文宽度
    ]{ctex}
\ctexset{
    today=small,%小写样式的日期
    contentsname={目录},
    % contentsname={\hspace{-\ccwd}目录},
    listfigurename={插图},
    listtablename={表格},
    figurename={图},
    tablename={表},
    abstractname={简{\quad}介},
    indexname={索引},
    appendixname={附录},
    bibname={参考文献},
    proofname={证明},
    % refname={参考文献},%只适用于beamer
    % algorithmname={算法},
    % continuation={(续)},%beamer续页的标识
    section={
        format+ = \Large\heiti\raggedright,
        name = {,\num\textbf{.}\hspace{1ex}},
        number={\num\thesection},
        nameformat={},
        numberformat={},
        aftername={},
        titleformat={},
        aftertitle={},
        runin=false,%对section级以下有用,标题是否和正文在同一段上
        beforeskip={3.5ex plus 1ex minus .2ex},%标题前垂直间距
        afterskip={2.3ex plus .2ex}%标题后垂直间距
    },
    subsection={
        format+ = \large\heiti\raggedright,
        name = {,\num\textbf{.}\hspace{1ex}},
        number={\num\thesubsection},
        nameformat={},
        numberformat={},
        aftername={},
        titleformat={},
        aftertitle={},
        runin=false,%对section级以下有用,标题是否和正文在同一段上
        beforeskip={3.5ex plus 1ex minus .2ex},%标题前垂直间距
        afterskip={2.3ex plus .2ex}%标题后垂直间距
    },
    subsubsection={
        format+ = \normalsize\heiti\raggedright,
        name = {,\num\textbf{.}\hspace{1ex}},
        number={\num\thesubsubsection},
        nameformat={},
        numberformat={},
        aftername={},
        titleformat={},
        aftertitle={},
        runin=false,%对section级以下有用,标题是否和正文在同一段上
        beforeskip={3.5ex plus 1ex minus .2ex},%标题前垂直间距
        afterskip={2.3ex plus .2ex}%标题后垂直间距
    },
    }

\title{\textbf{HW2}}
\author{范潇\quad2254298}
\date{\today}
\linespread{1.5}
\newcounter{problemname}
\newenvironment{problem}[1]{\stepcounter{problemname}\par\noindent\textbf{题目\arabic{problemname}. (#1)}}{}
\newenvironment{solution}{\par\noindent\textbf{解答. }}{}
\newenvironment{note}{\par\noindent\textbf{题目\arabic{problemname}的注记. }}{}
\usepackage{amsfonts}
\usepackage{lmodern}%解决报错
% 中文默认字体: 思源宋体,粗体为思源宋体半粗体,斜体为方正楷体_GBK
\setCJKmainfont{Source Han Serif SC}[BoldFont={Source Han Serif SC Heavy}, ItalicFont=FZKai-Z03S]
% 中文无衬线字体:思源黑体,粗体为思源黑体粗体
\setCJKsansfont{Source Han Sans CN}[BoldFont={Source Han Sans CN Heavy}]
% 中文等宽字体:微软雅黑light
\setCJKmonofont{Microsoft YaHei}[ItalicFont={Microsoft YaHei Light}]

\newCJKfontfamily\songti{Source Han Serif SC}[BoldFont={Source Han Serif SC Heavy}]
\newCJKfontfamily\xbsong{Source Han Serif SC SemiBold} % 小标宋
\newCJKfontfamily\dbsong{Source Han Serif SC Bold} % 大标宋
\newCJKfontfamily\cusong{Source Han Serif SC Heavy} % 粗宋
\newCJKfontfamily\heiti{Source Han Sans CN}[BoldFont={Source Han Sans CN Heavy}]
\newCJKfontfamily\dahei{Source Han Sans CN Medium} % 大黑
\newCJKfontfamily\cuhei{Source Han Sans CN Heavy} % 粗黑
\newCJKfontfamily\fangsong{FZFangSong-Z02S}
\newCJKfontfamily\kaiti{FZKai-Z03S}[ItalicFont={Microsoft YaHei Light}]%这个斜体只是用于lstlisting环境中的中文注释
% \newCJKfontfamily\kaiti{FZKai-Z03S}[ItalicFont={FZZJ-LZXTFSJW}]%这个斜体只是用于lstlisting环境中的中文注释
\setsansfont{Arial}
\setmonofont{Consolas}%设置西文等宽字体
\newfontfamily\code{Consolas}
\newfontfamily\num{Arial}

\usepackage{geometry}%设置整体页面布局
\geometry{a4paper}
\geometry{left=2cm,right=2cm,top=2.54cm,bottom=2.54cm}%word常规页边距
% \geometry{left=1.27cm,right=1.27cm,top=1.27cm,bottom=1.27cm}%word窄页边距
\setlength{\headheight}{13pt}%避免warning
\usepackage{fancyhdr}%必须在geometry包之后使用
\fancyhf{}
\makeatletter
\lhead{\sffamily\bfseries{2254298 范潇}}%可以使用thepage,CTEXthechapter,CTEXthesection
\makeatother
\chead{\sffamily\bfseries{\mytitle}}
\rhead{\sffamily\bfseries{- \thepage{} -}}
\renewcommand\headrulewidth{2pt}%设置眉头宽度
\pagestyle{fancy}
\usepackage[ruled,lined,longend,fillcomment,linesnumbered,resetcount,titlenotnumbered]{algorithm2e}
%参数解释:带框,按section编码,有竖线,end前带if等关键词,注释占满整行,代码部分编号(不包括输入输出、注释),每个代码块重新编号,可以调用TitleOfAlgo来打印算法标题但不作为单独的算法编码
%附带algorithm,function,procedure环境,其中function,procedure环境下,设置caption时,必须带有(),
%()之前的字符会被视为宏,可以在接下来的部分用\名字()来调用,所以推荐辅助函数用function,其中的某些展开部分用procedure,描述算法整体使用algorithm
\DontPrintSemicolon
\SetAlCapSkip{2ex}
\SetSideCommentRight
\SetFillComment
\newcommand{\forcond}{$i=0$ \KwTo $n$}
\SetKw{macro}{text}%自定义关键词
\SetKwFunction{funcmacro}{text}%自定义函数名,实际上function环境是在定义宏的同时说明了其内容
\SetKwProg{procedmacro}{text}{begin text}{end text}%自定义步骤,和function类似,但是后面两个参数可以设置开始和结尾的标志,和if等环境一样
\SetKwData{datamacro}{text}%可以用于突出特殊的变量,例如数据结构
\SetKwFunction{FRecurs}{FnRecursive}
\SetKwProg{Fn}{Function}{begin}{end}
\begin{document}
\maketitle
\begin{problem}{主元素问题}
    设$T[0..n-1]$是$n$个元素的数组。对任一元素$x$,设$S(x)=\{i|T[i]=x\}$。当$|S(x)|>n/2$时,称$x$为$T$的主元素。
    \begin{enumerate}
        \item 如果T中元素是有序的,按分治策略设计并实现一个线性时间算法,确定$T[0..n-1]$中是否有主元素;
        \item 若$T$中元素不是有序的,试设计并实现一个$O(n\log n)$有效算法,确定$T$是否有主元素。进一步,能找到一个线性时间算法吗?
    \end{enumerate}
\end{problem}
\begin{solution}
    \begin{enumerate}
        \item 因为$n-\lceil n/2\rceil\leq\lceil (n+1)/2\rceil\leq\lceil (n+1)/2\rceil$,所以若$T$存在主元素,则$T[\lceil (n+1)/2\rceil]$一定等于该主元素。
        因此只需要根据定义判断$T[\lceil (n+1)/2\rceil]$是否为主元素即可。由于$T$保证是有序的,从而相同的元素连在一起,可以利用二分查找来获取值为$T[\lceil (n+1)/2\rceil]$的
        元素下标的最值,从而可以直接得到对应的元素数量,然后再根据定义进行判断即可。伪代码在第2页中给出。时间复杂度为$\Theta(\log n)$.
        \begin{procedure}
            \SetKw{in}{in}
            \KwIn{升序数组T}
            \tcp*[h]{下标从0开始}\;
            pivot = T[$\lceil (n+1)/2\rceil$]\;
            l = 0\;
            r = n-1\;
            \While{l<r}
            {
                mid = $\lfloor (l+r)/2\rfloor$\;
                \leIf{T[mid]$\geq$pivot}{r = mid}{l = mid+1}
            }
            left = l\;
            l = 0\;
            r = n-1\;
            \While{l<r}
            {
                mid = $\lfloor (l+r+1)/2\rfloor$\;
                \leIf{T[mid]$\leq$pivot}{l = mid}{r = mid-1}
            }
            right = l\;
            \eIf{right-left+1 > n/2}{\Return{True}}{\Return{False}}
            \caption{checkPivotElementRecursively(T)}
        \end{procedure}
    \item 若$T$不保证有序,只需要先对其排序后再输入至\checkPivotElementRecursively 即可得到$O(n\log n)$的有效算法,伪代码在第2页中给出。如果想要进一步得到线性时间算法,可以使用线性选择函数
        来直接从$T$中选取第$\lceil (n+1)/2\rceil+1$大的元素,从而避免了排序,使得时间复杂度降为$\Theta(n)$,伪代码在第3页中给出。
        \begin{algorithm}
            \SetKw{in}{in}
            \SetKwFunction{Sort}{Sort}
            \KwIn{任意数组T}
            \Sort(T)\tcp*[h]{任意高效排序算法,如堆排序}\;
            \Return{\checkPivotElementRecursively(T)}\;
            \caption{checkPivotElement(T)}
        \end{algorithm}
         \begin{algorithm}
            \SetKw{in}{in}
            \SetKwFunction{linearSelect}{linearSelect}
            \KwIn{任意数组T}
          \tcp*[h]{下标从0开始}\;
            pivot = \linearSelect(T,$\lceil (n+1)/2\rceil+1$)\tcp*[h]{选取第$\lceil (n+1)/2\rceil+1$大的元素}\;
            cnt = 0\;
            \ForEach{x \in T}
            {\lIf{x == pivot}{cnt += 1}}
            \eIf{cnt > n/2}{\Return{True}}{\Return{False}}
            \caption{linearCheckPivotElement(T)}
        \end{algorithm}
    \end{enumerate}
\end{solution}
\end{document}