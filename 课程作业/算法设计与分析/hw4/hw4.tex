\documentclass[12pt, a4paper, oneside]{article}
\usepackage{amsmath, amsthm, amssymb, bm, graphicx, hyperref, mathrsfs}
\makeatletter
\newcommand{\mytitle}{\@title}
\makeatother

\usepackage[
    fontset=none,%设置中文支持,并自定义字体
    zihao=5,%默认字号为五号
    heading=true,%允许后续自定义标题样式
    scheme=chinese,%自动将文档样式中文化,例如图标标题
    punct=quanjiao,%全角式标点符号
    space=auto,%中文后接换行不会添加空格,但是英文会添加空格,需要用%手动取消
    linespread=1.3,%行距倍数是1.3
    autoindent=true,%自动缩进两个中文宽度
    ]{ctex}
\ctexset{
    today=small,%小写样式的日期
    contentsname={目录},
    % contentsname={\hspace{-\ccwd}目录},
    listfigurename={插图},
    listtablename={表格},
    figurename={图},
    tablename={表},
    abstractname={简{\quad}介},
    indexname={索引},
    appendixname={附录},
    bibname={参考文献},
    proofname={证明},
    % refname={参考文献},%只适用于beamer
    % algorithmname={算法},
    % continuation={(续)},%beamer续页的标识
    section={
        format+ = \Large\heiti\raggedright,
        name = {,\num\textbf{.}\hspace{1ex}},
        number={\num\thesection},
        nameformat={},
        numberformat={},
        aftername={},
        titleformat={},
        aftertitle={},
        runin=false,%对section级以下有用,标题是否和正文在同一段上
        beforeskip={3.5ex plus 1ex minus .2ex},%标题前垂直间距
        afterskip={2.3ex plus .2ex}%标题后垂直间距
    },
    subsection={
        format+ = \large\heiti\raggedright,
        name = {,\num\textbf{.}\hspace{1ex}},
        number={\num\thesubsection},
        nameformat={},
        numberformat={},
        aftername={},
        titleformat={},
        aftertitle={},
        runin=false,%对section级以下有用,标题是否和正文在同一段上
        beforeskip={3.5ex plus 1ex minus .2ex},%标题前垂直间距
        afterskip={2.3ex plus .2ex}%标题后垂直间距
    },
    subsubsection={
        format+ = \normalsize\heiti\raggedright,
        name = {,\num\textbf{.}\hspace{1ex}},
        number={\num\thesubsubsection},
        nameformat={},
        numberformat={},
        aftername={},
        titleformat={},
        aftertitle={},
        runin=false,%对section级以下有用,标题是否和正文在同一段上
        beforeskip={3.5ex plus 1ex minus .2ex},%标题前垂直间距
        afterskip={2.3ex plus .2ex}%标题后垂直间距
    },
    }

\title{\textbf{HW4}}
\author{范潇\quad2254298}
\date{\today}
\linespread{1.5}
\newcounter{problemname}
\newenvironment{problem}[1]{\stepcounter{problemname}\par\noindent\textbf{题目\arabic{problemname}. (#1)}}{}
\newenvironment{solution}{\par\noindent\textbf{解答. }}{}
\newenvironment{note}{\par\noindent\textbf{题目\arabic{problemname}的注记. }}{}
\usepackage{amsfonts}
\usepackage{lmodern}%解决报错
% 中文默认字体: 思源宋体,粗体为思源宋体半粗体,斜体为方正楷体_GBK
\setCJKmainfont{Source Han Serif SC}[BoldFont={Source Han Serif SC Heavy}, ItalicFont=FZKai-Z03S]
% 中文无衬线字体:思源黑体,粗体为思源黑体粗体
\setCJKsansfont{Source Han Sans CN}[BoldFont={Source Han Sans CN Heavy}]
% 中文等宽字体:微软雅黑light
\setCJKmonofont{Microsoft YaHei}[ItalicFont={Microsoft YaHei Light}]

\newCJKfontfamily\songti{Source Han Serif SC}[BoldFont={Source Han Serif SC Heavy}]
\newCJKfontfamily\xbsong{Source Han Serif SC SemiBold} % 小标宋
\newCJKfontfamily\dbsong{Source Han Serif SC Bold} % 大标宋
\newCJKfontfamily\cusong{Source Han Serif SC Heavy} % 粗宋
\newCJKfontfamily\heiti{Source Han Sans CN}[BoldFont={Source Han Sans CN Heavy}]
\newCJKfontfamily\dahei{Source Han Sans CN Medium} % 大黑
\newCJKfontfamily\cuhei{Source Han Sans CN Heavy} % 粗黑
\newCJKfontfamily\fangsong{FZFangSong-Z02S}
\newCJKfontfamily\kaiti{FZKai-Z03S}[ItalicFont={Microsoft YaHei Light}]%这个斜体只是用于lstlisting环境中的中文注释
% \newCJKfontfamily\kaiti{FZKai-Z03S}[ItalicFont={FZZJ-LZXTFSJW}]%这个斜体只是用于lstlisting环境中的中文注释
\setsansfont{Arial}
\setmonofont{Consolas}%设置西文等宽字体
\newfontfamily\code{Consolas}
\newfontfamily\num{Arial}

\usepackage{geometry}%设置整体页面布局
\geometry{a4paper}
\geometry{left=2cm,right=2cm,top=2.54cm,bottom=2.54cm}%word常规页边距
% \geometry{left=1.27cm,right=1.27cm,top=1.27cm,bottom=1.27cm}%word窄页边距
\setlength{\headheight}{13pt}%避免warning
\usepackage{fancyhdr}%必须在geometry包之后使用
\fancyhf{}
\makeatletter
\lhead{\sffamily\bfseries{2254298 范潇}}%可以使用thepage,CTEXthechapter,CTEXthesection
\makeatother
\chead{\sffamily\bfseries{\mytitle}}
\rhead{\sffamily\bfseries{- \thepage{} -}}
\renewcommand\headrulewidth{2pt}%设置眉头宽度
\pagestyle{fancy}
\usepackage[ruled,lined,longend,fillcomment,linesnumbered,resetcount,titlenotnumbered]{algorithm2e}
%参数解释:带框,按section编码,有竖线,end前带if等关键词,注释占满整行,代码部分编号(不包括输入输出、注释),每个代码块重新编号,可以调用TitleOfAlgo来打印算法标题但不作为单独的算法编码
%附带algorithm,function,procedure环境,其中function,procedure环境下,设置caption时,必须带有(),
%()之前的字符会被视为宏,可以在接下来的部分用\名字()来调用,所以推荐辅助函数用function,其中的某些展开部分用procedure,描述算法整体使用algorithm
\DontPrintSemicolon
\SetAlCapSkip{2ex}
\SetSideCommentRight
\SetFillComment
\newcommand{\forcond}{$i=0$ \KwTo $n$}
\SetKw{macro}{text}%自定义关键词
\SetKwFunction{funcmacro}{text}%自定义函数名,实际上function环境是在定义宏的同时说明了其内容
\SetKwProg{procedmacro}{text}{begin text}{end text}%自定义步骤,和function类似,但是后面两个参数可以设置开始和结尾的标志,和if等环境一样
\SetKwData{datamacro}{text}%可以用于突出特殊的变量,例如数据结构
\SetKwFunction{FRecurs}{FnRecursive}
\SetKwProg{Fn}{Function}{begin}{end}
\begin{document}
\maketitle
\begin{problem}{4-1}
按$l_1\geq l_2\geq\cdots\geq l_n$的次序来考虑程序时的一个反例输入为:10,2,6,6。

如果按照贪心算法,会将10分配给A,然后将2,6,6都分配给B。此时$\max\{\sum_{i\in A}l_i,\sum_{i\in B}l_i\}=14$。但是最优解之一为$A=\{10,2\},B=\{6,6\}$,使得
$\max\{\sum_{i\in A}l_i,\sum_{i\in B}l_i\}=12$。

按$l_1\leq l_2\leq\cdots\leq l_n$的次序来考虑程序时的一个反例输入为:1,1,2。

如果按照贪心算法,会将1分配给A,然后将1都分配给B,最后将2分配给A。此时$\max\{\sum_{i\in A}l_i,\sum_{i\in B}l_i\}=3$。但是最优解之一为$A=\{1,1\},B=\{2\}$,使得
$\max\{\sum_{i\in A}l_i,\sum_{i\in B}l_i\}=2$。

显然,对于给定输入,$\sum_{1\leq i\leq n}l_i$是定值,又由于对称性,不失一般性,可以只考虑使得$\sum_{i\in A}l_i\leq\sum_{i\in B}l_i$的解。此时,
$\sum_{1\leq i\leq n}l_i=\sum_{i\in A}l_i+\sum_{i\in B}l_i\geq 2\sum_{1\in A}l_i$,不妨设$l_i$为整数,记$\sum_{1\leq i\leq n}l_i=M$则
\[\sum_{i\in A}l_i\leq \lfloor\frac{M}{2}\rfloor\]
又因为
\[\max\{\sum_{i\in A}l_i,\sum_{i\in B}l_i\}=\max\{\sum_{i\in A}l_i,M-\sum_{i\in A}l_i\}=M-\sum_{i\in A}l_i\]
所以原问题等价于,求$A\subseteq\{1,2,\cdots,n\}$
\[\text{minimize}\quad M-\sum_{i\in A}l_i\]
\[\text{subject to}\quad \sum_{i\in A}l_i\leq \lfloor\frac{M}{2}\rfloor,l_i\in \mathbb{Z}^+ \]
也等价于
\[\text{maximize}\quad \sum_{i\in A}l_i\]
\[\text{subject to}\quad \sum_{i\in A}l_i\leq \lfloor\frac{M}{2}\rfloor,l_i\in \mathbb{Z}^+ \]

将问题“使得$\sum_{i\in A}l_i\leq x$
且$\sum_{i\in A}l_i$尽可能地大($A\subseteq\{1,2,\cdots,y\}$)”的最优解对应的$\sum_{i\in A}l_i$记为$m[x,y]$,则有
\[m[i,j] = \begin{cases}
    m[i,j-1]&v_{j}>i\\
    \max \{m[i-v_j,j-1]+v_j,m[i,j-1]\}&else\\
\end{cases}\]
显然,当$l_j>i$时,m[i,j]中不可能包含$l_j$,因为这违反了不等式约束,从而m[i,j]与m[i,j-1]对应的合法的解相同,
最优解也相同,从而m[i,j]=m[i,j-1]。
当$l_j\leq i$时,由反证法易得,若$j\notin A$,则$A\subseteq\{1,2,\cdots,j-1\}$,对应的最优解为$m[i,j-1]$的最优解;若$j \in A$,则对应的最优解为$\{j\}\cup A^{\prime}$,
其中$A^{\prime}$对应$m[i-l_j,j-1]$的最优解。

 原问题的最优解对应的和便是$m[\lfloor\frac{M}{2}\rfloor,n]$。想要计算得到$m[\lfloor\frac{M}{2}\rfloor,n]$,共需计算$\lfloor\frac{M}{2}\rfloor\cdot n$个值,所以时间复杂度为$\Theta(nM)$。

为了回溯还原最优解,需要保存过程中计算得到的$m[i,j]$值。回溯过程从$m[\lfloor\frac{M}{2}\rfloor,n]$开始,若$m[x,y]==m[x,y-1]$,则说明$m[x,y]$的最优解的构成和$m[x,y-1]$相同。
否则,说明$m[x,y]$对应的最优解中,$l_y\in A$,剩余部分由是$m[x-l_y,y-1]$对应的最优解。回溯过程所需要的时间复杂度为$\Theta(M+n)$

当$j=0$时,说明$m[i,j]$对应的最优解中$A=\varnothing$,所以$m[i,0]=0$。当$i=0$时,由于$v_k>0$,最优解中$A=\varnothing$,
从而$m[0,j]=0$。

\begin{algorithm}[H]
    \SetKw{sum}{sum}
    \SetKw{max}{max}
    \SetKw{len}{len}
    \SetKw{in}{in}
    \SetKw{range}{range}
    \SetKw{break}{break}
    \tcp*[h]{values下标从1开始}\;
    M = $\lfloor$\sum(values)/2$\rfloor$\;
    l = \len(values)\;
    初始化一个(M+1)$\times$(l+1)的全零数组dp\;
    \For{i \in [1..M]}
    {
        \For{j \in [1..l]}
        {
            \eIf{values[j]>i}
            {
                dp[i][j] = dp[i][j-1]\;
            }{
                dp[i][j] = max(dp[i][j-1],dp[i-values[j]][j-1]+values[j])
            }
        }
    }
    A=[]\tcp*[h]{开始回溯}\;
    x = M\;
    y = l\;
    \While{
        x and y
    }
    {
        \lWhile{y and dp[x][y] == dp[x][y-1]}
        {
            y -= 1
        }
        \lIf{y==0}{\break}
        A.append(y)\;
        x -= values[y-1]\;
        y -= 1\;
    }
    \Return{A}
    \caption{DynamicProgrammingApproach(values)}
\end{algorithm}
% 对于该问题的一个最优解,其中$A=\{i_1,i_2,\cdot,i_r\},i_1<i_2<\cdots<i_r$,
% $A^{\prime}=\{i_1,i_2,\cdot,i_{r}\}\backslash\{n\}$也构成问题
% “使得$\lfloor\frac{1}{2}\sum_{i\leq n-1}v_i\rfloor\geq\sum_{i\in A^{\prime}}v_i$
% 且$\sum_{i\in A^{\prime}}v_i$尽可能地大($A^{\prime}\subseteq\{1,\cdots,n-1\}$)”的最优解。事实上,如果
% $A^{\prime\prime}\neq A^{\prime}$在该子问题上比$A^{\prime}$更优,那么显然$A^{\prime\prime}\cup\{i_r\}$在原问题上
% 比$A=A^{\prime}\cup\{i_r\}$还要优,从而产生矛盾。因此该问题具有最优子结构性质,可以采用动态规划方法解决。

% 记“使得$\lfloor\frac{1}{2}\sum_{i\leq n}v_i\rfloor\geq\sum_{i\in A}v_i$
% 且$\sum_{i\in A}v_i$尽可能地大($A\subseteq\{1,2,\cdots,n\}$)”的最优解为$A_n$。显然,
% 如果$v_n$
\end{problem}
\end{document}