\documentclass[a4paper]{article}
\usepackage{ctex}
\title{HW5}
\author{范潇\quad 2254298}
\date{\today}
\begin{document}
\maketitle
现实生活中的许多事情可以抽象为我们所感兴趣的特定属性,或状态。相应地,许多问题也可以因此抽象为一张图——每个结点代表着一个状态,结点与结点之间的弧代表着
促使状态发生转移的动作,而我们的目标便是找到一条从当前状态至目标状态的解路径。

寻求解路径通常有两种方法:数学方法和启发式方法。\cite{ref1}前者更多地是站在理论和抽象的角度去寻找最优解,但是往往无法在实际中应用;后者
则结合特定的领域知识,虽然和前者相比进一步提高了算法的效率,但是通常不能保证得到的是最优解。而$A^*$算法的出现则打破了这一局面,它能够充分利用
已知的领域知识来优化计算效率,同时其背后的数学原理又保证所得解的最优性。

$A^*$算法的核心是公式
\[\hat{f}(n)=\hat{g}(n)+\hat{h}(n),\]
其中$\hat{g}(n)$是对从初始结点到当前结点$n$的最小代价$g(n)$的估计,启发式函数$\hat{h}(n)$则是对从当前结点到目标结点的最小代价$h(n)$的估计。
该算法利用上式来确定边缘队列中下一个展开的结点——优先展开$\hat{f}$值小的结点。
我们已知的领域知识完全体现在$\hat{h}(n)$中。特别地,当我们没有任何已知信息,即$\hat{h}=const$时,$A^*$算法便退化为了贪心算法。
要想在特定问题中体现出$A^*$的优势,必须结合已知的领域知识设计出合适的启发式函数$\hat{h}$。

更一般地,任何一个合格的启发式函数首先需要满足可达性和一致性这两个条件。

可达性是指算法总是能够返回最优解。可以证明,对于$A^*$算法,只要保证
\[\hat{h}(n)\leq h(n)\]
即可。也就是说,只需要保证启发式函数的估计总是“乐观的”——不会高估所需代价。对于实际问题,通过放宽问题的限制得到的解通常满足这一性质。例如,
两个城市之间直线距离作为两个城市之间的步行距离的估计便满足这一性质。

一致性对于启发式函数的要求更高,要求不等式
\[h(m,n)+\hat{h}(n)\geq\hat{h}(m)\]
恒成立。这一不等式反映的是:“对于结点$m$的估计$\hat{h}(m)$,只需考虑其自身的信息即可,如果用其他结点的估计$\hat{g(n)}$以及它们之间的联系$h(m,n)$,将不会带来提升”。
为了满足这一要求,通常需要启发式函数是“一致”的,即不会对某些结点进行特殊处理。如果启发式函数中包含了一些独立于问题本身的参数,例如随机变量,则它往往不满足一致性。\cite{ref1}
一致性要求对于$A^*$算法的意义在于,它确保了$A^*$算法在利用相同的领域知识的条件下,是所有可达性算法中最优的一个算法,即展开结点数最少的算法。

$A^*$算法具有较高的灵活性。通过改变启发式函数$\hat{h}$,它可以退化为BFS,DFS,UCS等算法。结合特定的领域知识设计出来的启发式函数也使得它可以用于
众多领域。同时,如果原先的启发式函数所需要的计算开销过大,可以适当对其进行化简,例如从$\sqrt{x^2+x^2}$化简为$(x+y)/2$,虽然这可能使得算法给出的并不是最优解。
\begin{thebibliography}{99}
\bibitem{ref1}E. Hart, N. J. Nilsson, and B. Raphael. A formal basis for the heuristic determination of minimum cost paths in graphs. IEEE Trans. Syst. Sci. and Cybernetics, SSC-4(2):100-107, 1968.
\end{thebibliography}
\end{document}