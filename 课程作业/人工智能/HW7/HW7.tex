\documentclass[12pt, a4paper, oneside]{ctexart}
\usepackage{amsmath, amsthm, amssymb, bm, graphicx, hyperref, mathrsfs}
\usepackage{makecell}
\usepackage{color}
\usepackage{geometry}%设置整体页面布局
\usepackage{pdfpages}
\geometry{a4paper}
\geometry{left=2cm,right=2cm,top=2.54cm,bottom=2.54cm}%word常规页边距
% \geometry{left=1.27cm,right=1.27cm,top=1.27cm,bottom=1.27cm}%word窄页边距
\setlength{\headheight}{13pt}%避免warning
\title{\textbf{HW7}}
\author{范潇\quad2254298}
\date{\today}
\linespread{1.5}
\newcounter{problemname}
\newenvironment{problem}[1]{\stepcounter{problemname}\par\noindent\textbf{题目\arabic{problemname}. (#1)}}{}
\newenvironment{solution}{\par\noindent\textbf{解答. }}{\\\par}
\newenvironment{note}{\par\noindent\textbf{题目\arabic{problemname}的注记. }}{\\\par}

\begin{document}
\maketitle
\begin{problem}{7.20}
    \begin{enumerate}
        \item \begin{align*}
       A\Leftrightarrow (B\lor E)&\equiv(A\land(B\lor E))\lor(\lnot A\land \lnot(B\lor E))\\
        &\equiv  (A\land(B\lor E))\lor(\lnot A\land \lnot B\land \lnot E) \\
        &\equiv (A\lor(\lnot A\land \lnot B\land \lnot E))\land((B\lor E)\lor(\lnot A\land \lnot B\land \lnot E))  \\
       & \equiv  (A\lor\lnot A)\land(A\lor\lnot B)\land(A\lor\lnot E)\land\\
       &\quad (B\lor E\lor\lnot A)\land(B\lor E\lor\lnot B)\land(B\lor E\lor\lnot E)\\
       & \equiv  (A\lor\lnot B)\land(A\lor\lnot E)\land (\lnot A\lor B\lor E)\\
        % &\equiv   (A\lor\lnot B)\land(A\lor\lnot E)\land(\lnot A\lor B)\land(B\lor\lnot E)\land(\lnot A\lor E)\land(\lnot B\lor E)\\
        % &\equiv   (A\lor\lnot B\lor E)\land(A\lor\lnot B\lor\lnot E)\land(A\lor B\lor\lnot E)\land(A\lor\lnot B\lor\lnot E)\land\\
        % &\quad(\lnot A\lor B\lor E)\land(\lnot A\lor B\lor \lnot E)\land(A\lor B\lor\lnot E)\land(\lnot A\lor \lnot B\lor E)\\
        \end{align*}
        \item \[ E\rightarrow D\equiv\lnot E\lor D\]
        \item\begin{align*}
            C\land F\rightarrow\lnot B&\equiv \lnot(C\land F)\lor\lnot B\\
            &\equiv \lnot C\lor\lnot F\lor\lnot B\\
        \end{align*}
        \item \[E\rightarrow B\equiv \lnot E\lor B\]
        \item \[B\rightarrow F\equiv \lnot B\lor F\]
        \item \[B\rightarrow C\equiv \lnot B\lor C\]
    \end{enumerate}
    所以合取范式为
    \[(A\lor\lnot B)\land(A\lor\lnot E)\land (\lnot A\lor B\lor E)\land(\lnot E\lor D)\land( \lnot C\lor\lnot F\lor\lnot B)\land(\lnot E\lor B)\land(\lnot B\lor F)\land(\lnot B\lor C)\]
\end{problem}
\begin{problem}{7.12}
   \begin{align*}
    &(A\lor\lnot B)\land(A\lor\lnot E)\land (\lnot A\lor B\lor E)\land(\lnot E\lor D)\land( \lnot C\lor\lnot F\lor\lnot B)\\
    &\land(\lnot E\lor B)\land(\lnot B\lor F)\land(\lnot B\lor C)\land\lnot(\lnot A\land\lnot B)\\
    \equiv&(A\lor\lnot B)\land(A\lor\lnot E)\land (\lnot A\lor B\lor E)\land(\lnot E\lor D)\land( \lnot C\lor\lnot F\lor\lnot B)\\
    &\land(\lnot E\lor B)\land(\lnot B\lor F)\land(\textcolor{red}{\lnot B\lor C})\land(A\lor B)\\
   \end{align*} 
   运用归结规则,得到新子句
   \[(\lnot B\lor B\lor E),(\lnot E\lor B\lor E),(B\lor E)\]
   \[A\lor\lnot A\lor E,A\lor\lnot E,\textcolor{red}{A},\lnot A\lor E\lor\lnot C\lor\lnot F,\lnot E\lor \lnot C\lor\lnot F,F\lor\lnot A\lor E,\lnot E\lor C,A\lor C\]
   \[\lnot B\lor\lnot F\]
   \[A\lor \lnot A\lor B,\textcolor{red}{\lnot A\lor B}\]
   \[\textcolor{red}{\lnot B\lor\lnot C}\]
   其中由$A,\lnot A\lor B$得到$B$,进而由$\lnot B\lor\lnot C$得到$\lnot C$,但是同时又可以由$\lnot B\lor C$得到$C$,从而得到空语句,所以得证。
\end{problem}
\end{document}