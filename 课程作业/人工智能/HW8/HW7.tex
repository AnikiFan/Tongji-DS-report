\documentclass[12pt, a4paper, oneside]{ctexart}
\usepackage{amsmath, amsthm, amssymb, bm, graphicx, hyperref, mathrsfs}
\usepackage{makecell}
\usepackage{color}
\usepackage{geometry}%设置整体页面布局
\usepackage{pdfpages}
\geometry{a4paper}
\geometry{left=2cm,right=2cm,top=2.54cm,bottom=2.54cm}%word常规页边距
% \geometry{left=1.27cm,right=1.27cm,top=1.27cm,bottom=1.27cm}%word窄页边距
\setlength{\headheight}{13pt}%避免warning
\title{\textbf{HW7}}
\author{范潇\quad2254298}
\date{\today}
\linespread{1.5}
\newcounter{problemname}
\newenvironment{problem}[1]{\stepcounter{problemname}\par\noindent\textbf{题目\arabic{problemname}. (#1)}}{}
\newenvironment{solution}{\par\noindent\textbf{解答. }}{\\\par}
\newenvironment{note}{\par\noindent\textbf{题目\arabic{problemname}的注记. }}{\\\par}

\begin{document}
\maketitle
\begin{problem}{破案问题}
    从给定事实中可以得到以下逻辑语句
    \begin{enumerate}
        % \item 论域$U=\{A,B,C\}$
        \item $Victim = A,Murder(A)\lor Murder(B)\lor Murder(C)$
        \item $\forall x( Murder(x)\Rightarrow Hate(x,Victim))$
        \item $\forall x( Hate(A,x)\Rightarrow\lnot Hate(C,x))$
        \item $\forall x (\lnot (x=B)\Rightarrow Hate(A,x))$
        \item $\forall x( Richer(A,x)\Rightarrow Hate(B,x))$
        \item $\forall x( Hate(A,x)\Rightarrow Hate(B,x))$
        \item $\lnot\exists x\forall y Hate(x,y)$
        \item $\forall x( Murder(x)\Rightarrow Richer(Victim,x))$
        \item $\lnot(A=B)$
    \end{enumerate}
    下面将其化为子句集
    \begin{align*}
       &(Victim = A) \land (Murder(A)\lor Murder(B)\lor Murder(C)) \land (\forall x (Murder(x)\Rightarrow Hate(x,Victim))) \land\\
      & (\forall x (Hate(A,x)\Rightarrow\lnot Hate(C,x)))\land (\forall x (\lnot (x=B)\Rightarrow Hate(A,x)))\land\\
      &(\forall x (Richer(A,x)\Rightarrow Hate(B,x)))\land (\forall x( Hate(A,x)\Rightarrow Hate(B,x)))\land \\
        & (\lnot\exists x\forall y Hate(x,y)) \land (\forall x( Murder(x)\Rightarrow Richer(Victim,x)))\land(\lnot(A=B))\\
      \equiv&(Victim = A) \land ( Murder(A)\lor Murder(B)\lor Murder(C)) \land (\forall x (\lnot Murder(x)\lor Hate(x,Victim))) \land\\
      & (\forall x (\lnot Hate(A,x)\lor\lnot Hate(C,x)))\land (\forall x ( (x=B)\lor Hate(A,x)))\land\\
      &(\forall x (\lnot Richer(A,x)\lor Hate(B,x)))\land (\forall x(\lnot Hate(A,x)\lor Hate(B,x)))\land \\
        & (\lnot\exists x\forall y Hate(x,y)) \land (\forall x(\lnot Murder(x)\lor Richer(Victim,x)))\land(\lnot(A=B))\\
    \equiv&(Victim = A) \land (Murder(A)\lor Murder(B)\lor Murder(C)) \land (\forall y (\lnot Murder(y)\lor Hate(y,Victim))) \land\\
      & (\forall w (\lnot Hate(A,w)\lor\lnot Hate(C,w)))\land (\forall v ( (v=B)\lor Hate(A,v)))\land\\
      &(\forall t (\lnot Richer(A,t)\lor Hate(B,t)))\land (\forall r(\lnot Hate(A,r)\lor Hate(B,r)))\land \\
        & (\forall m\exists n \lnot Hate(m,n)) \land (\forall z(\lnot Murder(z)\lor Richer(Victim,z)))\land(\lnot(A=B))\\
    \end{align*}
    消去存在量词后得
    \begin{align*}
     &(Victim = A) ( Murder(A)\lor Murder(B)\lor Murder(C))  \land (\forall y (\lnot Murder(y)\lor Hate(y,Victim))) \\
      \land& (\forall w (\lnot Hate(A,w)\lor\lnot Hate(C,w)))\land (\forall v ( (v=B)\lor Hate(A,v)))\\
      \land &(\forall t (\lnot Richer(A,t)\lor Hate(B,t)))\land (\forall r(\lnot Hate(A,r)\lor Hate(B,r)))\\
       \land &  (\forall m Hate(m,f(m))) \land (\forall z(\lnot Murder(z)\lor Richer(Victim,z)))\land(\lnot(A=B))\\
    \end{align*}
    转为前束形并略去全称量词得
    \begin{align*}
     &(Victim = A) \land  (Murder(A)\lor Murder(B)\lor Murder(C)) \land (\lnot Murder(y)\lor Hate(y,Victim)) \\
      \land&  (\lnot Hate(A,w)\lor\lnot Hate(C,w))\land  ( (v=B)\lor Hate(A,v))\\
      \land &(\lnot Richer(A,t)\lor Hate(B,t))\land (\lnot Hate(A,r)\lor Hate(B,r))\\
       \land &  ( Hate(m,f(m))) \land (\lnot Murder(z)\lor Richer(Victim,z))\land(\lnot(A=B))\\
    \end{align*}
    已经化为合取范式,消去合取符号并子句变量标准化,把$Victim$换为$A$后得
     \begin{align*}
       Murder(A)\lor Murder(B)\lor Murder(C)\\ \lnot Murder(y)\lor Hate(y,A) \\
        \lnot Hate(A,w)\lor\lnot Hate(C,w)\\  (v=B)\lor Hate(A,v)\\
      \lnot Richer(A,t)\lor Hate(B,t)\\ \lnot Hate(A,r)\lor Hate(B,r)\\
          Hate(m,f(m)) \\ \lnot Murder(z)\lor Richer(A,z)\\\lnot(A=B)\\
    \end{align*}
    设凶手为$u$,将$\lnot Murder(u)\lor ANSWER(u)$加入子句集中。再考虑事实“没有人比自己富有”,$\forall x(\lnot Richer(x,x))$,化为子句即$\lnot Richer(s,s)$
      \begin{align}
      Murder(A)\lor Murder(B)\lor Murder(C) \\ \lnot Murder(y)\lor Hate(y,A) \\
        \lnot Hate(A,w)\lor\lnot Hate(C,w)\\  (v=B)\lor Hate(A,v)\\
      \lnot Richer(A,t)\lor Hate(B,t)\\ \lnot Hate(A,r)\lor Hate(B,r)\\
          Hate(m,f(m)) \\ \lnot Murder(z)\lor Richer(A,z)\\\lnot(A=B)\\
          \lnot Murder(u)\lor ANSWER(u)\\\lnot Richer(s,s)
    \end{align}

    取置换$\sigma=\{A/v\}$,由(4),(9)得$Hate(A,A)$(12)。

    取置换$\sigma=\{A/w\}$,由(12),(3)得$\lnot Hate(C,A)$(13)。

    取置换$\sigma=\{C/y\}$,由(13),(2)得$\lnot Murder(C)$(14)。

    由(13),(1)得$Murder(B)\lor Murder(A)$(15)。

    取置换$\sigma=\{A/z,A/s\}$,由(8),(11)得$\lnot Murder(A)$(16)。

    由(16),(15)得$Murder(B)$(17)。

    取置换$\sigma=\{B/u\}$,由(10),(17)得$ANSWER(B)$(18)。

    所以凶手是B。
\end{problem}
\end{document}