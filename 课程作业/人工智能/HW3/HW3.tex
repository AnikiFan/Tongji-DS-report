\documentclass[12pt, a4paper, oneside]{ctexart}
\usepackage{amsmath, amsthm, amssymb, bm, graphicx, hyperref, mathrsfs}
\usepackage{makecell}
\usepackage{color}
\usepackage{geometry}%设置整体页面布局
\usepackage{pdfpages}
\geometry{a4paper}
\geometry{left=2cm,right=2cm,top=2.54cm,bottom=2.54cm}%word常规页边距
% \geometry{left=1.27cm,right=1.27cm,top=1.27cm,bottom=1.27cm}%word窄页边距
\setlength{\headheight}{13pt}%避免warning
\title{\textbf{HW3}}
\author{范潇\quad2254298}
\date{\today}
\linespread{1.5}
\newcounter{problemname}
\newenvironment{problem}[1]{\stepcounter{problemname}\par\noindent\textbf{题目\arabic{problemname}. (#1)}}{}
\newenvironment{solution}{\par\noindent\textbf{解答. }}{\\\par}
\newenvironment{note}{\par\noindent\textbf{题目\arabic{problemname}的注记. }}{\\\par}

\begin{document}
\maketitle
当遇到边缘队列中有多个优先级相同的结点时,按照字典序选择下一个展开的结点。
\section{DFS}
\begin{figure}[htbp]
    \centering
    \begin{tabular}{ccc}
       展开结点&边缘队列&goal test\\
              &S      &          \\
       S      &SA,SB,SC     &         \\
       SA      &SB,SC,SAD,SAE,SAG     &         \\
       SAD      &SB,SC,SAE,SAG     &         \\
       SAE      &SB,SC,SAG     &         \\
       SAG      &SB,SC     &   $\surd$      \\
    \end{tabular}
\end{figure}
解序列为SAG
\section{深度受限搜索与IDS}
IDS本质上是深度上限逐渐增加的深度受限搜索。深度受限搜索中的 $l$对应的便是IDS中的 $limit$。

l = 1时,无法得到解序列:
\begin{figure}[!h]
    \centering
    \begin{tabular}{ccc}
       展开结点&边缘队列&goal test\\
              &S      &          \\
       S      &SA,SB,SC     &         \\
       SA      &SB,SC     &         \\
       SB      &SC     &         \\
       SC      &     &         \\
    \end{tabular}
    \caption{l = 1}
\end{figure}

l = 2时,由于题中的搜索树的深度为2,所以得到的搜索序列与DFS给出的一致,解序列为SAG:
\begin{figure}[htbp]
    \centering
    \begin{tabular}{ccc}
       展开结点&边缘队列&goal test\\
              &S      &          \\
       S      &SA,SB,SC     &         \\
       SA      &SB,SC,SAD,SAE,SAG     &         \\
       SAD      &SB,SC,SAE,SAG     &         \\
       SAE      &SB,SC,SAG     &         \\
       SAG      &SB,SC     &   $\surd$      \\
    \end{tabular}
    \caption{l = 2}
\end{figure}

IDS算法先得到l=1时的搜索序列,由于没有达到目标,会继续得到l=2的搜索序列,最后得到解序列 SAG。
\section{BFS}
\begin{table}[!hb]
    \centering 
    \begin{tabular}{ccc}
   展开结点&边缘队列&goal test\\
              &S      &          \\
       S      &SA,SB,SC     &         \\
       SA      &SB,SC,SAD,SAE,SAG     &         \\
       SB      &SC,SAD,SAE,SAG,SBG     &         \\
       SC      &SAD,SAE,SAG,SBG,SCG     &         \\
       SAD      &SAE,SAG,SBG,SCG     &         \\
       SAE      &SAG,SBG,SCG     &         \\
       SAG      &SBG,SCG     &   $\surd$      \\
    \end{tabular}
\end{table}
解序列为SAG
\section{UCS}
解序列为SBG,对应的路径耗散为9
\begin{table}[!h]
    \centering 
    \begin{tabular}{ccc}
   展开结点&边缘队列&goal test\\
              &S $^0$      &          \\
       S $^0$     &SA $^1$,SB $^5$,SC $^8$     &         \\
        SA $^1$    &SB $^5$,SC $^8$,SAD $^4$,SAE $^8$,SAG $^{10}$     &         \\
        SAD $^4$   &SB $^5$,SC $^8$,SAE $^8$,SAG $^{10}$     &         \\
        SB $^5$ &SC $^8$,SAE $^8$,SAG $^{10}$,SBG $^9$     &         \\
        SAE $^8$ &SC $^8$,SAG $^{10}$,SBG $^9$     &         \\
         SC $^8$&SAG $^{10}$,SBG $^9$,SCG $^{13}$     &         \\
         SBG $^9$&SAG $^{10}$,SCG $^{13}$     &     $\surd$    \\
    \end{tabular}
\end{table}

\end{document}