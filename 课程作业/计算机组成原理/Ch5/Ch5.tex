\documentclass[12pt, a4paper, oneside]{ctexart}
\usepackage{amsmath, amsthm, amssymb, bm, graphicx, hyperref, mathrsfs}

\title{\textbf{第五章}}
\author{范潇\quad2254298}
\date{\today}
\linespread{1.5}
\newcounter{problemname}
\newenvironment{problem}[1]{\stepcounter{problemname}\par\noindent\textbf{题目\arabic{problemname}. (#1)}}{}
\newenvironment{solution}{\par\noindent\textbf{解答. }}{\\\par}
\newenvironment{note}{\par\noindent\textbf{题目\arabic{problemname}的注记. }}{\\\par}
\begin{document}
\maketitle
\begin{problem}{5.1}
\end{problem}
\begin{solution}
    理论上最多可以有$2^4$条双操作数指令,多余的$2^4-K$条用于扩展为单操作数指令,最多可以有$(2^4-K)\times 2^{6}$条单操作数指令,使用了$x$条后,
    剩余的用于拓展为无操作数指令,最多有$((2^4-K)\times 2^{6}-x)\times 2^6$条。因此
    \[((2^4-K)\times 2^{6}-x)\times 2^6\geq L\]
    解得
    \[x\leq 2^6\times(2^4-K)-\frac{L}{2^6}\]
    所以最多有$\lfloor 2^6\times(2^4-K)-\frac{L}{2^6}\rfloor$条。
\end{solution}
\begin{problem}{5.2}
\end{problem}
\begin{solution}
    2000H+03A0H+3FH=23DFH.

    2B00H+3FH=2B3FH

    所以变址编址和相对编址的访存有效地址分别为23DFH和2B3FH。
\end{solution}
\begin{problem}{5.3}
\end{problem}
\begin{solution}
    \begin{enumerate}
        \item 取出的数据为2800H,转移地址为2B3FH
        \item 取出的数据为2300H
    \end{enumerate}.
\end{solution}
\begin{problem}{5.4}
\end{problem}
\begin{solution}
    加法指令会根据运算结果改变状态位NZVC。如果结果为零,则将Z置1;如果为无符号数相加,若最高位产生进位或借位,则将C置1;如果为有符号数相加,如果结果为负数,则将N置1,如果溢出,则将V置1.
\end{solution}
\begin{problem}{5.11}
\end{problem}
\begin{solution}
    \begin{enumerate}
        \item CISC指令系统复杂庞大,指令数目一般有200-300条;RISC优先选取使用频率最高的一些简单指令,和一些很有用但不复杂的指令,避免复杂指令,且指令条数较少,一般小于100条
        \item CISC寻址方式多,指令格式多,指令长度不固定;RISC指令长度固定,指令格式种类少,寻址方式种类少,指令之间各字段的划分比较一致,各字段的功能也比较规整
        \item CISC可访存指令不受限制;RISC只有取数/存数指令访问存储器,其余指令都在寄存器之间进行
        \item CISC各种指令使用频率和执行时间相差很大;RISC大部分指令在一个或小于一个机器周期内完成
        \item CISC大多数采用微程序控制器;RISC以硬布线控制逻辑为主,不用或少用微指令码控制
    \end{enumerate}
    .
\end{solution}
\begin{problem}{5.12}
\end{problem}
\begin{solution}
    X+Y = 01111,结果为正数,无溢出和进位,状态位NZVC为0000
    
    X-Y = X+(-Y)= 00101+10110=11011结果为负数,无溢出和进位。状态位NZVC为1000
\end{solution}
\begin{problem}{5.13}
\end{problem}
\begin{solution}
    \begin{table}[h]
        \centering
        \begin{tabular}{cc}
           指令&编码\\
           \hline $I_1$&00\\
           $I_2$&10\\
           $I_3$&010\\
           $I_4$&110\\
           $I_5$&0110\\
           $I_6$&0111\\
            $I_7$&1110\\
            $I_8$&11110\\
            $I_9$&111110\\
            $I_{10}$&111111\\
        \end{tabular}
    \end{table}

    平均长度为
    \[2\times  0.55+3\times 0.2+4\times 0.19+5\times 0.03+6\times 0.03=2.79\]
\end{solution}
\end{document}