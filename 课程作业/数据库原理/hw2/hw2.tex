% this is a template for papers or notes written in Chinese
\documentclass[a4paper]{article}

\usepackage[
    fontset=none,%设置中文支持,并自定义字体
    zihao=5,%默认字号为五号
    heading=true,%允许后续自定义标题样式
    scheme=chinese,%自动将文档样式中文化,例如图标标题
    punct=quanjiao,%全角式标点符号
    space=auto,%中文后接换行不会添加空格,但是英文会添加空格,需要用%手动取消
    linespread=1.3,%行距倍数是1.3
    autoindent=true,%自动缩进两个中文宽度
    ]{ctex}
\ctexset{
    % tday=small,%小写样式的日期
    contentsname={目录},
    listfigurename={插图},
    listtablename={表格},
    figurename={图},
    tablename={表},
    abstractname={摘要},
    indexname={索引},
    appendixname={附录},
    bibname={参考文献},
    proofname={证明},
    % refname={参考文献},%只适用于beamer
    % algorithmname={算法},
    % continuation={(续)},%beamer续页的标识
    section={
        format+ = \Large\heiti\raggedright,
        name = {,\num\textbf{.}\hspace{1ex}},
        number={\num\thesection},
        nameformat={},
        numberformat={},
        aftername={},
        titleformat={},
        aftertitle={},
        runin=false,%对section级以下有用,标题是否和正文在同一段上
        beforeskip={3.5ex plus 1ex minus .2ex},%标题前垂直间距
        afterskip={2.3ex plus .2ex}%标题后垂直间距
    },
    subsection={
        format+ = \large\heiti\raggedright,
        name = {,\num\textbf{.}\hspace{1ex}},
        number={\num\thesubsection},
        nameformat={},
        numberformat={},
        aftername={},
        titleformat={},
        aftertitle={},
        runin=false,%对section级以下有用,标题是否和正文在同一段上
        beforeskip={3.5ex plus 1ex minus .2ex},%标题前垂直间距
        afterskip={2.3ex plus .2ex}%标题后垂直间距
    },
    subsubsection={
        format+ = \normalsize\heiti\raggedright,
        name = {,\num\textbf{.}\hspace{1ex}},
        number={\num\thesubsubsection},
        nameformat={},
        numberformat={},
        aftername={},
        titleformat={},
        aftertitle={},
        runin=false,%对section级以下有用,标题是否和正文在同一段上
        beforeskip={3.5ex plus 1ex minus .2ex},%标题前垂直间距
        afterskip={2.3ex plus .2ex}%标题后垂直间距
    },
    }

% 中文默认字体: 思源宋体,粗体为思源宋体半粗体,斜体为方正楷体_GBK
\setCJKmainfont{Source Han Serif SC}[BoldFont={Source Han Serif SC Heavy}, ItalicFont=FZKai-Z03S]
% 中文无衬线字体:思源黑体,粗体为思源黑体粗体
\setCJKsansfont{Source Han Sans CN}[BoldFont={Source Han Sans CN Heavy}]
% 中文等宽字体:微软雅黑light
\setCJKmonofont{Microsoft YaHei}[ItalicFont={Microsoft YaHei Light}]

\newCJKfontfamily\songti{Source Han Serif SC}[BoldFont={Source Han Serif SC Heavy}]
\newCJKfontfamily\xbsong{Source Han Serif SC SemiBold} % 小标宋
\newCJKfontfamily\dbsong{Source Han Serif SC Bold} % 大标宋
\newCJKfontfamily\cusong{Source Han Serif SC Heavy} % 粗宋
\newCJKfontfamily\heiti{Source Han Sans CN}[BoldFont={Source Han Sans CN Heavy}]
\newCJKfontfamily\dahei{Source Han Sans CN Medium} % 大黑
\newCJKfontfamily\cuhei{Source Han Sans CN Heavy} % 粗黑
\newCJKfontfamily\fangsong{FZFangSong-Z02S}
\newCJKfontfamily\kaiti{FZKai-Z03S}[ItalicFont={Microsoft YaHei Light}]%这个斜体只是用于lstlisting环境中的中文注释
% \newCJKfontfamily\kaiti{FZKai-Z03S}[ItalicFont={FZZJ-LZXTFSJW}]%这个斜体只是用于lstlisting环境中的中文注释
\setsansfont{Arial}
\setmonofont{Consolas}%设置西文等宽字体
\newfontfamily\code{Consolas}
\newfontfamily\num{Arial}

\usepackage{geometry}%设置整体页面布局
\geometry{a4paper}
\geometry{left=2cm,right=2cm,top=2.54cm,bottom=2.54cm}%word常规页边距
% \geometry{left=1.27cm,right=1.27cm,top=1.27cm,bottom=1.27cm}%word窄页边距
\setlength{\headheight}{13pt}%避免warning


\usepackage{fancyhdr}%必须在geometry包之后使用
\fancyhf{}
\makeatletter
\lhead{{\dahei \@title}}%可以使用thepage,CTEXthechapter,CTEXthesection
\makeatother
\rhead{\textbf{\num- \thepage{} -}}
\renewcommand\headrulewidth{1.5pt}%设置眉头宽度
\pagestyle{fancy}

\usepackage[ruled,algosection,lined,longend,fillcomment,linesnumbered,resetcount,titlenotnumbered]{algorithm2e}
%参数解释:带框,按section编码,有竖线,end前带if等关键词,注释占满整行,代码部分编号(不包括输入输出、注释),每个代码块重新编号,可以调用TitleOfAlgo来打印算法标题但不作为单独的算法编码
%附带algorithm,function,procedure环境,其中function,procedure环境下,设置caption时,必须带有(),
%()之前的字符会被视为宏,可以在接下来的部分用\名字()来调用,所以推荐辅助函数用function,其中的某些展开部分用procedure,描述算法整体使用algorithm
\DontPrintSemicolon
\SetAlCapSkip{2ex}
\SetSideCommentRight
\SetFillComment
\newcommand{\forcond}{$i=0$ \KwTo $n$}
\SetKw{downto}{downto}%自定义关键词
\SetKwFunction{funcmacro}{text}%自定义函数名,实际上function环境是在定义宏的同时说明了其内容
\SetKwProg{procedmacro}{text}{begin text}{end text}%自定义步骤,和function类似,但是后面两个参数可以设置开始和结尾的标志,和if等环境一样
\SetKwData{datamacro}{text}%可以用于突出特殊的变量,例如数据结构
\SetKwFunction{FRecurs}{FnRecursive}
\SetKwProg{Fn}{Function}{begin}{end}

\usepackage[strict]{changepage}
\usepackage{framed}%色块支持
\definecolor{formalshade}{rgb}{0.95,0.95,1} % 文本框颜色
% ------------------******-------------------
% 注意行末需要把空格注释掉,不然画出来的方框会有空白竖线
\newenvironment{formal}{%
\def\FrameCommand{%
\hspace{1pt}%
{\color{DarkBlue}\vrule width 2pt}%
{\color{formalshade}\vrule width 4pt}%
\colorbox{formalshade}%
}%
\MakeFramed{\advance\hsize-\width\FrameRestore}%
\noindent\hspace{-4.55pt}% disable indenting first paragraph
\begin{adjustwidth}{}{7pt}%
\vspace{2pt}\vspace{2pt}%
}
{%
\vspace{2pt}\end{adjustwidth}\endMakeFramed%
}

% 自定义标题格式
\makeatletter
\renewcommand{\maketitle}{
  \begin{center}
    \thispagestyle{fancy}
    {\quad}\\
    \vspace{0.1\textheight}
    {\huge\sffamily\bfseries\@title}\\ % 标题字体大小、粗体、颜色
    \vspace{2em} % 标题与作者名之间的垂直空间
    {\large\sffamily\@author} \\
  \end{center}
}
\makeatother

\usepackage{graphicx}

\title{作业{\hspace{1ex}}HW2}
\author{姓名:范潇{\quad}学号:2254298{\quad}日期:\today}
\date{}
\begin{document}
\maketitle
% \thispagestyle{fancy}%用于单独设置某页的样式,此处用于设置标题页的格式
在本次作业中我遵循\emph{SQLite}的语法。
\section{(3.9)}
\begin{lstlisting}
--a
select ID,person_name,city
from employee natural join works
where company_name = 'First Bank Corporation';   
\end{lstlisting}
\begin{lstlisting}
--b
select ID,person_name,city
from employee natural join works
where company_name = 'First Bank Corporation' and salary > 100000;
\end{lstlisting}
\begin{lstlisting}
--c
-- to make sure that those who don't work will be included in the resulting relation,
-- I join relation employee with relation works instead of carrying out the query directly on the relation works only
select ID
from employee natural left join works
where company_name <> 'First Bank Corporation' or company_name is null ; 
\end{lstlisting}
\begin{lstlisting}
--d
with target(ID,salary) as (
    select ID,salary
    from employee natural join works
    where company_name = 'Small Bank Corporation'
)
select ID
from target
where salary = (select max(salary) from target);    
\end{lstlisting}
\begin{lstlisting}
--e
with target(city) as (
    select city from company where company_name = 'Small Bank Corporation'
)
select company_name
from(
    select company_name,city
    from company
    where city in target
    -- pick out the valid companies
)
group by company_name
having count(*) = (select count(*) from target);    
\end{lstlisting}
\begin{lstlisting}
--f
with total(company_name,num) as (
    select company_name,count(*)
    from company natural join works
    group by company_name
)
select company_name
from total
where num = (select max(num) from total);    
\end{lstlisting}
\begin{lstlisting}
--g
with avg_salary(company_name,salary)as (
    select company_name, avg(works.salary)
    from company natural join works
    group by company_name
)
select company_name from avg_salary
where salary > (select salary from avg_salary where company_name = 'First Bank Corporation');
\end{lstlisting}
\section{(3.15)}
\begin{lstlisting}
--a
with target(branch_name) as (
    select branch_name
    from branch
    where branch_city = 'Brooklyn'
)
select customer_name
from customer natural join depositor natural join account
where branch_name in target
group by customer_name
having count(distinct branch_name) = (select count(*) from target);
\end{lstlisting}

\begin{lstlisting}
--b
select sum(amount)
from loan ;
\end{lstlisting}

\begin{lstlisting}
--c
select branch_name
from branch
where assets > (
    select min(assets)
    from branch
    where branch_city = 'Brooklyn'
    );
\end{lstlisting}
\section{(3.16)}
\begin{lstlisting}
--a
select ID,person_name
from employee natural join works natural join company ;
\end{lstlisting}

\begin{lstlisting}
--b
select E.ID,E.person_name
from employee as E natural join manages ,employee as M
where manager_id = M.ID and M.city = E.city and M.street = E.street;
\end{lstlisting}

\begin{lstlisting}
--c
with target(company_name,avg_salary) as (
    select company_name,avg(salary)
    from  works
    group by company_name
)
select ID,person_name
from employee natural join works natural join target
where salary>avg_salary;
\end{lstlisting}

\begin{lstlisting}
--d
with target(company_name,tot_salary) as (
    select company_name,sum(salary)
    from works
    group by company_name
    -- assume there is no company without anyone working for it
)
select company_name
from target
where tot_salary = (select min(tot_salary)from target);
\end{lstlisting}
\end{document}