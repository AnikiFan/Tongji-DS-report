\documentclass[12pt]{article}
\usepackage[margin=1in]{geometry}
\usepackage[all]{xy}


\usepackage{amsmath,amsthm,amssymb,color,latexsym}
\usepackage{geometry}        
\geometry{letterpaper}    
\usepackage{graphicx}

\newtheorem{problem}{Problem}

\newenvironment{solution}[1][\it{Solution}]{\textbf{#1. } }{$\square$}


\begin{document}
\noindent DataBase System Concept \hfill Homework \# 1\\
Student ID : 2254298

\hrulefill


\begin{problem}[2.7]
    Consider the bank database of Figure 2.18. Give an expression in the relational algebra for each of the following queries:
    \begin{enumerate}
        \item Find the name of each branch located in “Chicago”.
        \item Find the ID of each borrower who has a loan in branch “Downtown”.
    \end{enumerate}
\end{problem}
\begin{solution}
    The information we need for the first question are in the table \emph{branch}.
    \[\Pi_{branch\_name}(\sigma_{branch\_city="Chicago"}(branch))\]

    For the second question, we need to use the table \emph{borrower} and \emph{loan}.
    Since they share the attribute \emph{loan\_number}, I choose to use natural join.
    \[\Pi_{ID}((\sigma_{branch\_name="Downtown"}(loan))\Join borrower)\]
\end{solution} 

\begin{problem}[2.14]
    Consider the employee database of Figure 2.17. Give an expression in the relational algebra to express each of the following queries:
    \begin{enumerate}
        \item Find the ID and name of each employee who works for “BigBank”.
        \item Find the ID, name, and city of residence of each employee who works for “BigBank”.
        \item Find the ID, name, street address, and city of residence of each employee who works for “BigBank” and earns more than \$10000.
        \item Find the ID and name of each employee in this database who lives in the same city as the company for which she or he works.
    \end{enumerate}
\end{problem}
\begin{solution}
    \[\Pi_{person\_name}(employee\Join\sigma_{company\_name="BigBank"}(works))\]
    \[\Pi_{person\_name,city}(employee\Join\sigma_{company\_name="BigBank"}(works))\]
    \[\Pi_{person\_name,street,city}(employee\Join\sigma_{company\_name="BigBank"\land salary>10000}(works))\]
    \[\Pi_{person\_name}((employee\Join works)\Join company)\]
    where the second natural join guarentee that the selected employee live in the same city as the compnany she or he works for.
\end{solution}
\begin{problem}[2.15]
    Consider the bank database of Figure 2.18. Give an expression in the relational algebra for each of the following queries:
    \begin{enumerate}
        \item Find each loan number with a loan amount greater than \$10000.
        \item Find the ID of each depositor who has an account with a balance greater than \$6000.
        \item Find the ID of each depositor who has an account with a balance greater than \$6000 at the “Uptown” branch.
    \end{enumerate}
\end{problem}
\begin{solution}
    \[\Pi_{loan\_number}(\sigma_{amount>10000}(loan))\]
    \[\Pi_{ID}(\sigma_{account\_balance>6000}(account)\Join depositor)\]
    \[\Pi_{ID}(\sigma_{account\_balance>6000\land branch\_name="Downtown"}(account)\Join depositor)\]
    to keep the relation relatively small given by the join operations, I use the select operation first.
\end{solution}
%%%%%%%%%%%%%%%%%%%%%%%%%%%%%%%%%%%%%%%%%%%%%%%%%%%%%%%%
%%%%%Continue with this pattern if there are more%%%%%%%
%%%%%%%%%%%%%%%%%homework problems%%%%%%%%%%%%%%%%%%%%%%
%%%%%%%%%%%%%%%%%%%%%%%%%%%%%%%%%%%%%%%%%%%%%%%%%%%%%%%%
 
\end{document}