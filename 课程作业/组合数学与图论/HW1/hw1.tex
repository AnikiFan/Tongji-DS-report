\documentclass[12pt, a4paper, oneside]{article}
\usepackage{amsmath, amsthm, amssymb, bm, graphicx, hyperref, mathrsfs}
\makeatletter
\newcommand{\mytitle}{\@title}
\makeatother

\usepackage[
    fontset=none,%设置中文支持,并自定义字体
    zihao=5,%默认字号为五号
    heading=true,%允许后续自定义标题样式
    scheme=chinese,%自动将文档样式中文化,例如图标标题
    punct=quanjiao,%全角式标点符号
    space=auto,%中文后接换行不会添加空格,但是英文会添加空格,需要用%手动取消
    linespread=1.3,%行距倍数是1.3
    autoindent=true,%自动缩进两个中文宽度
    ]{ctex}
\ctexset{
    today=small,%小写样式的日期
    contentsname={目录},
    % contentsname={\hspace{-\ccwd}目录},
    listfigurename={插图},
    listtablename={表格},
    figurename={图},
    tablename={表},
    abstractname={简{\quad}介},
    indexname={索引},
    appendixname={附录},
    bibname={参考文献},
    proofname={证明},
    % refname={参考文献},%只适用于beamer
    % algorithmname={算法},
    % continuation={(续)},%beamer续页的标识
    section={
        format+ = \Large\heiti\raggedright,
        name = {,\num\textbf{.}\hspace{1ex}},
        number={\num\thesection},
        nameformat={},
        numberformat={},
        aftername={},
        titleformat={},
        aftertitle={},
        runin=false,%对section级以下有用,标题是否和正文在同一段上
        beforeskip={3.5ex plus 1ex minus .2ex},%标题前垂直间距
        afterskip={2.3ex plus .2ex}%标题后垂直间距
    },
    subsection={
        format+ = \large\heiti\raggedright,
        name = {,\num\textbf{.}\hspace{1ex}},
        number={\num\thesubsection},
        nameformat={},
        numberformat={},
        aftername={},
        titleformat={},
        aftertitle={},
        runin=false,%对section级以下有用,标题是否和正文在同一段上
        beforeskip={3.5ex plus 1ex minus .2ex},%标题前垂直间距
        afterskip={2.3ex plus .2ex}%标题后垂直间距
    },
    subsubsection={
        format+ = \normalsize\heiti\raggedright,
        name = {,\num\textbf{.}\hspace{1ex}},
        number={\num\thesubsubsection},
        nameformat={},
        numberformat={},
        aftername={},
        titleformat={},
        aftertitle={},
        runin=false,%对section级以下有用,标题是否和正文在同一段上
        beforeskip={3.5ex plus 1ex minus .2ex},%标题前垂直间距
        afterskip={2.3ex plus .2ex}%标题后垂直间距
    },
    }

\title{\textbf{HW1}}
\author{范潇\quad2254298}
\date{\today}
\linespread{1.5}
\newcounter{problemname}
\newenvironment{problem}[1]{\stepcounter{problemname}\par\noindent\textbf{题目\arabic{problemname}. (#1)}}{}
\newenvironment{solution}{\par\noindent\textbf{解答. }}{}
\newenvironment{note}{\par\noindent\textbf{题目\arabic{problemname}的注记. }}{}
\usepackage{amsfonts}
\usepackage{lmodern}%解决报错
% 中文默认字体: 思源宋体,粗体为思源宋体半粗体,斜体为方正楷体_GBK
\setCJKmainfont{Source Han Serif SC}[BoldFont={Source Han Serif SC Heavy}, ItalicFont=FZKai-Z03S]
% 中文无衬线字体:思源黑体,粗体为思源黑体粗体
\setCJKsansfont{Source Han Sans CN}[BoldFont={Source Han Sans CN Heavy}]
% 中文等宽字体:微软雅黑light
\setCJKmonofont{Microsoft YaHei}[ItalicFont={Microsoft YaHei Light}]

\newCJKfontfamily\songti{Source Han Serif SC}[BoldFont={Source Han Serif SC Heavy}]
\newCJKfontfamily\xbsong{Source Han Serif SC SemiBold} % 小标宋
\newCJKfontfamily\dbsong{Source Han Serif SC Bold} % 大标宋
\newCJKfontfamily\cusong{Source Han Serif SC Heavy} % 粗宋
\newCJKfontfamily\heiti{Source Han Sans CN}[BoldFont={Source Han Sans CN Heavy}]
\newCJKfontfamily\dahei{Source Han Sans CN Medium} % 大黑
\newCJKfontfamily\cuhei{Source Han Sans CN Heavy} % 粗黑
\newCJKfontfamily\fangsong{FZFangSong-Z02S}
\newCJKfontfamily\kaiti{FZKai-Z03S}[ItalicFont={Microsoft YaHei Light}]%这个斜体只是用于lstlisting环境中的中文注释
% \newCJKfontfamily\kaiti{FZKai-Z03S}[ItalicFont={FZZJ-LZXTFSJW}]%这个斜体只是用于lstlisting环境中的中文注释
\setsansfont{Arial}
\setmonofont{Consolas}%设置西文等宽字体
\newfontfamily\code{Consolas}
\newfontfamily\num{Arial}

\usepackage{geometry}%设置整体页面布局
\geometry{a4paper}
\geometry{left=2cm,right=2cm,top=2.54cm,bottom=2.54cm}%word常规页边距
% \geometry{left=1.27cm,right=1.27cm,top=1.27cm,bottom=1.27cm}%word窄页边距
\setlength{\headheight}{13pt}%避免warning
\usepackage{fancyhdr}%必须在geometry包之后使用
\fancyhf{}
\lhead{\sffamily\bfseries{范潇 2254298}}%可以使用thepage,CTEXthechapter,CTEXthesection
\chead{\sffamily\bfseries{\mytitle}}
\rhead{\sffamily\bfseries{- \thepage{} -}}
\renewcommand\headrulewidth{2pt}%设置眉头宽度
\pagestyle{fancy}
\begin{document}

\maketitle

\begin{problem}{1.3}
    任取 $n+1$个整数,求证其中至少有两个数,它们的差是 $n$的倍数。
\end{problem}
\begin{solution}
    记这 $n+1$个整数按升序排列为 $a_i,i = 0,\cdots, n$,令 $d_i = a_i-a_0 \mod n,i = 1,\cdots,n$,
    若 $\exists i,s.t. d_i = 0$ ,则 $a_i,a_0$之差为 $n$的倍数。否则,因为 $\forall i,0< d_i<n$,
    由鸽笼原理可知, $d_1,\cdots,d_n$这 $n$个数中,必有两个数 $d_s,d_t(s\neq t)$取值相同,则 $a_s,a_t$的差为 $n$的倍数。
    \qed
\end{solution}
\begin{problem}{1.6}
    从 $1,2,\cdots,200$中任取 $100$个整数,其中之一小于 $16$,那么必有两个数,一个能被另一个整除。
\end{problem}
\begin{problem}{1.8}
    任意给定 $52$个数,它们之中有两个数,其和或差是100的倍数。
\end{problem}
\begin{solution}
    记这 $52$个整数按升序排列为 $a_i,i = 1,\cdots, 52$,令 $d_i = a_i \mod 100,i = 1,\cdots,52$。
    其中每个数恰好属于下列 $51$个集合中的一个
    \[\{0\},\{1,99\},\{2,98\},\cdots,\{50\}\]
    由鸽笼原理可知,必有两个数 $d_i,d_j$ 属于同一集合。若 $i=j$,则 $a_i,a_j$之差为 $100$的倍数,否则 它们之和为 $100$的倍数。
    \qed
\end{solution}
\begin{problem}{1.10}
    在坐标平面上任意给定 $9$个整点,是否必有一个以它们中的三个点为顶点的三角形,其重心也是整点?
\end{problem}
\begin{solution}
    只需证明任给 $9$个整数,其中必有三个数之和为 $3$的倍数即可。
    
    任取 $9$个整数 $a_1,\cdots,a_9$, 令 $d_i=a_1\mod 3$,则 $d_i\in \{0,1,2\}$。
    由鸽笼原理可知,必有 $d_1,\cdots,d_9$中必有三个数相同,它们对应的 $a_i$之和便是 $3$的倍数。
\end{solution}
\begin{problem}{1.12}
    对任意的整数 $N$,存在着 $N$的一个倍数,使得它仅由数字 $0$和 $7$组成。
\end{problem}
\begin{solution}
    令 $a_i = \sum_{j = 0}^i7\times 10^j, i = 0,\cdots, N-1$, 令 $d_i= a_i\mod N, i = 0,\cdots,N-1$。
    若存在 $i$ \hspace{1ex}s.t.\hspace{1ex} $d_i=0$,则已经找到了所要求的数 $a_i$ 。否则
    $d_i\in\{1,\cdots,N-1\},i = 0,\cdots,N-1$,由鸽笼原理可知,必有两个数 $d_s,d_t$相同,则 $\lvert a_s-a_t\rvert$便是所要求的数。
\end{solution}
\begin{problem}{2.2}
   比 $5400$大的四位整数中, 数字 $2,7$不出现,且各位数字不同的整数由多少个? 
\end{problem}
\begin{solution}
    在 $6000-9999$中,符合条件的整数的个数为
    \[3\cdot 7\cdot 6\cdot 5\]
    各因子代表各位上能取的数字个数。

    在 $5500-5999$中,符合条件的整数的个数为
    \[1\cdot 4\cdot 6\cdot 5\]
    各因子代表各位上能取的数字个数。

    在 $5401-5499$中,符合条件的整数的个数为
    \[5+5\cdot 5 \]
    根据十位是否为 $0$使用加法原理。

    综上,共有 $780$种。
\end{solution}
\begin{problem}{2.5}
    现有 $100$件产品,从其中任意抽出 $3$件。
    \begin{enumerate}
        \item 共有多少种抽法?
        \item 如果 $100$件产品中有 $2$件次品,那么抽出的产品中至少有 $1$件次品的概率是多少?
        \item 如果 $100$件产品中有 $2$件次品,那么抽出的产品中恰好有 $1$件是次品的概率是多少?
    \end{enumerate}
\end{problem}
\begin{solution}
    \begin{enumerate}
        \item $\binom{100}{3}$
        \item $1-\binom{98}{3}/\binom{100}{3}$
        \item $2\binom{98}{2}/\binom{100}{3}$
    \end{enumerate}
\end{solution}
\begin{problem}{2.7}
   $8$个棋子大小相同, 其中 $5$个红的, $3$个蓝的。把它们放在 $8\times 8$的棋盘上,每行、每列只放一个,问有多少种方法?若在 $12\times 12$的棋盘上,结果如何? 
\end{problem}
\begin{solution}
    \begin{enumerate}
        \item $8!\binom{8}{3}$,先确定各个放置棋的位置,再确定哪些列放蓝棋
        \item $\binom{12}{8}^28!\binom{8}{3}$,先缩小棋盘范围
    \end{enumerate}
\end{solution}
\begin{problem}{2.8}
   有纪念章 $4$枚、纪念册 $6$本,赠送给 $10$位同学,每人得一件,共有多少种不同的送法?
\end{problem}
\begin{solution}
    $\binom{10}{4}$
\end{solution}
\begin{problem}{2.9}
   \begin{enumerate}
    \item 从整数 $1,2,\cdots,100$中选出两个数, 使得它们的差正好是 $7$,有多少种不同的选法?
    \item 如果要求选出的两个数之差小于等于 $7$,又有多少种不同的选法?
   \end{enumerate} 
\end{problem}
\begin{solution}
    \begin{enumerate}
        \item $93$
        \item $93+94+\cdots +99=672$
    \end{enumerate}
\end{solution}
\begin{problem}{2.14}
   有 $n$个不同的整数,从中取出两组来,要求第一组里的最小数大于第二组里的最大数,问有多少种方案? 
\end{problem}
\begin{solution}
    \[\sum_{i=1}^n2^{i-1}(2^{n-i}-1)=\sum_{i=1}^n2^{n-1}-2^{i-1}=n2^{n-1}-2^{n-1}+1\]
    按照第二组的最大数的取值分类,此时要求第一组非空。
\end{solution}
\begin{problem}{2.16}
   凸 $10$边形的任意 $3$条对角线不共点,试求该凸 $10$边形的对角线交于多少个点?又把所有的对角线分割成多少段?
\end{problem}
\begin{solution}
   记满足题中条件的凸 $n+2$边形中,由所有对角线交点个数记为 $h_n$。则 $h_n$满足以下递推关系式:
   \[
   \begin{cases}
   h_n=h_{n-1}+\binom{n+1}{3}+n-1&n\geq 2\\
   h_1 = 0&\\ 
   \end{cases} 
   \]
   下面在该$n+2$边形中任取一个顶点进行分析:

   第一项显然,第三项是由所取顶点的两侧的顶点的连线上的交点个数(被由所选顶点所引出的$n-1$条弦相交)。
   
   第二项的解释如下:

在其余 $n+1$个顶点中任取三个组成三角形,在其中,必有且仅有一边,能被其所对的顶点与所选顶点的连线相交,从而形成一个新的交点。由此,新的交点数等于由$n+1$个顶点组成的三角形的个数,也就是第二项。而新增的块数又等于新增交点数,从而得证。

类似地,对角线被分割的段的数量 $f_n$满足的递推关系式为:
\[
\begin{cases}
  f_n=f_{n-1}+3\binom{n+1}{3}+2*(n-1)&n\geq 2\\
  f_1=0&\\  
\end{cases}    
\]
由此,得出凸 $10$边形的对角线交点个数和对角线被分割的个数分别为$238,686$。
\end{solution}
\begin{problem}{2.20}
   考虑集合 $\{1,2,\cdots,n+1\}$的非空子集
   \begin{enumerate}
    \item 证明最大元素恰好是 $j$的子集数为 $2^{j-1}$
    \item 利用(1)的结论证明
    \[1+2+2^2+\cdots+2^n=2^{n+1}-1.\]
   \end{enumerate} 
\end{problem}
\begin{solution}
    \begin{enumerate}
    \item 显然, $\{1,2,\cdots,n+1\}$的最大元素恰好为 $j$的非空子集与 $\{1,2,\cdots,j-1\}$的子集之间形成一一映射 $S\longmapsto S\backslash\{j\}$
    \item 等式右边为 $\{1,2,\cdots,n+1\}$的非空子集数,而这些非空子集可以按照最大元素大小 $j,1\leq j\leq n+1$进行分类,由加法定理便可得到等式左边。
    \end{enumerate}
\end{solution}
\begin{problem}{2.22}
   \begin{enumerate}
    \item 在由 $5$个 $0$和 $4$个 $1$组成的字符串中, 出现 $01$或 $10$的总次数为 $4$的字符串有多少个?
    \item 在由 $m$个 $0$和 $n$个 $1$组成的字符串中,出现 $01$或 $10$的总次数为 $k$的字符串有多少个?
   \end{enumerate} 
\end{problem}
\begin{solution}
    使用插空法,即将 $1$插入 一排 $0$中。

    \begin{enumerate}
        \item 当 $k$为 奇数时,由于插入给定两个 $0$之间的所有 $1$对于出现次数的总贡献为 $2$ ,字符串一端的 $1$串的总贡献为 $1$,所以
        此时,有且只有一端以 $1$结尾。剩余的 $k-1$次出现需要由 $(k-1)/2$处 $0$之间的 $1$串贡献。所以总个数为 $2\binom{m-1}{(k-1)/2}\binom{n-1}{(k-1)/2}$,其中
        各因子的含义为:从两端中选取一端;从 $m-1$个位置中选取 $(k-1)/2$个;将 $n$个 $1$放入上述两步挑选的 $(k+1)/2$个位置上,使得每个位置至少有一个 $1$.
        \item 当 $k$为偶数时,显然,要么左右两端要么均以 $1$结尾,要么均以 $0$结尾\begin{enumerate}
            \item 如果左右两端均以 $1$为结尾,类似的有 $1\cdot \binom{m-1}{(k-2)/2}\binom{n-1}{k/2}$
            \item 如果左右两端均以 $0$为结尾,类似的有 $1\cdot \binom{m-1}{k/2}\binom{n-1}{((k-2)/2)}$
        \end{enumerate}
        由加法定理可知,总方案数为 $\binom{m-1}{(k-2)/2}\binom{n-1}{k/2}+\binom{m-1}{k/2}\binom{n-1}{((k-2)/2)}$。
    \end{enumerate}
    因此,第一小问所求个数为 $30$。
\end{solution}
\begin{problem}{2.25}
   从 $1$至 $100$的整数中不重复地选取两个数组成有序对 $(x,y)$,使得 $x$与 $y$的乘积 $xy$不能被 $3$整除,共可组成多少对? 
\end{problem}
\begin{solution}
    因为 
    \[
    \begin{array}[H]{rcl}
        (3a+1)(3b+1)&=&9ab+3(a+b)+1 \\
        (3a+1)(3b+2)&=&9ab+3(2a+b)+2 \\
        (3a+2)(3b+2)&=&9ab+3(2a+2b)+4 \\
    \end{array}    
    \]
    所以
     \[xy \equiv 0 \mod 3\Leftrightarrow x\equiv 0 \mod 3\lor y \equiv 0 \mod 3\]
    所以所求对数为 $100\cdot 99-(33\cdot 67+67\cdot 33+33\cdot 32)=4422$。
\end{solution}
\begin{problem}{2.28}
   证明下列组合恒等式:
   \begin{enumerate}
    \item \[\sum_{k=0}^n(-1)^k\cdot k^2\cdot\binom{n}{k}=0;\]
    \item \[\sum_{k=0}^n \frac{k+2}{k+1}\binom{n}{k}=\frac{(n+3)\cdot 2^n-1}{n+1};\]
    \item \[\sum_{k=0}^n \frac{1}{k+2}\binom{n}{k}=\frac{n\cdot 2^{n+1}+1}{(n+1)(n+2)};\]
   \end{enumerate} 
\end{problem}
\begin{solution}
    令
\[f(x)=(1-x)^n=\sum_{k=0}^n\binom{n}{k}x^k\cdot(-1)^k\]
    \begin{enumerate}
        \item 
\[(xf^{\prime})^{\prime}(1)= (f^{\prime}+xf^{\prime\prime})(1)=0=\sum_{k=0}^n(-1)^k\cdot k^2\cdot\binom{n}{k},n\geq 3\]
$n=0,1,2$时显然也成立。
\item 
\[(x\int_{0}^xf)^{\prime}(1)=\frac{(n+3)\cdot 2^n-1}{n+1}=\sum_{k=0}^n \frac{k+2}{k+1}\binom{n}{k}\]
\item 
\[(\int_{0}^xxf)(1)=\frac{n\cdot 2^{n+1}+1}{(n+1)(n+2)}=\sum_{k=0}^n \frac{1}{k+2}\binom{n}{k}\]
    \end{enumerate}
\end{solution}
\begin{problem}{3.4}
   给出
   \[\binom{n}{m}\binom{r}{0}+\binom{n-1}{m-1}\binom{r+1}{1}+\cdot+\binom{n-m}{0}\binom{r+m}{m}=\binom{n+r+1}{m}\] 
   的组合意义。
\end{problem}
\begin{solution}
    等号右边的含义便是从 $\{1,2,\cdot,n+r+1\}$中取 $m$个元素的取法个数。

    等号左边则是根据不包含在所取的 $m$个元素中的 第 $n-m+1$个元素的取值的不同情况使用加法定理。
    这个元素的可能取值为 $n-m+1,\cdots,n+1$。当取值为 $j$时,$\{1,\cdots,j-1\}$中有 $j-1-(n-m)$个元素被选中,而 $\{j+1,\cdots,n+r+1\}$中有 $m-(j-1-n+m)=m-j+n+1$个元素被选中。从而有等式左侧。
\end{solution}
\begin{problem}{3.6}
   证明 
   \[\binom{m}{0}\binom{m}{n}+\binom{m}{1}\binom{m-1}{n-1}+\cdots+\binom{m}{n}\binom{m-n}{0}=2^n\binom{m}{n}.\] 
\end{problem}
\begin{solution}
    \[
        \binom{m}{i}\binom{m-i}{n-i}=\frac{m!}{i!(m-i)!}\cdot\frac{(m-i)!}{(n-i)!(m-n)!}=\frac{n!}{i!(n-i)!}\cdot\frac{m!}{n!(m-n)!}=\binom{n}{i}\binom{m}{n}\]
        所以
        \[LHS = \sum_{i=0}^n\binom{n}{i}\binom{m}{n}=2^n\binom{m}{n}\]
\end{solution}
\begin{problem}{3.8}
   求整数 $a,b,c$,使得 
   \[m^3=a\binom{m}{3}+b\binom{m}{2}+c\binom{m}{1},\]
   并计算 $1^3+2^3+\cdots+n^3$的值。 
\end{problem}
\begin{solution}
    \[m^3=(a/6)m^3+(-a/2+b/2)m^2+(a/3-b/2+c)m\]
    比较系数后可得
\[a=6\quad b=-6\quad c = -5\]
所以 
\[\sum_{i = 1}^ni^3=\sum_{i = 1}^n6\binom{i}{3}-6\binom{i}{2}-5\binom{i}{1}= 6\binom{n+1}{4}-6\binom{n+1}{3}-5\binom{n+1}{2}=(\frac{n(n+1)}{2})^2\]
\end{solution}
\begin{problem}{3.15}
   用组合学方法证明恒等式
   \[\binom{n}{k}-\binom{n-3}{k}=\binom{n-1}{k-1}+\binom{n-2}{k-1}+\binom{n-3}{k-1}.\] 
\end{problem}
\begin{solution}
    左右两侧描述的都是从 $\{1,\cdots,n\}$中挑选至少包含 $1,2,3$中的一个的 $k$组合的方法数。
    左侧是运用减法法则。右侧则是根据组合中的最小值使用加法法则。
\end{solution}
\begin{problem}{3.18}
   令 $n$和 $k$为正整数,给出以下恒等式的组合学证明:
   \[n(n+1)2^{n-2}=\sum_{k=1}^nk^2\binom{n}{k}.\] 
\end{problem}
\begin{solution}
    等式两边是“从一个$n$元集合中取一个子集,然后取一个由子集中的元素构成的有序对“这一个行为的方案数。

    等式左边是先取有序对,当有序对中的元素相同时,这个元素有$n$种可能,所取子集便是从剩下的$n-1$个元素组成的集合的子集并上有序对中的元素,个数为$2^{n-1}$,当有序对中的元素不同时,有$n\left(n-1\right)$种可能,子集数则是$2^{n-2}$,从而有
    \[n2^{n-1}+n\left(n-1\right)2^{n-2}=n\left(n+1\right)2^{n-2}\]
    种可能。

    而等号右边则是先取子集,然后在子集中取有序对。根据所取子集中的元素个数使用加法法则。
\end{solution}
\end{document}