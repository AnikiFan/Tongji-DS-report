\documentclass[12pt, a4paper, oneside]{article}
\usepackage{amsmath, amsthm, amssymb, bm, graphicx, hyperref, mathrsfs}
\makeatletter
\newcommand{\mytitle}{\@title}
\makeatother

\usepackage[
    fontset=none,%设置中文支持,并自定义字体
    zihao=5,%默认字号为五号
    heading=true,%允许后续自定义标题样式
    scheme=chinese,%自动将文档样式中文化,例如图标标题
    punct=quanjiao,%全角式标点符号
    space=auto,%中文后接换行不会添加空格,但是英文会添加空格,需要用%手动取消
    linespread=1.3,%行距倍数是1.3
    autoindent=true,%自动缩进两个中文宽度
    ]{ctex}
\ctexset{
    today=small,%小写样式的日期
    contentsname={目录},
    % contentsname={\hspace{-\ccwd}目录},
    listfigurename={插图},
    listtablename={表格},
    figurename={图},
    tablename={表},
    abstractname={简{\quad}介},
    indexname={索引},
    appendixname={附录},
    bibname={参考文献},
    proofname={证明},
    % refname={参考文献},%只适用于beamer
    % algorithmname={算法},
    % continuation={(续)},%beamer续页的标识
    section={
        format+ = \Large\heiti\raggedright,
        name = {,\num\textbf{.}\hspace{1ex}},
        number={\num\thesection},
        nameformat={},
        numberformat={},
        aftername={},
        titleformat={},
        aftertitle={},
        runin=false,%对section级以下有用,标题是否和正文在同一段上
        beforeskip={3.5ex plus 1ex minus .2ex},%标题前垂直间距
        afterskip={2.3ex plus .2ex}%标题后垂直间距
    },
    subsection={
        format+ = \large\heiti\raggedright,
        name = {,\num\textbf{.}\hspace{1ex}},
        number={\num\thesubsection},
        nameformat={},
        numberformat={},
        aftername={},
        titleformat={},
        aftertitle={},
        runin=false,%对section级以下有用,标题是否和正文在同一段上
        beforeskip={3.5ex plus 1ex minus .2ex},%标题前垂直间距
        afterskip={2.3ex plus .2ex}%标题后垂直间距
    },
    subsubsection={
        format+ = \normalsize\heiti\raggedright,
        name = {,\num\textbf{.}\hspace{1ex}},
        number={\num\thesubsubsection},
        nameformat={},
        numberformat={},
        aftername={},
        titleformat={},
        aftertitle={},
        runin=false,%对section级以下有用,标题是否和正文在同一段上
        beforeskip={3.5ex plus 1ex minus .2ex},%标题前垂直间距
        afterskip={2.3ex plus .2ex}%标题后垂直间距
    },
    }

\title{\textbf{HW4}}
\author{范潇\quad2254298}
\date{\today}
\linespread{1.5}
\newcounter{problemname}
\newenvironment{problem}[1]{\stepcounter{problemname}\par\noindent\textbf{题目\arabic{problemname}. (#1)}}{}
\newenvironment{solution}{\par\noindent\textbf{解答. }}{}
\newenvironment{note}{\par\noindent\textbf{题目\arabic{problemname}的注记. }}{}
\usepackage{amsfonts}
\usepackage{lmodern}%解决报错
% 中文默认字体: 思源宋体,粗体为思源宋体半粗体,斜体为方正楷体_GBK
\setCJKmainfont{Source Han Serif SC}[BoldFont={Source Han Serif SC Heavy}, ItalicFont=FZKai-Z03S]
% 中文无衬线字体:思源黑体,粗体为思源黑体粗体
\setCJKsansfont{Source Han Sans CN}[BoldFont={Source Han Sans CN Heavy}]
% 中文等宽字体:微软雅黑light
\setCJKmonofont{Microsoft YaHei}[ItalicFont={Microsoft YaHei Light}]

\newCJKfontfamily\songti{Source Han Serif SC}[BoldFont={Source Han Serif SC Heavy}]
\newCJKfontfamily\xbsong{Source Han Serif SC SemiBold} % 小标宋
\newCJKfontfamily\dbsong{Source Han Serif SC Bold} % 大标宋
\newCJKfontfamily\cusong{Source Han Serif SC Heavy} % 粗宋
\newCJKfontfamily\heiti{Source Han Sans CN}[BoldFont={Source Han Sans CN Heavy}]
\newCJKfontfamily\dahei{Source Han Sans CN Medium} % 大黑
\newCJKfontfamily\cuhei{Source Han Sans CN Heavy} % 粗黑
\newCJKfontfamily\fangsong{FZFangSong-Z02S}
\newCJKfontfamily\kaiti{FZKai-Z03S}[ItalicFont={Microsoft YaHei Light}]%这个斜体只是用于lstlisting环境中的中文注释
% \newCJKfontfamily\kaiti{FZKai-Z03S}[ItalicFont={FZZJ-LZXTFSJW}]%这个斜体只是用于lstlisting环境中的中文注释
\setsansfont{Arial}
\setmonofont{Consolas}%设置西文等宽字体
\newfontfamily\code{Consolas}
\newfontfamily\num{Arial}

\usepackage{geometry}%设置整体页面布局
\geometry{a4paper}
\geometry{left=2cm,right=2cm,top=2.54cm,bottom=2.54cm}%word常规页边距
% \geometry{left=1.27cm,right=1.27cm,top=1.27cm,bottom=1.27cm}%word窄页边距
\setlength{\headheight}{13pt}%避免warning
\usepackage{fancyhdr}%必须在geometry包之后使用
\fancyhf{}
\makeatletter
\lhead{\sffamily\bfseries{2254298 范潇}}%可以使用thepage,CTEXthechapter,CTEXthesection
\makeatother
\chead{\sffamily\bfseries{\mytitle}}
\rhead{\sffamily\bfseries{- \thepage{} -}}
\renewcommand\headrulewidth{2pt}%设置眉头宽度
\pagestyle{fancy}
\begin{document}
\maketitle
\begin{problem}{1}
在平面上花$n$条直线,每对直线都在不同的点相交,它们构成的无限区间数记为$f(n)$,求$f(n)$满足的递推关系。
\end{problem}
\begin{solution}
    由平面几何知识可知:
    \[\begin{cases}
        f(n) = f(n-1) + 2(n\geq 2)\\
        f(0) = 1,f(1) = 2\\
    \end{cases}\]
\end{solution}
\begin{problem}{3}
$n$位四进制数中,若:
\begin{enumerate}
    \item 有偶数个$0$的序列共有$f(n)$个;
    \item 有偶数个$0$且有偶数个$1$的序列共有$g(n)$个。
\end{enumerate}
求$f(n),g(n)$满足的递推关系。
\end{problem}
\begin{solution}
    显然$f(0)=1,f(1)=3$,(空串视为合法序列)。当$n\geq 2$时,按照第$n$位上是否为$0$进行讨论。如果为0,那么前$n-1$位上只能有奇数个0,由减法法则可知,方案数为$4^{n-1}-f(n-1)$;
    否则,前$n-1$位上仍有偶数个0,由乘法法则可知,方案数为$3f(n-1)$。综上
    \[f(n)=\begin{cases}
       4^{n-1}+2f(n-1),n\geq 2\\
       1,n=0,1\\ 
    \end{cases}\]
\end{solution}
\begin{problem}{6}
求解下列递推关系:
\begin{enumerate}
    \item\[ \begin{cases}
        f(n)=f(n-1)+9f(n-2)-9f(n-3)(n\geq3)\\
        f(0)=0,f(1)=1,f(2)=2\\
    \end{cases}\]
    \item \[\begin{cases}
        f(n)=4f(n-1)-3f(n-2)+3^n\\
        f(0)=1,f(1)=2\\
    \end{cases}\]
\end{enumerate}
\end{problem}
\begin{solution}
    \begin{enumerate}
        \item 特征方程为
        \[(x^2-9)(x-1)=0\]
        所以设通解为
        \[f(n)=a_13^n+a_2(-3)^n+a_3\]
        将初值带入后可解得
        \[f(n)=3^{n-1}+\frac{(-3)^{n-1}}{4}-\frac{1}{4}\]
         \item 特征方程为
        \[(x-3)(x-1)=0\]
        3为一个根,所以设特解为
        \[an3^n\]
        带入递推式解得$a=\frac{3}{2}$.

        设解为
        \[f(n)=a_13^n+a_2+\frac{n3^{n+1}}{2}\]
        解得
        \[f(n)=-\frac{7}{4}3^n+\frac{11}{4}+\frac{n3^{n+1}}{2}\]
    \end{enumerate}
\end{solution}
\begin{problem}{7}
求解下列递推关系:
\begin{enumerate}
    \item \[\begin{cases}
        f^2(n)-2f(n-1)=0(n\geq1)\\
        f(0)=4\\
    \end{cases}\]
    \item \[ 
    f(n)=nf(n-1)+n!(n\geq 1)\\
    f(0)=2\\    
    \]
\end{enumerate}
\begin{solution}
    \begin{enumerate}
        \item 设$g(n)=\ln f(n)$,则有
        \[\begin{cases}
            2g(n)=\ln 2+g(n-1)&n\geq 1\\
                g(0)=2\ln2\\
        \end{cases}\]
        从而
        \[2(g(n)-\ln2)=g(n-1)-\ln2\]
        因此
        \[g(n)=(2^n+1)\ln2\]
        \[f(n)=2^{2^n+1}\]
    \item 设$g(n)=f(n)/n!$,则有
        \[\begin{cases}
            g(n)=g(n-1)+1&n\geq 1\\
                g(0)=2\ln2\\
        \end{cases}\]
        从而
        \[g(n)=2+n\]
        \[f(n)=(2+n)n!\]
    
    \end{enumerate}
\end{solution}
\end{problem}
\begin{problem}{8}
求解下列递推关系:
\begin{enumerate}
    \item \[\begin{cases}
        f(n)=f(n-1)+\frac{1}{n(n+1)}(n\geq1)\\
        f(0)=1\\
    \end{cases}\]
    \item \[
        \begin{cases}
            f(n+2)-f(n)=3\cdot2^n+4\cdot(-1)^n\\
            f(0)=0,f(1)=1\\
        \end{cases}\]
\end{enumerate}
\end{problem}
\begin{solution}
    \begin{enumerate}
     \item 设$g(n)=f(n)+\frac{1}{n+1}$,则有
     \[g(n)=g(0)=2\]
        \[f(n)=2-\frac{1}{n+1}\]
         \item 特征方程为
        \[x^2-1=0\]
        根为$x=\pm 1$。因此设$3\cdot 2^n$对应的特解为$a2^n$,$4\cdot(-1)^n$对应的特解为$bn(-1)^n$.
        令\[f(n)=a2^n+bn(-1)^n\]
        得
        \[3a\cdot2^n+2b(-1)^n=3\cdot2^n+4(-1)^n\]
        比较系数后得$a=1,b=2$。
        设通解为
        \[f(n)=2^n+2n(-1)^n+a_1+a_2(-1)^n\]
        带入初值后可得
        \[f(n)=2^n+2n(-1)^n+(-1)^{n+1}\]
    \end{enumerate}
\end{solution}
\begin{problem}{11}
$1\times n$棋盘用红、白、蓝三种颜色着色,不允许相邻两个都着红色,求着色方案数满足的递推关系,并求出着色方案数。
\end{problem}
\begin{solution}
    记方案数为$f(n)$,显然$f(0)=1,f(1)=3$,当$n\geq 2$时,按照第$n$块的颜色进行讨论。若为红色,那么第$n-1$块只能选择白色或蓝色,对应$2f(n-2)$种方案;
    否则,第$n$块为蓝色或白色,又乘法原理可知,对应$2f(n-1)$种方案。所以得到递推关系
    \[f(n)=\begin{cases}
       2f(n-1)+2f(n-2)&n\geq2\\
       1&n=0\\
       3&n=1\\ 
    \end{cases}\]
        特征方程为
        \[x^2-2x-2=0\]
        根为$x=1\pm\sqrt{3}$.
        令\[f(n)=a(1+\sqrt{3}^n)+b(1-\sqrt{3})^n\]
        带入初值后可得
 \[f(n)=(\frac{1}{2}+\frac{\sqrt{3}}{3})(1+\sqrt{3}^n)+(\frac{1}{2}-\frac{\sqrt{3}}{3})(1-\sqrt{3})^n\]
\end{solution}
\begin{problem}{14}
    \[\begin{cases}
        f(n)=f(n-1)+9f(n-2)-9f(n-3)\\
        f(0)=0,f(1)=1,f(2)=2\\
    \end{cases}\]
\end{problem}
\begin{solution}
    令
    \[G(x)=\sum_{n=0}^{\infty}f(n)x^n\]
    则
    \[(1-x-9x^2+9x^3)G(x)=\sum_{n=3}^{\infty}9f(n-3)x^n-\sum_{n=2}^{\infty}2f(n-2)x^n-\sum_{n=1}^{\infty}f(n-1)x^n+\sum_{n=0}^{\infty}f(n)x^n\]
    \[(9x^2-1)(x-1)G(x)=x^2+x\]
    \[G(x)=\frac{x^2+x}{(9x^2-1)(x-1)}=(x^2+x)(-\frac{3}{4}\frac{1}{3x-1}+\frac{3}{8}\frac{1}{3x+1}+\frac{1}{8}\frac{1}{x-1})=(x^2+x)(\frac{3}{4}\sum_{n=0}^{\infty}(3x)^n+\frac{3}{8}\sum_{n=0}^{\infty}(-3x)^n-\frac{1}{8}\sum_{n=0}^{\infty}x^n)\]
    整理后比较系数可得
        \[f(n)=3^{n-1}+\frac{(-3)^{n-1}}{4}-\frac{1}{4}\]
\end{solution}
\begin{problem}{17}
    在圆周上任取$n$个不相同的点,过每两点作一条弦。假设这些弦中没有三条在圆内相交于一点,令$a_n$表示这些弦将圆分成的区域数。证明
    \[a_n=\binom{n}{4}+\binom{n}{2}+1\]
\end{problem}
\begin{solution}
    由上一章的第18题可知,
   令$h_n$表示具有$n+2$条的凸边形区域被其对角线所分成的区域,假设没有三条对角线共点。定义$h_0=0.$
   则
   \[h_n=\binom{n+2}{4}+\binom{n+1}{2}\]
   而$a_n$在$h_{n-2}$的基础上增加了由弦和弧围成的区域,即
   \[a_n=h_{n-2}+n=\binom{n}{4}+\binom{n-1}{2}+n=\binom{n}{4}+\frac{(n-1)(n-2)}{2}+n=\binom{n}{4}+\frac{n^2-3n+2}{2}+n=\binom{n}{4}+\binom{n}{2}+1\]
\end{solution}
\begin{problem}{19}
   确定$f(n)$的一个递推关系式,$f(n)$是一个平面被$n$个圆划分的区域数量,期中每一对圆正好相交于两点,且没有三个圆相交于一点。 
\end{problem}
\begin{solution}
    对于第$n$个圆,它和前$n-1$个圆产生$2(n-1)$个交点,将其划分为$2(n-1)$条弧,而每对弧又会将原本的一个区域一分为二,从而有
    \[f(n)=2(n-1)+f(n-1),n\geq2\]
    显然$f(1)=2$,展开递推关系式后解得
    \[f(n)=n^2-n+2\]
\end{solution}
\end{document}