\documentclass[12pt, a4paper, oneside]{article}
\usepackage{amsmath, amsthm, amssymb, bm, graphicx, hyperref, mathrsfs}
\makeatletter
\newcommand{\mytitle}{\@title}
\makeatother

\usepackage[
    fontset=none,%设置中文支持,并自定义字体
    zihao=5,%默认字号为五号
    heading=true,%允许后续自定义标题样式
    scheme=chinese,%自动将文档样式中文化,例如图标标题
    punct=quanjiao,%全角式标点符号
    space=auto,%中文后接换行不会添加空格,但是英文会添加空格,需要用%手动取消
    linespread=1.3,%行距倍数是1.3
    autoindent=true,%自动缩进两个中文宽度
    ]{ctex}
\ctexset{
    today=small,%小写样式的日期
    contentsname={目录},
    % contentsname={\hspace{-\ccwd}目录},
    listfigurename={插图},
    listtablename={表格},
    figurename={图},
    tablename={表},
    abstractname={简{\quad}介},
    indexname={索引},
    appendixname={附录},
    bibname={参考文献},
    proofname={证明},
    % refname={参考文献},%只适用于beamer
    % algorithmname={算法},
    % continuation={(续)},%beamer续页的标识
    section={
        format+ = \Large\heiti\raggedright,
        name = {,\num\textbf{.}\hspace{1ex}},
        number={\num\thesection},
        nameformat={},
        numberformat={},
        aftername={},
        titleformat={},
        aftertitle={},
        runin=false,%对section级以下有用,标题是否和正文在同一段上
        beforeskip={3.5ex plus 1ex minus .2ex},%标题前垂直间距
        afterskip={2.3ex plus .2ex}%标题后垂直间距
    },
    subsection={
        format+ = \large\heiti\raggedright,
        name = {,\num\textbf{.}\hspace{1ex}},
        number={\num\thesubsection},
        nameformat={},
        numberformat={},
        aftername={},
        titleformat={},
        aftertitle={},
        runin=false,%对section级以下有用,标题是否和正文在同一段上
        beforeskip={3.5ex plus 1ex minus .2ex},%标题前垂直间距
        afterskip={2.3ex plus .2ex}%标题后垂直间距
    },
    subsubsection={
        format+ = \normalsize\heiti\raggedright,
        name = {,\num\textbf{.}\hspace{1ex}},
        number={\num\thesubsubsection},
        nameformat={},
        numberformat={},
        aftername={},
        titleformat={},
        aftertitle={},
        runin=false,%对section级以下有用,标题是否和正文在同一段上
        beforeskip={3.5ex plus 1ex minus .2ex},%标题前垂直间距
        afterskip={2.3ex plus .2ex}%标题后垂直间距
    },
    }

\title{\textbf{图论作业1}}
\author{范潇\quad2254298}
\date{\today}
\linespread{1.5}
\newcounter{problemname}
\newenvironment{problem}[1]{\stepcounter{problemname}\par\noindent\textbf{题目\arabic{problemname}. (#1)}}{}
\newenvironment{solution}{\par\noindent\textbf{解答. }}{}
\newenvironment{note}{\par\noindent\textbf{题目\arabic{problemname}的注记. }}{}
\usepackage{amsfonts}
\usepackage{lmodern}%解决报错
% 中文默认字体: 思源宋体,粗体为思源宋体半粗体,斜体为方正楷体_GBK
\setCJKmainfont{Source Han Serif SC}[BoldFont={Source Han Serif SC Heavy}, ItalicFont=FZKai-Z03S]
% 中文无衬线字体:思源黑体,粗体为思源黑体粗体
\setCJKsansfont{Source Han Sans CN}[BoldFont={Source Han Sans CN Heavy}]
% 中文等宽字体:微软雅黑light
\setCJKmonofont{Microsoft YaHei}[ItalicFont={Microsoft YaHei Light}]

\newCJKfontfamily\songti{Source Han Serif SC}[BoldFont={Source Han Serif SC Heavy}]
\newCJKfontfamily\xbsong{Source Han Serif SC SemiBold} % 小标宋
\newCJKfontfamily\dbsong{Source Han Serif SC Bold} % 大标宋
\newCJKfontfamily\cusong{Source Han Serif SC Heavy} % 粗宋
\newCJKfontfamily\heiti{Source Han Sans CN}[BoldFont={Source Han Sans CN Heavy}]
\newCJKfontfamily\dahei{Source Han Sans CN Medium} % 大黑
\newCJKfontfamily\cuhei{Source Han Sans CN Heavy} % 粗黑
\newCJKfontfamily\fangsong{FZFangSong-Z02S}
\newCJKfontfamily\kaiti{FZKai-Z03S}[ItalicFont={Microsoft YaHei Light}]%这个斜体只是用于lstlisting环境中的中文注释
% \newCJKfontfamily\kaiti{FZKai-Z03S}[ItalicFont={FZZJ-LZXTFSJW}]%这个斜体只是用于lstlisting环境中的中文注释
\setsansfont{Arial}
\setmonofont{Consolas}%设置西文等宽字体
\newfontfamily\code{Consolas}
\newfontfamily\num{Arial}

\usepackage{geometry}%设置整体页面布局
\geometry{a4paper}
\geometry{left=2cm,right=2cm,top=2.54cm,bottom=2.54cm}%word常规页边距
% \geometry{left=1.27cm,right=1.27cm,top=1.27cm,bottom=1.27cm}%word窄页边距
\setlength{\headheight}{13pt}%避免warning
\usepackage{fancyhdr}%必须在geometry包之后使用
\fancyhf{}
\makeatletter
\lhead{\sffamily\bfseries{2254298 范潇}}%可以使用thepage,CTEXthechapter,CTEXthesection
\makeatother
\chead{\sffamily\bfseries{\mytitle}}
\rhead{\sffamily\bfseries{- \thepage{} -}}
\renewcommand\headrulewidth{2pt}%设置眉头宽度
\pagestyle{fancy}
\begin{document}
\maketitle
\begin{problem}{1}
    (1)共7个顶点,若能简单图化,则度为6的两个顶点均分别与其他所有顶点相邻。特别地,它们与度为2的顶点相邻。此时对于两个度为5的顶点,它们必须分别再与
    对方和两个度为3的顶点相连。但是这会使得两个度数为3的顶点与4条边邻接(其中两条与度为6的点邻接),从而矛盾。

    (2)
    \begin{figure}[!h]
        \centering
        \includegraphics*[scale = 0.7]{q11.png}
        \caption{(2)第一种简单图}
    \end{figure}
    \begin{figure}[!h]
        \centering
        \includegraphics*[scale = 0.7]{q12.png}
        \caption{(2)第一种简单图}
    \end{figure}

    第一张图中,两个度为3的顶点相邻,而在第二张图中则不相邻。所以这两张图非同构。

    (3)
    \begin{figure}[!h]
        \centering
        \includegraphics*[scale = 0.7]{q13.png}
        \caption{(3)第一种简单图}
    \end{figure}
    \begin{figure}[!h]
        \centering
        \includegraphics*[scale = 0.7]{q14.png}
        \caption{(3)第一种简单图}
    \end{figure}
\end{problem}
\newpage
\begin{problem}{2}
    \[\sum_{i=1}^kd_i\]
    可以视为由两部分组成,一部分由边$d_id_j,(i<j\leq k)$贡献,另一部分由$d_id_j(i\leq k<j)$贡献。前者
    最多贡献$k(k-1)$,此时$\{d_1,\cdots,d_k\}$对应的诱导子图为完全图,其中的每一条边都对它的两个端点分别贡献了1。

    另一部分,将$d_id_j(i\leq k<j)$按照
    $j$的取值划分为集合$A_k+1,\cdot,A_n$。对于$A_r(k+1\leq r\leq n)$,因为以$v_r$为
    一端的边的个数小于$d_r$,所以$\lvert A_r\rvert\leq d_r$。又因为另一端互异,且下标小于等于$k$,所以$\lvert A_r\rvert\leq k$。
    因此$\lvert A_r\rvert \leq\min\{d_r,k\}$。而$A_r$中的每一条边,都因为它的一段的下标小于$k$,所以对合式贡献了1,所以
    合在一起第二部分为
    \[\sum_{i=k+1}^n\lvert A_i\rvert=\min\{k,d_i\}\]

    综上
    \[\sum_{i=1}^k\leq k(k-1)+\sum_{i=k+1}^n\min\{k,d_i\}\]
\end{problem}
\begin{problem}{3}
因为$G$为自补图,所以它和它的补图的边数相同。又因为它和它的补图之并为完全图,且两者的边集交集为空,所以
\[\frac{\lvert V\rvert(\lvert V\rvert-1)}{2}=2\lvert E\rvert\]
有
\[\lvert V\rvert(\lvert V\rvert-1)\equiv 0(\mod 4)\]
所以$\lvert V\rvert = 4k$或$\lvert V\rvert=4k+1$,其中$k$为正整数。
\end{problem}
\begin{problem}{4}
   $K_n$有$\frac{n(n-1)}{2}$条边,所以$G$是在$K_n$的基础上至多删去$\frac{n(n-1)}{2}-\frac{(n-1)(n-2)}{2}-2=n-3$条边得到的图。
   假设有两个不相邻的顶点$u,v$,满足$d(u)+d(v)<n$,则以$u$或$v$为一端的边至多有$n-1$条,而在$K_n$中,这样的边有$2n-3$条,
   所以需要至少删掉$n-2$条边,与“至多删去$n-3$条边”产生矛盾。所以对任意两个不相邻的顶点$u,v$有$d(u)+d(v)\geq n$,从而$G$有哈密顿圈,为哈密顿图。
   
   显然,$K_{n-1}$加上另外一个叶子顶点组成的$n$阶图为非哈密顿图,且满足$\lvert E\rvert =\binom{n-1}{2}+1$。
\end{problem}
\begin{problem}{5}
   设$T$有$n$个顶点,则有$n-1$条边,顶点度数之和由握手定理可知为$2(n-1)$。因此 
   \[8+6+4(n-10)=2(n-1)\]
   解得 $n=12$,从而有$2$个$4$度点。
    \begin{figure}[!h]
        \centering
        \includegraphics*[scale = 0.7]{q51.png}
    \end{figure}
    \begin{figure}[!h]
        \centering
        \includegraphics*[scale = 0.7]{q52.png}
    \end{figure}
    \begin{figure}[!h]
        \centering
        \includegraphics*[scale = 0.7]{q53.png}
    \end{figure}
    \begin{figure}[!h]
        \centering
        \includegraphics*[scale = 0.7]{q53.png}
    \end{figure}
\end{problem}
\newpage
\begin{problem}{6}
   任取一个顶点$A$,有鸽笼原理可知,在$G$或其补图中,有$3$条边与其关联。不妨设在$G$中,有边$AB,AC,AD$。若$\bar{G}$中有三角形$BCD$,则命题成立,
   否则,在$G$中,缺失的边与$A$一起组成了一个圈,从而命题也成立。 
\end{problem}
\begin{problem}{7}
   显然,树的中心一定存在。
   
   下面先证明,如果树的中心中至少有两个顶点,则任意两个顶点均相邻。假设$u,v$为树的中心中的两个互异顶点,若它们不相邻,
   则以$u,v$为两端的路径中至少还有另外一点$w$。设$x$为树中距离$w$最远的点,即$d(x,w)$为$w$对应的$r$值。下面考察以$x,w$为两端的路径,
   如果$u,v$均不在该路径上,则说明以$x,u$和以$x,v$为两端的路径上都有$w$,否则会在树上形成圈。此时$u,v$对应的$r$值均大于$d(x,w)$,
   即大于$w$对应的$r$值,而这与$u,v$在树的中心内矛盾;当$u$或$v$在这条路径上时,不妨设$u$在这条路径上,此时$v$必不可能也在这条路径上,
   否则会在树上形成圈。此时$d(v,x)>d(w,x)$,与$v$在树的中心内矛盾。

   因此,如果树的中心中至少有两个顶点,则任意两个顶点均相邻。当树的中心有两个顶点时,它们相邻,即树的中心为一条边。若
   树的中心有三个或以上个顶点,则它们之间会有圈,产生矛盾。所以树的中心内不可能有两个以上顶点。即树的中心为一个顶点或一条边。
\end{problem}
\begin{problem}{9}
    由定义可知,$\forall v,\chi{G-v}\leq \chi(G)-1$。若$\exists v,\chi{G-v}\leq \chi(G)-2$,则将$\chi{G-v}$用至多$\chi(G)-2$
    种颜色
    着色后,
    再将$v$及其相邻接的边添加回去,得到$G$。因为与其相邻的点的颜色至多只有$\chi(G)-2$个,所以可以将它染为一种新颜色,这样得到的$G$的染色方案
    只用了$\chi(G)-1$种颜色,与定义产生矛盾。所以$\chi{G-v}=\chi(G)-1$。

    若$G$不为连通图,则它至少有两个联通分量。若$\exists v,\chi{G-v}=\chi(G)-1$,设$v$在联通分量$C_i$内,则$G-v$中,其他连通分量$C_j(j\neq i)$
    并未发生改变,说明$\chi{C_j}\leq \chi(G),(j\neq i)$。显然,$\chi{C_i}=\chi(G)$,否则可以用至多$\chi(G)-1$中颜色对$G$着色。此时,如果选择
    $u\in C_j(j\neq i)$,则$\chi(G-u)=\chi(G)$,因为$\chi(C_i)=\chi(G)$。从而与$G$为色临界图产生矛盾。

    若$\exists v\in V,d(v)\leq \chi(G)-2$,则至多$\chi(G)-1$中颜色将$G-v$着色后,再将其复原未$G$。由于$d(v)\leq \chi(G)-2$,
    所以可以将它着色,使它的颜色与其相邻的顶点都不同,同时总共使用的颜色至多为$\chi(G)-1$种,而这与$\chi(G)$的定义矛盾。
\end{problem}
\begin{problem}{10}
    \begin{align*}
    \lvert E\rvert & = \lvert V\rvert +\lvert F\rvert -2\\
   2\lvert E\rvert & = 3f_3+4f_4+\cdots\\
   2\lvert E\rvert & = \sum_{u\in V}d(u)\geq\delta(G)\lvert V\rvert\geq 3\lvert V\rvert
    \end{align*}
    可得
    \[2\lvert E\rvert\geq 3\lvert E\rvert-3\lvert F\rvert +6\]
    若不存在边数小于等于4的面,即$f_3=0,f_4=0$
    \[6\lvert F\rvert\geq 2\lvert E\rvert +12\geq 12+3f_3+4f_4+\cdots\geq 5\lvert F\rvert+12\]
    所以$\lvert F\rvert \geq 12$,与面数小于12产生矛盾。所以存在边数小于等于4的面。
\end{problem}
\end{document}