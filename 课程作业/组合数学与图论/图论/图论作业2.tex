\documentclass[12pt, a4paper, oneside]{article}
\usepackage{amsmath, amsthm, amssymb, bm, graphicx, hyperref, mathrsfs}
\makeatletter
\newcommand{\mytitle}{\@title}
\makeatother

\usepackage[
    fontset=none,%设置中文支持,并自定义字体
    zihao=5,%默认字号为五号
    heading=true,%允许后续自定义标题样式
    scheme=chinese,%自动将文档样式中文化,例如图标标题
    punct=quanjiao,%全角式标点符号
    space=auto,%中文后接换行不会添加空格,但是英文会添加空格,需要用%手动取消
    linespread=1.3,%行距倍数是1.3
    autoindent=true,%自动缩进两个中文宽度
    ]{ctex}
\ctexset{
    today=small,%小写样式的日期
    contentsname={目录},
    % contentsname={\hspace{-\ccwd}目录},
    listfigurename={插图},
    listtablename={表格},
    figurename={图},
    tablename={表},
    abstractname={简{\quad}介},
    indexname={索引},
    appendixname={附录},
    bibname={参考文献},
    proofname={证明},
    % refname={参考文献},%只适用于beamer
    % algorithmname={算法},
    % continuation={(续)},%beamer续页的标识
    section={
        format+ = \Large\heiti\raggedright,
        name = {,\num\textbf{.}\hspace{1ex}},
        number={\num\thesection},
        nameformat={},
        numberformat={},
        aftername={},
        titleformat={},
        aftertitle={},
        runin=false,%对section级以下有用,标题是否和正文在同一段上
        beforeskip={3.5ex plus 1ex minus .2ex},%标题前垂直间距
        afterskip={2.3ex plus .2ex}%标题后垂直间距
    },
    subsection={
        format+ = \large\heiti\raggedright,
        name = {,\num\textbf{.}\hspace{1ex}},
        number={\num\thesubsection},
        nameformat={},
        numberformat={},
        aftername={},
        titleformat={},
        aftertitle={},
        runin=false,%对section级以下有用,标题是否和正文在同一段上
        beforeskip={3.5ex plus 1ex minus .2ex},%标题前垂直间距
        afterskip={2.3ex plus .2ex}%标题后垂直间距
    },
    subsubsection={
        format+ = \normalsize\heiti\raggedright,
        name = {,\num\textbf{.}\hspace{1ex}},
        number={\num\thesubsubsection},
        nameformat={},
        numberformat={},
        aftername={},
        titleformat={},
        aftertitle={},
        runin=false,%对section级以下有用,标题是否和正文在同一段上
        beforeskip={3.5ex plus 1ex minus .2ex},%标题前垂直间距
        afterskip={2.3ex plus .2ex}%标题后垂直间距
    },
    }

\title{\textbf{图论作业2}}
\author{范潇\quad2254298}
\date{\today}
\linespread{1.5}
\newcounter{problemname}
\newenvironment{problem}[1]{\stepcounter{problemname}\par\noindent\textbf{题目\arabic{problemname}. (#1)}}{}
\newenvironment{solution}{\par\noindent\textbf{解答. }}{}
\newenvironment{note}{\par\noindent\textbf{题目\arabic{problemname}的注记. }}{}
\usepackage{amsfonts}
\usepackage{lmodern}%解决报错
% 中文默认字体: 思源宋体,粗体为思源宋体半粗体,斜体为方正楷体_GBK
\setCJKmainfont{Source Han Serif SC}[BoldFont={Source Han Serif SC Heavy}, ItalicFont=FZKai-Z03S]
% 中文无衬线字体:思源黑体,粗体为思源黑体粗体
\setCJKsansfont{Source Han Sans CN}[BoldFont={Source Han Sans CN Heavy}]
% 中文等宽字体:微软雅黑light
\setCJKmonofont{Microsoft YaHei}[ItalicFont={Microsoft YaHei Light}]

\newCJKfontfamily\songti{Source Han Serif SC}[BoldFont={Source Han Serif SC Heavy}]
\newCJKfontfamily\xbsong{Source Han Serif SC SemiBold} % 小标宋
\newCJKfontfamily\dbsong{Source Han Serif SC Bold} % 大标宋
\newCJKfontfamily\cusong{Source Han Serif SC Heavy} % 粗宋
\newCJKfontfamily\heiti{Source Han Sans CN}[BoldFont={Source Han Sans CN Heavy}]
\newCJKfontfamily\dahei{Source Han Sans CN Medium} % 大黑
\newCJKfontfamily\cuhei{Source Han Sans CN Heavy} % 粗黑
\newCJKfontfamily\fangsong{FZFangSong-Z02S}
\newCJKfontfamily\kaiti{FZKai-Z03S}[ItalicFont={Microsoft YaHei Light}]%这个斜体只是用于lstlisting环境中的中文注释
% \newCJKfontfamily\kaiti{FZKai-Z03S}[ItalicFont={FZZJ-LZXTFSJW}]%这个斜体只是用于lstlisting环境中的中文注释
\setsansfont{Arial}
\setmonofont{Consolas}%设置西文等宽字体
\newfontfamily\code{Consolas}
\newfontfamily\num{Arial}

\usepackage{geometry}%设置整体页面布局
\geometry{a4paper}
\geometry{left=2cm,right=2cm,top=2.54cm,bottom=2.54cm}%word常规页边距
% \geometry{left=1.27cm,right=1.27cm,top=1.27cm,bottom=1.27cm}%word窄页边距
\setlength{\headheight}{13pt}%避免warning
\usepackage{fancyhdr}%必须在geometry包之后使用
\fancyhf{}
\makeatletter
\lhead{\sffamily\bfseries{2254298 范潇}}%可以使用thepage,CTEXthechapter,CTEXthesection
\makeatother
\chead{\sffamily\bfseries{\mytitle}}
\rhead{\sffamily\bfseries{- \thepage{} -}}
\renewcommand\headrulewidth{2pt}%设置眉头宽度
\pagestyle{fancy}
\begin{document}
\maketitle
\begin{problem}{1}
   对$\lvert V\rvert$进行数学归纳。当$\lvert V\rvert=0,1$时,显然命题成立。假设当$\lvert V\rvert\leq k$时命题成立,
   当$\lvert V\rvert=k+1$时,由于至少有一个叶子结点$v$,设它与顶点$u$相邻。若对于该树存在完全匹配$P_{V}$,显然$P_{V}=P_{V\backslash{\{u,v\}}}\cup\{uv\}$ ,且由于归纳假设$P_{V\backslash\{u,v\}}$是唯一的,所以$P_V$也是唯一的。

   因此,任何树至多有一个完美匹配。
\end{problem}
\begin{problem}{2}
   只需证明当$\lvert E\rvert = \lfloor n^2/4\rfloor +1$时,$t(G)\geq \lfloor n/2\rfloor$即可。 

    当$n\geq 3$时,无论$n$的奇偶性,均有$\lfloor n/2\rfloor\lceil n/2\rceil = \lfloor n^2/4\rfloor$。因此,
    取完全二部图$K_{\lfloor n/2\rfloor,\lceil n/2\rceil}$,因为此时$\lceil n/2\rceil\geq 2$,所以可以在顶点数$\lceil n/2\rceil$的一侧
    取两个点,并将它们连接起来,得到一个$m=\lfloor n^2/4\rfloor +1$的图,同时,由于该图是在完全二部图的基础上得到的,选取的两个点都与
    另一侧的顶点相邻,从而有$t(G)\geq \lfloor n/2\rfloor$
\end{problem}
\begin{problem}{3}
   由于图\ref*{q31}中的着色方案的存在,所以$R(P_3,C_4)\geq 5$。下面证明$R(P_3,C_4)\leq 5$。只需证明对于任意一个$K_5$的红蓝二着色,必能得到一个
   蓝色的$P_3$或红色的$C_4$。

   \begin{figure}[!h]
    \centering
    \includegraphics*[scale = 0.3]{q31.png}
    \caption{$K_4$的一种二着色方案}\label{q31}
   \end{figure} 

   $K_5$中,必存在一个顶点,有偶数个蓝边与其邻接。事实上,如果不存在,则蓝边与顶点相邻接的次数为奇数,与握手定理矛盾。不妨设这个顶点为$v_1$。

   当有$4$条蓝边与其邻接时,若不存在蓝色的$P_3$,则没有蓝边与$v_2,v_3,v_4,v_5$邻接,从而它们形成一个红色的$C_4(v_2\rightarrow v_5\rightarrow v_3\rightarrow v_4 \rightarrow v_2)$。
     \begin{figure}[!h]
    \centering
    \includegraphics*[scale = 0.3]{q32.png}
    \caption{有四条蓝边与$v_1$邻接}\label{q32}
   \end{figure} 

   当有$2$条蓝边与其邻接时,若不存在蓝色的$P_3$,则没有蓝边与$v_2,v_5$邻接,从而它们形成一个红色的$C_4(v_2\rightarrow v_5\rightarrow v_3\rightarrow v_4 \rightarrow v_2)$。
  \begin{figure}[!h]
    \centering
    \includegraphics*[scale = 0.3]{q33.png}
    \caption{有两条蓝边与$v_1$邻接}\label{q33}
   \end{figure} 

   当有$0$条蓝边与其邻接时,若不存在红色的$P_3$,显然对于$v_2,v_3,v_4,v_5$,在图\ref*{q34}的基础上至多分别新增一条红边与它们邻接。但是如果
   只与两条红色边邻接,则回到了上一种情况。所以在图\ref*{q34}中,其余边都是蓝边,但是这时显然存在蓝色$P_3$。
  \begin{figure}[!h]
    \centering
    \includegraphics*[scale = 0.3]{q34.png}
    \caption{有零条蓝边与$v_1$邻接}\label{q34}
   \end{figure} 

   综上,$R(P_3,C_4)\geq 5$,因此$R(P_3,C_4)=5$。
\end{problem}
\begin{problem}{4}
    
\end{problem}
\begin{problem}{22}
    若一条边为某个三角形的最长边,则将其染为红色,剩余边染为蓝色。由于$R(3,3)=6$,所以必有一个蓝色三角形或红色三角形。但是因为每个三角形必有一条最长边,即
    必有一条红边,所以一定能得到一个红色三角形。该红色三角形的最短边便是另一个三角形的最长边。
\end{problem}
\begin{problem}{23}
    对于$K_{n(r_{n-1}-1)+2}$中的任一顶点$v$,由鸽笼原理可知,至少存在一种颜色$\alpha$,有$r_{n-1}$条这种颜色的边与其邻接。
    记这些边的另一端的顶点为$v_1,\cdots,v_{r_{n-1}},\cdots$。若存在$1\leq i<j$,使得$v_iv_j$的颜色也为$\alpha$,则
    得到一个同色三角形$vv_iv_j$。否则,$v_1,\cdots,v_{r_{n-1}}$的诱导子图中边的颜色至多有$n-1$种,因此其中必有一个同色三角形。综上,$r_n\leq n(r_{n-1}-1)+2$。

    因此
    \begin{align*}
        r_n&\leq n(r_{n-1}-1)+2\\
           &\leq n[(n-1)(r_{n-2}-1)+1]+2\\
           &= n(n-1)(r_{n-2}-1)+n+2\\
           &\leq \cdots\\
           &\leq n(n-1)\cdots 2(r_1-1)+n(n-2)\cdots 3+\cdots +n+2\\
           &= 2n!+\frac{n!}{2!}+\cdots+\frac{n!}{(n-1)!}+1\\
           &= n![\frac{1}{n!}+\frac{1}{(n-1)!}+\cdots+\frac{1}{2!}+\frac{1}{1!}+\frac{1}{0!}]+1\\
           &\leq \lfloor en!\rfloor +1=\lceil en!\rceil\\
    \end{align*}

    在三着色的$K_{17}$中,任取一个顶点$v$,至少有6条与它邻接的边同色,记该颜色为$\alpha$,这些边另一端上的顶点为$v_1,\cdots,v_6,\cdots$。
    若存在$1\leq i<j$,使得$v_iv_j$的颜色也为$\alpha$,则$vv_iv_j$为同色三角形。否则,$v_1,\cdots,v_6$的诱导子图中边的颜色至多有$2$种, 
    而$r(3,3)=6$,所以必有一个同色三角形。综上,三着色的$K_{17}$中必有一个同色三角形,所以$r_3\leq 17$.
\end{problem}
\begin{problem}{25}
    $\{1,2,\cdots,s_{n-1}-1\}$可以划分为$n-1$个子集,各子集中没有$x+y=z$的解。
    记这些子集为$A_1,\cdots,A_{n-1}$。

    下面把$\{1,2,\cdots,3s_{n-2}-2\}$划分为不满足要求的$n$个子集。

    令$B_{i} = \{j\mid j\in A_i\lor j-(2s_{n-1}-1)\in A_i\},i=1,\cdots,n-1;B_{n} = \{s_{n-1},s_{n-1}+1,\cdots,2s_{n-1}-1\}$.

    因为任取$i,j\in B_n,i+j\geq 2s_{n-1}$,所以$B_n$中没有$x+y=z$的解。
    对于$B_i,i\neq n$,任取$a,b\in B_i$,若$a,b\in B_i-A_i$,则$a+b\geq 4s_{n-1}$,所以$a+b\notin B_i$;
    若$a,b\in A_i$,$a+b\leq 2s_{n-1}-2$,所以$a+b\notin B_i-A_i$又因为$A_i$中没有$x+y=z$的解,所以$a+b\notin B_i$;若$a,b$一个属于$A_i$,
    令一个属于$B_i-A_i$,不妨设$x\in B_i-A_i,y\in A_i$若存在$z\in B_i,s.t.x+y=z$,则$(x-(2s_{n-1}-1))+y=z-(2s_{n-1}-1)$,而$x-(2s_{n-1}-1)\in A_i$,$z\leq 2s_{n-1}-2$,所以$z\in A_i$,
    而这与$A_i$的构造相矛盾。

    因此,可以把$\{1,2,\cdots,3s_{n-2}-2\}$划分为不满足要求的$n$个子集。从而$s_n\geq 3s_{n-1}-1$。

    当$n=1$时,$s_1=2\geq \frac{1}{2}(3+1)$。假设当$n=k$时不等式成立,则当$n=k+1$时有 
    \[s_{k+1}\geq 3s_k-1\geq \frac{3}{2}(3^n+1)-1=\frac{1}{2}(3^{n+1}+1)\]
    因此不等式成立。
\end{problem}
\end{document}