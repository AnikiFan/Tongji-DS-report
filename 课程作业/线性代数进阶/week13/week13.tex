\documentclass[12pt, a4paper, oneside]{ctexart}
\usepackage{amsmath, amsthm, amssymb, bm, graphicx, hyperref, mathrsfs}

\title{\textbf{HW13}}
\author{范潇\quad2254298}
\date{\today}
\linespread{1.5}
\newcounter{problemname}
\newenvironment{problem}[1]{\stepcounter{problemname}\par\noindent\textbf{题目\arabic{problemname}. (#1)}}{}
\newenvironment{solution}{\par\noindent\textbf{解答. }}{\\\par}
\newenvironment{note}{\par\noindent\textbf{题目\arabic{problemname}的注记. }}{\\\par}

\begin{document}
\maketitle
\begin{problem}{9.3.4}
    设
    \[A =\begin{pmatrix}
        \bm{\alpha}_1^{\top}\\
        \bm{\alpha}_2^{\top}\\
        \vdots\\
        \bm{\alpha}_m^{\top}\\
    \end{pmatrix}\]
    则
    \[\bm{W(A)}=span\{\bm{\alpha}_1,\bm{\alpha}_2,\cdots,\bm{\alpha}_m\}\]
    记方程$\bm{AX=0}$的基础解系为
    \[\bm{\xi}_1,\bm{\xi}_2,\cdots,\bm{\xi}_r\]
    则
    \[\bm{N(A)}=span\{\bm{\xi}_1,\bm{\xi}_2,\cdots,\bm{\xi}_r\}\]
    \[(\bm{\xi_i,\alpha_j})=0,\forall 1\leq i\leq r,1\leq j\leq m\]
    所以$\forall \bm{\alpha}\in \bm{W(A)},\bm{\xi}\in\bm{N(A)}$
    \[(\bm{\alpha,\xi})=(\sum_{i}k_i\bm{\alpha_i},\sum_jt_j\bm{\xi_j})=\sum_{i}\sum_jk_it_j(\bm{\alpha_i},\bm{\xi_j})=0\]
    即
    \[\bm{W(A)}\perp \bm{N(A)}\]
    从而$\bm{W(A)},N(A)$之和为直和。又因为$\dim(W(A))+\dim(N(A))=n$,所以
    \[\mathbb{R} ^n=\bm{W(A)}\oplus \bm{N(A)}\]
\end{problem}
\end{document}