\documentclass[12pt, a4paper, oneside]{ctexart}
\usepackage{amsmath, amsthm, amssymb, bm, graphicx, hyperref, mathrsfs}

\title{\textbf{HW14}}
\author{范潇\quad2254298}
\date{\today}
\linespread{1.5}
\newcounter{problemname}
\newenvironment{problem}[1]{\stepcounter{problemname}\par\noindent\textbf{题目\arabic{problemname}. (#1)}}{}
\newenvironment{solution}{\par\noindent\textbf{解答. }}{\\\par}
\newenvironment{note}{\par\noindent\textbf{题目\arabic{problemname}的注记. }}{\\\par}

\begin{document}
\maketitle
\begin{problem}{9.5.6}
    假设$\mathscr{F}$为反称的,则任给一个标准正交基 $\bm{e}_1,\bm{e}_2,\cdots,\bm{e}_n$,记$\mathscr{F}$在其下的矩阵为$(a_{ij})$,则
    % 对于$\bm{\alpha}=\sum_ik_i\bm{e_i},\bm{\beta}=\sum_jt_j\bm{e_j}$
    % \[(\mathscr{F}\bm{\alpha},\bm{\beta})\]
    \[(\mathscr{F}\bm{e}_i,\bm{e}_j)=(\sum_{k}a_{ki}\bm{e}_k,\bm{e}_j)=a_{ji}\]
    \[(\bm{e}_i,\mathscr{F}\bm{e}_j)=(\bm{e}_i,\sum_{k}a_{kj}\bm{e}_k)=a_{ij}\]
    又因为
    \[(\mathscr{F}\bm{\alpha},\bm{\beta})=-(\bm{\alpha},\mathscr{F}\mathbf{\beta})\]
    且$i,j$取值任意,所以有
    \[(a_{ij})=(-a_{ji})\]
    即$(a_{ij})$为反称矩阵。

    若$V_1$是反称线性变换的不变子空间,任取$\bm{\alpha}\in V_1^{\perp}$,对于任意$\bm{\beta}\in V_1$,
    \[(\mathscr{F}\bm{\alpha},\bm{\beta})=-(\bm{\alpha},\mathscr{F}\bm{\beta})=0\]
    最后一个等号是因为$\mathscr{F}\bm{\beta}\in V_1=(V_1^{\perp})^{\perp}$。因为$\bm{\alpha},\bm{\beta}$的任意性,所以
    $\mathscr{F}\bm{\alpha}\in V_1^{\perp}$,即$V_1^{\perp}$也为$\mathscr{F}$的不变子空间。
\end{problem}
\end{document}