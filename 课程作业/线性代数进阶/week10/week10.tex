\documentclass[12pt, a4paper, oneside]{ctexart}
\usepackage{amsmath, amsthm, amssymb, bm, graphicx, hyperref, mathrsfs}

\title{\textbf{HW10}}
\author{范潇\quad2254298}
\date{\today}
\linespread{1.5}
\newcounter{problemname}
\newenvironment{problem}[1]{\stepcounter{problemname}\par\noindent\textbf{题目\arabic{problemname}. (#1)}}{}
\newenvironment{solution}{\par\noindent\textbf{解答. }}{\\\par}
\newenvironment{note}{\par\noindent\textbf{题目\arabic{problemname}的注记. }}{\\\par}

\begin{document}
\maketitle
\begin{problem}{7.4.4}
    显然
    \[f(A)=0\Leftrightarrow f(A_i)=0,i=1,\cdots,s\]
    所以 
    \[m_A(A_i)=0,i=1,\cdots,s\]
    又因为
    \[f(A_i)=0\Rightarrow m_{A_i}\mid f,i=1,\cdots,s\]
    所以 
    \[m_{A_i}\mid m_A\]
    即$m_A$为$m_i,i=1,\cdots,s$的公倍式。任取$m_i,i=1,\cdots,s$的一个公倍式$f$,存在$g_i$使得$f=g_im_i$,因此 
    \[f(A_i)=g_i(A_i)m_i(A_i)=g_i(A_i)\cdot 0=0,i=1,\cdots,s\]
    从而 
    \[f(A)=0\]
    所以$f$为$A$的一个零化多项式。因此
    \[m_A\mid f\]
    即$m_A$为$m_i,i=1,\cdots,s$的最小公倍式。
\end{problem}
\end{document}