\documentclass[12pt, a4paper, oneside]{ctexart}
\usepackage{amsmath, amsthm, amssymb, bm, graphicx, hyperref, mathrsfs}

\title{\textbf{HW8}}
\author{范潇\quad2254298}
\date{\today}
\linespread{1.5}
\newcounter{problemname}
\newenvironment{problem}[1]{\stepcounter{problemname}\par\noindent\textbf{题目\arabic{problemname}. (#1)}}{}
\newenvironment{solution}{\par\noindent\textbf{解答. }}{\\\par}
\newenvironment{note}{\par\noindent\textbf{题目\arabic{problemname}的注记. }}{\\\par}

\begin{document}
\maketitle
\begin{problem}{6.5.2}
    设 $W_1$与 $W_2$分别是齐次线性方程组 $x_1+x_2+\cdots+x_n=0$与 $x_1=x_2=\cdots=x_n$的解空间。证明 $\mathbb{R}^n=W_1\oplus W_2$
\end{problem}
\begin{solution}
    显然 $W_1+W_2\subseteq \mathbb{R}^n$。任取 $\mathbf{x}=(x_1,\cdots,x_n)\in \mathbb{R}^n,$有
    \[\mathbf{x}=\bar{\mathbf{x}}+\mathbf{x}^{\prime}\]
    其中 $\bar{\mathbf{x}}=(\bar{x},\cdots,\bar{x})\in W_2,\bar{x}=\frac{1}{n}\sum_ix_i,\mathbf{x}^{\prime}=(x_1-\bar{x},\cdots,x_n-\bar{x})\in W_1$.

    所以$\mathbb{R}^n\subseteq W_1+W_2$,从而$\mathbb{R}^n=W_1+W_2$。

    任取 $\mathbf{x}=(x_1,\cdots,x_n)\in W_1\cap W_2$,则
    \[x_1+x_2+\cdots+x_n=0\]
    \[x_1=x_2=\cdots=x_n\]
    易得
    \[x_1=x_2=\cdots=x_n=0\]
    即 $W_1\cap W_2=\{\mathbf{0}\}$。因此和为直和,从而$\mathbb{R}^n=W_1\oplus W_2$。
\end{solution}
\end{document}