\documentclass[12pt, a4paper, oneside]{ctexart}
\usepackage{amsmath, amsthm, amssymb, bm, graphicx, hyperref, mathrsfs}

\title{\textbf{HW1}}
\author{范潇\quad2254298}
\date{\today}
\linespread{1.5}
\newcounter{problemname}
\newenvironment{problem}[1]{\stepcounter{problemname}\par\noindent\textbf{题目\arabic{problemname}. (#1)}}{\\\par}
\newenvironment{solution}{\par\noindent\textbf{解答. }}{\\\par}
\newenvironment{note}{\par\noindent\textbf{题目\arabic{problemname}的注记. }}{\\\par}

\begin{document}

\maketitle

\begin{problem}{1.2.7}
    如果等式$f(x)=q(x)g(x)+r(x)$成立,则$f(x),g(x)$与$g(x),r(x)$有相同的公因式。
\end{problem}

\begin{solution}
    设 $h(x)$为$f(x)$与 $g(x)$的一个公因式,则 $h(x)\mid f(x),h(x)\mid g(x)$,
    从而有
    \[h(x)\mid[f(x)-q(x)g(x)]=r(x),\]
    又因为
    \[h(x)\mid g(x)\]
    所以 $h(x)$也是 $g(x)$与 $r(x)$的公因式。

    因为 $r(x)=(-q(x))g(x)+f(x)$,由对称性可知, $g(x)$与 $r(x)$的公因式也是 $g(x)$与 $f(x)$的公因式。 

    从而
    $f(x),g(x)$与$g(x),r(x)$有相同的公因式。
    \qed
\end{solution}
\end{document}