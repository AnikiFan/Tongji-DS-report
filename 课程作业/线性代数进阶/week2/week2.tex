\documentclass[12pt, a4paper, oneside]{ctexart}
\usepackage{amsmath, amsthm, amssymb, bm, graphicx, hyperref, mathrsfs}

\title{\textbf{HW3}}
\author{范潇\quad2254298}
\date{\today}
\linespread{1.5}
\newcounter{problemname}
\newenvironment{problem}[1]{\stepcounter{problemname}\par\noindent\textbf{题目\arabic{problemname}. (#1)}}{}
\newenvironment{solution}{\par\noindent\textbf{解答. }}{\\\par}
\newenvironment{note}{\par\noindent\textbf{题目\arabic{problemname}的注记. }}{\\\par}

\begin{document}

\maketitle

\begin{problem}{2.2.9}
    证明
    \[[\mathbf{a},\mathbf{b},\mathbf{c}\times \mathbf{d}]+[\mathbf{b},\mathbf{c},\mathbf{a}\times \mathbf{d}]+[\mathbf{c},\mathbf{a},\mathbf{b}\times \mathbf{d}]=\mathbf{0}\]
\end{problem}
\begin{solution}
    因为对于任意三维空间向量 $\mathbf{a},\mathbf{b},\mathbf{c},\mathbf{d}$有:
    \[
    \begin{array}[H]{rcl}
       (\mathbf{a}\times \mathbf{b})\cdot(\mathbf{c}\times \mathbf{d})&=&\mathbf{a}\cdot \mathbf{b}\times(\mathbf{c}\times \mathbf{d})\\
       &=&\mathbf{a}\cdot((\mathbf{b}\cdot \mathbf{d})\mathbf{c}-(\mathbf{b}\cdot \mathbf{c})\mathbf{d}) \\
       &=&(\mathbf{b}\cdot \mathbf{d})(\mathbf{a}\cdot \mathbf{c})-(\mathbf{b}\cdot \mathbf{c})(\mathbf{a}\cdot \mathbf{d})\\
    \end{array}    
    \]
    所以
    \[
    \begin{array}[H]{rl}
    &[\mathbf{a},\mathbf{b},\mathbf{c}\times \mathbf{d}]+[\mathbf{b},\mathbf{c},\mathbf{a}\times \mathbf{d}]+[\mathbf{c},\mathbf{a},\mathbf{b}\times \mathbf{d}]\\
    =& (\mathbf{a}\times \mathbf{b})\cdot(\mathbf{c}\times \mathbf{d})+(\mathbf{b}\times \mathbf{c})\cdot(\mathbf{a}\times \mathbf{d})+(\mathbf{c}\times \mathbf{a})\cdot(\mathbf{b}\times \mathbf{d}) \\
    =& (\mathbf{b}\cdot \mathbf{d})(\mathbf{a}\cdot \mathbf{c})-(\mathbf{b}\cdot \mathbf{c})(\mathbf{a}\cdot \mathbf{d})+\\
    & (\mathbf{c}\cdot \mathbf{d})(\mathbf{b}\cdot \mathbf{a})-(\mathbf{c}\cdot \mathbf{a})(\mathbf{b}\cdot \mathbf{d})+\\
    & (\mathbf{a}\cdot \mathbf{d})(\mathbf{c}\cdot \mathbf{b})-(\mathbf{a}\cdot \mathbf{b})(\mathbf{c}\cdot \mathbf{d})\\
    =&\mathbf{0}\\ 
    \end{array}    
    \]
    \qed
\end{solution}
\end{document}