\chapter{概述}
\thispagestyle{empty}
\section{课题背景}
共享单车作为共享经济的新形态,
是指企业与政府进行合作,在居民区、商业区、地铁站、公交车
站、校园等公共场所提供自行车骑行的共享服务。它是借助互
联网技术推出的一种分时租赁业务。在整个运营过程中由智能
技术提供支持,企业负责整个运营过程中的管理\cite{article1}。

消费者在选择共享单车品牌时,除了定价、押金等经济因素,还会着重考量以下体验因素:
\begin{itemize}
    \item 单车性能:例如单车重量,可调节性等;
    \item 维护质量:是否有大量未及时回收的损坏单车;
    \item 单车分布:单车的时空分布是否符合使用需求的时空分布。
\end{itemize}

同时,由于共享单车市场的快速发展,交通运输部等10部门也对共享单车(互联网租凭自行车)行业发布指导意见,要求“引导有序投放车辆”、“加强互联网租凭自行车标准化建设”、“加强停放管理和监督执法”,“引导用户安全文明用车”等。

\section{编写目的}

本课题旨在为一个虚构的新创共享单车公司开发单车管理与调度平台,以帮助公司的工作人员对该公司的共享单车进行管理与调度,达到提升公司市场竞争力、符合政府规范的最终目的。

结合当今各大共享单车公司,如美团单车、哈喽单车等,的运营模式,我们对于该公司作以下假设:
\begin{enumerate}
    \item 出于存储代价的考量,该公司的数据库中不保存历史数据。例如,对于“单车”实体,只保存其最近上传的坐标;
    \item 该公司使用的共享单车只会在开关锁的时候向系统上传信息;
    \item 运维人员的运维任务由组长通过微信、钉钉等工具下发,或者自行根据实际情况实施调度(因此不在数据库中进行记录);
    \item 要求用户将车辆停放在指定停车区域内(因此需要对于停车区域进行建模)。
\end{enumerate}

由于本次设计的平台用于共享单车企业的共享单车管理与调度。因此,所用数据库中存储的数据均围绕“共享单车”这一主体,
并不涉及共享单车用户的余额、购买的月卡套餐等信息,从而并未对共享单车用户进行建模。

