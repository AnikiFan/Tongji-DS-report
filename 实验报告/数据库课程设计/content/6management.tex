\chapter{项目管理}
\thispagestyle{empty}
\section{框架选择}
本项目采用Next.js作为全栈框架。它是一个开源的 JavaScript 框架,基于 React 构建,旨在简化和优化现代 Web 应用的开发。它提供了很多开箱即用的功能,能够帮助开发者构建快速、可扩展的 Web 应用,尤其适用于静态网站生成(SSG)、服务器端渲染(SSR)和客户端渲染(CSR)。

Next.js具有以下特点:
\begin{enumerate}
    \item \textbf{服务器端渲染(SSR)}:
    Next.js 支持服务器端渲染,这意味着页面的 HTML 可以在服务器上预先渲染,然后传送给客户端。这有助于提高 SEO(搜索引擎优化)性能,因为搜索引擎可以直接读取渲染好的 HTML。
    
    \item \textbf{静态站点生成(SSG)}:
    Next.js 支持通过静态站点生成(SSG)功能,在构建时生成静态 HTML 页面。这对于内容更新不频繁的应用非常有效,能显著提高性能。
    
    \item \textbf{API 路由}:
    Next.js 可以用来构建 API 服务,允许开发者在同一个项目中同时处理前端和后端逻辑。开发者可以在 \texttt{pages/api} 目录下创建 API 路由,从而处理服务器端的请求。
    
    \item \textbf{增量静态生成(ISR)}:
    Next.js 提供了增量静态生成功能,允许开发者在生产环境中动态地更新静态页面。开发者可以设置页面的重新生成周期,这对于需要定期更新内容的应用非常有用。
    
    \item \textbf{自动代码拆分}:
    Next.js 会自动拆分 JavaScript 代码,按需加载页面,确保每个页面只加载它需要的 JavaScript,提升页面加载速度。
    
    \item \textbf{文件系统路由}:
    在 Next.js 中,路由是基于文件系统的。只要在 \texttt{pages} 目录下创建相应的文件或目录,就会自动生成对应的路由。
    
    \item \textbf{静态文件支持}:
    Next.js 允许开发者将静态文件(如图片、字体、JSON 等)放置在 \texttt{public} 目录中,并自动生成可访问的 URL。
    
    \item \textbf{类型支持}:
    Next.js 完全支持 TypeScript,可以直接在项目中使用 TypeScript,无需额外配置。
    
    \item \textbf{React 特性}:
    由于 Next.js 基于 React,开发者可以利用 React 的所有特性,如组件化开发、Hooks、Context 等。
\end{enumerate}
\section{开发平台}
本项目使用的开发平台如下:
\begin{itemize}
    \item \textbf{开发语言}:TypeScript;
    \item \textbf{开发框架}:采用Next.js作为全栈框架;
    \item \textbf{集成开发环境}:VSCode,WebStorm;
    \item \textbf{版本控制工具}:Git;
    \item \textbf{前期设计工具}:Adobe XD;
    \item \textbf{代码托管平台}:GitHub;
    \item \textbf{依赖管理工具}:pnpm;
    \item \textbf{数据库管理工具}:DataGrip,pgAdmin4;
    \item \textbf{数据库}:PostgreSQL+PostGIS;
    \item \textbf{ORM框架}:DrizzleORM;
    \item \textbf{CSS框架}:Tailwind CSS;
    \item \textbf{组件库}:NextUI;
    \item \textbf{地理空间数据可视化平台}:Deck.gl,Mapbox;
\end{itemize}