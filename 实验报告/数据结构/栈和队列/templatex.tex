\documentclass[a4paper,11pt]{article}%必须以此为开头,可以在[]内设置栏数,单双面,横竖向
\usepackage{ctex}%提供中文支持
\usepackage{graphicx}%用于插入图片
\usepackage{verbatim}%使用\verbatiminput{filename}来直接导入文件中的文本内容
\usepackage{layouts}%用于设置页面布局
\usepackage{makecell}%允许表格的单元格内换行
%伪代码
\usepackage{algorithm}
\usepackage{algorithmicx}
\usepackage{algpseudocode}
\usepackage{amsmath}
%伪代码
%代码块
\usepackage{ listings}
\usepackage[x11names]{xcolor}
\definecolor{mygreen}{rgb}{0,0.6,0}
\definecolor{mygray}{rgb}{0.5,0.5,0.5}
\definecolor{mymauve}{rgb}{0.58,0,0.82}
\lstset{ %
	backgroundcolor=\color{white},      % choose the background color
	basicstyle=\footnotesize\ttfamily,  % size of fonts used for the code
	columns=fullflexible,
	tabsize=4,
	breaklines=true,               % automatic line breaking only at whitespace
	captionpos=b,                  % sets the caption-position to bottom
	commentstyle=\color{mygreen},  % comment style
	escapeinside={\%*}{*)},        % if you want to add LaTeX within your code
	keywordstyle=\color{blue},     % keyword style
	stringstyle=\color{mymauve}\ttfamily,  % string literal style
	frame=single,
	rulesepcolor=\color{red!20!green!20!blue!20},
	% identifierstyle=\color{red},
	language=c,
}
%代码块
%引用
\usepackage{framed}
\usepackage{quoting}
 \colorlet{shadecolor}{Turquoise1!20}
\newenvironment{shadedquotation}
 {\begin{shaded*}
  \quoting[leftmargin=0pt, vskip=0pt]
 }
 {\endquoting
 \end{shaded*}
}
%引用
%跨页伪代码
\makeatletter
\newenvironment{breakablealgorithm}
  {% \begin{breakablealgorithm}
   \begin{center}
     \refstepcounter{algorithm}% New algorithm
     \hrule height.8pt depth0pt \kern2pt% \@fs@pre for \@fs@ruled
     \renewcommand{\caption}[2][\relax]{% Make a new \caption
       {\raggedright\textbf{\ALG@name~\thealgorithm} ##2\par}%
       \ifx\relax##1\relax % #1 is \relax
         \addcontentsline{loa}{algorithm}{\protect\numberline{\thealgorithm}##2}%
       \else % #1 is not \relax
         \addcontentsline{loa}{algorithm}{\protect\numberline{\thealgorithm}##1}%
       \fi
       \kern2pt\hrule\kern2pt
     }
  }{% \end{breakablealgorithm}
     \kern2pt\hrule\relax% \@fs@post for \@fs@ruled
   \end{center}
  }
\makeatother
%跨页伪代码
\newcommand*{\abs}[1]{\lvert #1 \rvert}
\floatname{algorithm}{算法}
\renewcommand{\algorithmicrequire}{\textbf{输入:}}
\renewcommand{\algorithmicensure}{\textbf{输出:}}
\author{姓名: 范潇\phantom{11}学号: 2254298}
\title{作业 HW2 试验报告}
\date{2023年 10月 15日}
\begin{document}
\lstset{breaklines}%这条命令可以让LaTeX自动将长的代码行换行排版
		\lstset{extendedchars=false}%这一条命令可以解决代码跨页时,章节标题,页眉等汉字不显
		%示的问题
		\lstset{numbers=left,numberstyle=\tiny,keywordstyle=\color{blue!70},
			commentstyle=\color{red!50!green!50!blue!50},frame=shadowbox,
			rulesepcolor=\color{red!20!green!20!blue!20},escapeinside=``,
			%xleftmargin=-10em,xrightmargin=-23em,
            aboveskip=1em} 
\pagestyle{plain}%页面风格,plain为中下方有页码.heading是页眉中间有页数,同时有章节名,empty是空页眉页尾
%\thispagestyle{pagestyle}%本页页面风格
% 实验报告格式要求按照模板(使用Markdown等也请保证报告内包含模板中的要素)
% 对字体大小、缩进、颜色等不做强制要求(但尽量代码部分和文字内容有一定区分,可参考vscode配色)
% 实验报告要求在文字简洁的同时将内容表示清楚
% 报告内不要大段贴代码,尽量控制在20页以内
\maketitle
\section{涉及数据结构和相关背景}
本次实验主要涉及栈和队列这两个数据结构。

栈,即限定只在表的一段(表尾)进行插入和删除操作的线性表。
我们可以用顺序表来实现顺序栈,或者用不带头结点的单链表来实现链栈,也可以用固定大小的数组来模拟栈。

队列,即限定在表的一端进行删除,在表的另一端进行插入操作的线性表,允许插入的一端称为队头,允许插入的一端称为队尾。
队列的存储结构有链队列和循环队列。其中链队列的本质即带头结点的线性链表。循环队列则是队列的顺序存储结构,只能用静态数组来实现。

下面给出的是利用静态分配整型指针的栈的功能实现。
\begin{lstlisting}[language={[ANSI]C},keywordstyle=\color{blue!70},commentstyle=\color{red!50!green!50!blue!50},frame=shadowbox,
				rulesepcolor=\color{red!20!green!20!blue!20}]
//初始化栈
Status InitStack(SqStack &S)
{
	S.base = (ElemType*)malloc(sizeof(ElemType)*INIT_STACK_SIZE);
	if(!S.base)//检查是否申请成功
	  return OVERFLOW;
	S.stacksize = INIT_STACK_SIZE;
	S.top = S.base;
	return OK;
}
\end{lstlisting}
\begin{lstlisting}[language={[ANSI]C},keywordstyle=\color{blue!70},commentstyle=\color{red!50!green!50!blue!50},frame=shadowbox,
				rulesepcolor=\color{red!20!green!20!blue!20}]
//获取栈顶数据
Status GetTop (SqStack S,ElemType &e)
{
	if(S.top == S.base)//检查是否为空
	  return ERROR;
	e = *(S.top -1);
	return OK;
}
\end{lstlisting}
\begin{lstlisting}[language={[ANSI]C},keywordstyle=\color{blue!70},commentstyle=\color{red!50!green!50!blue!50},frame=shadowbox,
				rulesepcolor=\color{red!20!green!20!blue!20}]
//压栈
Status Push(SqStack &S,ElemType e)
{
	if(S.top-S.base >= S.stacksize){//检查是否上溢
		S.base = (ElemType*)realloc(S.base,sizeof(ElemType)*(S.stacksize+STACK_SIZE_INCREMENT));
		if(!S.base)
		  return OVERFLOW;
		S.top = S.base + S.stacksize;
		S.stacksize += STACK_SIZE_INCREMENT;
	}
	*S.top++=e;
	return OK;
}
\end{lstlisting}
\begin{lstlisting}[language={[ANSI]C},keywordstyle=\color{blue!70},commentstyle=\color{red!50!green!50!blue!50},frame=shadowbox,
				rulesepcolor=\color{red!20!green!20!blue!20}]
//出栈
Status Pop(SqStack &S,ElemType &e)
{
	if(S.base == S.top)//判断是否为空
	  return ERROR;
	e = *--S.top;
	return OK;
}
\end{lstlisting}
\begin{lstlisting}[language={[ANSI]C},keywordstyle=\color{blue!70},commentstyle=\color{red!50!green!50!blue!50},frame=shadowbox,
				rulesepcolor=\color{red!20!green!20!blue!20}]
//判断是否栈空
Status StackEmpty(SqStack S)
{
	return S.top ==S.base;
}
\end{lstlisting}
\begin{lstlisting}[language={[ANSI]C},keywordstyle=\color{blue!70},commentstyle=\color{red!50!green!50!blue!50},frame=shadowbox,
				rulesepcolor=\color{red!20!green!20!blue!20}]
//重置栈
Status EmptyStack(SqStack &S)
{
	S.top = S.base;
	S.stacksize =0;
	return OK;
}
\end{lstlisting}
下面给出的是链队列的一些功能实现
\begin{lstlisting}[language={[ANSI]C},keywordstyle=\color{blue!70},commentstyle=\color{red!50!green!50!blue!50},frame=shadowbox,
				rulesepcolor=\color{red!20!green!20!blue!20}]
//初始化队列
Status InitQueue(LinkQueue &Q)
{
	Q.front = Q.rear = (QueuePtr)malloc(sizeof(Node));
	if(!Q.front)
	  return OVERFLOW; 
	Q.front ->next = NULL;
	return OK;
}
\end{lstlisting}
\begin{lstlisting}[language={[ANSI]C},keywordstyle=\color{blue!70},commentstyle=\color{red!50!green!50!blue!50},frame=shadowbox,
				rulesepcolor=\color{red!20!green!20!blue!20}]
//销毁队列
Status DestroyQueue(LinkQueue &Q)
{
	while(Q.front)
	{
		Q.rear = Q.front->next;
		free(Q.front);
		Q.front=Q.rear;
	}
	return OK;
}
\end{lstlisting}
\begin{lstlisting}[language={[ANSI]C},keywordstyle=\color{blue!70},commentstyle=\color{red!50!green!50!blue!50},frame=shadowbox,
				rulesepcolor=\color{red!20!green!20!blue!20}]
//进入队列
Status EnQueue(LinkQueue&Q,ElemType e)
{
	Q.rear->next =  (Node*) malloc(sizeof(Node));
	if(!Q.rear->next)
	  return OVERFLOW;
	Q.rear = Q.rear->next;
	Q.rear->next = NULL;
	Q.rear->data = e;
	return OK;
}
\end{lstlisting}

\begin{lstlisting}[language={[ANSI]C},keywordstyle=\color{blue!70},commentstyle=\color{red!50!green!50!blue!50},frame=shadowbox,
				rulesepcolor=\color{red!20!green!20!blue!20}]
//出队列
Status DeQueue(LinkQueue&Q,ElemType &e)
{
	if(Q.front == Q.rear)
	  return ERROR;
	Node * temp = Q.front->next;
	Q.front ->next = temp->next;
	if(!temp->next)
	  Q.rear = Q.front;
	e = temp->data;
	free(temp);
	return OK;
}
\end{lstlisting}

\begin{lstlisting}[language={[ANSI]C},keywordstyle=\color{blue!70},commentstyle=\color{red!50!green!50!blue!50},frame=shadowbox,
				rulesepcolor=\color{red!20!green!20!blue!20}]
//判断队列是否为空
Status QueueEmpty(LinkQueue Q)
{
	return Q.front == Q.rear;
}
\end{lstlisting}
\section{实验内容}
\subsection{列车进站}
\subsubsection{问题描述}
\begin{shadedquotation}
    \emph{问题描述:}
    有一队编号不同的火车依次入站。把火车站视为一个栈,请问哪些出站的顺序是合法的?
\end{shadedquotation}
\subsubsection{基本要求}
\begin{shadedquotation}
    \emph{输入要求:}
    第1行,一个串,入站序列。
后面多行,每行一个串,表示出栈序列
当输入=EOF时结束。
\end{shadedquotation}
\begin{shadedquotation}
    \emph{输出要求:}
    多行,若给定的出栈序列可以得到,输出yes,否则输出no。
\end{shadedquotation}
\subsubsection{数据结构设计}
\begin{lstlisting}[language={[ANSI]C},keywordstyle=\color{blue!70},commentstyle=\color{red!50!green!50!blue!50},frame=shadowbox,
				rulesepcolor=\color{red!20!green!20!blue!20}]
char in [LIST_INIT_SIZE];//存储进站次序
char out[LIST_INIT_SIZE];//存储出站顺序
typedef char ElemType;
typedef struct{
	ElemType* top;
	ElemType* base;
	int stacksize;
}SqStack;//利用静态分配指针来实现
SqStack station;//模拟火车站
\end{lstlisting}
\subsubsection{功能说明}
\begin{lstlisting}[language={[ANSI]C},keywordstyle=\color{blue!70},commentstyle=\color{red!50!green!50!blue!50},frame=shadowbox,
				rulesepcolor=\color{red!20!green!20!blue!20}]
int main()
{
	InitStack(station);//初始化模拟栈
	scanf("%s",&in);//存储进栈顺序
	while(scanf("%s",&out)==1&&out[0]){//循环读取出栈顺序
	int i =0,j=0;
	char temp;
	EmptyStack(station);//重置栈
		while(in[i]!='\0'){//模拟进栈
			Push(station,in[i]);//进栈
			temp = in[i];//读取待出栈编号
			while(out[j] == temp){//循环判断当前的出栈编号是否等于待出栈编号
				Pop(station,temp);//出栈
			 GetTop(station,temp);//读取下一个待出栈编号
				j++;
			}
			i++;//循环进栈
		}
		if(StackEmpty(station))//栈空则说明该出栈顺序是合法的。
		  printf("yes\n");
		else
		  printf("no\n");
	}
	return 0;
}
\end{lstlisting}
\subsubsection{调试分析}
在调试该程序时,我发现如果当一个出栈次序被判定为不合法后,后续的都被判为不合法,依次
判断问题出在循环时没有重置变量,进行了针对性排查后发现是每次读入出栈次序时,没有将栈进行重置,
在循环开头添加了对应语句后便成功通过。
\subsubsection{总结与体会}
在涉及处理多个输入时,要确保相互之间不会影响,即要在循环开头对于变量进行重置。
\subsection{布尔表达式}
\subsubsection{问题描述}
\begin{shadedquotation}
    \emph{问题描述:}
    求解布尔表达式。
\end{shadedquotation}
\subsubsection{基本要求}
\begin{shadedquotation}
    \emph{输入要求:}
    文件输入,有若干(A$<$=20)个表达式,其中每一行为一个表达式。 表达式有(N$<$=100)个符号,符号间可以用任意空格分开,或者没有空格,所以表达式的总长度,即字符的个数,是未知的。
\end{shadedquotation}
\begin{shadedquotation}
    \emph{输出要求:}
    对测试用例中的每个表达式输出“Expression”,后面跟着序列号和“: ”,然后是相应的测试表达式的结果(V或F),每个表达式结果占一行(注意冒号后面有空格)。
\end{shadedquotation}
\subsubsection{数据结构设计}
\begin{lstlisting}[language={[ANSI]C},keywordstyle=\color{blue!70},commentstyle=\color{red!50!green!50!blue!50},frame=shadowbox,
				rulesepcolor=\color{red!20!green!20!blue!20}]
typedef struct {
	ElemType base[MAX_SIZE];//由于长度上限已知,且较小,所以用静态数组来实现
	int top;
}SqStack;
SqStack Operand;//存储VF
SqStack Opt;//存储其他运算符
char expression[MAX_SIZE];//存储表达式
\end{lstlisting}
\subsubsection{功能说明}
\begin{breakablealgorithm}
    \caption{运算符优先级}
    \begin{algorithmic}[1]
        \Require char\,in待进栈运算符,char\,cur栈顶运算符
        \Ensure 比较结果
        \Function {compare}{char\,in,char\,cur}
        \State{利用switch和if语句以及运算符之间的优先级和结合性来得出结果}
        \State \Return 比较结果
        \EndFunction
    \end{algorithmic}
\end{breakablealgorithm}
\begin{breakablealgorithm}
    \caption{运算}
    \begin{algorithmic}[1]
        \Require SqStack\,\&Operand布尔值栈,SqStack\,\&Opt运算符栈
        \Ensure void
        \Function {calc}{SqStack\,\&Operand,SqStack\,\&Opt}
        \If{运算符栈顶为!}
            \State{运算符栈出栈}
            \State{布尔值栈出栈}
            \State{相反的布尔值压进布尔值栈}
        \Else 
        \State{运算符栈出栈}
        \State{布尔值栈出栈两次}
        \State{进行运算}
        \State{将结果对应的布尔值压入布尔值栈}
        \EndIf
        \EndFunction
    \end{algorithmic}
\end{breakablealgorithm}
\begin{breakablealgorithm}
    \caption{求解布尔表达式}
    \begin{algorithmic}[1]
        \Require 布尔表达式
        \Ensure 布尔值
        \Function {SolveExpression}{char\,Expression[\,] 布尔表达式}
        \While{遍历Expression中的每个字符,直至尾零}
            \If{是布尔值}
                \State{压入布尔值栈}
            \Else 
                \If{布尔运算符栈非空}
                \State{temp取运算符栈顶部}
                     \While{该布尔运算符比栈顶布尔运算符temp优先级高}
                        \State{计算}
                         \If{运算符栈为空}
                        \State{break}
                        \Else 
                        \State{temp取运算符栈顶部}
                        \EndIf 
                     \EndWhile
                \EndIf 
            \EndIf
        \EndWhile
        \While{运算符栈非空}
        \State{计算}
        \EndWhile
        \State{布尔值栈出栈并打印}
        \EndFunction
    \end{algorithmic}
\end{breakablealgorithm}
\subsubsection{调试分析}
在调试该程序的过程中,我不仅仅打印出中间变量的值,还将中间所调用的函数名也给打印出来,
方便查找问题的原因。
\subsubsection{总结与体会}
本题是求解涉及四则运算的表达式的变体。易错点在于取反运算符是右结合,而非左结合。同时,
取反运算符是一元运算符,在实现时需要多加注意。

本例也体现出栈的实用性。
\subsection{最长字串}
\subsubsection{问题描述}
\begin{shadedquotation}
    \emph{问题描述:}
    已知一个长度为n,仅含有字符'('和')'的字符串,请计算出最长的正确的括号子串的长度及起始位置,若存在多个,取第一个的起始位置。
子串是指任意长度的连续的字符序列。
\end{shadedquotation}
\subsubsection{基本要求}
\begin{shadedquotation}
    \emph{输入要求:}
    一行字符串。
\end{shadedquotation}
\begin{shadedquotation}
    \emph{输出要求:}
    子串长度,及起始位置。
\end{shadedquotation}
\subsubsection{数据结构设计}
\begin{lstlisting}[language={[ANSI]C},keywordstyle=\color{blue!70},commentstyle=\color{red!50!green!50!blue!50},frame=shadowbox,
				rulesepcolor=\color{red!20!green!20!blue!20}]
typedef struct {
	ElemType base[MAX_SIZE];//由于长度上限固定,且只有单一测试样例,所以选择使用静态数组实现。
	int top;
}SqStack;//模拟配对过程的栈
SqStack index;
int maxlen=0,maxpos=0;//最大长度,最大长度的子串开始位置
int len[MAX_SIZE]={0};//存储以对应位置为结尾的最长子串的长度
\end{lstlisting}
\subsubsection{功能说明}
\begin{lstlisting}[language={[ANSI]C},keywordstyle=\color{blue!70},commentstyle=\color{red!50!green!50!blue!50},frame=shadowbox,
				rulesepcolor=\color{red!20!green!20!blue!20}]
int main()
{
	int n =0;
	int tempindex;
	char temp = getchar();
	InitStack(index);
	while(temp !='\r'&&temp !='\n'&&temp !=EOF){
		if(temp == '(')//读取的是左括号
		  Push(index,n);//等待与之匹配的右括号
		if(temp ==')'&&!StackEmpty(index)){//读取的是右括号,且有与之匹配的左括号
			Pop(index,tempindex);
			len[n]= n-tempindex+1+len[tempindex-1];
            //计算匹配的左右括号之间的长度,并加上在此之前与之相邻的子串长度
			if(len[n]>maxlen){//检查是否要更新最大值,及其位置
				maxlen = len[n];
				maxpos =n-len[n]+1;//通过右括号的位置以及字串长度来求得左括号的位置
			}
        }//否则读取的是非法的右括号
		temp = getchar();
		n++;
	}
	printf("%d %d",maxlen,maxpos);
	return 0;
}
\end{lstlisting}
\subsubsection{调试分析}
一开始提交测评后,反馈第1,4,7,8,10个样例没通过。后来将自己的输出和正确输出进行比较,发现一般数值十分接近。
最后发现是自己对题目的理解有误,我一开始以为合法的字串的两端是一对相互配对的括号,而实际上只要是一串均能够
相互配对的左右括号即可。
\subsubsection{总结与体会}
栈这一数据结构适合处理符合FILO的场景。
\subsection{队列的应用}
\subsubsection{问题描述}
\begin{shadedquotation}
    \emph{问题描述:}
    输入一个n*m的0 1矩阵,1表示该位置有东西,0表示该位置没有东西。所有四邻域联通的1算作一个区域,仅在矩阵边缘联通的不算作区域。
\end{shadedquotation}
\subsubsection{基本要求}
\begin{shadedquotation}
    \emph{输入要求:}
    第1行2个正整数n,m, 表示要输入的矩阵行数和列数。第2—n+1行为n*m的矩阵,每个元素的值为0或1。
\end{shadedquotation}
\begin{shadedquotation}
    \emph{输出要求:}
    1行,代表区域数。
\end{shadedquotation}
\subsubsection{数据结构设计}
\begin{lstlisting}[language={[ANSI]C},keywordstyle=\color{blue!70},commentstyle=\color{red!50!green!50!blue!50},frame=shadowbox,
				rulesepcolor=\color{red!20!green!20!blue!20}]
typedef struct{
	int i;
	int j;
}ElemType,index;//二维数组中的下标
typedef struct Node{//元素结点
	ElemType data;
	struct Node * next;
}Node,*QueuePtr;//采用链队列
typedef struct {//特殊结点
	QueuePtr front;//队头指针
	QueuePtr rear;//队尾指针
}LinkQueue;
LinkQueue Queue;//用于搜索的队列
int mark[MAX_SIZE][MAX_SIZE]={0};//用于记录是否已被计数
int mat[MAX_SIZE][MAX_SIZE]={0};//用于存储输入矩阵
\end{lstlisting}
\subsubsection{功能说明}
\begin{breakablealgorithm}
    \caption{查找相邻区域}
    \begin{algorithmic}[1]
        \Require index\,cur搜索的起始下标,int\,\&valid是否为合法区域
        \Ensure void
        \Function {Search}{index\,cur,int\,\&valid}
        \State{cur进搜索队列}
        \While{搜索队列非空}
        \State{退队列,用cur来储存}
        \State{在将mark中的对应元素置1}
        \If{mat中与cur相邻的位置为1,且在mark中为0}
        \State{将该位置进队列}
        \If{该位置不为矩阵边缘}
        \State{valid置1}
        \EndIf
        \EndIf 
        \EndWhile
        \EndFunction
    \end{algorithmic}
\end{breakablealgorithm}
\begin{lstlisting}[language={[ANSI]C},keywordstyle=\color{blue!70},commentstyle=\color{red!50!green!50!blue!50},frame=shadowbox,
				rulesepcolor=\color{red!20!green!20!blue!20}]
int main()
{
	scanf("%d%d",&n,&m);
	for(int i =1;i<=n;i++)//存储输入矩阵
	  for(int j = 1;j<=m;j++)//下标起始为1
		scanf("%d",&mat[i][j]);
	for(int i = 1;i<=n;i++)
	  for(int j=1;j<=m;j++)
		if(!mark[i][j]&&mat[i][j]){//未被搜索过且为1
			int valid=0;//需要有在中间的元素
			InitQueue(Queue);//初始化搜索队列
			index index{i,j};
			Search(index,valid);//进行搜索
			DestroyQueue(Queue);//销毁队列
			if(valid)//如果合法,则计数器+1
			  num++;
		}
	printf("%d",num);
	return 0;
}
\end{lstlisting}
\subsubsection{调试分析}
程序结构较清晰,没有在调试过程中遇到困难。
\subsubsection{总结与体会}
可以利用循环和队列来代替一部分递归的使用。

这道题的一个易错点在于矩阵边缘部分的搜索,为了避免发生越界访问,以及为了
统一处理,在存储输入二维矩阵时,下标应该从1开始,使得存储的二维矩阵有一圈为0的“边框”。
\subsection{队列中的最大值问题}
\subsubsection{问题描述}
\begin{shadedquotation}
    \emph{问题描述:}
    给定一个队列,有下列3个基本操作:
    \begin{enumerate}
        \item Enqueue(v): v 入队
        \item Dequeue(): 使队首元素删除,并返回此元素
        \item GetMax(): 返回队列中的最大元素
    \end{enumerate}
请设计一种数据结构和算法,让GetMax操作的时间复杂度尽可能地低。
\end{shadedquotation}
\subsubsection{基本要求}
\begin{shadedquotation}
    \emph{输入要求:}
    第1行1个正整数n, 表示队列的容量(队列中最多有n个元素)。接着读入多行,每一行执行一个动作。
\end{shadedquotation}
\begin{shadedquotation}
    \emph{输出要求:}
    多行,分别是执行每次操作后的结果。

    若输入"dequeue",表示出队,当队空时,输出一行“Queue is Empty”;否则,输出出队的元素;

    若输入"enqueue m",表示将元素m入队,当队满时(入队前队列中元素已有n个),输出"Queue is Full",否则,不输出;

    若输入"max",输出队列中最大元素,若队空,输出一行“Queue is Empty”。

    若输入"quit",结束输入,输出队列中的所有元素。
\end{shadedquotation}
\subsubsection{数据结构设计}
\begin{lstlisting}[language={[ANSI]C},keywordstyle=\color{blue!70},commentstyle=\color{red!50!green!50!blue!50},frame=shadowbox,
				rulesepcolor=\color{red!20!green!20!blue!20}]
typedef long long Status,ElemType;
typedef struct Node{//元素结点
	ElemType data;
	struct Node * next;
}Node,*QueuePtr;
typedef struct {//特殊结点
	ElemType maxdata;//用于存储当前链表中的最大值
	long long max_size;
	QueuePtr front;//队头指针
	QueuePtr rear;//队尾指针
	long long Length;
}LinkQueue;
LinkQueue Q;
\end{lstlisting}
\subsubsection{功能说明}
\begin{breakablealgorithm}
    \caption{初始化队列}
    \begin{algorithmic}[1]
        \Require LinkQueue\,\&Q待初始化队列,long\,long\,max\_size队列最大值
        \Ensure 操作状态
        \Function {InitQueue}{LinkQueue\,\&Q,long\,long\,max\_size}
        \State 为头尾指针申请内存空间,并判断是否成功
        \State 将队列的长度置零
        \State 将队列的最大长度置为max\_size
        \State \Return OK
        \EndFunction
    \end{algorithmic}
\end{breakablealgorithm}
\begin{breakablealgorithm}
    \caption{数据进入队列}
    \begin{algorithmic}[1]
        \Require LinkQueue\,\&Q队列,ElemType\,e待进入元素
        \Ensure 操作状态
        \Function {InitQueue}{LinkQueue\,\&Q,long\,long\,max\_size}
        \State 判断队列是否已满
        \State 为普通结点申请内存空间,并判断是否成功
        \If{队列为空}
        \State 将特殊结点中的maxdata置为e
        \Else
        \If {e大于maxdata}
        \State   将maxdata置为e
        \EndIf
        \EndIf
        \State 特殊结点的Length加一
        \State 将新普通节点进入队列
        \State \Return OK
        \EndFunction
    \end{algorithmic}
\end{breakablealgorithm}
\begin{breakablealgorithm}
    \caption{搜寻队列中的最大值}
    \begin{algorithmic}[1]
        \Require LinkQueue\,\&Q队列
        \Ensure 操作状态
        \Function {FindMax}{LinkQueue\,\&Q}
        \State 判断队列是否为空
        \State 将工作指针指向开始结点
        \State 将max置为工作指针所指结点的数据
        \While{工作指针遍历队列}
        \If{工作指针所指结点数据大于max}
        \State{更新max}
        \EndIf
        \EndWhile
        \State{将特殊节点中的maxdata置为max}
        \State \Return OK
        \EndFunction
    \end{algorithmic}
\end{breakablealgorithm}
\begin{breakablealgorithm}
    \caption{出队列}
    \begin{algorithmic}[1]
        \Require LinkQueue\,\&Q队列,ElemType\,e
        \Ensure 操作状态
        \Function {DeQueue}{LinkQueue\,\&Q,ElemType\,\&e}
        \State 判断队列是否为空
        \State 将开始结点移出队列,对应的数据赋给e
        \If{e等于maxdata}
        \State{FindMax}
        \EndIf 
        \State 释放移除结点的内存空间。
        \State \Return OK
        \EndFunction
    \end{algorithmic}
\end{breakablealgorithm}
\subsubsection{调试分析}
采用输出中间变量以及调用函数的函数名的方式来辅助调试,过程较为顺利。 
\subsubsection{总结与体会}
可以在特殊结点或头结点中存储关于表的额外信息,本题中我便在特殊结点中增加了maxdata这一数据域来
记录表中的最大值。易错点在于初始化链队列时,不能将maxdata置1,因为后续的数据可能为负数。
正确的操作应该是将第一个进入的元素赋给maxdata。有了这一数据域后,便能降低遍历整个队列来获取最大值的频率。
\section{实验总结}
栈和列表本质上是操作受限的线性表,它们将操作限制在了表的两端,但是仍然有很广泛的应用场景。
操作的简化也使得栈和列表的实现过程相对简单。

根据使用场景中的数据是FIFO还是FILO,应该分别选用队列和栈。

在某些情况下,可以使用队列和循环的组合来代替递归函数。
\end{document}