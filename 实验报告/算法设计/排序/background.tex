\section{涉及数据结构和相关背景}
现实中的许多应用需要对数据进行排序。同时,许多算法中也包含了“排序”这一子步骤。因此,研究排序算法十分重要。
有许多排序算法,评价它们的质量好坏主要依据一下几个指标:时间复杂度,所需附加存储,复杂性。

排序算法的时间复杂度主要体现在所需的平均比较次数和平均移动次序上。
如果一个排序算法在排序过程中,会直接将待排序的最大(小)元素调整至它的最终位置,则称该排序算法具有部分排序功能。如果一个排序
算法不会改变排序依据——关键字——相同的数据的相对位置,则称该排序算法具有稳定性。