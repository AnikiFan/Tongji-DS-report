% this is a template for papers or notes written in Chinese
\documentclass[a4paper]{article}

\usepackage[
    fontset=none,%设置中文支持,并自定义字体
    zihao=5,%默认字号为五号
    heading=true,%允许后续自定义标题样式
    scheme=chinese,%自动将文档样式中文化,例如图标标题
    punct=quanjiao,%全角式标点符号
    space=auto,%中文后接换行不会添加空格,但是英文会添加空格,需要用%手动取消
    linespread=1.3,%行距倍数是1.3
    autoindent=true,%自动缩进两个中文宽度
    ]{ctex}
\ctexset{
    % tday=small,%小写样式的日期
    contentsname={目录},
    listfigurename={插图},
    listtablename={表格},
    figurename={图},
    tablename={表},
    abstractname={摘要},
    indexname={索引},
    appendixname={附录},
    bibname={参考文献},
    proofname={证明},
    % refname={参考文献},%只适用于beamer
    % algorithmname={算法},
    % continuation={(续)},%beamer续页的标识
    section={
        format+ = \Large\heiti\raggedright,
        name = {,\num\textbf{.}\hspace{1ex}},
        number={\num\thesection},
        nameformat={},
        numberformat={},
        aftername={},
        titleformat={},
        aftertitle={},
        runin=false,%对section级以下有用,标题是否和正文在同一段上
        beforeskip={3.5ex plus 1ex minus .2ex},%标题前垂直间距
        afterskip={2.3ex plus .2ex}%标题后垂直间距
    },
    subsection={
        format+ = \large\heiti\raggedright,
        name = {,\num\textbf{.}\hspace{1ex}},
        number={\num\thesubsection},
        nameformat={},
        numberformat={},
        aftername={},
        titleformat={},
        aftertitle={},
        runin=false,%对section级以下有用,标题是否和正文在同一段上
        beforeskip={3.5ex plus 1ex minus .2ex},%标题前垂直间距
        afterskip={2.3ex plus .2ex}%标题后垂直间距
    },
    subsubsection={
        format+ = \normalsize\heiti\raggedright,
        name = {,\num\textbf{.}\hspace{1ex}},
        number={\num\thesubsubsection},
        nameformat={},
        numberformat={},
        aftername={},
        titleformat={},
        aftertitle={},
        runin=false,%对section级以下有用,标题是否和正文在同一段上
        beforeskip={3.5ex plus 1ex minus .2ex},%标题前垂直间距
        afterskip={2.3ex plus .2ex}%标题后垂直间距
    },
    }

% 中文默认字体: 思源宋体,粗体为思源宋体半粗体,斜体为方正楷体_GBK
\setCJKmainfont{Source Han Serif SC}[BoldFont={Source Han Serif SC Heavy}, ItalicFont=FZKai-Z03S]
% 中文无衬线字体:思源黑体,粗体为思源黑体粗体
\setCJKsansfont{Source Han Sans CN}[BoldFont={Source Han Sans CN Heavy}]
% 中文等宽字体:微软雅黑light
\setCJKmonofont{Microsoft YaHei}[ItalicFont={Microsoft YaHei Light}]

\newCJKfontfamily\songti{Source Han Serif SC}[BoldFont={Source Han Serif SC Heavy}]
\newCJKfontfamily\xbsong{Source Han Serif SC SemiBold} % 小标宋
\newCJKfontfamily\dbsong{Source Han Serif SC Bold} % 大标宋
\newCJKfontfamily\cusong{Source Han Serif SC Heavy} % 粗宋
\newCJKfontfamily\heiti{Source Han Sans CN}[BoldFont={Source Han Sans CN Heavy}]
\newCJKfontfamily\dahei{Source Han Sans CN Medium} % 大黑
\newCJKfontfamily\cuhei{Source Han Sans CN Heavy} % 粗黑
\newCJKfontfamily\fangsong{FZFangSong-Z02S}
\newCJKfontfamily\kaiti{FZKai-Z03S}[ItalicFont={Microsoft YaHei Light}]%这个斜体只是用于lstlisting环境中的中文注释
% \newCJKfontfamily\kaiti{FZKai-Z03S}[ItalicFont={FZZJ-LZXTFSJW}]%这个斜体只是用于lstlisting环境中的中文注释
\setsansfont{Arial}
\setmonofont{Consolas}%设置西文等宽字体
\newfontfamily\code{Consolas}
\newfontfamily\num{Arial}

\usepackage{geometry}%设置整体页面布局
\geometry{a4paper}
\geometry{left=2cm,right=2cm,top=2.54cm,bottom=2.54cm}%word常规页边距
% \geometry{left=1.27cm,right=1.27cm,top=1.27cm,bottom=1.27cm}%word窄页边距
\setlength{\headheight}{13pt}%避免warning


\usepackage{fancyhdr}%必须在geometry包之后使用
\fancyhf{}
\makeatletter
\lhead{{\dahei \@title}}%可以使用thepage,CTEXthechapter,CTEXthesection
\makeatother
\rhead{\textbf{\num- \thepage{} -}}
\renewcommand\headrulewidth{1.5pt}%设置眉头宽度
\pagestyle{fancy}

\usepackage[ruled,algosection,lined,longend,fillcomment,linesnumbered,resetcount,titlenotnumbered]{algorithm2e}
%参数解释:带框,按section编码,有竖线,end前带if等关键词,注释占满整行,代码部分编号(不包括输入输出、注释),每个代码块重新编号,可以调用TitleOfAlgo来打印算法标题但不作为单独的算法编码
%附带algorithm,function,procedure环境,其中function,procedure环境下,设置caption时,必须带有(),
%()之前的字符会被视为宏,可以在接下来的部分用\名字()来调用,所以推荐辅助函数用function,其中的某些展开部分用procedure,描述算法整体使用algorithm
\DontPrintSemicolon
\SetAlCapSkip{2ex}
\SetSideCommentRight
\SetFillComment
\newcommand{\forcond}{$i=0$ \KwTo $n$}
\SetKw{downto}{downto}%自定义关键词
\SetKwFunction{funcmacro}{text}%自定义函数名,实际上function环境是在定义宏的同时说明了其内容
\SetKwProg{procedmacro}{text}{begin text}{end text}%自定义步骤,和function类似,但是后面两个参数可以设置开始和结尾的标志,和if等环境一样
\SetKwData{datamacro}{text}%可以用于突出特殊的变量,例如数据结构
\SetKwFunction{FRecurs}{FnRecursive}
\SetKwProg{Fn}{Function}{begin}{end}

\usepackage[strict]{changepage}
\usepackage{framed}%色块支持
\definecolor{formalshade}{rgb}{0.95,0.95,1} % 文本框颜色
% ------------------******-------------------
% 注意行末需要把空格注释掉,不然画出来的方框会有空白竖线
\newenvironment{formal}{%
\def\FrameCommand{%
\hspace{1pt}%
{\color{DarkBlue}\vrule width 2pt}%
{\color{formalshade}\vrule width 4pt}%
\colorbox{formalshade}%
}%
\MakeFramed{\advance\hsize-\width\FrameRestore}%
\noindent\hspace{-4.55pt}% disable indenting first paragraph
\begin{adjustwidth}{}{7pt}%
\vspace{2pt}\vspace{2pt}%
}
{%
\vspace{2pt}\end{adjustwidth}\endMakeFramed%
}

% 自定义标题格式
\makeatletter
\renewcommand{\maketitle}{
  \begin{center}
    \thispagestyle{fancy}
    {\quad}\\
    \vspace{0.1\textheight}
    {\huge\sffamily\bfseries\@title}\\ % 标题字体大小、粗体、颜色
    \vspace{2em} % 标题与作者名之间的垂直空间
    {\large\sffamily\@author} \\
  \end{center}
}
\makeatother

\usepackage{graphicx}
\usepackage{amsmath}

\title{作业{\hspace{1ex}}HW6{\hspace{1ex}}实验报告}
\author{姓名:范潇{\quad}学号:2254298{\quad}日期:\today}
\date{}
\begin{document}
\maketitle
% \thispagestyle{fancy}%用于单独设置某页的样式,此处用于设置标题页的格式
\section{涉及数据结构和相关背景}
现实中的许多应用需要对数据进行排序。同时,许多算法中也包含了“排序”这一子步骤。因此,研究排序算法十分重要。
有许多排序算法,评价它们的质量好坏主要依据一下几个指标:时间复杂度,所需附加存储,复杂性。

排序算法的时间复杂度主要体现在所需的平均比较次数和平均移动次序上。
如果一个排序算法在排序过程中,会直接将待排序的最大(小)元素调整至它的最终位置,则称该排序算法具有部分排序功能。如果一个排序
算法不会改变排序依据——关键字——相同的数据的相对位置,则称该排序算法具有稳定性。

\newpage
\section{实验内容}
\section{题目一}
\subsection{实验目的}
安装配置UNIX V6++的运行环境。
\subsection{实验内容}
\begin{enumerate}
    \item 安装后端服务根证书;
    \item 登录实验平台;
    \item  运行远程桌面环境;
    \item 初始化代码仓库;
    \item 启动UNIX V6++运行环境。
\end{enumerate}
\subsection{实验过程}
安装完后端服务根证书后,登录网站http://vesper-system.pages.tongji.edu.cn/vesper-front/,然后如图\ref{changeCode}所示,修改密码。接着,如图
\ref{changeName}所示,修改主机名。之后,我在GitLab中创建了自己的代码分支。等待桌面环境启用后,如图\ref{genCode}所示,生成密钥,然后如图\ref{setCode}所示,将公钥上传至GitLab。
当我尝试下载实验工具包时,要求我输入用户名和密码,为此,为如图\ref{setToken}所示,申请了个人访问令牌用作密码。下载完成后,将工作目录设置为unix-v6pp-tongji,然后如图\ref{init},\ref{makeQemu},\ref{qemu}所示,依次执行init.sh和make qemu命令,
最终启动了UNIX V6++界面并如图\ref{helloWorld}所示,执行了echo命令。

\begin{figure}[!htbp]
    \centering
    \includegraphics[scale=0.4]{fig/changeCode.png}
    \caption{修改密码}\label{changeCode}
\end{figure}
\begin{figure}[!htbp]
    \centering
    \includegraphics[scale=0.4]{fig/changeName.png}
    \caption{修改主机名}\label{changeName}
\end{figure}

\begin{figure}[!htbp]
    \centering
    \includegraphics[scale=0.7]{fig/genCode.png}
    \caption{生成密钥}\label{genCode}
\end{figure}

\begin{figure}[!htbp]
    \centering
    \includegraphics[scale=1]{fig/setCode.png}
    \caption{在GitLab中上传公钥}\label{setCode}
\end{figure}

\begin{figure}[!htbp]
    \centering
    \includegraphics[scale=0.4]{fig/setToken.png}
    \caption{设置个人访问令牌}\label{setToken}
\end{figure}

\begin{figure}[!htbp]
    \centering
    \includegraphics[scale=0.4]{fig/downloadTool.png}
    \caption{下载实验工具包}\label{downloadTool}
\end{figure}

\begin{figure}[!htbp]
    \centering
    \includegraphics[scale=1]{fig/init.png}
    \caption{执行init.sh}\label{init}
\end{figure}

\begin{figure}[!htbp]
    \centering
    \includegraphics[scale=1]{fig/makeQemu.png}
    \caption{执行make qemu}\label{makeQemu}
\end{figure}

\begin{figure}[!htbp]
    \centering
    \includegraphics[scale=0.5]{fig/qemu.png}
    \caption{UNIX V6++界面}\label{qemu}
\end{figure}

\begin{figure}[!htbp]
    \centering
    \includegraphics[scale=0.5]{fig/helloWorld.png}
    \caption{执行echo命令}\label{helloWorld}
\end{figure}

\newpage
\subsection{求逆序对数}
\begin{formal}
    {\cuhei 问题描述:}
请求出整数序列$A$的所有逆序对个数。
\end{formal}
\begin{formal}
    {\cuhei 输入要求:}

    输入包含多组测试数据,每组测试数据有两行。
第一行为整数$N(1 \leq N \leq 20000)$,当输入0时结束;
第二行为$N$个整数,表示长为$N$的整数序列。
\end{formal}
\begin{formal}
    {\cuhei 输出要求:}

    每组数据对应一行,输出逆序对的个数。
\end{formal}
\subsubsection{数据结构设计}
\begin{lstlisting}[name=Q1]
typedef struct {
    int data[MAXN] ;//整数序列
    int length ;//整数序列长度
}array ;
\end{lstlisting}
\subsubsection{功能描述}

% 在直接插入排序中,一次位置调整便能消去一对逆序对,而当排序结束时,逆序对数为零。
% 逆序对数等于直接插入排序中位置交换的次数。为了降低时间复杂度,使用折半插入排序,并
% 直接利用插入位置计算出交换次数。
% \begin{procedure}
%     \tcp*[h]{A为给定整数序列,i为待排序元素下标}\;
%     \tcp*[h]{返回待插入位置}\;
%     l = 0\;
%     h = i-1\;
%     \While(\tcp*[h]{确保mid+1合法}){l<h}
%     {
%         mid = $\lfloor$(l+h)/2$\rfloor$\;
%         \lIf(\tcp*[h]{确保稳定性}){A.data[mid]$\leq$A.data[i]<A.data[mid+1]}{\KwRet{mid+1}}
%         \leIf{A.data[mid]>A.data[i]}{h=mid}{l=mid+1}
%     }
%     \leIf{A.data[l]$\leq$A.data[i]}{\KwRet{l+1}}{\KwRet{l}}
%     \caption{BinarySearch(A,i)}
% \end{procedure}
% \begin{algorithm}
%     \tcp*[h]{A为给定整数序列}\;
%     \For{j = 2 \KwTo A.length}
%     {
%         key = A.data[j]\;
%         dest = \BinarySearch(A,j)\;
%         \For{i = j-1 \downto dest}
%         {
%             A.data[i + 1] = A.data[i]\;
%             i = i - 1\;
%         }
%         A.data[dest] = key\;
%         A.counter = A.counter + j - dest\;
%     }
%     \KwRet{A.counter}
%     \caption{Count-Inversion}
% \end{algorithm}

\begin{function}[H]
    mid = $\lfloor$(r+l)/2$\rfloor$\;
    $n_1$ = mid - l + 1 \;
    $n_2$ = r - mid\;
    创建大小分别为 $n_1,n_2$的数组 $L,R$\;
    \lFor{i=0 \KwTo $n_1$-1}{L[i]=A[l+i]}
    \lFor{i=0 \KwTo $n_2$-1}{R[i]=A[mid+1+i]}
    m = 0\;
    n = 0\;
    counter = 0\;
    \For(\tcp*[h]{合并子序列}){i=0 \KwTo $n_1+n_2-1$}
    {
        \eIf{(L[m] $\leq$R[n] and m <$n_1$) or (n $\geq n_2$)}
        {
            A[l+i] = L[m]\;
            m = m + 1\;
        }
        {
            A[l+i] = R[n]\;
            counter = counter + $n_1$ + n - i\tcp*[h]{需要移动的“步数”}\;
            n = n + 1\;
        }
    }
    \KwRet{counter}
    \caption{Merge(A,r,l)}
\end{function}
\begin{function}
    \SetKwFunction{CountInversion}{CountInversion}
    \tcp*[h]{A为给定整数序列}\;
    \tcp*[h]{最终解答调用\CountInversion(A,0,A.length-1)}\;
    \lIf{r == l}{\KwRet{0}}
    mid = $\lfloor$(r+l)/2$\rfloor$\;
    \KwRet{\CountInversion(A,l,mid) + \CountInversion(A,mid+1,r) + \Merge(A,l,r)}\;
    \caption{CountInversion(A,r,l)}
\end{function}
\subsubsection{调试分析}
本题采用输出关键变量进行调试。选用的关键变量是\Merge()中涉及到的数组部分以及对应的\emph{counter}。

在调试的过程中,遇到了多次相同输入,但是输出不同的情况,因此判断是内存出现问题。查看数组部分的输出信息,
发现在运行过程中数组内部出现了异常元素。经过排查后发现是“合并子序列”步骤中,当一个子序列以及全部转移完毕时,
仍有可能继续转移,从而产生内存溢出,加上 $m<n_1,n\geq n_2$条件后便恢复正常。
\subsubsection{总结和体会}
我一开始使用折半插入排序,但是最终有两个样例由于超时而无法通过,这是因为折半插入排序移动元素的平均时间复杂度仍为 $O(n^2)$。
最后我在归并排序的基础上进行调整,时间复杂度维持在了 $\Theta(n\lg n)$。使用该方法时,我仍然有两个样例没有通过,原因是
内存使用过多,经过排查,是因为只申请了内存,而没有释放。在调试的过程中,我也体会到了处理边界情况的重要性,如果能够
合理设置哨兵,例如在数组 $L,R$结尾设置两个哨兵,便可以在保持程序简洁的同时确保正确性。
\newpage
\subsection{三数之和}
\begin{formal}
    {\cuhei 问题描述:}

    给你一个整数数组 $nums$ ,判断是否存在三元组 $[nums[i], nums[j], nums[k]]$ 满足 $i \neq j,i \neq k$ 且 $j \neq k$ ,同时还满足 $nums[i] + nums[j] + nums[k] == 0$ 。请你返回所有和为 0 且不重复的三元组,每个三元组占一行。
\end{formal}
\begin{formal}
    {\cuhei 输入要求:}

    2行。第1行一个整数,表示数组元素个数;
    第2行输入一组整数,中间以空格分隔。
\end{formal}
\begin{formal}
    {\cuhei 输出要求:}

    输出所有和为0的三元组,每个三元组一行,中间以空格分隔。
    对于每一个三元组,你需要按从小到大的顺序依次返回三个元素;
    对于所有三元组,你需要按三元组中最小元素从小到大的顺序依次打印每一组三元组。
\end{formal}
\subsubsection{数据结构设计}
\begin{lstlisting}[name=Q1]
int nums[MAXN];
int sums[MAXN]; //存储所有2组合之和
\end{lstlisting}
\subsubsection{功能描述}
\begin{function}
    \SetKwFunction{Swap}{Swap}
    \SetKwFunction{Heapify}{Heapify}
    max = i\;
    \lIf{$2i\leq size$ and nums[2i]>nums[max]}{max = 2i}
    \lIf{$2i+1\leq size$ and nums[2i+1]>nums[max]}{max = 2i+1}
    \If{$max\neq i$}
    {\Swap(nums[max],nums[i])\;
        \Heapify(nums,max,size) \;
    }
    \caption{Heapify(nums,i,size)}
\end{function}
\begin{function}
    \SetKw{downto}{downto}
    \lFor{i = $\lfloor n/2\rfloor$ \downto 1}
    {\Heapify(nums,i)}
    \caption{Build-Heap(nums)}
\end{function}
\begin{algorithm}[H]
    \SetKwFunction{build}{Build-Heap}
    \build(nums)\;
    \For{i = nums.length \downto 2}
    {
        \Swap(nums[1],nums[i])\;
        \Heapify(nums,1,i-1)
    }
    \caption{Heap-Sort(nums)}
\end{algorithm}
\begin{function}[H]
    \SetKwFunction{Heap}{Heap-Sort}
    \SetKwFunction{Binary}{Binary-Search}
    \SetKwFunction{print}{print}
    \Heap(nums)\;
    \SetKw{NIL}{NIL}
    i = 1\;
    \While{nums[i]$\leq$0}
    {
        l = i+1\;
        r = nums.length\;
        \While{l<r}
        {
            \uIf{nums[i]+num[l]+num[r]==0}
            {
                \print(nums[i],num[l],num[r])\;
                \lWhile(\tcp*[h]{确保不会重复}){nums[l]==nums[l+1]}{l=l+1}
                l = l+1\;
                \lWhile(\tcp*[h]{确保不会重复}){nums[r]==nums[r-1]}{r=r-1}
                r = r-1\;
            }
            \uElseIf{nums[i]+nums[l]+nums[r]<0}
            {
                l = l +1\;
            }
            \Else{r = r-1\;}
        }
        \lWhile(\tcp*[h]{确保不会重复}){nums[i]==nums[i+1]}{i=i+1}
        i = i+1\;
    }
    \caption{Triple-Sum(nums)}
\end{function}
\subsubsection{调试分析}
一开始调试遇到的主要问题:即使已经在9,11行处确保了l,r不会重复,但是仍然会输出重复的三元组。经过分析后发现还需要用同样的
方式确保i,添加了19行后便能正确运行。
\subsubsection{总结和体会}
该题的难点在于如何在较低的时间复杂度内求出所有的符合要求的三元组。如果使用暴力枚举,则需要 $O(n^3)$的时间复杂度。这里我使用了
双指针的方法,使得时间复杂度降为 $O(n^2)$。还有一个难点是满足“不重复”的需求,代码中有三处需要进行相应的修改。
\newpage
\subsection{最大数}
\begin{formal}
    {\cuhei 问题描述:}

    给定一组非负整数 nums,重新排列每个数的顺序(每个数不可拆分)使之组成一个最大的整数。
\end{formal}
\begin{formal}
    {\cuhei 输入要求:}

    输入包含两行:第一行包含一个整数$n$,表示组数$nums$的长度;第二行包含$n$个整数$nums[i]$。
\end{formal}
\begin{formal}
    {\cuhei 输出要求:}

    输出包含一行,为重新排列后得到的数字。
\end{formal}
\subsubsection{数据结构设计}
\begin{lstlisting}[name=Q4]
string nums[MAXN]; //利用to_string函数将非负整数转化为string.
\end{lstlisting}
\subsubsection{功能描述}
\begin{function}
    \SetKwFunction{Min}{Min}
    \SetKw{Break}{Break}
    \tcp*[h]{判断$S_1$是否排在$S_2$左侧}\;
    s1 = S1 + S2\;
    s2 = S2 + S1\;
    \For{i=1 \KwTo $s_1.length$}
    {
        \lIf{$s_1[i] > s_2[i]$}{\Return{true}}
        \lIf{$s_1[i] < s_2[i]$}{\Return{false}}
    }
    \KwRet{true}
    \caption{StringCompare($S_1,S_2$)}
\end{function}
\begin{function}[H]
    \SetKwFunction{Swap}{Swap}
    \tcp*[h]{默认以最后一个元素作为pivot;l,r分别为子序列的左右端点下标}\;
    \tcp*[h]{返回pivot的最终下标}\;
    i = l - 1\;
    \For{j = l \KwTo r-1}
    {
        \If(\tcp*[h]{使用自定义的比较规则}){StringCompare(A[j],A[r])}
        {
            i = i + 1\;
            \Swap(A[j],A[i])\;
        }
    }
    \Swap(A[i+1],A[r])\;
    \KwRet{i+1}
    \caption{Partition(A,l,r)}
\end{function}
\begin{algorithm}
    \SetKwFunction{Quick}{Quick-Sort}
    \If{l<r}
    {
        p = Partition(A,l,r)\;
        \Quick(A,l,p-1)\;
        \Quick(A,p+1,r)\;
    }
   \caption{Quick-Sort(A,l,r)} 
\end{algorithm}
\subsubsection{调试分析}
一开始我自定义的排序规则是:当一个字符串以另一个字符串开头时,哪个小,哪个便排在前面。但是这样的无法正确排序3,30,34,5,9,
会给出9533034而非9534330.后来我将该比较函数调整为:当一个字符串以另一个字符串开头时,将剩余的字符串再一次与较短的字符串进行比较,以
得到最终的结果,但是这样得到的时9534303.最后我将自定义比较函数改为上述的逻辑,才成功通过测试。
\subsubsection{总结和体会}
本题的基本框架便是排序,唯一需要调整的便是如何判断两个数字串哪个排在前面。而难点便是自定义的比较函数。
在本题中我使用的是快速排序,平均时间复杂度为 $\Theta(n\log n)$,这使得在使用了自定的比较函数后,时间复杂度仍能接受。
如果使用冒泡排序等简单排序算法,则可能超出时间限制。
\newpage
\subsection{题目名称}
\begin{formal}
    {\cuhei 问题描述:}

    测试\\
    测试
\end{formal}
\begin{formal}
    {\cuhei 输入要求:}

    测试\\
    测试
\end{formal}
\begin{formal}
    {\cuhei 输出要求:}

    测试\\
    测试
\end{formal}
\subsubsection{数据结构设计}
\begin{lstlisting}[name=Q1]
//
test
test
\end{lstlisting}
\subsubsection{功能描述}
\begin{lstlisting}[name=F11]
//
test
test
\end{lstlisting}
\begin{function}
    \tcp*[h]{describe your function}\;
    \KwIn{input}
    \KwOut{output}
    \caption{NameOfFunction()}
\end{function}
\subsubsection{调试分析}
\subsubsection{总结和体会}
\newpage
\section{实验总结}
用于排序的各个算法各有特点,即使是高效排序算法,在处理极端和特殊情况时的表现差异也很明显。因此,在实际应用时,
需要结合对于输入数据的已有知识,选取最为合适的排序算法。同时,如果只是把排序作为其他算法中的子步骤,在保证时间复杂度
的前提下,可以考虑排序算法的稳定排序、部分排序等特性对于算法设计是否有用,从而选取最合适的排序算法。
\end{document}