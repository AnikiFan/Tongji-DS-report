\subsection{最大数}
\begin{formal}
    {\cuhei 问题描述:}

    给定一组非负整数 nums,重新排列每个数的顺序(每个数不可拆分)使之组成一个最大的整数。
\end{formal}
\begin{formal}
    {\cuhei 输入要求:}

    输入包含两行:第一行包含一个整数$n$,表示组数$nums$的长度;第二行包含$n$个整数$nums[i]$。
\end{formal}
\begin{formal}
    {\cuhei 输出要求:}

    输出包含一行,为重新排列后得到的数字。
\end{formal}
\subsubsection{数据结构设计}
\begin{lstlisting}[name=Q4]
string nums[MAXN]; //利用to_string函数将非负整数转化为string.
\end{lstlisting}
\subsubsection{功能描述}
\begin{function}
    \SetKwFunction{Min}{Min}
    \SetKw{Break}{Break}
    \tcp*[h]{判断$S_1$是否排在$S_2$左侧}\;
    s1 = S1 + S2\;
    s2 = S2 + S1\;
    \For{i=1 \KwTo $s_1.length$}
    {
        \lIf{$s_1[i] > s_2[i]$}{\Return{true}}
        \lIf{$s_1[i] < s_2[i]$}{\Return{false}}
    }
    \KwRet{true}
    \caption{StringCompare($S_1,S_2$)}
\end{function}
\begin{function}[H]
    \SetKwFunction{Swap}{Swap}
    \tcp*[h]{默认以最后一个元素作为pivot;l,r分别为子序列的左右端点下标}\;
    \tcp*[h]{返回pivot的最终下标}\;
    i = l - 1\;
    \For{j = l \KwTo r-1}
    {
        \If(\tcp*[h]{使用自定义的比较规则}){StringCompare(A[j],A[r])}
        {
            i = i + 1\;
            \Swap(A[j],A[i])\;
        }
    }
    \Swap(A[i+1],A[r])\;
    \KwRet{i+1}
    \caption{Partition(A,l,r)}
\end{function}
\begin{algorithm}
    \SetKwFunction{Quick}{Quick-Sort}
    \If{l<r}
    {
        p = Partition(A,l,r)\;
        \Quick(A,l,p-1)\;
        \Quick(A,p+1,r)\;
    }
   \caption{Quick-Sort(A,l,r)} 
\end{algorithm}
\subsubsection{调试分析}
一开始我自定义的排序规则是:当一个字符串以另一个字符串开头时,哪个小,哪个便排在前面。但是这样的无法正确排序3,30,34,5,9,
会给出9533034而非9534330.后来我将该比较函数调整为:当一个字符串以另一个字符串开头时,将剩余的字符串再一次与较短的字符串进行比较,以
得到最终的结果,但是这样得到的时9534303.最后我将自定义比较函数改为上述的逻辑,才成功通过测试。
\subsubsection{总结和体会}
本题的基本框架便是排序,唯一需要调整的便是如何判断两个数字串哪个排在前面。而难点便是自定义的比较函数。
在本题中我使用的是快速排序,平均时间复杂度为 $\Theta(n\log n)$,这使得在使用了自定的比较函数后,时间复杂度仍能接受。
如果使用冒泡排序等简单排序算法,则可能超出时间限制。