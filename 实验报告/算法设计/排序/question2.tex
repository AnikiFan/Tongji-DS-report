\subsection{求逆序对数}
\begin{formal}
    {\cuhei 问题描述:}
请求出整数序列$A$的所有逆序对个数。
\end{formal}
\begin{formal}
    {\cuhei 输入要求:}

    输入包含多组测试数据,每组测试数据有两行。
第一行为整数$N(1 \leq N \leq 20000)$,当输入0时结束;
第二行为$N$个整数,表示长为$N$的整数序列。
\end{formal}
\begin{formal}
    {\cuhei 输出要求:}

    每组数据对应一行,输出逆序对的个数。
\end{formal}
\subsubsection{数据结构设计}
\begin{lstlisting}[name=Q1]
typedef struct {
    int data[MAXN] ;//整数序列
    int length ;//整数序列长度
}array ;
\end{lstlisting}
\subsubsection{功能描述}

% 在直接插入排序中,一次位置调整便能消去一对逆序对,而当排序结束时,逆序对数为零。
% 逆序对数等于直接插入排序中位置交换的次数。为了降低时间复杂度,使用折半插入排序,并
% 直接利用插入位置计算出交换次数。
% \begin{procedure}
%     \tcp*[h]{A为给定整数序列,i为待排序元素下标}\;
%     \tcp*[h]{返回待插入位置}\;
%     l = 0\;
%     h = i-1\;
%     \While(\tcp*[h]{确保mid+1合法}){l<h}
%     {
%         mid = $\lfloor$(l+h)/2$\rfloor$\;
%         \lIf(\tcp*[h]{确保稳定性}){A.data[mid]$\leq$A.data[i]<A.data[mid+1]}{\KwRet{mid+1}}
%         \leIf{A.data[mid]>A.data[i]}{h=mid}{l=mid+1}
%     }
%     \leIf{A.data[l]$\leq$A.data[i]}{\KwRet{l+1}}{\KwRet{l}}
%     \caption{BinarySearch(A,i)}
% \end{procedure}
% \begin{algorithm}
%     \tcp*[h]{A为给定整数序列}\;
%     \For{j = 2 \KwTo A.length}
%     {
%         key = A.data[j]\;
%         dest = \BinarySearch(A,j)\;
%         \For{i = j-1 \downto dest}
%         {
%             A.data[i + 1] = A.data[i]\;
%             i = i - 1\;
%         }
%         A.data[dest] = key\;
%         A.counter = A.counter + j - dest\;
%     }
%     \KwRet{A.counter}
%     \caption{Count-Inversion}
% \end{algorithm}

\begin{function}[H]
    mid = $\lfloor$(r+l)/2$\rfloor$\;
    $n_1$ = mid - l + 1 \;
    $n_2$ = r - mid\;
    创建大小分别为 $n_1,n_2$的数组 $L,R$\;
    \lFor{i=0 \KwTo $n_1$-1}{L[i]=A[l+i]}
    \lFor{i=0 \KwTo $n_2$-1}{R[i]=A[mid+1+i]}
    m = 0\;
    n = 0\;
    counter = 0\;
    \For(\tcp*[h]{合并子序列}){i=0 \KwTo $n_1+n_2-1$}
    {
        \eIf{(L[m] $\leq$R[n] and m <$n_1$) or (n $\geq n_2$)}
        {
            A[l+i] = L[m]\;
            m = m + 1\;
        }
        {
            A[l+i] = R[n]\;
            counter = counter + $n_1$ + n - i\tcp*[h]{需要移动的“步数”}\;
            n = n + 1\;
        }
    }
    \KwRet{counter}
    \caption{Merge(A,r,l)}
\end{function}
\begin{function}
    \SetKwFunction{CountInversion}{CountInversion}
    \tcp*[h]{A为给定整数序列}\;
    \tcp*[h]{最终解答调用\CountInversion(A,0,A.length-1)}\;
    \lIf{r == l}{\KwRet{0}}
    mid = $\lfloor$(r+l)/2$\rfloor$\;
    \KwRet{\CountInversion(A,l,mid) + \CountInversion(A,mid+1,r) + \Merge(A,l,r)}\;
    \caption{CountInversion(A,r,l)}
\end{function}
\subsubsection{调试分析}
本题采用输出关键变量进行调试。选用的关键变量是\Merge()中涉及到的数组部分以及对应的\emph{counter}。

在调试的过程中,遇到了多次相同输入,但是输出不同的情况,因此判断是内存出现问题。查看数组部分的输出信息,
发现在运行过程中数组内部出现了异常元素。经过排查后发现是“合并子序列”步骤中,当一个子序列以及全部转移完毕时,
仍有可能继续转移,从而产生内存溢出,加上 $m<n_1,n\geq n_2$条件后便恢复正常。
\subsubsection{总结和体会}
我一开始使用折半插入排序,但是最终有两个样例由于超时而无法通过,这是因为折半插入排序移动元素的平均时间复杂度仍为 $O(n^2)$。
最后我在归并排序的基础上进行调整,时间复杂度维持在了 $\Theta(n\lg n)$。使用该方法时,我仍然有两个样例没有通过,原因是
内存使用过多,经过排查,是因为只申请了内存,而没有释放。在调试的过程中,我也体会到了处理边界情况的重要性,如果能够
合理设置哨兵,例如在数组 $L,R$结尾设置两个哨兵,便可以在保持程序简洁的同时确保正确性。