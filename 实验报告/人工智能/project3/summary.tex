\chapter{总结与分析}
%
%实验中遇到的问题及解决方案,收获和思考:对算法的理解、优缺点的评价、算法的适用场景
%
在实验的过程中,问题主要出现在后半部分的智能体的规划算法的实现上。例如在实现foodLogicPlan时,在测试过程中,算法返回的方案过短,
经过排查后发现是因为foodSuccessorAxiom中,并没有考虑全所有的三种情况:1)之前无食物;2)之前有食物,吃豆人当前在该位置;3)之前有食物,
吃豆人当前不在该位置。这导致约束过弱,导致SAT在求满足的模型时可以不受约束地对食物相关的语句进行赋值,例如没有考虑第三种情况,那么即使吃豆人
当前不在该位置,SAT返回的模型中该位置上的食物也可以被设置为已收集,从而导致吃豆人过早地认为收集食物的任务已经完成了。

在测试的过程中,我还直观地感受到了基于逻辑的规划算法的缺点——耗时过长。基于逻辑的规划算法虽然强大,能够帮助智能体进行路径规划、制图、SLAM等功能,
且相应的源代码都十分模块化,但是随着时间的推移,知识库KB中包含的语句数量快速增长,导致耗时较长。即使是在一个较小的迷宫场景下,想要完成一个任务
都需要耗时将近半分钟,这在充满不确定性以及多变性的现实场景中往往是不可接受的。