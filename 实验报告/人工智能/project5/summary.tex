\chapter{总结与分析}
%
%实验中遇到的问题及解决方案,收获和思考:对算法的理解、优缺点的评价、算法的适用场景
%
通过本次实验,我深刻体会到了机器学习以及神经网络的强大之处。

在实验过程中,我主要遇到了三个问题。

其一,在完成问题二的过程中,起初,我并没有在线性层中增加偏置值。从可视化的结果中可以看出,由模型得到的曲线只是两段从原点引出的射线。而当我
添加了偏置值后,由模型得到的曲线很快便接近正弦函数的形态。

其二,在完成神经网络的相关题目时,我发现如果编程过程中把有关问题的参数以硬编码的方式保留下来,将会大大降低代码的可移植性和可读性,
同时,还不便于推导所需矩阵的形状。为此,我重构了自己的代码,将参数在\_\_init\_\_函数中便初始化,而非以魔数的形式在后续的训练过程中使用。
这使得我在完成第三题时,可以直接使用第二题的代码,只需修改输入输出的相关参数即可。

其三,在完成第四题的过程中,我体会到了超参数的重要性。一开始,我沿用前两题的两层线性层的神经网络架构,但是尝试了如0.5,0.1,0.01等多个学习率
后,结果都是在达到所需的准确率之前便产生了梯度消失或梯度爆炸的现象。为此,我将神经网络架构调整为了总共有三层线性层的架构。在
相同的其他超参数的条件下,新得到的神经网络便能成功达到所需的准确率。

在训练神经网络的过程中,我也观察到神经网络的准确率不一定时随着训练时间递增的,会有一定的波动存在,为此,在最后一题中,我设定停止训练的准确率阈值为87\%,
以增加在测试集上达到81\%准确率的概率。当然这样做的代价便是训练时间会增长一倍左右。我最终使用的神经网络能够在训练了60个epoch后达到80\%左右的准确率,
训练了100个epoch后能达到85\%左右的准确率,而为了达到87\%的准确率,需要训练180个epoch左右,这需要20分钟左右的时间。