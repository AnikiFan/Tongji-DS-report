\chapter{项目概述}
\section{主要内容与目标}
本次项目的主要内容为神经网络的搭建与应用。

首先,项目要求我们实现感知机,然后搭建神经网络以完成非线性回归、数字分类以及语言识别。
\section{已有代码}
下面对已有代码中与本项目的具体实现相关的代码进行分析。
\subsection{nn.py}
该文件提供了用于搭建神经网络的一系列节点:
\begin{enumerate}
    \item Consant:由浮点数组成的二维数组;
    \item Parameter:可训练的感知机或神经网络,提供了用于更新参数的方法update;
    \item DotProduct:对它的输入进行点积运算;
    \item Add:逐元素的矩阵加法;
    \item AddBias:为特征向量添加一个偏置值;
    \item Linear:对输入进行线性变换;
    \item ReLU:ReLU激活函数;
    \item SquareLoss:平方损失函数;
    \item SoftmaxLoss:SoftmaxLoss损失函数。
\end{enumerate}

同时,还提供了一下两个函数:
\begin{enumerate}
    \item gradients:给定损失函数对于给定参数的梯度;
    \item as\_scalar:将给定节点以Python中的数值类型返回。
\end{enumerate}
\subsection{backend.py}
该文件提供了本项目中所需要使用的数据集,提供了以下方法:
\begin{enumerate}
    \item iterate\_once:返回数据集中的一批数据,返回的类型为一个二元组(x,y),其中x,y分别为特征和标签,类型均为Constant结点,大小分别为“批的大小$\times$特征数量”,“批的大小$\times$输出大小”;
    \item iterate\_forever:一个不断从数据集中返回数据的迭代器;
    \item get\_validation\_accuracy:返回在验证集上得到的准确率。
\end{enumerate}