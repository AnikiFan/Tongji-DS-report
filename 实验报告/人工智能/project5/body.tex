\makeatletter
\newcommand{\mytitle}{\@title}
\makeatother

\usepackage[
    fontset=none,%设置中文支持,并自定义字体
    zihao=5,%默认字号为五号
    heading=true,%允许后续自定义标题样式
    scheme=chinese,%自动将文档样式中文化,例如图标标题
    punct=quanjiao,%全角式标点符号
    space=auto,%中文后接换行不会添加空格,但是英文会添加空格,需要用%手动取消
    % linespread=1.3,%行距倍数是1.3
    autoindent=true,%自动缩进两个中文宽度
    ]{ctex}

\ctexset{
    today=small,%小写样式的日期
    contentsname={目录},
    % contentsname={\hspace{-\ccwd}目录},
    listfigurename={插图},
    listtablename={表格},
    figurename={图},
    tablename={表},
    abstractname={简{\quad}介},
    indexname={索引},
    appendixname={附录},
    bibname={参考文献},
    proofname={证明},
    % refname={参考文献},%只适用于beamer
    % algorithmname={算法},
    % continuation={(续)},%beamer续页的标识
    chapter={
        format+ = \Huge\heiti\raggedright,
        name = {,\num\textbf{.}\hspace{1ex}},
        number={\num\thechapter},
        nameformat={},
        numberformat={},
        aftername={},
        titleformat={},
        aftertitle={},
        runin=false,%对section级以下有用,标题是否和正文在同一段上
        beforeskip={3.5ex plus 1ex minus .2ex},%标题前垂直间距
        afterskip={2.3ex plus .2ex}%标题后垂直间距
    },
    section={
        format+ = \Large\heiti\raggedright,
        name = {,\num\textbf{.}\hspace{1ex}},
        number={\num\thesection},
        nameformat={},
        numberformat={},
        aftername={},
        titleformat={},
        aftertitle={},
        runin=false,%对section级以下有用,标题是否和正文在同一段上
        beforeskip={3.5ex plus 1ex minus .2ex},%标题前垂直间距
        afterskip={2.3ex plus .2ex}%标题后垂直间距
    },
    subsection={
        format+ = \large\heiti\raggedright,
        name = {,\num\textbf{.}\hspace{1ex}},
        number={\num\thesubsection},
        nameformat={},
        numberformat={},
        aftername={},
        titleformat={},
        aftertitle={},
        runin=false,%对section级以下有用,标题是否和正文在同一段上
        beforeskip={3.5ex plus 1ex minus .2ex},%标题前垂直间距
        afterskip={2.3ex plus .2ex}%标题后垂直间距
    },
    subsubsection={
        format+ = \normalsize\heiti\raggedright,
        name = {,\num\textbf{.}\hspace{1ex}},
        number={\num\thesubsubsection},
        nameformat={},
        numberformat={},
        aftername={},
        titleformat={},
        aftertitle={},
        runin=false,%对section级以下有用,标题是否和正文在同一段上
        beforeskip={3.5ex plus 1ex minus .2ex},%标题前垂直间距
        afterskip={2.3ex plus .2ex}%标题后垂直间距
    },
    }

\usepackage{amsfonts}
\usepackage{lmodern}%解决报错
% 中文默认字体: 思源宋体,粗体为思源宋体半粗体,斜体为方正楷体_GBK
\setCJKmainfont{Source Han Serif SC}[BoldFont={Source Han Serif SC Heavy}, ItalicFont=FZKai-Z03S]
% 中文无衬线字体:思源黑体,粗体为思源黑体粗体
\setCJKsansfont{Source Han Sans CN}[BoldFont={Source Han Sans CN Heavy}]
% 中文等宽字体:微软雅黑light
\setCJKmonofont{Microsoft YaHei}[ItalicFont={Microsoft YaHei Light}]

\newCJKfontfamily\songti{Source Han Serif SC}[BoldFont={Source Han Serif SC Heavy}]
\newCJKfontfamily\xbsong{Source Han Serif SC SemiBold} % 小标宋
\newCJKfontfamily\dbsong{Source Han Serif SC Bold} % 大标宋
\newCJKfontfamily\cusong{Source Han Serif SC Heavy} % 粗宋
\newCJKfontfamily\heiti{Source Han Sans CN}[BoldFont={Source Han Sans CN Heavy}]
\newCJKfontfamily\dahei{Source Han Sans CN Medium} % 大黑
\newCJKfontfamily\cuhei{Source Han Sans CN Heavy} % 粗黑
\newCJKfontfamily\fangsong{FZFangSong-Z02S}
\newCJKfontfamily\kaiti{FZKai-Z03S}[ItalicFont={Microsoft YaHei Light}]%这个斜体只是用于lstlisting环境中的中文注释
% \newCJKfontfamily\kaiti{FZKai-Z03S}[ItalicFont={FZZJ-LZXTFSJW}]%这个斜体只是用于lstlisting环境中的中文注释
\setsansfont{Arial}
\setmonofont{Consolas}%设置西文等宽字体
\newfontfamily\code{Consolas}
\newfontfamily\num{Arial}

\usepackage{geometry}%设置整体页面布局
\geometry{a4paper}
\geometry{left=2cm,right=2cm,top=2.54cm,bottom=2.54cm}%word常规页边距
% \geometry{left=1.27cm,right=1.27cm,top=1.27cm,bottom=1.27cm}%word窄页边距
\setlength{\headheight}{13pt}%避免warning


\usepackage{fancyhdr}%必须在geometry包之后使用
\fancyhf{}
\lhead{\sffamily\bfseries{范潇 2254298}}%可以使用thepage,CTEXthechapter,CTEXthesection
\chead{\sffamily\bfseries{课程设计作业Project5}}
\rhead{\sffamily\bfseries{- \thepage{} -}}
\renewcommand\headrulewidth{2pt}%设置眉头宽度
\pagestyle{fancy}


\usepackage{tikz}
\usepackage{graphicx}
\usepackage{longtable}
\usepackage{amssymb}
\newcommand*{\abs}[1]{\lvert #1 \rvert}
\setlength{\baselineskip}{18pt}
\usepackage{float}