\usepackage[ruled,algosection,lined,longend,fillcomment,linesnumbered,resetcount,titlenotnumbered]{algorithm2e}
%参数解释:带框,按section编码,有竖线,end前带if等关键词,注释占满整行,代码部分编号(不包括输入输出、注释),每个代码块重新编号,可以调用TitleOfAlgo来打印算法标题但不作为单独的算法编码
%附带algorithm,function,procedure环境,其中function,procedure环境下,设置caption时,必须带有(),
%()之前的字符会被视为宏,可以在接下来的部分用\名字()来调用,所以推荐辅助函数用function,其中的某些展开部分用procedure,描述算法整体使用algorithm
\DontPrintSemicolon
\SetAlCapSkip{2ex}
\SetSideCommentRight
\SetFillComment
\newcommand{\forcond}{$i=0$ \KwTo $n$}
\SetKw{macro}{text}%自定义关键词
\SetKwFunction{funcmacro}{text}%自定义函数名,实际上function环境是在定义宏的同时说明了其内容
\SetKwProg{procedmacro}{text}{begin text}{end text}%自定义步骤,和function类似,但是后面两个参数可以设置开始和结尾的标志,和if等环境一样
\SetKwData{datamacro}{text}%可以用于突出特殊的变量,例如数据结构
\SetKwFunction{FRecurs}{FnRecursive}
\SetKwProg{Fn}{Function}{begin}{end}