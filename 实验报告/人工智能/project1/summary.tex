\chapter{总结与分析}
%
%实验中遇到的问题及解决方案,收获和思考:对算法的理解、优缺点的评价、算法的适用场景
%
在完成第一部分时,我选择实现了通用图搜索算法generalGraphSearch,这使得我在实现具体的搜索算法时,只需在额外实现对应的优先级函数即可。这一便利的代价便是
在实现通用算法时,需要注意通用性,以兼容不同的搜索问题以及搜索算法。在这一过程中,我所遇到的主要问题便是由于我用字典来存放reached表而导致的,使得我必须将
状态修改为可以进行散列的数据结构。将来可以将其改进为用列表存储reached表,并且是将整个结点直接放入,简化了所需要的数据结构。

在设计第二、第三部分的启发式函数的过程中,我体会到在忽略一定约束后所得到的最优解或者是完整约束下的部分解均有可能成为一个比较好的启发式函数。例如
在CornersProblem中,我所设计的启发式函数便是在忽略了墙壁的约束下得到的最优解路径的长度;而在AnyFoodSearchProblem中,我所设计的启发式函数是在忽略
墙壁的约束后,各个食物以及吃豆人当前位置之间的距离的最大值,这实际上相当于只考虑了吃豆人收集其中一个或两个食物所需的最短距离。

同时,我在设计启发式函数的过程中也意识到:启发式函数所需要的代价应该尽量小,即可以快速得到。在设计AnyFoodSearchProblem的启发式函数时,我曾尝试在忽略墙壁的约束后
将使用UCS算法得到的解路径的长度作为函数值,但是经过实验发现当食物个数略多时,所需要的时间便变得不可接受,而这是因为该问题的状态数量随着食物的数量指数级增长。
实际上,实验证明,只需把各个食物到当前位置的曼哈顿距离的最大值当作启发函数值便可以获得不错的效果。

在设计启发式函数的过程中,我也意识到:要想设计的启发式函数是一致的,往往需要启发式函数值不随着状态发生突变。例如我在设计AnyFoodSearchProblem的启发式函数时,
尝试用以剩余食物位置以及吃豆人当前位置为结点的最小生成树的总代价作为启发式函数值,但是由于随着吃豆人的位置改变,最小生成树可能会发生多处变化,所以这一启发式函数是不一致的。

从实验的最后一题中我体会到了即使状态空间很大,如果不追求最优解的话,可以在较短时间内得到一个解路径——即使不是最优的,但通常也是也是可以接受的。而像UCS和BFS这类算法,
当面对较多的状态时,即使能够保证给出最优解,但是由于展开的结点数很多,所需要的时间代价往往是不可接受的。