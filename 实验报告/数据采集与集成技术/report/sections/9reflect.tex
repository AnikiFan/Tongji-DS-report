\section{数据采集与集成的国内外现状认识与体会}

在数据采集方面,国内外的发展情况有较大的差别,其主要原因是我国互联网行业起步相对较慢。
一方面,在数据源头层面,我国各地政府以及官方机构
公开数据的渠道较少,且多数存在质量低、数据量少、更新频率低等问题,使得我国缺少了一高质量数据来源;
另一方面,在软件生态层面,常用的requests,beautiful soup,scrapy等库都来自国外,并且经过了长期的开发,具有完善的功能。
例如requests库有Kenneth Reitz于2011年创建、BeautifulSoup由Leonard Richardson开发,初版发布于2004年;Scrapy由Scrapinghub开发,发布时间为2008。
近年来,由于大模型的兴起而导致的对于大规模数据的需求推动了数据采集的发展,
例如LAION数据集的采集催生出了img2dataset这一高性能的网络图像爬取库。


在数据集成方面,我国以阿里巴巴、字节跳动等公司推出了多款开源的数据集成工具,例如阿里巴巴的Addax、字节跳动开源的BitSail、云雀等。

我认为我国需加强对于数据公开平台的规范化建设,应该把握住由于庞大的人口所带来的数据资源,加速发展数据采集与集成技术。