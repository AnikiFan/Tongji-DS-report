\section{数据库逻辑设计}

\subsection{关系模式设计}

\subsubsection{规范化数据层关系模式}




\begin{enumerate}
  \item 用户关系模式
  \begin{listing}[htbp]
    \begin{minted}{sql}
User(
    id: Integer (PK),
    username: String(50) NOT NULL UNIQUE,
    password: String(50) NOT NULL,
    role: Enum('user', 'admin') NOT NULL DEFAULT 'user',
    created_at: DateTime NOT NULL
)
    \end{minted}
    \caption{用户表结构}\label{lst:user_schema}
  \end{listing}

  \item 公司关系模式
  \begin{listing}[htbp]
    \begin{minted}{sql}
Company(
    id: Integer (PK),
    name: String(500) NOT NULL UNIQUE,
    industry: String(200),
    size: String(50),
    created_at: DateTime NOT NULL,
    updated_at: DateTime NOT NULL
)
    \end{minted}
    \caption{公司表结构}\label{lst:company_schema}
  \end{listing}

  \item 职位关系模式
  \begin{listing}[htbp]
    \begin{minted}{sql}
Job(
    id: Integer (PK),
    title: String(500) NOT NULL,
    type_level3: String(100),
    salary: String(100),
    experience: String(100),
    education: String(100),
    skill: Text,
    benefits: Text,
    source: String(512),
    company_id: Integer (FK -> Company.id),
    raw_job_id: Integer (FK -> StagedRecruitmentData.id),
    created_at: DateTime NOT NULL,
    updated_at: DateTime NOT NULL
)
    \end{minted}
    \caption{职位表结构}\label{lst:job_schema}
  \end{listing}

  \item 地址关系模式
  \begin{listing}[htbp]
    \begin{minted}{sql}
Address(
    id: Integer (PK),
    address_str: String(200) NOT NULL,
    latitude: Float,
    longitude: Float,
    created_at: DateTime NOT NULL,
    updated_at: DateTime NOT NULL
)
    \end{minted}
    \caption{地址表结构}\label{lst:address_schema}
  \end{listing}

  \item 公司-地址关系模式
  \begin{listing}[htbp]
    \begin{minted}{sql}
company_address(
    company_id: Integer (PK, FK -> Company.id),
    address_id: Integer (PK, FK -> Address.id),
    created_at: DateTime NOT NULL
)
    \end{minted}
    \caption{公司-地址关系表结构}\label{lst:company_address_schema}
  \end{listing}

  \item 职位-地址关系模式
  \begin{listing}[htbp]
    \begin{minted}{sql}
job_address(
    job_id: Integer (PK, FK -> Job.id),
    address_id: Integer (PK, FK -> Address.id),
    created_at: DateTime NOT NULL
)
    \end{minted}
    \caption{职位-地址关系表结构}\label{lst:job_address_schema}
  \end{listing}
\end{enumerate}

\subsubsection{数据处理层关系模式}

\begin{enumerate}
  \item BOSS直聘原始数据关系模式
  \begin{listing}[htbp]
    \begin{minted}{sql}
RawRecruitmentData_Boss(
    id: Integer (PK),
    title: String(500),
    type_level3: String(100),
    salary: String(255),
    address_str: String(500),
    experience: String(255),
    education: String(255),
    skill: Text,
    benefits: Text,
    company_name: String(500),
    company_info: Text,
    source_url: String(1000) UNIQUE,
    raw_data: JSON,
    created_at: DateTime NOT NULL,
    processed: Boolean DEFAULT FALSE
)
    \end{minted}
    \caption{BOSS直聘原始数据表结构}\label{lst:raw_boss_schema}
  \end{listing}

  \item 猎聘网原始数据关系模式
  \begin{listing}[htbp]
    \begin{minted}{sql}
RawRecruitmentData_Liepin(
    id: Integer (PK),
    title: String(500),
    address_str: String(500),
    salary: String(255),
    tags: JSON,
    company_name: String(500),
    company_info: Text,
    raw_data: JSON,
    created_at: DateTime NOT NULL,
    processed: Boolean DEFAULT FALSE
)
    \end{minted}
    \caption{猎聘网原始数据表结构}\label{lst:raw_liepin_schema}
  \end{listing}

  \item 暂存数据关系模式
  \begin{listing}[htbp]
    \begin{minted}{sql}
StagedRecruitmentData(
    id: Integer (PK),
    job_title: String(255),
    job_salary: String(255),
    job_type_level3: String(100),
    job_experience: String(255),
    job_education: String(255),
    job_skill: Text,
    job_benefits: Text,
    address_str: String(255),
    company_name: String(255),
    company_listed: String(50),
    company_industry: Text,
    company_size: String(50),
    source: String(512),
    record_hash_for_linkage: String(64),
    duplicate_group_id: Integer,
    is_master_record: Boolean DEFAULT FALSE,
    linked_record_ids: JSON,
    processed: Boolean DEFAULT FALSE,
    created_at: DateTime NOT NULL
)
    \end{minted}
    \caption{暂存数据表结构}\label{lst:staged_data_schema}
  \end{listing}
\end{enumerate}

\subsection{规范化分析}

\subsubsection{函数依赖分析}

规范化数据层的函数依赖分析如下。对于每个表,我们需要分析其属性之间的函数依赖关系,以确保数据库设计满足规范化要求。函数依赖表示一个或多个属性的值可以唯一确定另一个属性的值。通过分析这些依赖关系,我们可以评估数据库设计是否存在数据冗余和异常,从而优化表结构。以下将详细分析每个表的函数依赖:

\begin{itemize}
    \item User表函数依赖:
    \begin{itemize}
        \item id $\rightarrow$ \{username, password, role, created\_at\}
        \item username $\rightarrow$ \{id, password, role, created\_at\}
    \end{itemize}
    \item Company表函数依赖:
    \begin{itemize}
        \item id $\rightarrow$ \{name, industry, size, created\_at, updated\_at\}
        \item name $\rightarrow$ \{id, industry, size, created\_at, updated\_at\}
    \end{itemize}
    \item Job表函数依赖:
    \begin{itemize}
        \item id $\rightarrow$ \{title, type\_level3, salary, experience, education, skill, benefits, source, company\_id, raw\_job\_id, created\_at, updated\_at\}
        \item \{company\_id, title, created\_at\} $\rightarrow$ \{id, type\_level3, salary, experience, education, skill, benefits, source, raw\_job\_id, updated\_at\}
    \end{itemize}
    \item Address表函数依赖:
    \begin{itemize}
        \item id $\rightarrow$ \{address\_str, latitude, longitude, created\_at, updated\_at\}
        \item \{address\_str, latitude, longitude\} $\rightarrow$ \{id, created\_at, updated\_at\}
    \end{itemize}
\end{itemize}

数据处理层的函数依赖分析较为简单,由于这些表不需要满足3NF,主要关注主键依赖

\begin{itemize}
    \item RawRecruitmentData\_Boss: id $\rightarrow$ 所有其他属性
    \item RawRecruitmentData\_Liepin: id $\rightarrow$ 所有其他属性
    \item StagedRecruitmentData: id $\rightarrow$ 所有其他属性
\end{itemize}

\subsubsection{规范化过程说明}

\begin{enumerate}
  \item 第一范式(1NF)
  \begin{itemize}
      \item 所有表的属性都是原子的
    \begin{itemize}
      \item 每个字段只包含单一类型的数据,如 username、password 等字段各自独立
      \item 不存在可以进一步拆分的列,避免出现类似 name\_and\_phone 这样的复合字段
      \item 每个字段的取值都是单一的,不允许多值
    \end{itemize}
    
    \item 使用JSON类型存储非结构化数据
    \begin{itemize}
      \item RawRecruitmentData\_Boss 和 RawRecruitmentData\_Liepin 表中使用 raw\_data (JSON类型) 存储原始爬取数据
      \item 保持原始数据的完整性,同时不破坏数据库表结构的规范性
      \item JSON字段便于存储变化的数据结构,适合存放第三方平台的原始数据
    \end{itemize}

    \item 复杂的多值属性通过关系表处理
    \begin{itemize}
      \item 使用 company\_address 表处理公司与地址的多对多关系
      \item 使用 job\_address 表处理职位与工作地点的多对多关系
      \item 避免在主表中出现 address1、address2 等多值字段
    \end{itemize}
  \end{itemize}

  \item 第二范式(2NF)
  \begin{itemize}
    \item 所有非主键属性完全依赖于主键
    \begin{itemize}
      \item 确保表中的每个非主键字段都依赖于整个主键,而不是主键的一部分
      \item 特别注意具有复合主键的表,如 company\_address 表中的 (company\_id, address\_id)
      \item 避免出现部分依赖,即某些字段只依赖于复合主键中的某一部分
    \end{itemize}

    \item 通过关系表处理多对多关系,避免部分依赖
    \begin{itemize}
      \item 关系表仅包含必要的关联字段(如 company\_id 和 address\_id)
      \item 添加 created\_at 等必要的时间戳字段用于追踪记录创建时间
      \item 不在关系表中存储可能导致部分依赖的其他业务字段
    \end{itemize}

    \item 每个关系表的设计原则
    \begin{itemize}
      \item 主键通常是参与关系的两个表的主键组合
      \item 只包含必要的关联字段和时间戳信息
      \item 避免添加与单个主键相关的冗余信息
    \end{itemize}
  \end{itemize}

  \item 第三范式(3NF)
  \begin{itemize}
    \item 消除非主键属性对主键的传递依赖
    \begin{itemize}
      \item 确保所有非主键属性直接依赖于主键
      \item 避免出现 A 依赖 B,B 又依赖主键的情况
      \item 对于发现的传递依赖,通过建立新表来消除
    \end{itemize}

    \item 公司信息独立为Company表
    \begin{itemize}
      \item 包含公司基本信息:name、industry、size 等字段
      \item 使用 company\_id 作为主键,在其他表中作为外键引用
      \item 避免公司信息在职位表等其他表中重复出现
    \end{itemize}

    \item 地址信息独立为Address表
    \begin{itemize}
      \item 存储完整的地址信息:address\_str、latitude、longitude 等
      \item 通过 company\_address 和 job\_address 关系表与其他实体关联
      \item 实现地址信息的统一管理和复用
    \end{itemize}

    \item 职位信息独立为Job表
    \begin{itemize}
      \item 只包含与职位直接相关的属性
      \item 通过外键 company\_id 关联到公司信息
      \item 通过 job\_address 关系表关联地址信息
    \end{itemize}
  \end{itemize}

  \item 规范化设计的优势
  \begin{itemize}
    \item 减少数据冗余,提高数据一致性
    \item 便于数据的维护和更新
    \item 支持系统的可扩展性
    \item 为后续的性能优化提供良好基础
  \end{itemize}
\end{enumerate}

\subsection{完整性约束设计}

\subsubsection{实体完整性约束}

\begin{enumerate}
  \item 主键约束
  \begin{itemize}
    \item 所有表都使用自增整数id作为主键,确保每条记录的唯一标识
    \item 关系表使用联合主键:
    \begin{itemize}
      \item company\_address表使用(company\_id, address\_id)作为联合主键
      \item job\_address表使用(job\_id, address\_id)作为联合主键
      \item 联合主键可以防止重复关联关系的产生
    \end{itemize}
  \end{itemize}

  \item 唯一性约束
  \begin{itemize}
    \item User表的username字段设置唯一约束,确保用户名不重复
    \item Company表的name字段设置唯一约束,避免重复录入公司信息
    \item RawRecruitmentData\_Boss表的source\_url字段设置唯一约束,防止重复爬取
    \item StagedRecruitmentData表的record\_hash\_for\_linkage字段设置唯一约束,用于数据去重
  \end{itemize}
\end{enumerate}

\subsubsection{参照完整性约束}

\begin{enumerate}
  \item 外键约束设计
  
  数据库中的外键约束主要体现在三个方面:Job表与Company表及StagedRecruitmentData表的关联、company\_address关系表的双向关联、以及job\_address关系表的双向关联。如\cref{lst:foreign_keys}所示,这些外键约束确保了数据的引用完整性。其中,Job表通过company\_id关联到Company表,通过raw\_job\_id关联到StagedRecruitmentData表;company\_address表通过company\_id和address\_id分别关联到Company表和Address表;job\_address表则通过job\_id和address\_id分别关联到Job表和Address表。

  \begin{listing}[htbp]
    \begin{minted}{sql}
-- Job表的外键约束
ALTER TABLE Job ADD CONSTRAINT fk_job_company
    FOREIGN KEY (company_id) REFERENCES Company(id);
ALTER TABLE Job ADD CONSTRAINT fk_job_staged
    FOREIGN KEY (raw_job_id) REFERENCES StagedRecruitmentData(id);

-- company_address表的外键约束
ALTER TABLE company_address ADD CONSTRAINT fk_company_address_company
    FOREIGN KEY (company_id) REFERENCES Company(id);
ALTER TABLE company_address ADD CONSTRAINT fk_company_address_address
    FOREIGN KEY (address_id) REFERENCES Address(id);

-- job_address表的外键约束
ALTER TABLE job_address ADD CONSTRAINT fk_job_address_job
    FOREIGN KEY (job_id) REFERENCES Job(id);
ALTER TABLE job_address ADD CONSTRAINT fk_job_address_address
    FOREIGN KEY (address_id) REFERENCES Address(id);
    \end{minted}
    \caption{外键约束定义}\label{lst:foreign_keys}
  \end{listing}

  \item 级联操作策略
  
  为了确保数据的安全性和一致性,所有外键约束均采用RESTRICT策略。这意味着当尝试删除被其他表引用的记录时,数据库将阻止该操作(ON DELETE RESTRICT);同样,当尝试更新被引用的主键值时,也会被阻止(ON UPDATE RESTRICT)。系统不采用级联删除(CASCADE)机制,以防止因误操作导致大规模的数据丢失。所有需要进行的更新操作都通过应用层代码严格控制,以确保数据的一致性和完整性。
\end{enumerate}

\subsubsection{用户定义完整性约束}
对于用户定义完整性约束,主要包含三类约束:非空约束、默认值约束和检查约束。非空约束(如\cref{lst:not_null_constraints}所示)主要应用于User表的username、password和role字段,Company表的name字段,Job表的title和company\_id字段,以及Address表的address\_str字段,确保这些关键字段不能为空。默认值约束(如\cref{lst:default_constraints}所示)设置了User表role字段的默认值为'user',各数据处理表的processed字段默认值为FALSE,以及Company表和User表的时间戳字段默认值为当前时间。检查约束(如\cref{lst:check_constraints}所示)则对数据的有效性进行验证,包括User表role字段的取值范围限制,Address表经纬度的有效范围检查,以及Company表和Job表的时间戳先后顺序验证。

\begin{listing}[htbp]
    \begin{minted}{sql}
-- User表非空约束
ALTER TABLE User MODIFY username VARCHAR(50) NOT NULL;
ALTER TABLE User MODIFY password VARCHAR(50) NOT NULL;
ALTER TABLE User MODIFY role ENUM('user', 'admin') NOT NULL;

-- Company表非空约束
ALTER TABLE Company MODIFY name VARCHAR(500) NOT NULL;

-- Job表非空约束
ALTER TABLE Job MODIFY title VARCHAR(500) NOT NULL;
ALTER TABLE Job MODIFY company_id INTEGER NOT NULL;

-- Address表非空约束
ALTER TABLE Address MODIFY address_str VARCHAR(200) NOT NULL;
    \end{minted}
    \caption{非空约束定义}\label{lst:not_null_constraints}
  \end{listing}

\begin{listing}[htbp]
    \begin{minted}{sql}
-- 角色默认值
ALTER TABLE User ALTER role SET DEFAULT 'user';

-- 处理状态默认值
ALTER TABLE RawRecruitmentData_Boss ALTER processed SET DEFAULT FALSE;
ALTER TABLE RawRecruitmentData_Liepin ALTER processed SET DEFAULT FALSE;
ALTER TABLE StagedRecruitmentData ALTER processed SET DEFAULT FALSE;

-- 时间戳默认值
ALTER TABLE User ALTER created_at SET DEFAULT CURRENT_TIMESTAMP;
ALTER TABLE Company ALTER created_at SET DEFAULT CURRENT_TIMESTAMP;
ALTER TABLE Company ALTER updated_at SET DEFAULT CURRENT_TIMESTAMP;
    \end{minted}
    \caption{默认值约束定义}\label{lst:default_constraints}
  \end{listing}

\begin{listing}[htbp]
    \begin{minted}{sql}
-- 角色检查
ALTER TABLE User ADD CONSTRAINT check_user_role 
    CHECK (role IN ('user', 'admin'));

-- 经纬度范围检查
ALTER TABLE Address ADD CONSTRAINT check_latitude
    CHECK (latitude BETWEEN -90 AND 90);
ALTER TABLE Address ADD CONSTRAINT check_longitude
    CHECK (longitude BETWEEN -180 AND 180);

-- 时间戳有效性检查
ALTER TABLE Company ADD CONSTRAINT check_company_timestamps
    CHECK (updated_at >= created_at);
ALTER TABLE Job ADD CONSTRAINT check_job_timestamps
    CHECK (updated_at >= created_at);
    \end{minted}
    \caption{检查约束定义}\label{lst:check_constraints}
  \end{listing}
