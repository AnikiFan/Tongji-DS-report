\section{需求分析}

\subsection{系统目标}

本系统旨在构建一个多源招聘数据分析平台。在当前就业市场中,招聘信息分散在BOSS直聘、猎聘网等多个平台,给求职者和研究人员带来了信息获取和分析的困难。本系统通过整合多个招聘网站的数据,建立统一的数据仓库,为就业市场研究提供数据支持。

系统的核心目标包括:
\begin{itemize}
    \item 实现多源数据的自动采集与整合,确保数据的及时性和完整性
    \item 建立规范化的数据处理流程,保证数据质量
    \item 提供多维度的数据分析和可视化功能,支持就业市场研究
    \item 构建高效可靠的数据仓库,实现数据的长期积累和价值挖掘
\end{itemize}

\subsection{功能需求}

\subsubsection{数据采集需求}

系统的数据采集模块是整个平台的基础,需要解决多源异构数据的获取和存储问题。根据项目代码实现,主要包含以下功能:

\paragraph{1. 多源数据采集}
系统支持从BOSS直聘、猎聘网等多个招聘网站采集数据。采集过程采用异步方式实现,通过\texttt{BossZhipinCrawler}和相应的爬虫组件执行数据获取。采集的数据包括:
\begin{itemize}
    \item 职位基本信息:职位名称、薪资范围、工作经验要求等
    \item 公司详细信息:公司名称、行业类别、公司规模等
    \item 工作地点信息:详细地址、所属区域等
\end{itemize}

\paragraph{2. 数据存储架构}
系统采用分层存储架构,包括:
\begin{itemize}
    \item 原始数据层:通过\texttt{RawRecruitmentData\_Boss}和\texttt{RawRecruitmentData\_Liepin}表存储爬虫直接获取的数据
    \item 暂存数据层:使用\texttt{StagedRecruitmentData}表存储统一格式后的中间数据
    \item 规范化数据层:包含\texttt{Job}、\texttt{Company}、\texttt{Address}等标准化的业务实体表
\end{itemize}

\paragraph{3. 增量更新机制}
系统实现了数据的增量更新机制:
\begin{itemize}
    \item 使用处理标记(\texttt{processed})追踪数据处理状态
    \item 通过时间戳(\texttt{created\_at}、\texttt{updated\_at})记录数据变更
    \item 支持数据的版本控制和历史追溯
\end{itemize}

\subsubsection{数据处理需求}

数据处理是确保数据质量的关键环节,系统实现了完整的ETL(提取、转换、加载)流程:

\paragraph{1. 数据清洗和转换}
系统通过\texttt{transform.py}模块实现数据的标准化处理:
\begin{itemize}
    \item 统一数据格式:将不同来源的数据转换为统一的结构
    \item 数据规范化:对薪资、工作经验、学历等字段进行标准化处理
    \item 空值处理:对缺失数据进行合理的填充或标记
\end{itemize}

\paragraph{2. 数据质量控制}
系统实现了多层次的数据质量控制机制:
\begin{itemize}
    \item 记录链接:通过\texttt{record\_linkage.py}实现重复数据的识别和处理
    \item 数据验证:使用数据库约束和应用层验证确保数据完整性
    \item 异常检测:实现对异常值的识别和处理
\end{itemize}

\paragraph{3. 数据增强}
系统提供了多种数据增强功能:
\begin{itemize}
    \item 地理编码:通过\texttt{geo\_encoder.py}将地址信息转换为经纬度坐标
    \item 数据分类:实现职位类型的多级分类
    \item 统计分析:通过\texttt{salary\_stats.py}等模块实现各类统计指标的计算
\end{itemize}

\subsection{数据需求}

\subsubsection{核心数据实体}

系统的数据模型设计基于业务实体的特点和关系,主要包括:

\paragraph{1. 用户数据(User)}
用户实体用于系统访问控制,包含:
\begin{itemize}
    \item 基本信息:用户名、密码哈希、电子邮件等
    \item 角色信息:用户角色(普通用户/管理员)
    \item 权限控制:基于角色的访问控制(RBAC)
\end{itemize}

\paragraph{2. 公司数据(Company)}
公司实体是招聘信息的重要维度:
\begin{itemize}
    \item 基本信息:公司名称、所属行业、公司规模等
    \item 关联信息:与职位、地址的多重关联关系
    \item 元数据:创建时间、更新时间等
\end{itemize}

\paragraph{3. 职位数据(Job)}
职位是系统的核心业务实体:
\begin{itemize}
    \item 基本信息:职位名称、职位类型、薪资范围等
    \item 要求信息:工作经验、学历要求、技能要求等
    \item 关联信息:所属公司、工作地点等
    \item 溯源信息:数据来源、原始ID等
\end{itemize}

\paragraph{4. 地址数据(Address)}
地址实体支持位置相关的分析:
\begin{itemize}
    \item 基本信息:地址文本、所属区域
    \item 地理信息:经纬度坐标
    \item 关联信息:与公司、职位的多对多关系
\end{itemize}

\subsubsection{数据关系需求}

系统通过复杂的关系模型支持多维度的数据分析:

\paragraph{1. 多对多关系}
系统实现了灵活的多对多关系处理:
\begin{itemize}
    \item 公司与地址:通过\texttt{company\_address}关联表实现
    \item 职位与地址:通过\texttt{job\_address}关联表实现
    \item 支持关系的时间追踪和历史记录
\end{itemize}

\paragraph{2. 一对多关系}
系统维护了实体间的层级关系:
\begin{itemize}
    \item 公司与职位:一个公司可发布多个职位
    \item 原始数据与规范化数据:保持数据处理的可追溯性
\end{itemize}

\subsection{性能需求}

\subsubsection{数据库性能需求}

系统对数据库性能提出了较高要求:

\paragraph{1. 并发处理}
系统采用异步数据库操作提升并发性能:
\begin{itemize}
    \item 使用\texttt{AsyncSession}支持异步数据库访问
    \item 实现连接池管理,优化资源使用
    \item 通过事务控制保证数据一致性
\end{itemize}

\paragraph{2. 查询性能}
系统实现了多层次的查询优化:
\begin{itemize}
    \item 使用物化视图(\texttt{materialized\_view.py})优化常用查询
    \item 实现高效的索引策略
    \item 支持复杂的统计分析查询
\end{itemize}

\paragraph{3. 存储性能}
系统采用高效的存储策略:
\begin{itemize}
    \item 实现数据压缩存储
    \item 使用合理的分区策略
    \item 优化索引设计和存储结构
\end{itemize}

\subsubsection{系统性能需求}

系统整体性能需求包括:

\paragraph{1. 响应时间}
系统定义了明确的响应时间要求:
\begin{itemize}
    \item 普通查询响应时间控制在1秒以内
    \item 复杂统计分析响应时间不超过3秒
    \item 支持数据处理任务的异步执行
\end{itemize}

\paragraph{2. 可扩展性}
系统设计考虑了未来的扩展需求:
\begin{itemize}
    \item 支持数据量的持续增长
    \item 便于新数据源的接入
    \item 支持新分析功能的扩展
\end{itemize}

\subsection{数据库特殊需求}

\subsubsection{数据一致性}

系统实现了多层次的数据一致性保障:

\paragraph{1. 实体完整性}
通过主键和唯一约束确保实体的唯一性:
\begin{itemize}
    \item 使用自增主键标识每条记录
    \item 实现业务层面的唯一性检查
    \item 保证实体间的引用完整性
\end{itemize}

\paragraph{2. 参照完整性}
系统通过外键约束维护数据关系:
\begin{itemize}
    \item 确保关联数据的有效性
    \item 实现级联更新和删除
    \item 防止孤立数据的产生
\end{itemize}

\paragraph{3. 业务规则约束}
系统实现了复杂的业务规则验证:
\begin{itemize}
    \item 薪资范围的合理性检查
    \item 地址信息的有效性验证
    \item 数据关联的业务逻辑验证
\end{itemize}

\subsubsection{数据安全性}

系统实现了完整的数据安全保护机制:

\paragraph{1. 访问控制}
基于角色的访问控制:
\begin{itemize}
    \item 实现用户认证和授权
    \item 控制数据访问权限
    \item 记录用户操作日志
\end{itemize}

