\section{数据库物理设计}

\subsection{存储结构设计}

\subsubsection{数据库选型}
\begin{itemize}
  \item 选用PostgreSQL 数据库系统
  \begin{itemize}
    \item 支持JSON类型,用于存储爬虫原始数据(raw\_data字段)和地址聚合(addresses字段)
    \item 支持物化视图,用于预计算recruitment\_mv,提升查询性能
    \item 支持gin\_trgm\_ops操作符,用于地址和公司名称的模糊匹配
    \item 支持异步SQLAlchemy,实现高并发数据处理
    \item 内置的MVCC机制,保证数据一致性
  \end{itemize}
  \item 数据库配置
  \begin{itemize}
    \item 连接池:使用SQLAlchemy的异步会话管理
    \item 环境变量:通过.env文件配置数据库连接参数
    \item 时区处理:统一使用UTC时区存储时间戳
  \end{itemize}
\end{itemize}

\subsubsection{表空间组织}

\begin{enumerate}
  \item 规范化数据层表空间
  \begin{itemize}
    \item 核心业务表:users, companies, jobs, addresses
    \begin{itemize}
      \item users表:用户认证和权限管理
      \item companies表:公司基本信息和元数据
      \item jobs表:职位详情和关联信息
      \item addresses表:地理位置和坐标数据
    \end{itemize}
    \item 关系映射表:company\_address, job\_address
    \begin{itemize}
      \item company\_address:多对多关系,支持一个公司多个地址
      \item job\_address:多对多关系,支持一个职位多个工作地点
    \end{itemize}
    \item 物化视图:recruitment\_mv
    \begin{itemize}
      \item 预聚合职位、公司、地址信息
      \item 支持高性能的分页和统计查询
      \item 定时刷新保证数据一致性
    \end{itemize}
  \end{itemize}

  \item 数据处理层表空间
  \begin{itemize}
    \item 原始数据表:raw\_recruitment\_boss, raw\_recruitment\_liepin
    \begin{itemize}
      \item 存储爬虫直接获取的JSON数据
      \item 使用processed标志位跟踪处理状态
      \item 保留source\_url确保数据不重复爬取
    \end{itemize}
    \item 暂存数据表:staged\_recruitment\_data
    \begin{itemize}
      \item 统一格式的中间表,存储清洗后的数据
      \item 支持跨来源的记录去重和链接
      \item 使用record\_hash\_for\_linkage优化相似度计算
      \item 通过duplicate\_group\_id和is\_master\_record管理重复记录
    \end{itemize}
  \end{itemize}
\end{enumerate}

\subsubsection{字段类型选择}

在设计数据库字段类型时,我们需要根据数据的特征和使用场景选择合适的类型,以平衡存储空间、查询性能和数据完整性。表\ref{tab:field_types}展示了本系统中使用的主要字段类型及其应用场景。

对于标识符字段,我们选择Integer类型作为主键和外键,这不仅节省存储空间,还能通过自增特性避免键值碎片,提高数据库性能。在PostgreSQL中,Integer类型的自增主键会被自动创建聚集索引,有利于提升查询效率。

文本类型的选择上,我们根据内容长度的不同采用了String和Text两种类型。对于长度固定或者有明确上限的字段(如用户名、角色等),使用String类型并指定合适的长度限制;对于可能包含大量文本的字段(如职位描述、公司简介等),则使用Text类型以节省存储空间。特别地,对于枚举类型的字段(如用户角色),我们使用PostgreSQL的Enum类型来确保数据的有效性。

为了处理复杂的非结构化数据,我们充分利用了PostgreSQL对JSON类型的支持。原始爬虫数据和需要灵活扩展的字段(如标签数据、地址列表等)都使用JSON类型存储,这不仅提供了灵活的数据结构,还能通过PostgreSQL的JSON操作符实现高效的查询。

表\ref{tab:field_lengths}详细列出了各个表中字符串类型字段的长度限制。这些限制是基于实际数据分析得出的,既要确保能够容纳正常数据,又要避免过度分配空间。例如,公司名称和职位名称的长度限制设置为500,这是考虑到中文公司名称可能较长,而且可能包含附加信息;而用户名等字段则限制在50个字符以内,这对于标识用途来说已经足够。

对于时间相关的字段,我们统一使用带时区的DateTime类型,并通过SQLAlchemy的配置确保所有时间戳都以UTC时区存储。这样不仅能够准确记录时间信息,还能避免因时区差异导致的问题。created\_at和updated\_at这两个时间戳字段在大多数表中都会使用,用于追踪记录的创建和修改时间。

\begin{table}[!htbp]
  \caption{数据库字段类型设计}
  \label{tab:field_types}
  \centering
  \begin{tabular}{@{}llll@{}} \toprule
    \textbf{类型分类} & \textbf{具体类型} & \textbf{应用场景} & \textbf{优点} \\ \midrule
    标识符字段 & Integer & 主键、外键 & 存储空间小,连接效率高 \\
              &         & id字段自增 & 避免键值碎片 \\ \midrule
    文本字段   & String(50) & 用户名、角色枚举 & 定长存储,性能好 \\
              & String(500) & 公司名称、职位名称 & 适应较长文本 \\
              & Text & 技能描述、职位福利 & 变长存储,节省空间 \\
              & Enum & 用户角色(UserRole) & 约束取值范围 \\ \midrule
    JSON字段   & JSON & 原始数据(raw\_data) & 灵活的数据结构 \\
              &      & 标签数据(tags) & 支持复杂查询 \\
              &      & 地址列表(addresses) & 支持索引优化 \\ \midrule
    时间戳字段 & DateTime & created\_at & 支持时区(UTC) \\
              &          & updated\_at & 自动更新时间戳 \\ \bottomrule
  \end{tabular}
\end{table}

\begin{table}[!htbp]
  \caption{字段长度限制设计}
  \label{tab:field_lengths}
  \centering
  \begin{tabular}{@{}lll@{}} \toprule
    \textbf{表名} & \textbf{字段名} & \textbf{长度限制} \\ \midrule
    users & username & String(50) \\
          & password & String(50) \\ \midrule
    companies & name & String(500) \\
             & industry & String(200) \\
             & size & String(50) \\ \midrule
    jobs & title & String(500) \\
         & type\_level3 & String(100) \\
         & salary & String(100) \\
         & experience & String(100) \\
         & education & String(100) \\ \midrule
    addresses & address\_str & String(200) \\ \bottomrule
  \end{tabular}
\end{table}

\subsection{索引设计}

\subsubsection{基础索引}

\begin{enumerate}
  \item 主键索引
  \begin{itemize}
    \item 所有表的id字段
    \item 类型:B-tree
    \item 特点:聚集索引
  \end{itemize}

  \item 外键索引(如代码\ref{lst:foreign_key_indexes}所示)
  \item 唯一索引(如代码\ref{lst:unique_indexes}所示)
\end{enumerate}

\begin{listing}[htbp]
  \begin{minted}{sql}
-- Job表外键索引
CREATE INDEX ix_jobs_company_id ON jobs(company_id);
CREATE INDEX ix_jobs_raw_job_id ON jobs(raw_job_id);
  \end{minted}
  \caption{外键索引定义}\label{lst:foreign_key_indexes}
\end{listing}

\begin{listing}[htbp]
  \begin{minted}{sql}
-- User表唯一索引
CREATE UNIQUE INDEX ix_users_username ON users(username);

-- Company表唯一索引
CREATE UNIQUE INDEX ix_companies_name ON companies(name);
  \end{minted}
  \caption{唯一索引定义}\label{lst:unique_indexes}
\end{listing}

\subsubsection{复合索引}

\begin{enumerate}
  \item 业务查询优化索引(如代码\ref{lst:composite_indexes}所示)
  \item 数据处理优化索引(如代码\ref{lst:processing_indexes}所示)
\end{enumerate}

\begin{listing}[htbp]
  \begin{minted}{sql}
-- 公司查询优化
CREATE INDEX ix_companies_industry_size ON companies(industry, size);
CREATE INDEX ix_companies_name_industry ON companies(name, industry);

-- 职位查询优化
CREATE INDEX ix_jobs_title_type ON jobs(title, type_level3);
CREATE INDEX ix_jobs_created_updated ON jobs(created_at, updated_at);
  \end{minted}
  \caption{复合索引定义}\label{lst:composite_indexes}
\end{listing}

\begin{listing}[htbp]
  \begin{minted}{sql}
-- 原始数据处理状态索引
CREATE INDEX ix_raw_recruitment_boss_processed_created 
ON raw_recruitment_boss(processed, created_at);

-- 暂存数据去重索引
CREATE INDEX ix_staged_recruitment_data_record_hash_for_linkage 
ON staged_recruitment_data(record_hash_for_linkage);
  \end{minted}
  \caption{数据处理索引定义}\label{lst:processing_indexes}
\end{listing}

\subsubsection{特殊索引}

\begin{enumerate}
  \item GIN索引
  \begin{itemize}
    \item 用于全文搜索
    \item 支持模糊匹配
    \item 示例:
    \begin{minted}{sql}
-- 地址文本搜索
CREATE INDEX ix_addresses_address_str_trgm 
ON addresses USING gin (address_str gin_trgm_ops);
    \end{minted}
  \end{itemize}

  \item B-tree索引
  \begin{itemize}
    \item 用于精确匹配和范围查询
    \item 支持多列复合索引
    \item 示例:
    \begin{minted}{sql}
-- 公司查询优化
CREATE INDEX ix_companies_industry_size 
ON companies(industry, size);
    \end{minted}
  \end{itemize}

  \item 物化视图索引
  \begin{listing}[htbp]
    \begin{minted}{sql}
-- 物化视图优化索引
CREATE INDEX ix_recruitment_mv_title ON recruitment_mv(title);
CREATE INDEX ix_recruitment_mv_company_name ON recruitment_mv(company_name);
CREATE INDEX ix_recruitment_mv_created_at ON recruitment_mv(created_at);
    \end{minted}
    \caption{物化视图索引定义}\label{lst:materialized_view_indexes}
  \end{listing}
\end{enumerate}

\subsection{查询优化设计}

\subsubsection{物化视图优化}
\begin{enumerate}
  \item 物化视图设计
  \begin{itemize}
    \item 预聚合核心查询数据
    \begin{itemize}
      \item 合并jobs、companies和addresses表
      \item 使用json\_agg聚合多个地址
      \item 包含所有常用查询字段
    \end{itemize}
    \item 性能优化
    \begin{itemize}
      \item 创建覆盖索引避免回表
      \item 使用CONCURRENTLY并发刷新
      \item 通过调度器定期维护
    \end{itemize}
  \end{itemize}

  \item 刷新策略
  \begin{itemize}
    \item 定时刷新
    \begin{itemize}
      \item 默认5分钟刷新间隔
      \item 可通过环境变量配置
      \item 支持手动触发刷新
    \end{itemize}
    \item 并发刷新
    \begin{itemize}
      \item 使用CONCURRENTLY选项
      \item 不阻塞在线查询
      \item 自动处理失败重试
    \end{itemize}
  \end{itemize}
\end{enumerate}

\subsubsection{查询优化策略}

\begin{enumerate}
  \item 分页查询优化
  \begin{itemize}
    \item 使用物化视图加速
    \item 避免COUNT(*)带来的全表扫描
    \item 基于时间戳的游标分页
    \item 示例:
    \begin{minted}{sql}
-- 使用物化视图的分页查询
SELECT * FROM recruitment_mv
WHERE company_industry = :industry
  AND created_at < :last_timestamp
ORDER BY created_at DESC
LIMIT :limit;
    \end{minted}
  \end{itemize}

  \item 统计查询优化
  \begin{itemize}
    \item 利用物化视图预计算
    \item 使用array\_agg聚合数组
    \item 缓存热点统计数据
    \item 示例:
    \begin{minted}{sql}
-- 使用物化视图的统计查询
SELECT 
    company_industry,
    count(*) as job_count,
    array_agg(DISTINCT company_size) as company_sizes,
    avg(CASE 
        WHEN salary ~ '^[0-9]+-[0-9]+k$' 
        THEN (
            (regexp_replace(split_part(salary, '-', 1), 'k', '')::integer + 
             regexp_replace(split_part(salary, '-', 2), 'k', '')::integer) / 2
        )::float 
        ELSE null 
    END) as avg_salary
FROM recruitment_mv
GROUP BY company_industry;
    \end{minted}
  \end{itemize}

  \item 全文搜索优化
  \begin{itemize}
    \item 使用GIN索引加速模糊匹配
    \item 支持中文分词和相似度计算
    \item 优化字符串标准化处理
    \item 示例:
    \begin{minted}{sql}
-- 使用GIN索引的模糊搜索
SELECT * FROM addresses
WHERE address_str % :query
ORDER BY similarity(address_str, :query) DESC
LIMIT 10;
    \end{minted}
  \end{itemize}

  \item 记录链接优化
  \begin{itemize}
    \item 使用哈希值快速过滤
    \item 分批处理避免内存溢出
    \item 动态调整相似度阈值
    \item 示例:
    \begin{minted}{python}
def calculate_similarity_score(record1: Dict, record2: Dict) -> float:
    """计算两条记录的相似度得分,使用动态权重"""
    base_weights = {
        'company': 0.4,  # 公司名称权重
        'title': 0.4,    # 职位名称权重
        'address': 0.1,  # 地址权重
        'other': 0.1     # 其他特征权重
    }
    # 动态调整权重和计算相似度
    ...
    return final_score
    \end{minted}
  \end{itemize}
\end{enumerate}

\subsection{并发控制}

\subsubsection{事务隔离级别}
\begin{itemize}
  \item 读已提交(Read Committed)
  \begin{itemize}
    \item PostgreSQL默认隔离级别
    \item 防止脏读,允许不可重复读
    \item 支持MVCC,实现读写互不阻塞
  \end{itemize}
  \item 事务管理
  \begin{itemize}
    \item 使用SQLAlchemy的异步会话(AsyncSession)
    \item 支持嵌套事务(begin\_nested)
    \item 自动提交和回滚机制
  \end{itemize}
\end{itemize}

\subsubsection{并发访问优化}
\begin{itemize}
  \item ETL并发处理
  \begin{itemize}
    \item 使用batch\_size控制批处理大小
    \item 通过processed标志位追踪处理状态
    \item 支持断点续传和错误重试
  \end{itemize}
  \item 记录链接优化
  \begin{itemize}
    \item 使用record\_hash快速判断重复
    \item 动态调整相似度阈值
    \item 分批处理避免长事务
  \end{itemize}
  \item 物化视图刷新
  \begin{itemize}
    \item 支持CONCURRENTLY并发刷新
    \item 通过调度器自动维护
    \item 避免阻塞在线查询
  \end{itemize}
\end{itemize}