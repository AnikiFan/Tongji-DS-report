\section{实验目的}
本实验的目的是通过实际操作掌握数据仓库的基本概念、建立方法和步骤,
并使用 MySQL 数据库建立数据仓库,了解其对数据仓库的支持。通过数据的抽
取、转换和加载(ETL)过程,体验数据仓库在存储、管理和分析大数据中的作
用。具体目的如下:
\begin{enumerate}
    \item 掌握数据仓库的基本概念:理解数据仓库的定义、特点、作用及其在企
    业决策支持中的重要性。
    \item 学习数据仓库的建模方法:掌握维度建模的基本原理,包括星型模式和
    雪花型模式的设计。
    \item 实现 ETL 过程:通过 Python 脚本,从原始数据源抽取数据,进行清洗和
    转换,并加载到数据仓库中,理解 ETL 过程的关键步骤和方法。
    \item 建立数据仓库环境:使用 MySQL 数据库创建维度表和事实表,理解
    MySQL 在数据仓库中的应用和优势。
    \item 数据分析与查询:通过 SQL 查询对数据仓库中的数据进行分析,掌握数
    据查询、聚合和报告生成的方法。
    \item 提高数据管理能力:通过实验操作,提高对大数据环境中数据管理、存
    储和分析的实际操作能力。
    \item 应用数据仓库技术解决实际问题:通过智能建筑系统的数据集,模拟真
    实场景,解决数据管理和分析中的实际问题,提升对数据仓库技术的应
    用能力。
\end{enumerate}