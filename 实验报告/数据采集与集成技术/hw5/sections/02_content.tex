\section{实验内容}
本实验将整合来自多个数据源的信息,包括混凝土的成分配比、养护时间和
强度测试结果。通过数据清洗、集成和可视化分析,展示集成的数据。
\section{实验原理}
传统数据集成的方法解决语义歧义性、实例表示歧义性和数据不
一致性带来的挑战时使用的是一种流水线体系结构,主要包含三个步
骤:1)模式对齐;2)记录连接;3)数据融合。

传统数据集成中的第一个主要步骤是模式对齐。它主要针对的是
语义歧义性带来的问题,目标是理解哪些属性具有相同的含义而哪些属性的含义不同。其正式的定义如下。 

给定某一领域内的一组数据源模式,不同的模式用不同的方式描
述该领域。模式对齐步骤生成以下三种输出。 
\begin{enumerate}
    \item 中间模式。它为不同数据源提供一个统一的视图,并描述了给定领域的突出方面。
    \item 属性匹配。它将每个源模式中的属性匹配到中间模式的相应属性。
    \item 模式映射。每个源模式和中间模式之间的映射用来说明数据源的内容和中间数据的内容之间的语义关系。
\end{enumerate}

结果模式映射被用来在查询问答中将一个用户查询重新表达成一
组底层数据源上的查询。

传统数据集成中的第二个主要步骤是记录链接。它主要针对的是
实例表示歧义性所造成的问题,目标是理解哪些记录表示相同的实体
而哪些不是。其正式的定义如下。

给定一组数据源,每个包含了定义在一组属性上的一组记录。记录链接是计算出记录集上的一个划分,使得每个划分类包含描述同一
实体的记录。 

传统数据集成中的第三个主要步骤是数据融合。它主要针对的是
数据质量带来的挑战,目标是理解在数据源提供相互冲突的数据值时
在集成起来的数据中应该使用哪个值。其正式的定义如下。 

给定一组数据项,以及为其中一些数据项提供值的一组数据源。
数据融合决定每个数据项正确的值。 