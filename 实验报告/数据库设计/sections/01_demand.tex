\section{需求分析}\label{sec:demand}
\subsection{问题背景}
\subsubsection{共享单车的概念}
共享单车作为共享经济的新形态,
是指企业与政府进行合作,在居民区、商业区、地铁站、公交车
站、校园等公共场所提供自行车骑行的共享服务。它是借助互
联网技术推出的一种分时租赁业务。在整个运营过程中由智能
技术提供支持,企业负责整个运营过程中的管理\cite{article1}。
\subsubsection{共享单车企业面对的挑战}
消费者在选择共享单车品牌时,除了定价、押金等经济因素,还会着重考量以下体验因素:

\begin{itemize}
    \item 单车性能:例如单车重量,可调节性等;
    \item 维护质量:是否有大量未及时回收的损坏单车;
    \item 单车分布:单车的时空分布是否符合使用需求的时空分布。
\end{itemize}

同时,由于共享单车市场的快速发展,交通运输部等10部门也对共享单车(互联网租凭自行车)行业发布指导意见,要求“引导有序投放车辆”、“加强互联网租凭自行车标准化建设”、“加强停放管理和监督执法”,“引导用户安全文明用车”等。

为此,想要提高市场竞争力,占据主导地位,共享单车企业首先需要达成以下目标:

\begin{itemize}
    \item 及时更新迭代投入使用的单车型号;
    \item 追踪投入使用的单车的情况,如已使用时长;
    \item 记录并处理用户的反馈,例如单车损坏情况、用车需求;
    \item 及时回收损坏单车;
    \item 实时调度闲置单车至需求密集区域;
    \item 及时对违规停车用户进行处罚。
\end{itemize}

上述这些目标都可以纳入城市共享单车管理与调度的范畴之中。
\subsection{目标场景}
本次设计的数据库用于共享单车企业的共享单车管理与调度系统。因此,该数据库中存储的数据均围绕“共享单车”这一主体,
并不涉及用户的余额、购买的月卡套餐等信息。

下面列出的是本次设计的数据库的几个可能的使用场景:
\begin{itemize}
    \item 在共享单车手机应用中,需要使用该数据库中存储的数据来实时显示用户附近的可使用车辆信息;
    \item 共享单车运维团队定时根据数据库中存储的信息来回收并淘汰使用时间过长或型号过旧的单车;
    \item 共享单车运维团队定时从数据库中读取近期的用户反馈,并回收涉及的单车;
    \item 共享单车运维团队根据数据库中存储的行车记录,分析单车使用需求的时空分布,指定调度方案;
    % \item 共享单车企业根据数据库中存储的行车记录来动态调整推荐停车点的数量和位置;
    \item 分析团队利用数据库中存储的行车轨迹数据对用户使用习惯进行分析;
    \item 分析团队利用聚类等方式对数据库中存储的单车位置数据进行分析;
    \item 分析团队对数据库中存储的历史数据进行可视化。
\end{itemize}
\subsection{数据需求}
针对上述应用场景,本数据库存储和下列对象相关的数据:
\begin{itemize}
    \item 单车
    \item 单车轨迹
    \item 单车回收信息
    \item 单车调度信息
    \item 单车投放信息
    \item 用户
    \item 用户反馈
    % \item 停车点
    \item 仓库
\end{itemize}
% 具体属性已经各实体之间的关系将在后续的概念设计与逻辑设计中详细阐明。
\subsection{功能需求}
根据上述的应用场景,可知该数据库中需要进行的操作具有以下特点:
\begin{itemize}
    \item 经常需要根据时间属性进行范围选择,并且以附近的时间段为主;
    \item 需要针对空间数据,即坐标,进行范围查询,即获取一定范围内的数据;
    \item 更新单车回收、投放、调度信息时,需要同步更新单车坐标,仓库存储情况等数据;
    \item 特定数据更新频率较高,例如单车的当前坐标、状态信息;
    \item 有时需要读出大量数据,例如进行数据分析和可视化时;
    \item 较少进行删除操作。
\end{itemize}

在进行后续的设计时,将针对上述特点对数据库进行优化。