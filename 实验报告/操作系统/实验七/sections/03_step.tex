\section{实验步骤}
\subsection{实验4.1}
\begin{listing}[htbp]
    \begin{minted}{c}
#include <stdio.h>
#include <sys.h>
#include <file.h>
void main1()
{
    char data1[13]="Hello World!";
    char data2[13];
    int fd = 0;
    int count = 0;
    fd = creat("Jessy",0666);
    if (fd>0)
    {
        printf("The file %d is created.\n",fd);
    }
    else
    {
        printf("The file can not be created.\n");
    }
    count = write(fd, data1, 13);
    if (count == 13)
    {
        printf("%d characters have been written into the file %d.\n", count,fd);
    }
    else
    {
        printf("The file can not be written successfully.\n");
    }
    close(fd);
    fd = open("Jessy",3); //以可读可写的方式打开文件
    count = read(fd, data2, 12);
    printf("%d characters are read from file %d: %s.", count, fd, data2);
    printf("\n");
    close(fd);
}
    \end{minted}
    \caption{fileTest.c}\label{lst:fileTest}
\end{listing}

执行代码\ref{lst:fileTest}可以得到图\ref{p1true}中的输出结果。当注释掉第28,29行的重新打开文件操作后,可以得到如图\ref{p1false}的结果;
当把字符串数组 \texttt{data1,data2}的长度都改为12后,得到图\ref{p112}中的输出结果。

代码\ref{lst:fileTest}中以模式0666创建文件,
对应的是所有用户都有读写权限。由代码\ref{lst:Creat}中的\verb|__asm__ __volatile__ ( "int $0x80":"=a"(res):"a"(8),"b"(pathname),"c"(mode));|,和代码\ref{lst:creat}
中的\verb@unsigned int newACCMode=u.u_arg[1]&(Inode::IRWXU|Inode::IRWXG|Inode::IRWXO);@\newline 可以看到,该模式确实被传递进\texttt{Creat}中。但是,
从\verb|this->Open1(pInode, File::FWRITE, 2);|可以看到,\texttt{Creat}函数中还是只以写操作模式打开文件。因此,若想读文件,必须先关闭,然后重新以至少为读操作的模式打开后才能读取文件。
从代码\ref{lst:RdWr}中的\verb|if ( (pFile->f_flag & mode) == 0 )|语句可以看出,如果所作的读写操作与文件打开的模式不匹配,则会设置错误码后直接返回,此时User结构中的用于存放系统调用返回值的核心栈单元未被修改。
从代码\ref{lst:Trap}中的\verb|if( User::NOERROR != u.u_error )|可以看到,如果出错码不为\texttt{User::NOERROR},则会将系统调用的返回值设置为出错码的相反数。
从代码\ref{lst:read}中的\verb|if ( res >= 0 )|可以看出,如果系统调用返回值不为正数,则会返回-1,因此,在错误输出中,会打印出-1。

之所以将字符串数组长度都修改为12后会出现图\ref{p112}的结果,是因为如代码 \ref{lst:printf}所示,向标准输出的长度由 \texttt{\_sprintf}的返回值决定。而如代码\ref{lst:sprintf}所示,字符串输出对应的长度由 \texttt{sprintf\_string}的返回值决定。
如代码\ref{lst:sprintfstring}所示,如果并没有特殊的格式规定,返回值将由 \texttt{strlen}的返回值决定。
如代码\ref{lst:strlen}所示,\texttt{strlen}函数是通过字符串数组中值为0的位置来判断数组的结束位置的,因此
当两个字符串数组的长度都修改为12后,用户栈中连续24个字节都为非零值,\texttt{strlen}的返回值为24,而\texttt{printf}又是以该返回值来决定输出的长度的,因此会得到该输出。

\begin{listing}[htbp]
    \begin{minted}{c}
int creat(char* pathname, unsigned int mode)
{
    int res;
    __asm__ __volatile__ ( "int $0x80":"=a"(res):"a"(8),"b"(pathname),"c"(mode));
    if ( res >= 0 )
        return res;
    return -1;
}
    \end{minted}
    \caption{\texttt{creat}}\label{lst:creat}
\end{listing}

\begin{listing}[htbp]
    \begin{minted}{c}
 void FileManager::Creat()
{
    Inode* pInode;
    User& u = Kernel::Instance().GetUser();
    unsigned int newACCMode = u.u_arg[1] & (Inode::IRWXU|Inode::IRWXG|Inode::IRWXO);
    pInode = this->NameI(NextChar, FileManager::CREATE);
    if ( NULL == pInode )
    {
        if(u.u_error)
            return;
        pInode = this->MakNode( newACCMode & (~Inode::ISVTX) );
        if ( NULL == pInode )
        {
            return;
        }
        this->Open1(pInode, File::FWRITE, 2);
    }
    else
    {
        this->Open1(pInode, File::FWRITE, 1);
        pInode->i_mode |= newACCMode;
    }
}
    \end{minted}
    \caption{\texttt{Creat}}\label{lst:Creat}
\end{listing}

\begin{listing}[htbp]
    \begin{minted}{c}
void FileManager::Rdwr( enum File::FileFlags mode )
{
    File* pFile;
    User& u = Kernel::Instance().GetUser();
    pFile = u.u_ofiles.GetF(u.u_arg[0]);    /* fd */
    if ( NULL == pFile )
    {
        return;
    }
    if ( (pFile->f_flag & mode) == 0 )
    {
        u.u_error = User::EACCES;
        return;
    }
    u.u_IOParam.m_Base = (unsigned char *)u.u_arg[1];    /* 目标缓冲区首址 */
    u.u_IOParam.m_Count = u.u_arg[2];        /* 要求读/写的字节数 */
    u.u_segflg = 0;        /* User Space I/O,读入的内容要送数据段或用户栈段 */
    if(pFile->f_flag & File::FPIPE)
    {
        if ( File::FREAD == mode )
        {
            this->ReadP(pFile);
        }
        else
        {
            this->WriteP(pFile);
        }
    }
    else
    {
        pFile->f_inode->NFlock();
        /* 设置文件起始读位置 */
        u.u_IOParam.m_Offset = pFile->f_offset;
        if ( File::FREAD == mode )
        {
            pFile->f_inode->ReadI();
        }
        else
        {
            pFile->f_inode->WriteI();
        }
        pFile->f_offset += (u.u_arg[2] - u.u_IOParam.m_Count);
        pFile->f_inode->NFrele();
    }
    u.u_ar0[User::EAX] = u.u_arg[2] - u.u_IOParam.m_Count;
}
    \end{minted}
    \caption{\texttt{RdWr}}\label{lst:RdWr}
\end{listing}


\begin{listing}[htbp]
    \begin{minted}{c}
void SystemCall::Trap(struct pt_regs* regs, struct pt_context* context)
{    
    User& u = Kernel::Instance().GetUser();
    if ( u.u_procp->IsSig() )
    {
        u.u_procp->PSig(context);
        u.u_error = User::EINTR;
        regs->eax = -u.u_error;
        return;
    }
    u.u_ar0 = &regs->eax;
    if(regs->eax == 20)
        regs->eax = 20;
    u.u_error = User::NOERROR;
    SystemCallTableEntry *callp = &m_SystemEntranceTable[regs->eax];
    unsigned int * syscall_arg = (unsigned int *)&regs->ebx;
    for( unsigned int i = 0; i < callp->count; i++ )
    {
        u.u_arg[i] = (int)(*syscall_arg++);
    }
    u.u_dirp = (char *)u.u_arg[0];
    u.u_arg[4] = (int)context;
    Trap1(callp->call);        /* 系统调用处理子程序,如fork(), read()等等 */
    if ( u.u_intflg != 0 )
    {
        u.u_error = User::EINTR;
    }
    if( User::NOERROR != u.u_error )
    {
        regs->eax = -u.u_error;
        Diagnose::Write("regs->eax = %d , u.u_error = %d\n",regs->eax,u.u_error);
    }
    if ( u.u_procp->IsSig() )
    {
        u.u_procp->PSig(context);
    }
    u.u_procp->SetPri();
}
    \end{minted}
    \caption{\texttt{Trap}}\label{lst:Trap}
\end{listing}

\begin{listing}[htbp]
    \begin{minted}{c}
int read(int fd, char* buf, int nbytes)
{
    int res;
    __asm__ __volatile__ ("int $0x80":"=a"(res):"a"(3),"b"(fd),"c"(buf),"d"(nbytes));
    if ( res >= 0 )
        return res;
    return -1;
}
    \end{minted}
    \caption{\texttt{read}}\label{lst:read}
\end{listing}



\begin{listing}[htbp]
    \begin{minted}{c}
void printf(char* fmt, ... )
{
    char buffer[1024];
    va pva = (va) getvahead(fmt);
    int count = _sprintf(buffer, fmt, pva);    
    write(STDOUT, buffer, count);
}
    \end{minted}
    \caption{\texttt{printf}}\label{lst:printf}
\end{listing}

\begin{listing}[htbp]
    \begin{minted}{c}
int _sprintf(char* buffer, char* fmt, va pva)
{
    unsigned int va_offset = 0;
    struct print_spec spec;
    char* bp = buffer;
    if ( buffer == 0 ) return -1;
    spec.fmt = spec.start_fmt = spec.end_fmt = fmt;
    while( find_spec(&spec) >= 0 )
    {
        char* sbp = spec.end_fmt;        
        while ( sbp < spec.start_fmt ) *bp++ = *sbp++;
        parse_spec(&spec);        
        switch( spec.info.spec )
        {
        case 'c':
            bp += sprintf_char(bp, pva, &va_offset);            
            break;
        case 's':
            bp += sprintf_string(bp, &spec.info, pva, &va_offset);
            break;            
        case 'd':
        case 'i':
        case 'x':
        case 'X':
        case 'o':
        case 'u':
            bp += sprintf_interger(bp, &spec.info, pva,&va_offset);
            break;
        case 'f':
        case 'F':
        case 'e':
        case 'E':
        case 'g':
        case 'G':
            bp += sprintf_double(bp, &spec.info, pva, &va_offset);
            break;
        default:
            break;
        }        
    }
    while ( spec.end_fmt < spec.start_fmt ) *bp++ = *spec.end_fmt++;
    *bp = 0;
    return bp - buffer;
}
    \end{minted}
    \caption{\texttt{\_sprintf}}\label{lst:sprintf}
\end{listing}

\begin{listing}[htbp]
    \begin{minted}{c}
int sprintf_string(char* buffer, struct print_info* info, va pva, unsigned int* pva_offset)
{
    char* pstr = getva(char*, pva, *pva_offset);
    int strl = strlen(pstr);
    int padding = 0;
    char* bp = buffer;
    int i;
    *pva_offset += sizeof(char*);
    if ( info->width == -1 ) info->width = strl; /* 用户并没有设置宽度值 */
    if ( info->prec == -1 ) info->prec = strl;
    if ( info->prec > info->width) info->width = info->prec;
    padding = info->width - (strl > info->prec ? info->prec : strl);
    if ( !info->left )/* padding at left */
    {
        while ( bp < buffer + padding ) *bp++ = ' ';
    }
    for ( i = 0; *pstr && i < info->width - padding; *bp++ = *pstr++, i++ );
    if ( info->left ) 
    {
        while( bp < buffer + info->width ) *bp++ = ' ';
    }    
    return info->width;
}
    \end{minted}
    \caption{\texttt{sprintf\_string}}\label{lst:sprintfstring}
\end{listing}

\begin{listing}[htbp]
    \begin{minted}{c}
int strlen (char* str)
{
    int length = 0;
    while( *(str++) ) ++length;
    return( length );
}
    \end{minted}
    \caption{\texttt{strlen}}\label{lst:strlen}
\end{listing}

\clearpage

\subsection{实验4.2}

本实验所用代码如代码\ref{lst:fileTest2}所示。输出如图\ref{p2}所示。

父进程是通过调用 \texttt{fork}来创建子进程的过程中,将 \texttt{User}结构完全复制给子结构,其中包括指向 \texttt{File}结构的指针以及表明创建进程用户的用户标识的\texttt{u\_uid},\texttt{u\_gid},而 \texttt{File}结构中的
\texttt{f\_offset}描述的是文件的读写指针位置,\texttt{f\_flag}描述的是文件的工作模式, \texttt{f\_inode}的描述的是指向打开文件的内存Inode指针,而文件的读写权限记录在 \texttt{Inode}中的 \texttt{i\_mode}中。
综上,父进程通过复制指向 \texttt{File}结构的指针来实现对于 \texttt{File}结构的共享,从而实现对于读写指针和文件工作模式的共享;通过复制用户标识,来实现读写权限的共享。

之所以父进程要重新上台后要执行 \texttt{seek}语句,是因为其与子进程共享了读写指针位置,而子文件进行写操作后,读写指针位置位于写入内容的末尾,为了读入所写的内容,需要将读写指针置于写入文件的起始位置,即0。
如果没有 \texttt{seek}语句,输出如图\ref{p2noseek}所示,这是因为此时读入的内容为全0, \texttt{strlen}返回的值为0,从而没有输出。

父子进程执行的 \texttt{close}操作的区别在于:子进程先执行 \texttt{close}操作,此时,子进程只释放自己的 \texttt{u\_ofile}结构,因为父进程还在共享 \texttt{File}结构以及对应的内存\texttt{Inode}结构,所以不进行释放;
而当父进程执行 \texttt{close}操作时,因为没有其他进程共享 \texttt{File}和相应的内存 \texttt{Inode}结构,所以除了释放自己的 \texttt{u\_ofile}外,还会释放对应的 \texttt{File}结构和内存 \texttt{Inode}结构。

\begin{listing}[htbp]
    \begin{minted}{c}
#include <stdio.h>
#include <sys.h>
#include <file.h>
void main1()
{
    char data1[13]="Hello World!";
    char data2[13];
    int fd = 0;
    int count = 0;
    int i,j;
    fd = creat("Jessy",0666); //刚创建好的文件,访问方式是可写
    if (fd>0)
    {
        printf("The file %d is created.\n",fd);
    }
    else
    {
        printf("The file can not be created.\n");
    }
    close(fd);
    fd = open("Jessy",3); //以可读可写的方式打开文件
    if(fork())
    {
        i=wait(&j);
        seek(fd,0,0);
        count = read(fd, data2, 12);
        printf("%d characters are read from file %d: %s.", count, fd, data2);
        printf("\n");
        close(fd);
    }
    else
    {
        count = write(fd, data1, 13);
        if (count == 13)
        {
            printf("%d characters have been written into the file %d.\n", count,fd);
        }
        else
        {
            printf("The file can not be written successfully.\n");
        }
        close(fd);
        exit(0);
    }
}
    \end{minted}
    \caption{fileTest.c}\label{lst:fileTest2}
\end{listing}

\subsection{实验4.3}

本实验所用代码如代码\ref{lst:fileTest3}所示,对应输出结果如图\ref{p3}所示,与所要求的输出一致。

在代码\ref{lst:fileTest3}中,父进程创建文件后便创建子进程然后入睡,此时父子进程共享 \texttt{File}结构。
子进程上台后,关闭然后以写方式重新打开文件,这使得父子进程不再共享 \texttt{File},从而不共享文件读写指针。
父进程唤醒上台后,关闭然后以只读方式重新打开文件,此时文件读写指针重置,从而无需执行 \texttt{seek}函数。

在代码\ref{lst:fileTest3}中,子进程关闭了两次文件,第一次关闭时只释放了自己的 \texttt{o\_ufile}结构,第二次关闭时仍只释放了自己的 \texttt{o\_ufile}结构;
父进程也关闭了两次文件,这两次关闭时均释放了 \texttt{o\_ufile}结构、 \texttt{File}结构和内存\texttt{Inode}结构。

\begin{listing}[htbp]
    \begin{minted}{c}
#include <stdio.h>
#include <sys.h>
#include <file.h>
void main1()
{
    char data[20];
    int fd = 0;
    int count = 0;
    fd = creat("Jessy",0421); 
    if (fd>0)
    {
        printf("The file %d is created.\n",fd);
    }
    else
    {
        printf("The file can not be created.\n");
    }
    if(fork())
    {
        wait(&count);
        close(fd);
        open("Jessy",1);
        count = read(fd, data, count);
        printf("%d characters are read from file %d: %s.", count, fd, data);
        printf("\n");
        close(fd);
    }
    else
    {
        close(fd);
        open("Jessy",2);
        count = write(fd, "Hello World!", 12);
        if (count == 12)
        {
            printf("%d characters have been written into the file %d.\n", count,fd);
        }
        else
        {
            printf("The file can not be written successfully.\n");
        }
        close(fd);
        exit(count);
    }
}
    \end{minted}
    \caption{fileTest.c}\label{lst:fileTest3}
\end{listing}
