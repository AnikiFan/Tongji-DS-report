\section{实验结果}
\subsection{修改后的 \texttt{MemoryDescriptor.h}}
  修改后的 \texttt{MemoryDescriptor.h}如代码\ref{lst:h}所示。

\begin{listing}[htbp]
    \begin{minted}{c}
//略去注释
#ifndef MEMORY_DESCRIPTOR_H
#define MEMORY_DESCRIPTOR_H
#include "PageTable.h"
class MemoryDescriptor
{
public:
    static const unsigned int USER_SPACE_SIZE    = 0x800000; 
    static const unsigned int USER_SPACE_PAGE_TABLE_CNT = 0x2;
    static const unsigned long USER_SPACE_START_ADDRESS        = 0x0;
public:
    MemoryDescriptor();
    ~MemoryDescriptor();
public:
    void MapToPageTable();
    unsigned long GetTextStartAddress();
    unsigned long GetTextSize();
    unsigned long GetDataStartAddress();
    unsigned long GetDataSize();
    unsigned long GetStackSize();
private:
    unsigned int MapEntry(unsigned long virtualAddress, unsigned int size, unsigned long phyPageIdx, bool isReadWrite);
    PageTable*        UserPageTableArray;
public:
    unsigned long    m_TextStartAddress;    /* 代码段起始地址 */
    unsigned long    m_TextSize;            /* 代码段长度 */
    unsigned long    m_DataStartAddress; /* 数据段起始地址 */
    unsigned long    m_DataSize;            /* 数据段长度 */
    unsigned long    m_StackSize;        /* 栈段长度 */
};
#endif
    \end{minted}
    \caption{MemoryDescriptor.h}\label{lst:h}
\end{listing}
\subsection{修改后的 \texttt{MemoryDescriptor.cpp}}
  修改后的 \texttt{MemoryDescriptor.cpp}如代码\ref{lst:cpp}所示。

\begin{listing}[!h]
    \begin{minted}{c}
// 略去头文件
unsigned int MemoryDescriptor::MapEntry(unsigned long virtualAddress, unsigned int size, unsigned long phyPageIdx, bool isReadWrite)
{    
    unsigned long address = virtualAddress - USER_SPACE_START_ADDRESS;
    //计算从pagetable的哪一个地址开始映射
    unsigned long startIdx = address >> 12;
    unsigned long cnt = ( size + (PageManager::PAGE_SIZE - 1) )/ PageManager::PAGE_SIZE;
    PageTableEntry* entrys = (PageTableEntry*)this->UserPageTableArray;
    for ( unsigned int i = startIdx; i < startIdx + cnt; i++, phyPageIdx++ )
    {
        entrys[i].m_Present = 0x1;
        entrys[i].m_ReadWriter = isReadWrite;
        entrys[i].m_PageBaseAddress = phyPageIdx;
    }
    return phyPageIdx;
}
// 略去成员访问函数
void MemoryDescriptor::MapToPageTable()
{
    User& u = Kernel::Instance().GetUser();
    this->UserPageTableArray = Machine::Instance().GetUserPageTableArray();
    unsigned int textAddress = 0,dataAddress=0,stackAddress=0;
    if ( u.u_procp->p_textp != NULL )
    {
        textAddress = u.u_procp->p_textp->x_caddr;
    }
    if ( u.u_procp->p_addr != NULL )
    {
        dataAddress = u.u_procp->p_addr + PageManager::PAGE_SIZE;
    }
    for (unsigned int i = 0; i < Machine::USER_PAGE_TABLE_CNT; i++)
    {
        for ( unsigned int j = 0; j < PageTable::ENTRY_CNT_PER_PAGETABLE; j++ )
        {
            this->UserPageTableArray[i].m_Entrys[j].m_Present = 0;   //先清0
        }
    }
    unsigned int phyPageIndex = textAddress>>12;
    /* 以相对起始地址phyPageIndex为0,为正文段建立相对地址映照表 */
    this->MapEntry(m_TextStartAddress, m_TextSize, phyPageIndex, false);
    phyPageIndex = dataAddress >> 12;
    /* 以相对起始地址phyPageIndex为1,ppda区占用1页4K大小物理内存,为数据段建立相对地址映照表 */
    phyPageIndex = this->MapEntry(m_DataStartAddress, m_DataSize, phyPageIndex, true);
    /* 紧跟着数据段之后,为堆栈段建立相对地址映照表 */
    unsigned long stackStartAddress = (USER_SPACE_START_ADDRESS + USER_SPACE_SIZE - m_StackSize) & 0xFFFFF000;
    this->MapEntry(stackStartAddress, m_StackSize, phyPageIndex, true);
    this->UserPageTableArray[0].m_Entrys[0].m_Present = 1;
    this->UserPageTableArray[0].m_Entrys[0].m_ReadWriter = 1;
    this->UserPageTableArray[0].m_Entrys[0].m_PageBaseAddress = 0;
    FlushPageDirectory();
}
    \end{minted}
    \caption{MemoryDescriptor.cpp}\label{lst:cpp}
\end{listing}