\documentclass[11pt,
  % twoside,
  logo = {example-image},
  % logo height = 1cm,  % logo width = 2cm,
  title in boldface,
  % title in sffamily,
  theorem in new line,
  % remove qed,
  colored solution,
]{homework}

%% For highlighting the code in this document
\usepackage{newverbs}
\newverbcommand{\cverb}{\color{red!50!orange}}{}

%% Below is the main document


\UseLanguage{Spanish}


\title{El tema, semana 1}
\author{Nombre APELLIDO}
% \date{\today, Ubicación}
% \date{\today[only-year-month], Ubicación}
\date{\TheDate{2024-01-01}, Ubicación}


\begin{document}


\textcolor{gray!55}{(Si desea escribir la respuesta directamente...)}

\begin{problem}
    Aquí está la solución / la prueba.
\end{problem}


\bigskip\textcolor{gray!55}{(Si desea plantear el problema y luego escribir su respuesta...)}

\begin{problem}[Breve descripción]
    También puede plantear el problema aquí...
\end{problem}

\begin{solution}
    ... y escribir la solución aquí...
\end{solution}

\bigskip\textcolor{gray!55}{(Si prefiere \textquote{Prueba} en lugar de \textquote{Solución}...)}

\begin{solution}[Prueba]
    ... o una prueba como ésta...
    \begin{lemma}[Puede escribir alguna descripción aquí]\label{lem}
        Algún resultado auxiliar.
    \end{lemma}
    \begin{proof}
        La prueba \cref[de]{lem}, donde usamos la siguiente fórmula (tenga en cuenta el uso de \cverb|\qedhere|):
        \[
            \infty = \infty + 1.
            \qedhere % Para colocar el símbolo Q.E.D. en el lugar correcto.
        \]
    \end{proof}
    \begin{fact}[Este resultado no requiere prueba]
        \proofless
        Utilice \cverb|\proofless| para cambiar la caja hueca que marca el final de un entorno de tipo teorema en una caja sólida.
    \end{fact}
    ... y el resto de pasos...
\end{solution}


\bigskip\textcolor{gray!55}{(También puede escribir \texttt{answer} en lugar de \texttt{solution} si lo desea...)}

\begin{answer}
    El uso del entorno \verb|answer| es exactamente el mismo que \verb|solution|.
\end{answer}


\bigskip\textcolor{gray!55}{(Si prefiere el estilo clásico de prueba...)}

\begin{proof}
    El entorno habitual de \verb|proof| también funciona.
\end{proof}


\bigskip\textcolor{gray!55}{(Si desea responder cada subpregunta de un problema individualmente...)}

\begin{problem}[Un problema con muchas subpreguntas]
    \begin{enumerate}
        \item La primera pregunta.

        \begin{solution}
            La solución de la primera pregunta.
        \end{solution}

        \item La segunda pregunta.

        \begin{enumerate}
            \item La primera subpregunta.

            \begin{solution}
                La solución de la primera subpregunta.
            \end{solution}

            \item La segunda subpregunta.

            \begin{solution}
                La solución de la segunda subpregunta.
            \end{solution}

        \end{enumerate}

        \item La tercera pregunta.

        \begin{solution}
            La solución de la tercera pregunta.
        \end{solution}

    \end{enumerate}
    Utilice \cverb|\noqed| (o \cverb|\noQED|) al final para suprimir el símbolo Q.E.D. que marca el final del problema actual.
    \noQED
\end{problem}


\bigskip\textcolor{gray!55}{(Si desea numerar el ejercicio manualmente...)}

\ManualNumbering{exercise}{A.1.1}
\begin{exercise}[Un ejercicio con numeración especificada manualmente]
    Utilice \cverb|\ManualNumbering| para configurar manualmente la numeración. Esta numeración sólo se aplicará al siguiente entorno especificado.
\end{exercise}

\begin{exercise}
    Volver a la numeración normal.
\end{exercise}


\bigskip\textcolor{gray!55}{(Si hay una pregunta que no sabe cómo resolver en este momento...)}

\DNF<alguna descripción>


\end{document}
