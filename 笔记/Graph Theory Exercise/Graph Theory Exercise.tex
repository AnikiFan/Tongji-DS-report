\documentclass[a4paper,11pt]{article}%必须以此为开头,可以在[]内设置栏数,单双面,横竖向
\usepackage{latexsym}%符号字体
\usepackage{makeidx}%制作索引
\makeindex
\usepackage{ifthen}%提供分支语句
\usepackage{graphicx}%用于插入图片
\usepackage{amsmath}%用于数学公式
\usepackage{IEEEtrantools}%用于使用IEEE数学公式排版工具
\usepackage{amsfonts}%用于其他字体的数学符号
\usepackage{amsthm}%提供证明,定理等环境
\usepackage{amssymb}%用于提供各种数学符号
\usepackage{mathrsfs}%用于提供花体字母
\usepackage{verbatim}%使用\verbatiminput{filename}来直接导入文件中的文本内容
\usepackage{layouts}%用于设置页面布局
\usepackage{calc}%允许一些常量参量用算术表达式代替
\usepackage{indentfirst}
\usepackage{hyperref}
\usepackage{makecell}%允许表格的单元格内换行
\usepackage{bm}%使用bm来对希腊字母加粗
\usepackage{longtable}
\usepackage{slashed}
\theoremstyle{remark}
\newtheorem*{remark}{remark}
\theoremstyle{definition}
\newtheorem{problem}{Problem}[subsection]
\newcommand*{\abs}[1]{\lvert #1 \rvert}
\author{Fan}
\title{Graph Theory Exercises}
\date{2023.8.13}
\begin{document}
\maketitle
\pagestyle{plain}
\tableofcontents
\printindex
\section{Definitions and examples}
\subsection{Definitions}
\begin{problem}
   If $G$ is a simple  graph with at least two vertices, prove that $G$ 
   must contain two or more vertices of the same degree. 
   \begin{proof}
    Suppose $G$ has n vertices $(n\geq 2)$ and none of its vertices has the 
    same degree. We can also assume that there is no isolated vertice, since 
    there is atmost one isolated vertice and won't contribute to 
    other vertices' degrees. Thus, we have n vertices with at least 1 degree.
    However, since the graph is simple, each vertices can have n-1 degrees at 
    most. Thus, there must be two vertices with same degree.
   \end{proof}
\end{problem}
\begin{problem}
    \begin{enumerate}
        \item Show that there are exactly $2^{n(n-1)/2}$ labelled simple graph on v vertices.
        \item How many of these have exactly m edge. \index{unsolved problem 1.0.2.2.}
    \end{enumerate}
    \begin{proof}
        We will prove by induction.
        When $n=1$, the only vertice has to be isolated vertice. Thus there 
        is only one possibility, which is agree with $2^{n(n-1)/2}$.

        Suppose when $n=k$ then equation is valid. Then, when $n=k+1$, 
        we can regard this situation as adding one vertice into $n$ vertices.
        The new vertex can choose whether to adjacent with other vertices, 
        thus we have totally $2^{n(n-1)/2}\cdot 2^{n}=2^{n(n+1)/2}$.
    \end{proof}
\end{problem}
\begin{problem}
    Let $G$ be a graph with $n$ vertices and $m$ edges, and let $v$ be 
    a vertex of $G$ of degree $k$ and $e$ be an edge of $G$. How many vertices 
    and edges have $G-e,G-v$ and $G\verb|\|e$.

    $G-e$ has $n$ vertices and $m-1$ edges.
    $G-v$ has $n-1$ vertices and $m-k+l$ edges, where l is the number of the loops incident with $v$.
    $G\verb|\|e$ has $n-1$ vertices and $m-s$ edges, where $s$ is the number of edges that is in the same edge-family with $e$.
\end{problem}
\begin{problem}
    If $G$ is a graph without loops, what can you say about the sum of the 
    entries in 
    \begin{enumerate}
        \item any row or column of the adjacency matrix of $G$?
        \item any row of the incidence matrix $G$?
        \item any column of the incidence matrix of $G$?
    \end{enumerate}

        Nothing.
        Nothing.
        It must be 2.
\end{problem}
\subsection{Challenge problems}
\begin{problem}
A simple graph that is isomorphic to its complement is self-complementary.
Prove that, if $G$ is self-complementary, then $G$ has $4k$ or $4k+1$ 
vertices, where $k$ is an integer.
\begin{proof}
    $G\cup G'=K_n$. Since $G$ and $G'$ has same amounts of edges, the 
    number of $K_n$'s edge is even. Thus $n(n-1)/2$ must be an even number.
    Therefore, $n=4k$ or $n=4k+1$.
\end{proof}
\end{problem}
\begin{problem}
    If $G$ is a simple graph with edge-set $E(G)$, the vector space of $G$
    is the vector space over the field $Z_2=\{0,1\}$ of integers modulo 2,
    whose elements are subsets of $E(G)$. The sum $E+F$ of two such subsets 
    $E$ and $F$ is the set of edges in $E$ or $F$ but not in both, and 
    scalar multiplication is defined by $1\cdot E=E$ and $0\cdot E=\varnothing$.
    Show that this define a vector space over $Z_2$, and find a basis for it.
    \index{unsolved problem 1.2.2.}
\end{problem}
\begin{problem}
    Prove that, if $G$ is regular of degree $k$, then $L(G)$ is regular of degree $2k-2$.
    \begin{proof}
        Since $G$ is regular of degree $k$, for each edge $wv$, each of its end 
        has other $k-1$ edges that is incident with it. Thus, the vertex isomorphic to
        the edge $vw$ has $2k-2$ edge that is incident with it.
    \end{proof}
\end{problem}
\begin{problem}
    Find an expression for the number of edges of $L(G)$ in terms of the degrees of 
    the vertices of $G$.

    \[\sum_{V(G)}\frac{\text{deg}v(\text{deg}v-1)}{2}\]
\end{problem}
\begin{problem}
    Show that an infinite graph $G$ can be drawn in Euclidean 3-space if $V(G)$
    and $E(G)$ can each be put in one-one correspondence with a subset of the 
    set of real numbers.
    \begin{proof}
        \index{unsolved problem 1.2.5.}
    \end{proof}
\end{problem}
\section{Paths and cycles}
\subsection{Connectivity}
\begin{problem}
    Prove that a simple graph and its complement cannot both be disconnected.
    \begin{proof}
    Suppose graph $G$ has at least two component.
    For vertex $v$ and $w$, if they are in the different component, then 
    in $\bar{G}$, they are connected. If they are in the same component, 
    then we can choose vertex $z$ in other component, so that in $\bar{G}$,
    we have $w\rightarrow z\rightarrow w$. Thus, for every pair of vertices in 
    $\bar{G}$, there exists a path connect them. Therefore, $\bar{G}$ is connected.
    \end{proof}
\end{problem}
\begin{problem}
    Prove that a graph is $2-$edge-connected if and only if any two distinct 
    vertices are joined by at least two paths with no edges in common.
    \begin{proof}
        Since $G$ is connected, for any two distinct vertices $v$ and $w$ in 
        $G$, there exists a path $P$ that joins them. Suppose there is another 
        path $P'$ that also joins them. If there are no two paths that have no 
        edges in common. Then, $P$ and $P'$ 
        \index{unsolved problem 2.1.2.}
    \end{proof}
\end{problem}
\begin{problem}
    Prove that a graph with at least three vertices is $2-$connected if and only if 
    any two distinct vertices are joined by at least two path with no other 
    vertices in common.
    \index{unsolved problem 2.1.3.}
\end{problem}
\begin{problem}
    A tournament $T$ is irreducible if it is impossible to split 
    the set of vertices of $T$ into two disjoint set $V_1$ and 
    $V_2$ so that each arc joining a vertex of $V_1$ and a vertex of 
    $V_2$ is directed from $V_1$ to $V_2$.
    Prove that a tournament is irreducible if and only if it is strongly 
    connected.
    \index{unsolved problem 2.1.4.}
\end{problem}
\subsection{Euler graphs and digraphs}
\begin{problem}
    Let $G$ be a connected graph with $k(>0)$ vertices of odd degree.
    Show that the minimum number of trails, that together include every edge 
    of $G$ and that have no edges in common, is $k/2$.
    \begin{proof}
        By Handshaking theorem, $k$ must be an even number.    

    We then proof by induction.
    When $k=2$, there is a semi-Eulerian trail, thus the minimal number is $1$.
    Suppose then $k-2$ is an even number, the minimal number is $(k-2)/2$.
    Then, when there is $k$ vertices with odd degrees.
    Suppose there are no vertices with odd incidence are adjacent. Then, 
    the original graph can be separated into two graph:
    one of them consists of all the vertices with even degree and 
    $k-2$ vertices with odd degree and all the edges that is incident with them.
    By induction hypothesis, this part can have minimal number of $(k-2)/2$.
    Another graph consists of all the vertices of even degree and two 
    vertices with odd degree that are not included in another part and the 
    edges incident with them. By corollary 2.3. there is a semi-Eulerian.
    Thus, the minimal number of second part is $1$. Therefore, the minimal 
    number for the original graph is $k/2$.

    If there are at least two vertices with odd degree that are adjacent,
    then, by removing one edge incident with these two vertices, we can 
    get a graph with only $k-2$ vertices with odd degree. Thus, by induction 
    hypothesis, the total minimal number is $k/2$.

    \begin{remark}
        Actually, we only prove that the minimal number is less than $k/2$.
        We can also prove that the minimal number is greater than $k/2$.
        When $k=2$, obviously, the minimal number is greater than $1$.
        Then, on the basis of a given graph, we can add a pair of vertices that 
        is only adjacent with each other. Then, if when there is $k$ vertices 
        with odd degree, the minimal number is greater than $k/2$. Then, when there is 
        $k+2$ vertices with odd degree, the minimal number is greater than $k/2+1=(k+2)/2$.
        Thus we finish the proof.
    \end{remark}
    \end{proof}
    \begin{remark}
        The proof given by the answer is more elegent.

        Connect the $k$ vertices in pair by adding $k/2$ edges,
        then get the Eulerian trail. Finally, remove the added edges to 
        get $k/2$ trail.

        However, it seems that it doesn't prove it is the minimal number.
    \end{remark}
\end{problem}
\subsection{Hamiltonian graphs and digraphs}
\begin{problem}
    Prove that, for every $n$, $Q_n$ is Hamiltonian.
    \begin{proof}
        For $n=1,2,3$, it is obvious to give out the Hamiltonian cycle.
        Suppose when $n=k$, there is a Hamiltonian cycle for $Q_n$, which is 
        \[\cdots\rightarrow Y\rightarrow X\rightarrow Z\rightarrow\cdots\rightarrow Y\rightarrow X\rightarrow Z\rightarrow\cdots\]
        where $X,Y,Z$ are all a string consists of n's 1 or 0.
        Then we can construct the Hamiltonian cycle for $Q_{k+1}$:
        \[\overline{X0}\rightarrow\overline{X1}\rightarrow(\text{along the direction of the former cycle})\rightarrow\overline{Y1}\rightarrow\overline{Y0}\]
        \[\rightarrow \text{(against the direction of the former cycle)}\rightarrow\overline{Z0}\rightarrow\overline{X0}\]
    \end{proof}
\end{problem}
\begin{problem}
    Prove that, if $G$ is a bipartite graph with an odd number of vertices, 
    then $G$ is non-Hamiltonian.
    \begin{proof}
   If a graph is non-Hamiltonian, then, any subgraph of it is also non-Hamiltonian.
   Thus, we only need to prove that a complete bipartite is non-Hamiltonian,
   which is obvious. 
    \end{proof}
\end{problem}
\begin{problem}
    Let $G$ be a graph with $n$ vertices and (n-1)(n-2)/2+2 edges.
    Use Ore's theorem to prove that $G$ is Hamiltonian.
    \begin{proof}
        The graph can be obtained by removing $n-3$'s edges from 
        $K_n$. Suppose there is a pair of disjoint vertices $v,w$ that 
        doesn't satisfy the condition of theorem 2.13., then, $\deg(v)+\deg(w)\leq n-1$.
        Thus, we must remove at least $1+(2n-2)-(n-1)=n'$s edges from $K_n$.
        Thus, there is no such vertices in the given graph. Therefore, it is a 
        Hamiltonian. 
    \end{proof}
\end{problem}
\subsection{Challenge problems}
\begin{problem}
    Let $G$ be a simple garph on $2k$ vertices containing no triangles.
    Prove, by induction on $k$, that $G$ has at most $k^2$ edges.
    \begin{proof}
       When $k=1$, $G$ has two vertices. Since $G$ is a simple graph, it 
       is obvious that it can have at most $1$ edges. 

       Suppose when $n=k$, in order not to have triangle in $G$, 
       $G$ can have at most $k^2$ edges.
       When $n=k+1$, there is $2k+2$ vertices in $G$. For any $2k$ vertices 
       in the graph $G$, by induciton hypothesis, there are at most 
       $k^2$ edges that are incident with any two of them. 
       Then, we can separate the $2k+2$ vertices into $v,w$ and 
       other $2k$ vertices. Since there is no triangle, there is atmost
       $k^2$ edges incident with vertex in the latter group.
       And there are atmost $2k+1$ edges left: $vw$ and $vz,wz$ for every 
       vertex $z$ in the latter group. Thus, there is at most, $k^2+2k+1=(k+1)^2$ edges.
\begin{remark}
    Such upper bound can be achieved in $K_{k,k}$
\end{remark}
    \end{proof}
\end{problem}
\begin{problem}
    Let $G$ be a connected graph with vertex-set $\{v_1,v_2,\cdots,v_n\}$,
    m edges and $t$ triangles. 
    \begin{enumerate}
        \item If $A$ is the adjacency matrix of $G$, prove that the number 
        of walks of length 2 from $v_i$ to $v_j$ is the $ij$th entry of the 
        matrix $A^2$. Deduce that $2m=$ the sum of the diagonal entries of $A^2$.
        \item Obtain a corresponding result for the number of walks of length $3$ 
        from $v_i$ to $v_j$ and deduce that $6t=$ the sum of the diagonal entries of $A^3$. 
    \end{enumerate}
    \begin{proof}
        \[A^2(i;j)=\sum_{k=1}^{n}A(i;k)A(k;j)\]
        Since
        \[
        \begin{array}{rl}
            A(i;k)A(k;j)\neq 0&\Leftrightarrow A(i;k)\neq 0 \text{ and }A(k;j)\neq 0\\
            &\Leftrightarrow v_i \text{ is adjacent with }v_k \text{ and }v_k \text{ is adjacent with }v_j\\ 
        \end{array}
    \]
    If there is $n$ walks, then there is $n$ different $k,$ thus $A^2(i;j)=n$.
    If $A^2(i;j)=n,$ then there should be $n$ different $k,$ thus there is $n$ different walks.

    Since every edge provide 2 different walk by traversing along it twice,
    whose starting point is the two vertices incident with the edges respectively.

    We can regard $A^3$ as $A^2 A$, then we can proof it similarly.
    \end{proof}
\end{problem}
\begin{problem}
    \begin{enumerate}
    \item Prove that, if two distinct cycles of a graph $G$ each contains an edge $e$,
    then $G$ has a cycle that doesn't contain $e$.
    \item Prove a similar result with `cycle' replaced by `cutsets'.
    \end{enumerate}
    \begin{proof}
        Obviously, the union of two cycle without $e$ is the desitred cycle.

        Suppose $A,B$ is the distinct cutset but both have $e$.
        Then, $A\cup B$ is no a cutset but has a subset that is cutset.
        \index{unsolved problem 2.4.3.}
    \end{proof}
\end{problem}
\begin{problem}
    Prove that, if $C$ is a cycle and $C^*$ is 
    a cutset of a connected graph $G$, then $C$ and $C^*$ have an even number 
    of edges in common.
    \begin{proof}
        Since a cutset will separate a graph into two, say, $A$ and $B$,
        and cutset has no unnecessary edges. Then, every edge in the cutset 
        should incident with one vertex in $A$ and one vertex in $B$.
        Otherwise, if we remove that edge from cutset, the graph can still be
        separated into two part.

        Thus, if there is one edge that is both in $C$ and $C^*$. Then, 
        say that edge is $ab$ and $a$ is in $A$ and $b$ is in $B$.
        Then, the part of the circle that is connected to the vertex $a$ should 
        be separated into $A$, to the vertex $b$ separated into $B$.
        Thus, there must be even edges that is both in $C$ and $C^*$.
    \end{proof}
\end{problem}
\begin{problem}
    Prove that, if $S$ is any set of edges of $G$ with an even number 
    of edges in common with each cutset of $G$, then $S$ can be split into 
    edge-disjoint cycles.
    \index{unsolved problem 2.4.5.}
\end{problem}
\begin{problem}
    A set $E$ of edges of a graph is independent if $E$ contains no 
    cycle of $G$. Prove that 
    \begin{enumerate}
        \item any subset of an independent set is independent.
        \item if $I$ and $J$ are independent sets of edges with $\abs{J}>\abs{I}$,
        then there is an edge $e$ that lies in $J$ but not in $I$ with the 
        property that $I\cup\{e\}$ is independent.
    \end{enumerate}
    Prove also that (1) and (2) still hold if we replace the word `cycle' by `cutset'.
    \begin{proof}
        The first question is very easy to prove.

        For the second question, since $\abs{J}>\abs{I}$, there exists 
        an edge $e$ that lies in $J$ but not in $I$. Suppose there is no 
        such edges that statify the given conditons, then, for every edge $e$
         that lies in $J$ but not in $I$, $I\cup \{e\}$ is there is a circle.
         However, $I$ is independent, which means that $I$ has a semi-Eulerian
         trail and has two vertices with odd degrees. To form a circle, 
         every edge that lies in $J$ but not in $I$ is incident with those two 
         vertices whose degree are odd. Therefore, there can only be one such edge,
         since $J$ is independent and thus has no multiple edges. We denote that very
         edge by $e$. Thus, we can get $J$ by adding $e$ into $I$. However, 
         this means that there is an Eulerian trail in $J$, thus there must be 
         a circle in $J$, which is a contradiction.

        When we replace `cycle' by `cutset', (1) is still obvious.
        \index{unsolved problem 2.4.6.}
        \begin{remark}
            Maybe this problem is closely related to the previous two problems.
        \end{remark}
    \end{proof}
\end{problem}
\begin{problem}
    Let $V$ be the vector space of a graph $G$
    \begin{enumerate}
        \item Use corollary 2.4 to show that, if $C$ and $D$ are cycles of $G$,
        then their sum $C+D$ can be written as a union of edge-disjoint cycles.
        \item Deduce that the set of such unions of cycles of $C$ forms a subspace $W$
        of $V$ (the cycle subspace of $G$), and find its dimension.
        \item Show that the set of unions of edge-disjoint cutsets of $G$ forms 
        a subspace $W^*$ of $V$(the cutset subspace of $G$), and  find its dimension.
    \end{enumerate}
    \index{unsolved problem 2.4.7.}
\end{problem}
\begin{problem}
    Show that the line graph of a simple Eulerian graph is Eulerian.
    \begin{proof}
        Suppose $G$ is a simple Eulerian graph with Eulerian cycle.
        Then, every vertices in $G$ has even degree.
        Thus, for every edges, each vertex that it is incident with is 
        also adjacent to an odd number of edges. Thus, every edge is 
        adjacent with odd numbers of other edges. Therefore, in $L(G)$,
        every vertex has even degree. Thus, $L(G)$ is also a Eulerian graph.
    \end{proof}
\end{problem}
\begin{problem}
    If the line graph of a simple graph $G$ is Eulerian, must $G$ 
    be Eulerian?
    \begin{proof}
        We only need to prove that $L(L(G))=G$.

        Since the number of the vertices in $L(L(G))$ equal the number of 
        the edges in $L(G)$, which is equal to the number of the vertices in $G$.
        And each pair of vertices in $L(L(G))$ is adjacent if and only if the corresponding
        edges is adjacent in $L(G)$, which happens if and only if the corresponding pair of vertices is 
        adjacent in $G$. Thus, $L(L(G))$ is isomorphic to $G$. Thus, the answer is yes.
    \end{proof}    
\end{problem}
\begin{problem}
    Let $T$ be a tournament. The score of a vertex of $T$ is its out-degree,
    and the score sequence of $T$ is the sequence formed by arming the scores of its 
    vertices in non-decreasing order. Prove that, if $(s_1,s_2,\cdots,s_n)$ is the 
    score-sequence of a tournament $T$, then 
    \begin{enumerate}
        \item $s_1+s_2+\cdots+s_n=n(n-1)/2$;
        \item for each positive integer $k<n,s_1+s_2+\cdots+s_k\geq k(k-1)/2$,with strict inequility 
        for all $k$ if and only if $T$ is strong connected.
        \item $T$ is transitive if and only if $s_k=k-1$ for each $k$
    \end{enumerate}
    \begin{proof}
        The first question is trival, since there are $n(n-1)/2$ edges at all.

        For the second question, since for the match in the $k's$ team,
        no metter who wins, it will contribute to the sum, thus $s_1+\cdots+s_k\geq k(k-1)/2$.
        When the equation is valid, these $k$'s team don't win unless they are fighting
        with the team other than these $k's$ teams. Thus, obviously, the $k$'s team 
        is isolated from the outside, since there is no path from outside from inside.
        Such path only exists when it is a strict inequility.

        \index{unsolved problem 2.4.10.3.}
    \end{proof}
\end{problem}
\begin{problem}
    Let $G$ be a Hamiltonian graph and let $S$ be any set of $k$ vertices in $G$.
    Prove that the graph $G-S$ has at most $k$ component.
    \begin{proof}
        The deletion of $k$ vertices will at most cut the Hamiltonian cycle into 
        $k's$ path, thus at most $k's$ components.
    \end{proof}
\end{problem}
\begin{problem}
    What is the maximum number of edge-disjoint Hamiltonian cycles in $K_{2k+1}$?
\end{problem}
\section{Trees}
\subsection{ Properties of trees}
\begin{problem}
    Prove that every tree is a bipartite graph
    \begin{proof}
        Choose a vertes $v$ in the tree $T$, and color it 
        white. Then, since $T$ is connected, there is a shortest 
        path join each othere vertex in $T$ to $v$. Color the 
        vertex whose corresponding shortest path's leng is even 
        white. If odd, color black.
        When a vertex is about to color, there suppose to be 
        only one vertex adjacent to it has already been colored,
        otherwise, there will be a color. Thus, we can get a bipartite.
    \end{proof}
\end{problem}
\begin{problem}
    If $G$ is a connected graph, a centre of $G$ 
    is a vertex $v$ with the property that the maximum 
    of the distances betweenn $v$ and the other vertices of $G$
    is as small as possible. Prove that every tree has either one centre 
    or two adjacent centres.
    \begin{proof}
       Since we are talking about finite graph, there is at least one such 
    center. Even we remove one end-vertex from the graph, the center stay the 
    same, unless the center is exactly that end-vertex, which will only 
    only happen when there are only two vertices. So there can only be at most 
    2 center.
    \end{proof}
\end{problem}
\begin{problem}
    \begin{enumerate}
    \item Let $C^*$ be a set of edges of a graph $G$. Show that, if $C^*$
    has an edge in common with each spanning forest of $G$, then $C^*$
    contains a cutset.
    \item Obtain a corresponding result for cycles.
    \end{enumerate}
    \begin{proof}
        We only need to prove that if a edge is in every spanning tree of $G$, 
        then, it is a bridge.

        It is obvious that it is in every possible circle or in none of them.

        In the first case, then, by removing that edge we can get a 
        tree with $n-1$ edges, which means that there are $n$ edges in 
        the original graph and the circle rank of it is $1$. Thus there 
        is only one cicle. Obviously, that edge is not a loop, then,
        it is in a circle consists of many edges, which is contrary to the hypothesis.

        In the second case, suppose only one of the vertices incident with that edge 
        is in  a cycle. Then, the part of the graph on the other side of 
        the edge is a tree, and thus it is a bridge. 
        Soppose the two vertices incident with that edge are in the circle $C_1$
        and $C_2$, then, there is no other edges that is also connected with 
        both cycles, otherwise that edge will be in a cycle. Thus, it is a 
        bridge.

        The corresponding result is that `let $C$ be a set of edges of a graph 
        $G$, if $C$ has a edge that has no common edge with any spanning forest of $G$, then  
        $C$ contain a cycle' 

        We only need to prove that one of the edge of $C$ is a loop.
        Since that edge have no common edge with any spanning forest,
        then, that edge must be in a cycle with no more that one edge, which means 
        that it is a loop.
    \end{proof}
\end{problem}
\begin{problem}
    Let $T_1$ and $T_2$ be spanning trees of a connected graph $G$.
    \begin{enumerate}
        \item If $e$ is any edge of $T_1$, show that there exists 
        an edge $f$ of $T_2$ such that the graph $(T_1-\{e\})\cup \{f\}$
        (obtained from $T_1 $on replacing $e$ by $f$)
        is also a spanning tree.
        \item Deduce that $T_1$ can be `transformed' into $T_2$ by replacing the edges of $T_1$
        one at a time by edges of $T_2$ in such a way that a spanning tree is obtained at 
        each stage.
    \end{enumerate}
    \index{unsolved problem 3.1.4.}
\end{problem}
\subsection{Counting trees}
\begin{problem}
    Let $\tau(G)$ be the number of spanning trees in a connected graph $G$.
Prove that, for any edge $e$, $\tau(G)=\tau(G-e)+\tau(G\verb|\|e)$.
\begin{proof}
   First, it is easy to show that, if the deletion of a 
   edge doesn't break the spanning tree, then, it is still the spanning tree 
   of the graph with one edge less. 

   $\tau(G-e)$ equal the number of the spanning trees that won't be break 
   by the deletion of edge $e$. For those spanning tree, if we 
   let the original graph change into $G\verb|\|e$, we will get a circle, because
   there is already a path included in the spanning tree that join the 
   vertices joined by $e$.
   For the spanning trees break by the deletion, by getting $G\verb|\|e$,
   we can get a new spanning tree.

   Thus we finished the proof.
    
\end{proof}
\end{problem}
\subsection{Challenge problems}
\begin{problem}
   Show that if $H$ and $K$ are subgraphs of a connected graph $G$, and if $H\cup K$ and $H\cap K$ are defined in the obvious way, then 
   the cutset rank $\zeta$ satisfies:
   \begin{enumerate}
    \item $0\leq\zeta(H)\leq\abs{E(H)}$(the number of edges of $H$)
    \item if $H$ is a subgraph of $K$, then $\zeta(H)\leq\zeta(K)$
    \item $\zeta(H\cup K)+\zeta(H\cap K)\leq\zeta(H)+\zeta(K)$.
   \end{enumerate} 
   \begin{proof}
   (1) Since the cutset rank equal the $n-1$ for a connected graph with n vertices and the subgraph of a connected graph is also connected,
   $0=1-1\leq\zeta(H)\leq\abs{V(H)}-1$. What's more, since $H$ is connected, the minimal number of the edge is $\abs{V(H)}-1$, 
   achieved when $H$ is a tree. Thus we finished the proof of the first inequility. 

   A subgraph of a connected graph is still connected but may have less vertices, thus we have second inequility.

   Obviously, we the sum of the vertices of the graph on each sides are equal. Since $H,K$ are all 
   connected, $H\cap K$ are also connected. Since $H\cup K$ may be disconnected, consist of two component and 
   the formula for cutset tank is $\zeta(G)=n-k$. We got the inequility with $\leq$.
   \end{proof}
\end{problem}
\begin{problem}
    Let $V$ be the vector space associated with a simple connected graph $G$, and let $T$ be a spanning tree of $G$.
    \begin{enumerate}
        \item Show that the fundamental set of cycles associated with $T$ forms a basis for the cycle subspace $W$.
        \item Obtain a corresponding result for the cutset subspace $W^*$.
        \item Deduce that the dimensions of $W$ and $W^*$ are $\gamma(G)$ and $\zeta(G)$, respectively.
    \end{enumerate}
    \index{unsolved problem 3.3.2.}
\end{problem}
\begin{problem}
    Use the matrix-tree theorem to prove Cayley's theorem.
    \begin{proof}
       The matrix-tree theorem tells us about how many spanning tree can be construct in a given graph and 
       Cayley's theorem tells us how many tree can be constrcted, given a certain number of vertices. 
       \index{unsolved problem 3.3.3.}
    \end{proof}
\end{problem}
\begin{problem}
    Let $T(n)$ be the number of labelled trees on $n$ vertices.
    \begin{enumerate}
        \item By counting the number of ways of joining a labelled tree on $k$ vertices and one on $n-k$ vertices, prove that 
        \[2(n-1)T(n)=\sum_{k=1}^{n-1}\binom{n}{k} k(n-k)T(k)T(n-k)\]
        \item Deduce the identity 
        \[\sum_{k=1}^{n-1}\binom{n}{k}k^{k-1}(n-k)^{n-k-1}=2(n-1)n^{n-2}\]
    \end{enumerate}
    \begin{proof}
        If we want to get a tree with $n$ vertices after joining, then, we can get $T(n)$ different trees.

        Given $n$ labelled vertices, to run through all the condition, first, we have to classify all the condition 
        according to the number of the vertices in the each components, one is $k$ and the other one is $n-k$, where $k=1,2,\cdots,n-1$.
        Thus we have the $\sum_{k=1}^{n-1}$ on the RHS. Now, given a certain $k$, we need to run through all the possibility according to 
        the different trees with $k$ vertices an $n-k$ vertices, thus we have $T(k)T(n-k)$ on the RHS. Finally, we 
        have two choose one vertex from each component and join them to form the tree with $n$ vertices. Thus we have $k(n-k)$ on the RHS.
        Since we have $k$ from $1$ to $n-1$, we repeat once. THus we have $2$ on the LHS. What's more, since every tree with $n$ vertices 
        can be constrcted by join one vertex from a component with $k$ vertices and another one from a component with $n-k$ vertices,
        where $k$ keep the same, we actually repeat $(n-1)$ times on the basis of the former repeat, leading to the $(n-1)$ on the LHS. 
        Thus we finished the proof.

       By substituting $T(n)$ with $n^{n-2}$,  we can get the second formula. 
    \end{proof}
\end{problem}
\section{Planarity}
\subsection{Planar graph}
\begin{problem}
    Which complete graphs and complete bipartite garphs are planar?
    \begin{proof}
        Since by contracting one vertex to any other vertex of a complete graph, we can get 
        another complete graph with one less vertex, only $K_n(n\leq 4)$ are planar.

        In the same manner, we can know that $K_{a,b}$ is planar if and only if at least one of them is less than $3$.
    \end{proof}
\end{problem}
\begin{problem}
    \begin{enumerate}
        \item For which values of $k$ is the $k-$cube $Q_k$ planar?
        \item For which values of $r,s$ and $t$ is the complete tripartite graph $K_{r,s,t}$planar?
    \end{enumerate}
    \begin{proof}
        For $k$ less than 5, because if $k\geq 5$, then the cube graph can be concracted into a $K_5$.
        When $k\geq 5$, there are at least 32 vertices with degree at least 5. So we have enough vertices and 
        degrees to allow it to be contracted into $K_5$, which is easy to prove.

        For the second question, obviously, there can be at most one number that is greater than 2.

        Suppose $r\leq s\leq t$. When $r=s=1$, there is no constrain for $t$. When $r=1,s=2$, if $t\geq 3$, then, 
        one of its subgraph is $K_{3,3}$. Thus $t= 2$. When $r=s=2$, due to the former discussion, $t\leq 2$.
        It is easy to construct a planar complete graph $K_{2,2,2}$.
        Overall, only $K_{1,1,n},K_{1,2,2},K_{2,2,2}$ is planar.
    \end{proof}
\end{problem}
\begin{problem}
   If two homeomorphic graphs have $n_i$ vertices and $m_i$ edges $(i=1,2)$, show that 
   \[m_1-n_1=m_2-n_2\]
   \begin{proof}
    Suppose the original graph $G$ has $n_0$ vertices and $m_0$ edges. 
    When ever we add a vertex on its edge, we got a graph with an extra vertex and an extra edge, thus the 
    difference between the number of the vertices and the number of the edges stay same.
   \end{proof}
\end{problem}
\begin{problem}
    A graph is outerplanar if $G$ can be drawn in the plane so that all of its vertices lie on the exterior boundary.
    \begin{enumerate}
        \item Show that $K_4$ and $K_{2,3}$ are not outerplanar.
        \item Deduce that, if $G$ is an outerplanar graph, then $G$ contains no subgraph homeomorphic or contractible to $K_4$ or $K_{2,3}$.
    \end{enumerate}
    \begin{proof}
        The first question can be answered by experienment.

        For the second question, suppose $G$ is an outerplanar graph, and contains a subgraph that is homeomorphic or contractible to those two 
        graph. It is obviously that, a subgraph of a outerplanar graph is still an outerplanar graph. What's more, during the process of 
        homeomorphic and contractible, an outerplanar graph won't become a non-outerplanar graph. The rest of the proof is then obvious.
    \end{proof}
\end{problem}
\begin{problem}
    By placing the vertices at the point $(1,1^2,1^3),(2,2^2,2^3),\cdots$prove that any simple graph can be drawn without crossings in 
    Euclidean three-dimensional space so that each edge is represented by a straight line.
    \index{unsolved problem 4.1.5.}
\end{problem}
\subsection{Euler's formula}
\begin{problem}
    \begin{enumerate}
        \item Use Euler's formula to prove that, if $G$ is a connected planar graph of girth 5 with $n$ vertices and $m$ edges, then $m\leq \frac{5}{3}(n-2).$Deduce that the Petersen graph is non-planar.
        \item Obtain an inequility, generalizing that in part (1), for connected planar graphs of girth $r$.
    \end{enumerate}
    \begin{proof}
      We shall give the genral version directly.
      Since the girth of the planar graph is $r$, every face of that graph is at least bounded by $r$ edges. It follows that 
      \[2m\geq rf,\]  
      where the factor 2 is because of the fact that each edge is adjacent to 2 faces at the same time.

      Since 
      \[n-m+f=2,\]
      we have 
      \[ 2m\geq r(2+m-n)\]
      or 
      \[ m\leq \frac{r}{r-2}(n-2)\]
      When $r=5$, we get the inequility in $(1)$. Petersen graph has $10$ vertices and $15$ edges, which does not satisfy the inequility when $r=5$ 
    \end{proof}
\end{problem}
\begin{problem}
    Let $G$ be a polyhedron (or polyhedral graph), each of whose faces is bounded by a pentagon or hexagon.
    \begin{enumerate}
        \item Use Euler's formula to show that $G$ must have at least 12 pentagonal faces.
        \item Prove, in addition, that if $G$ is such a polyhedron with exactly three faces meeting at each vertex(such as a football), then $G$ has exactly 12 pentagonal faces.
    \end{enumerate}
    \begin{proof}
        \index{unsolved problem 4.2.2.}
    \end{proof}
\end{problem}
\begin{problem}
    Let $G$ be a siple plane graph with fewer than 12 faces, in which each vertex has degree at least 3.
    Use Euler's formula to show that $G$ has a face bounded by at most four edges.
\begin{proof}
    If the graph satisfying all the condition have no face bounded by less than 4 edges, then
    \[
    \begin{cases}
        n-m+f=2\\
        f<12\\
        3n\leq 2m\\
        5f\leq 2m\\
    \end{cases}
    \]
    We get 
    \[15n+15f=30+15m\leq 10m+6m\]
    or 
    \[30\leq m.\]
    However
    \[f=2+m-n\geq 2+m=\frac{2}{3}m=2+\frac{m}{3}=12\]
    which is contrary to the second inequility.
\end{proof}
\end{problem}
\begin{problem}
    Let $G$ be a simple connected cubic graph, and let $C_k$ be the number of $k-$sided faces.
    By counting the number of vertices and edges of $G$, prove that
    \[3C_3+2C_4+C_5-C_7-2C_8-3C_9-\cdots=12 \]
    \begin{proof}
        \[
    \begin{cases}
        3n=\sum_{k=3}^{\infty}kC_k\\
        n-m+f=2\\
        2m=\sum_{k=3}^{\infty}kC_k\\
        f=\sum_{k=3}^{\infty}C_k\\
    \end{cases}
    \]
    The first equation is about the relationship bewteen the number of vertices and the number of edges in cubic graph.
    The second equation is the Euler's formula.
    The third equation is about the number of the edges.
    The fourth equation is about the number of the faces.

    By combining these there equations, we have 
\[6\sum_{k=3}^{\infty}C_k-\sum_{k=3}^{\infty}kC_k=12.\]
    \end{proof}
\end{problem}
\begin{problem}
    Let $G$ be a simple graph with at least 11 vertices, and let $\bar{G}$ be its complement.
    Prove that $G$ and $\bar{G}$ cannot be both planar.
    \begin{proof}
        If both graph are planar, then
        \[
        \begin{cases}
            m+\bar{m}=\frac{n(n-1)}{2}\\
            n=\bar{n}\geq 11\\
            m\leq 3n-6\\
            \bar{m}\leq 3\bar{n}-6\\
        \end{cases}
        \]
        thus we have 
        \[\frac{2}{n(n-1)}\leq 6n-12\]
        or 
        \[n^2-13n+24\leq 0\]
        which is contrary to $n\geq 11$.
    \end{proof}
\end{problem}
\subsection{Dual graph}
\begin{problem}
    Use the duality to prove that there exists no plane graph with five faces, each pair of which shares  an edge in common.
    \begin{proof}
        If $G$ is a graph satisfying all the conditions, then, its dual graph is $K_5$, whose dual graph doesn't exists, which 
        is contrary to the fact that the dual graph of a dual graph is its original graph.
    \end{proof}
\end{problem}
\begin{problem}
    Prove that if $G$ is a disconnected plane graph, then $G^{**}$ is not isomorphic to $G$.
    \begin{proof}
        It is obvious that if $G$ is a disconnected plane graph, then, each of its components is a connected plane graph.
        Since the dual of a connected plane graph is a connected graph, because each pair of faces of the original graph is separated by at least one edge,
        and the vertex corresponding to the infinite graph of the original graph is obviously adjacent to at least one vertex of each 
        dual graph of the original components. Thus, $G^*$ is connected. So is $G^{**}$. As a result, $G$ and $G^{**}$ are not 
        isomorphic to each other.
    \end{proof}
\end{problem}
\begin{problem}
    Dualize the result of Exercises 4.2.2.
    \begin{proof}
        \index{unsolved problem 4.3.3.}
    \end{proof}
\end{problem}
\begin{problem}
    Prove that, if $G$ is a $3-connected $ plane graph, then its geometric dual is a simple graph.
    \begin{proof}
        If its geometric dual is not a simple graph, then there are loops or multiple edges.
        Since the original graph is also the dual graph of its dual graph, there should be a vertex with 2 degrees or 3 degrees,
        which is a contradiction.
    \end{proof}
\end{problem}
\begin{problem}
    Let $G$ be a connected plane graph, prove that $G$ is bipartite if and only if its dual $G^*$ is Eulerian.
    \begin{proof}
        If $G$ is a bipartite, then, every cycle of $G$ has even length. In other words, every face of $G$ is bounded edges of even number.
        Thus, every vertex in $G^*$ is of even degrees.

        If $G^*$ is Eulerian, then, it is can be split into separate cycles.
        Then, it is planar and thus have a dual graph. If we color white the vertex, in $G$, corresponding to the infinite face of $G^*$.
        Since the circle form a chain with inclusion relation, we can color the vertex corresponding to the face of the biggest circle,
        that is not overlapped by other samller cycle ,in each chain black, the second white and so 
        on. Thus we get a bipartite $G$.
    \end{proof}
    \begin{remark}
        In the above proof, we regard $G^*$ as geometric dual. It should be considered as abstract dual, because every geometric dual is a
        abstract dual, but not every abstract dual is a geometric dual.
    \end{remark}
\end{problem}
\begin{problem}
    Prove that if $G$ is a connected plane graph, then any spanning tree in $G$ correspond to the complement of a spanning tree in $G^*$.
    \begin{proof}
    Since every cycle in $G$ is corresponding to a cutset in $G^*$. Then, the number of the edges in the complement of the spanning tree in $G$
    equal to the one of the spanning tree in $G^*$. If we switch $G$ with $G^*$, the statement is still valid. 
    \index{unsolved problem 4.3.6.}
    \end{proof}
\end{problem}
\subsection{Graphs on other surfaces}
\begin{problem}
    Show that there is no graph of genus $g\geq 1$ such that is regular of degree 4 and in which each face is triangle. 
    \begin{proof}
        If there is such graph that is has genus greater than or equal 1 and satisfies other conditions, then 
        \[
        \begin{cases}
            n-m+f=2-2g\\
            g\geq 1\\
            2m=4n\\
            3f=2m\\
        \end{cases}
        \]
        It follows that 
        \[12n-12m+12f=24-24g=6m-12m+8m=2m\]
        or 
        \[12g=12-m< 12,\]
        which is contrary to the second inequility. 
    \end{proof}
\end{problem}
\begin{problem}
    Obtain a lower bound, analogous to that of Corollary 4.6. for a graph containing no triangles.
    \begin{proof}
        If in $G$, there is no triangles, then 
        \[\begin{cases}
           n-m+f=2-2g\\
           2m\geq 4f\\ 
        \end{cases}\]
        What's more, $g$ is an integer. Thus we have 
        \[\left\lceil 1+\frac{1}{4}(m-2n)\right\rceil \leq g.\]
    \end{proof}
\end{problem}
\begin{problem}
    Deduce that $g(K_{r,s})\geq\left\lceil \frac{1}{4}(r-2)(s-2)\right\rceil $.
    \begin{proof}
       We can deduce it directly from the result of the last question, since $n=r+s,m=rs$. 
    \end{proof}
\end{problem}
\subsection{Challenge problem}
\begin{problem}
Let $G$ be a planar graph with vertex-set $\{v_1,v_2,\cdots,v_n\}$, and let $p_1,p_2,\cdots,p_n$ be any $n$ distinct points in the plane.
Given a heuristic argument to show that $G$ can be drawn in the plane in such a way that the point $p_i$ represents the vertex $v_i$, for each $i$.
\begin{proof}
\index{unsolved problem 4.5.1.}    
\end{proof}
\end{problem}
\begin{problem}
    If $r$ and $s$ are both even, prove that 
    \[\text{cr}(K_{r,s})\leq\frac{1}{16}rs(r-2)(s-2),\]
    and obtain corresponding results when $r$ and/or $s$ is odd.
    \begin{proof}
        Place the $r$ vertices along the $x-$axis with $\frac{1}{2}r$ vertices on each side of the origin,
        and the $s$ vertices along the $y-$axis in a similar way; then join up the vertices by straightline segments and count the crossing.
        Then we have 
        \[\begin{array}{rl}
            \text{cr}(G)&=4\cdot[(\frac{r}{2}-1)(\frac{s}{2}-2)+(\frac{r}{2}-2)(\frac{s}{2}-3)+\cdots+2\cdot 1]\\
                        &=
        \end{array}
    \]        
    \index{unsolved problem 4.5.2.}
    \end{proof}
\end{problem}
\begin{problem}
    Prove that 
    \[t(K_{r,s})\geq\left\lceil \frac{rs}{2r+2s-4}\right\rceil \]
    \begin{proof}
        Since there is no triangles in bipartite, so is the each layer of it.
        It follows that, for each layer of the graph satisfies
        \[m_i\leq 2r+2s-4\]
        By summing up all $t(K_{r,s})$ inequility, we have 
        \[\sum_{i=1}^{t(K_{r,s})}m_i=rs\leq (2r+2s-4)t(K_{r,s})\]
        or 
        \[ t(K_{r,s})\geq \left\lceil \frac{rs}{2r+2s-4}\right\rceil \]
    \end{proof}
\end{problem}
\begin{problem}
    By splitting $K_{r,s}$ into a number of copies of $K_{2,s}$, prove that, if $r$ is even, then $t(K_{r,s})\leq r$, 
    and deduce from the last question that  
    \[t(K_{r,s})=\frac{1}{2}r \text{ if } s>\frac{1}{2}(r-2)^2\]
    \begin{proof}
        Obviously,
        \[t(K_{r,s})\leq \frac{r}{2}\]
        When $s>\frac{1}{2}(r-2)^2$
        \[\left\lceil \frac{rs}{2r+2s-4}\right\rceil> \left\lceil \frac{\frac{1}{2}r(r-2)^2}{2r+(r-2)^2-4}\right\rceil=\frac{r}{2}-1   \]
        Thus 
        \[\frac{r}{2}\geq t(K_{r,s})\geq \left\lceil \frac{rs}{2r+2s-4}\right\rceil\geq \frac{r}{2} \]
    \end{proof}
\end{problem}
\begin{problem}
    Let $G$ be a polyhedral graph and let $W$ be the cycle subspace of $G$.
    \begin{enumerate}
        \item Show that the polygons bounding the finite faces of $G$ form a basis for $W$.
        \item Deduce Corollary 4.1.
    \end{enumerate}
    \index{unsolved problem 4.5.5.}
\end{problem}
\begin{problem}
    A graph $G^*$ is a Whitney dual of $G$ is there is a one-one correspondence between $E(G)$ and $E(G^*)$ such that for each subgraph $H$
    or $G$ with $V(H)=V(G)$, the corresponding subgraph $H^*$ of $G^*$ satisfies 
    \[\gamma(H)+\zeta(\bar{H}^*)-\zeta(G^*)\]
    where $\bar{H}^*$ is obtained from $G^*$ by deleting the edges of $H^*$.
    \begin{enumerate}
        \item Show that this generalizes the idea of a geometric dual.
        \item Prove that, if $G^*$ is a Whitney dual, then $G$ is a Whitney dual of $G^*$
    \end{enumerate} 
    \index{unsolved problem 4.5.6.}
\end{problem}
\begin{problem}
    \begin{enumerate}
        \item Let $G$ be a non-planar graph that can be drawn without crossings on a Mobius strip. Prove that, with the usual notation, $n-m+f=1$
        \item Show that $K_5$ and $K_{3,3}$ can be drawn without crossings on the surface of a Mobius strip.
    \end{enumerate}
    \index{unsolved problem 4.5.7.}
\end{problem}
\section{Colouring graphs}
\subsection{Colouring vertices}
\begin{problem}
    What is the chromatic number of the complete tripartite graph $K_{r,s,t}?$
    \begin{proof}
        Is 3.
    \end{proof}
\end{problem}
\begin{problem}
    What is the chromatic number of the $k-$cube $Q_k$?
    \begin{proof}
        The chromatic number of the cube is $2$.
        When $k=1,2,3$, it is obvious, because there are two vertices in that graph.
        We are now going to finish the proof by induction.
        Suppose when $n=k$, $Q_k$'s chromatic number is 2.
        Since we have already prove that there is a Hamiltonian cycle in $Q_k$, 
        we can denote $Q_k$ by 
        \[X_0\rightarrow X_1\rightarrow\cdots\rightarrow X_0.\]
        $Q_{k+1}$ include two $Q_k$, 
        and the edge joining $\overline{X_i0}$ and $\overline{X_i1}$.
        Since every vertices in $Q_{k+1}$ is of degree $k+1$, there is no other 
        edges connecting the two $Q_k$.

        By induction hypothesis, we can only use two colours to colour the first $Q_{k}$, say white and black.
        Then, we can colour the second $Q_k$ with the same pattern but in the opposite colour.

        It is obvious that we colour the $Q_{k+1}$ with only two colours correctly.
    \end{proof}
\end{problem}
\begin{problem}
    Let $G$ be a simple graph with $n$ vertices, which is regular of $d$. By considering the number of vertices that can be 
    assigned the same colour, prove that $\chi(G)\geq \frac{n}{n-d}$
\end{problem}
\begin{proof}
   Since there are only $n$ vertices in the graph and each vertex is regular of $d$, the number of vertices that can be assigned the same 
   colour won't be greater than $n-d$, otherwise there must be two adjacent vertices in the same colour.
   Thus, then number of the colours in that graph is greater than or equal $\frac{n}{n-d}$
\end{proof}
\begin{problem}
    Let $G$ be a simple planar graph containing no triangles.
    \begin{enumerate}
        \item Using Euler's formula, show that $G$ contains a vertex of degree at most 3.
        \item Use induction to deduce that $G$ is $4-$colourable.
    \end{enumerate}
    \begin{proof}
        If $G$ is a simple planar graph containing no triangles and no vertices of degree less than 4, 
        then 
    \[\begin{cases}
        n-m+f=2\\
        2m\geq 4f\\
        2m\geq 4n\\
    \end{cases}\]
    It follows that 
    \[4m\geq 4n+4f=8+4m,\]
    which is absurd.

    We will prove that $G$ is $4-$colourable by the induction on the number of the vertices in $G$.
    When there are less than 5 vertices in $G$, the statement is trival.
    When there are $k$ vertices in $G$, there is a vertex $v$ whose degree is less than 4.
    By deleting vertex $v$, we get a graph with $k-1$ vertices. By the induction hypothesis, that graph 
    can be coloured by 4 vertices. Since $v$'s degree is less than 4, after adding that vertex into the graph,  
    we can always colour it properly.
    \end{proof}
\end{problem}
\subsection{Chromatic polynomials}
\begin{problem}
    \begin{enumerate}
        \item Prove that the chromatic polynomial of $K_{2,s}$ is 
        \[k(k-1)^s+k(k-1)(k-2)^s\]
        \item Prove that the chromatic polynomial of $C_n$ is $(k-1)^n+(-1)^n(k-1)$
    \end{enumerate}
    \begin{proof}
       Two of the vertices in $K_{2,s}$ can be coloured white, and the other coloured black.
       If we have other $k$ colours, we can first colour one of the white vertex. Thus we have $k$
       choice. If another white vertex is coloured the same, then, each of the black vertex have $k-1$ choices.
       Otherwise, another white vertex have $k-1$ choice and each of the black vertex have $k-2$ choices.
       Thus, the total choice we have is 
       \[k(k-1)^s+k(k-1)(k-2)^s\]

       For the second equation, we are going to prove by induction on the edge of the cycle graph.
       When $n=1,2$, the equation is obviously valid.
       When $n=i$,
       \[\begin{array}{rl}
        P_{C_i}(k)&=P_{C_i-e}(k)-P_{C_i/e}(k)\\
        &=k(k-1)^{i-1}-P_{C_{i-1}}\\
        &=k(k-1)^{i-1}-(k-1)^{i-1}-(-1)^{i-1}(k-1)\\
        &=(k-1)^i+(-1)^i(k-1)\\
       \end{array}
       \]
       Thus we finish the proof.
    \end{proof}
\end{problem}
\begin{problem}
    Prove that, if $G$ is a disconnected simple graph, then its chromatic polynomial $P_G(k)$ is the product of the chromatic polynomials 
    of its components. What can you say about the degree of the lowest non-vanishing term?
\begin{proof}
    Since the colouring of each components won't interfere each other, given certain kinds of colour, the 
 number of the colouring plan is exactly the product of each component's number of colouring plan.
 Thus we finished the proof. Since each component's chromatic polynomial's constant term is 0, and it is not a 
 zero polynomial, thus, the degree of the lowest non-vanishing term is greater than the number of the components.
\end{proof}
\end{problem}
\begin{problem}
    Let $G$ be a simple graph with $n$ vertices and $m$ edges. Use induction on $m$, together with Theorem 5.6. to prove that 
    \begin{enumerate}
        \item the coefficient of $k^{n-1}$ is $-m$
        \item the coefficients of $P_G(k)$ alternate in sign.
    \end{enumerate}
    \begin{proof}
       When $m=0$, the chromatic polynomial is $k^n$, when $m=1$, the chromatic polynomial is $k^{n-1}(k-1)$, which all satisfies the two 
       statements. Suppose when there are $i$ edges the statements are valid.
       Then, when we have a simple graph $G$ with $n+1$ vertices, since $P_G=P_{G-e}-P_{G/e}$, where $e$ is an edge in $G$ and thus 
       the two chromatic polynomial are belong to simple graph with $i$ edges. What's more, the highest degree of each chromatic polynomial is 
       $n$ and $n-1$ respectively. Since the $k^{n-1}$'s coefficients is $m-1$ and $1$ respectively, we have proved the first statement.
       Since the $n$th, $n-2$th,$\cdots$ terms of the first chromatic polynomial have positive coefficients, by induction hypothesis and 
       the fact that the highest term's coefficient is 1, and the other coefficient is negative. Meanwhile, in the second chromatic polynomial,
       the sign is inversed, which means that the first chromatic polynomial has the same sign as $-P_{G/e}$. Thus, the second statement is true.
       \begin{remark}
        It seems that when proving the second statement, we still need the inducion hypothesis on the number of vertices.
       \end{remark}
    \end{proof}
\end{problem}
\begin{problem}
    Prove that if 
    \[P_G(k)=k(k-1)^n,\]
    then $G$ is a tree on $n$ vertices.
    \begin{proof}
        If there are several components, when the degree of the lowest term is greater than 1, which is contrary to the condition.
        Thus, $G$ is a connected graph.

    The second coefficient is $n-1$, thus $G$ has $n-1$ edges and is a tree.
    \end{proof}
\end{problem}
\subsection{Colouring graph}
\begin{problem}
    The plane is divided into a finite number of regions by drawing infinite straight lines in an arbitrary manner. Show that 
    these regions can be 2-coloured.
\begin{proof}
We are going to prove this by induction on the number of line.

Since we place the lines in an arbitrary way, we can regard every vertices in the plane as the 
crossing of two line.
Therefore, whenever we place a new line in the plane, it will lie on a sequence of face the first and the last of which is infinite 
region. Thus, the original graph is separate into to part by the added line. If we inverse the colour of the regions in the one side,
then we can get a new 2-coloured plane.
\end{proof}
\end{problem}
\subsection{Colouring edges}
\begin{problem}
    Prove that if $G$ is a cubic Hamiltonian graph, then $\chi'(G)=3$.
    \begin{proof}
        By Handshaking theorem, there must be even number of vertices in $G$. Thus 
        we can 2-colour the Hamiltonian cycle in it and colour the other edges 
        with the third colour, which is valid since the vertices is of regular 3.
    \end{proof}
\end{problem}
\subsection{Challenge problems}
\begin{problem}
    A graph is $k-$critical if $\chi(G)=k$ and if the deletion of any vertex yields a graph with 
    smaller chromatic number. 
    Prove that if $G$ is $k-$critical, then 
    \begin{enumerate}
        \item every vertex of $G$ has degree at least $k-1$
        \item $G$ has no cut-vertices.
    \end{enumerate}
\end{problem}
\begin{proof}
    If there is a vertex, say $v$, with degree less than $k-1$. Then after the deletion of $v$, we could $k-1-$colour the 
    remaining graph. However, if we reinstate $v$, since $v$ is with degree less than $k-1$, among the $k-1$ colour, there 
    is at least one colour left for $v$. Thus the original graph can actully $k-1-$coloured, which is contrary to $\chi(G)=k$.

\index{unsolved problem 5.5.1.}
\end{proof}
\begin{problem}
    Give a upperbound for the least degree of vertices in $G$ and the number of colours needed to colour it,
    if 
    \begin{enumerate}
        \item $G$ has girth $r$
        \item $G$ has thickness $t$
    \end{enumerate}
    \begin{proof}
       Let $\Delta$ be the least degree of vertices in $G$, then 
       \[\begin{cases}
        \Delta n\leq 2m\\
        rf\leq 2m\\
        n-m+f=2\\
       \end{cases}\] 
       It follows that 
       \[\Delta\leq\frac{n-2}{n}\cdot\frac{r}{r-2}\cdot 2<\frac{2r}{r-2}.\] 

       We are now going to prove that $G$ is $\Delta+1-$colourable by induction.
       Say $v$ is the vertex in $G$ that is of degree less than or equal $\Delta$.
       After the deletion of $v$, we can $\Delta+1-$colour the remainging graph by induction hypothesis.
       However, since $v$ is less then $\Delta +1$ degree, when we reinstate $v$ in the graph, we can always colour it properly. 

       For the second condition, 
       \[\begin{cases}
        t\geq \left\lceil m/3n-6\right\rceil \geq m/3n-6\\
        \Delta n\leq 2m\\
        n-m+f=2\\
       \end{cases}\]
       It follows that 
       \[2t(3-\frac{6}{n})\geq\Delta\]
    \end{proof}
\end{problem}
\begin{problem}
    Let $G$ be a countable graph, each finite subgraph of which is $k-$colourable.
    \begin{enumerate}
        \item Prove that $G$ is $k-$colourable.
        \item Deduce that every countable planar graph is $4-$colourable.
    \end{enumerate}
    \begin{proof}
        We begin our proof by choosing a vertex from $G$, say $v$, and fix the colour of it in every coloured graph containing it.
        (We can alway achieve this by switch colours.) Then, we are going to use tree to represent the relation between each graph whose 
        center is $v$. One of the end-vertex of the tree is $v$, regarded as the zero level. In the $n$th level,  
        the vertices present the graph consists of all the vertices that can be join with $v$ by a path whose length is less than or equal 
        $n$, and the edges incident with them. What's more, each graph is corresponding to a unique $k-$colourable colouring, 
        which is assure by the condition.
        There is no edges join the two vertices in the same level of tree, and the vertex is adjacent to the vertex in the next level 
        if it is the subgraph of it.
        It is obviously that the tree is a countable graph and there is a path whose end-vertex is $v$ and extend to infinite level.
        The existence of the path show that we can the whole graph $G$ is $k-$colourable.
        
        The second follows immediately from the fact that every finite graph is $4-$colourable.
    \end{proof}
\end{problem}
\begin{problem}
    \begin{enumerate}
        \item Let $G$ be a simple graph which is not a null graph. Prove that $\chi'(G)=\chi(L(G))$.
        \item Prove theorem 5.12. in the case $\Delta=3.$
    \end{enumerate}
    \begin{proof}
       If $\chi'(G)=k$, then, the edge of $G$ can be $k-$coloured, but cannot $(k-1)-$coloured.
       Thus, we can colour the vertices in $L(G)$ the same colour as their corresponding edges in $G$, and get a $k-$coloured 
       $L(G)$, since the vertices in $L(G)$ are adjacent if and only if their corresponding edges in $G$ are also adjacent, which 
       follows that they don't have the same colour. The converse is also true, which means that $L(G)$ cannot be $(k-1)-$coloured.
       Thus we finish the proof.

       We are now going to prove that if $G$ is a simple graph with largest vertex-degree 3, then 
       \[3\leq\chi'(G) \leq 4\] 
       By the last statement we have proved, we only need to prove that 
       \[3\leq\chi(L(G))\leq 4.\]
       Since $G$ is a simple graph with largest vertex-degree 3, by theorem 5.2.
\[\chi(L(G))\leq \Delta L(G)\leq 2+2=4\]
What's more, suppose $v$ is a vertex in $G$ that is of degree 3. Then, $v$ and the edges incident with it correspond to 
a triangle in $L(G)$, thus 
\[\chi(L(G))\geq 3.\]
    \end{proof}
\end{problem}
\begin{problem}
    Prove that, if a toroidal graph is embeded on the surface of a torus, then its faces can be coloured with seven colours.
    \index{unsolved problem 5.5.5.}
\end{problem}
\begin{problem}
    Let $G$ be a simple graph with an odd number of vertices. Prove that if $G$ is regular of degree $\Delta$, then $\chi'(G)=\Delta+1$.
\end{problem}
\section{Matching, marriage and Menger's theorem}
\subsection{Hall's `marriage' theorem}
Let $B$ be a set of boys, and suppose that each boy in $B$ wishes to marry more than one of his girl friends. Find a necessary and sufficient 
condition for the harem problem to have a solution.
\begin{proof}
    If a boy want to marry $k$ gir friends, then, we can replace him with $k$ person who have the same set of girl friends.
\end{proof}
\begin{problem}
    Let $E$ be the set $\{1,2,\cdots,50\}$. How many different transverals has the family $\{\{1,2\},\{2,3\},\{3,4\},\cdots,\{50,1\}\}$?
\begin{proof}
   If we pick 1 from $\{1,2\}$, then we can only pick 50 from $\{50,1\}$, and then 49 from $\{49,50\}$ and so on. 
   If we pick $2$ from $\{1,2\}$, then we can only pick $3$ from $\{2,3\}$, $4$ from $\{3,4\}$ and so on.
   Therefore, we only have 2 possible transveral.
\end{proof}
\end{problem}
\begin{problem}
    Write the statement of Corollary 6.1. in marriage terminology.
    \begin{proof}
       If there are m girls, each of whom knows several boys, then a necessary and sufficient condition for 
       only $t$ girls can find proper boys who are different from each other is that any $k$ girls collectively know 
       at least $k+t-m$ boys.
    \end{proof}
\end{problem}
\subsection{Network flows}
\begin{problem}
    Consider a network with several sources and sinks. Show how the analysis of the flows in this network can be reduced to the 
    standard case by the addition of a new`source vertex' and `sink vertex.'
\end{problem}
\begin{proof}
    By adding a new source vertex as the source of all the original source and a new sink vertex as the sink of all the original sink.
\end{proof}
\subsection{Challenge problems}
\begin{problem}
    \begin{enumerate}
        \item Use the marriage condition to show that if each girl has $r(\geq 1)$ boy friends and each boy has $r$ girl friends, then 
        the marriage problem has a solution.
        \item Use the result of (1) to prove that, if $G$ is bipartite graph which is regular of degree $r$, then $G$ has a complete matching.
        Deduce that the chromatic index of $G$ is $r$.
    \end{enumerate}
    \begin{proof}
       For $k\leq m$, if any $k$ girls only collectively know $s(\leq k-1)$ boys. It follows that 
       \[r\cdot s=k\cdot r\] 
       or 
       \[s=k\]
       which is absurd.
       Therefore, for any $k$ girls, they must collectively know at least $k$ boys. By theorem 6.1., we have a solution.

       We can colour boys white and girls black, then the solution is the desired complete mathching.

       We only need to provide a $r-$colours drawing.
       We can index the white vertices with $a_1,a_2,\cdots,a_n$ and black vertices with $b_1,b_2,\cdots,b_n$.
       Then, we join $a_i$ with $b_i,b_{i+1},b_{i+2},b_{i+3},\cdots,b_{i+r}$ and colour the edges with colour $1,2,\cdot,r$
       respectively.
    \end{proof}
\end{problem}
\begin{problem}
    Suppose that the marriage condition is satisfied, and that each of the $m$ girls knows at 
    least $t$ boys. Show that the marriages can be arranged in at least $t!$ ways if $t\leq m$, and 
    in at least $\frac{t!}{(t-m)!}$ ways if $t>m$. 
    \begin{proof}
       When $m=1$, there is only $1$ girl and when she know $t$ boys, there is exactly $t=\frac{t!}{(t-1)!}$ choice.
       Suppose the statement is true when $m$ is less than $k$. Then, when $m=k$, 
       \index{unsolved problem 6.3.2.}
    \end{proof}
\end{problem}
\begin{problem}
    Let $E$ and $\mathcal{F}$ have their usual meanings, let $T_1$ and $T_2$ be transverals of $\mathcal{F}$, 
    and let $x$ be an element of $T_1$. Show that there exists an element $y$ of $T_2$ such that $(T_1-\{x\})\cup\{y\}$
    is also a transveral of $\mathcal{F}$.
    \begin{proof}
       \index{unsolved problem 6.3.3.} 
    \end{proof}
\end{problem}
\begin{problem}
    Let $\mathcal{F}$ be a family consisting of $m$ non-empty subsets of $E$, and let $A$ be a subset of $E$.
    By applying theorem 6.1. to the family consisting of $\mathcal{F}$, together with $\abs{E}-m$ copies of $E-A$,
    prove that there is a transveral of $\mathcal{F}$ containing $A$ if and only if 
    \begin{enumerate}
        \item $\mathcal{F}$ has a transveral
        \item $A$ is a partial transveral of $\mathcal{F}$
    \end{enumerate}
    \index{unsolved problem 6.3.4.}
\end{problem}
\begin{problem}
    The rank $r(A)$ of a subset $A$ of $E$ is the number of elements in the largest partial transveral of $\mathcal{F}$ contained in $A$.
    Show that 
    \begin{enumerate}
        \item $0\leq r(A)\leq\abs{A}$
        \item if $A\subseteq B\subseteq E$, then $r(A)\leq r(B)$
        \item if $A,B\subseteq E$, then $r(A\cup B)+r(A\cap B)\leq r(A)+r(B)$.
    \end{enumerate}
    \begin{proof}
            Since a partial transveral is a non-empty set and it is contained in $A$, the first inequility is obvious.

            Whenever a partial transveral is in $A$, it is also in $B$, thus the second inequility is also obvious.

            \index{unsolved problem 6.3.5.}
    \end{proof}
\end{problem}
\begin{problem}
    Let $E$ be a countable set, and let $\mathcal{F}=(S_1,S_2,\cdots)$ be a countable family of non-empty finite subsets of $E$
    Defining a transveral of $\mathcal{F}$ in the natural way, show, by theorem 2.7. that $\mathcal{F}$ has a transveral
    if and only if the union of any $k$ subset $S_i$ contains at least $k$ elements, for all finite $k$.
    \begin{proof}
       ($\rightarrow$) is obvious.
       
       ($\leftarrow$) we can construct a tree, in which every vertex stand for a transveral of finite family. 
       According to theorem 2.7., such tree reaches to infinity.
    \end{proof}
\end{problem}
\begin{problem}
    Prove theorem 6.4.
\index{unsolved problem 6.3.6.}
\end{problem}
\begin{problem}
    Show how the theorem 6.7.can be used to prove theorem 6.1.
    \index{unsolved problem 6.3.7.}
\end{problem}
\end{document}