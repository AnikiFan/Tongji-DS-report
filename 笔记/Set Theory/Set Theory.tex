\documentclass[a4paper,11pt]{article}%必须以此为开头,可以在[]内设置栏数,单双面,横竖向
\usepackage{latexsym}%符号字体
\usepackage{makeidx}%制作索引
\makeindex
\usepackage{ifthen}%提供分支语句
\usepackage{graphicx}%用于插入图片
\usepackage{amsmath}%用于数学公式
\usepackage{IEEEtrantools}%用于使用IEEE数学公式排版工具
\usepackage{amsfonts}%用于其他字体的数学符号
\usepackage{amsthm}%提供证明,定理等环境
\usepackage{amssymb}%用于提供各种数学符号
\usepackage{mathrsfs}%用于提供花体字母
\usepackage{verbatim}%使用\verbatiminput{filename}来直接导入文件中的文本内容
\usepackage{layouts}%用于设置页面布局
\usepackage{calc}%允许一些常量参量用算术表达式代替
\usepackage{indentfirst}
\usepackage{hyperref}
\usepackage{makecell}%允许表格的单元格内换行
\usepackage{bm}%使用bm来对希腊字母加粗
\usepackage{longtable}
\usepackage{slashed}%支持字母上加斜线
\theoremstyle{remark}
\newtheorem*{remark}{remark}
\theoremstyle{definition}
\newtheorem{theorem}{theorem}[section]
\theoremstyle{definition}
\newtheorem*{definition}{definition}
\theoremstyle{plain}
\newtheorem*{property}{性质}
\newcommand*{\abs}[1]{\lvert #1 \rvert}
\theoremstyle{definition}
\newtheorem{axiom}{Axiom}
\author{Fan}
\title{Set Theory}
\date{2023.7.21}
\begin{document}
\maketitle
\tableofcontents
\begin{abstract}
    This is a reading note for Robert's \emph{Set Theory}. It is my first 
    reading note written in English, so you may see many grammar mistakes. 
    This note consists of some important point that I am not familiar with 
    or considered worthing remembered and my answer to the question following
    each lesson, whose correctness are not guranteened, but I will point out
    where can be refined.
\end{abstract}
\pagestyle{plain}%页面风格,plain为中下方有页码.heading是页眉中间有页数,同时有章节名,empty是空页眉页尾
%\thispagestyle{pagestyle}%本页页面风格
\begin{longtable}{cc}
        \caption{various notation} \\
       \multicolumn{1}{c}{notations}&\multicolumn{1}{c}{meanings}\\
       \hline
       \endfirsthead
       \multicolumn{1}{c}{notations}&\multicolumn{1}{c}{meanings}\\
       \hline\endhead
 $\sim$& \emph{not, or it is not the case that}\\
$\lor$&  \emph{or}\\
$\land $&  \emph{and}\\
$A\rightarrow B$&  \emph{if} A \emph{then } B\\
$A\leftrightarrow B$ &  A \emph{if and only if } B (Also:`iff')\\
$\forall xA$ &  \emph{for all x such that } A\\
$\exists xA$ &  \emph{there exists an x such that } A\\
$\exists!xA$ &  \emph{there exists exactly one x such that }A\\
$\iota xA$& \emph{the unique x such that }A\\
$x=y$&  \emph{x equals y, or x and y are identical}\\
$A)(B$& $A$ and $B$ are disjointed\\
   $ A \cup B$ & $  \{x:x\in A \text{or} x \in B\}$\\
   $ A \cap B $& $ \{x:x\in A \text{and} x \in B\}$\\
    $A - B $&  $ \{x:x\in A \text{and} x \notin B\}$\\
   $ A \circleddash  B$ &$   (A-B)\cup(B-A)$\\
    $\varnothing $&$   \{x:x \neq x\}$\\
    $A\times B$& $\{(a,b):a\in A \text{and} b\in B\}$\\
    $f\upharpoonright C$& the restriction of $f$ to $C$\\
    $\Breve{R}$& $R$ converse\\
    $R/S$&relative product\\
    $R\circ S$&the composition of R and S\\
    $R[U]$&the R-image of $U$\\
    $Id_A$&the identity function for $A$\\
    $A\underset{f}{\sim} B$&$f$ is a one-to-one function on $A$ onto $B$\\
    $A \sim B$& $A$ and $B$ are equivalent\\
    $B^C$& the set of all functions on $C$ to $B$\\
    $\prod_{i\in I}\mathfrak{a}_i$&Cartesian product\\
    $P(A)$&power set of $A$\\ 
    $c_B$&characteristic function of B\\
    $\underbar{A}\underset{f}{\cong }\underbar{A}'$&an isomorphism between $\underbar{A}$ and $\underbar{A}'$\\
    $\underbar{A}|B$& the substructures of $\underbar{A}$\\
    $\sup A$& the least upper bound of $A$\\
    $\inf A$& the greatest lower bound of $A$\\ 
    $\bar{\bar{A}}$& the cardinal number of $A$\\
    $\hat{n} $& hyperexponentiation to the base 2 at n\\
\end{longtable}
\newpage
\section{SETS AND RELATIONS AND OPERATIONS AMONG THEM}
    Logical symbols and abbreviations are easy to write, but 
they are not meant to use throughout the whole proof. Even mathematics
only sometimes use them in theorems'writing. When writing a proof,
it is recommended to use ordinary language to explain your idea whenever it is
necessery.
\subsection{Set algebra and the set-builder}
In this textbook, we assume that non-set doesn't exists. In 
other words, we only consider what is so-called pure set.
\begin{axiom}[Axiom of Extensionality]
If, for any $x$,$x\in A$ if and only if $x\in B$,then $A=B$ .
\end{axiom}
All expression in language can be classified into 3 groups, namely 
asserting expression, naming expression and others. 
Asserting expression is a kind of expression that can specify the range or 
value of some variable, for example, an inequility.
Name expression is a kind of expression that describle some forms, for example,
a function.
When using set builder 
$\{x:---\}$, ---should always be replaced by asserting expression.
When using set buider $\{...:---\}$,...should be replaced by a naming expression,
and --- should be replaced by a asserting expression.

We defined following Boolean operations:
\[
\begin{cases}
    A \cup B & =  \{x:x\in A \text{or} x \in B\},\text{called the union of} A \text{and} B.\\
    A \cap B & =  \{x:x\in A \text{and} x \in B\},\text{called the intersection of} A \text{and} B.\\
    A - B & =  \{x:x\in A \text{and} x \notin B\},\text{called } A \text{minus} B.\\
    A \circleddash  B & =  (A-B)\cup(B-A),\text{called the symmetric difference of} A \text{and} B.\\
    \varnothing & =  \{x:x \neq x\}, \text{the empty set}\\
\end{cases}
\]
Particularly, $X-A$is also written as $\tilde{A}^{(X)}$, called the complement of $A$ with respect to $X$.

When we are discussing sets, we always assume that they are included in a set $X$ called `temporary universe'.
In this case, we write $\tilde{A}^{(X)}$as $\tilde{A}$.
However, there is no absolute universe.

If a law only involves $\sim,\cup,\cap,\varnothing,X$,not $\circleddash$,etc,its dual 
is obtained simply by replacing these,respectively by $\sim,\cap,\cup,X,\varnothing$.

\begin{theorem}
    Assume $A$,$B$,$C\subseteq X$and write $\tilde{U}$for $X-U$.
\[
\begin{cases}
    A \cup B = B \cup A & \text{and dual}\\
    A \cup (B \cup C)=(A\cup B)\cup C& \text{and dual}\\
    A \cup \varnothing =A & \text{and dual}(A\cap X =A)\\
    A \cap \varnothing=\varnothing& \text{and dual}.\\
    \tilde{\tilde{A}}=A & \\
    A \subseteq B \text{if and only if} \tilde{A}\supseteq \tilde{B}& \\
    A-B = A \cap \tilde{B}& \\
    A\cap B =\varnothing \text{if and only if} A \subseteq \tilde{B}& \\
\end{cases}    
\]
$A$ and  $B$ are called disjoint, written $A)(B$,if $A\cap B =\varnothing$.
\[A\cap(B\cup C)=(A\cap B)\cup (A\cap C),\text{and dual}\]
\[\widetilde{A\cup B}=\tilde{A}\cap \tilde{B},\text{and dual} \]
\[A\cap(B-C)=(A\cap B)-(A\cap C)\]
\end{theorem}
\begin{theorem}
    The family of all subsets of $X $,under $\circleddash$for+,and $\cap$for $\cdot$,
    with $\varnothing$for 0, and $X$ for 1,forms a communicative ring with unit.
    the ring also satisfies the two additional laws :$A+A=0$(every element is of order 2)
    and $A\cdot A=A$($\cdot$ is idempotent).(Such ring is called a Boolean ring with unit.)
\end{theorem}
\begin{remark}
    To remember this kind of equation, I think you can regarded them as difference translations
    for one same thing. For instance, $A-B=A\cap \tilde{B}$, this equation's both side 
    means that remove B from A, or the part of A that doesn't belong to B.
\end{remark}
\subsection{Russell's paradox}
\begin{remark}
    $\{x:x\notin x\}$ does not exist.
\end{remark}
In our intuitive set theory, we now accept these two axiom, namely:
\begin{axiom}[Separation Axiom]
   $ \{x:x\in B \text{ and}\mathfrak{S} x\}$always exists.
\end{axiom}
\begin{axiom}[Replacement Axiom]
    $\{\mathfrak{C} :x \in B\}$always exists.
\end{axiom}
\begin{theorem}
    There is no universe set, i.e., for any set $C$ there exists $t$ such that $t \notin C$.
\end{theorem}
\subsection{Infinite unions and intersections}
\begin{theorem}
    Assume $\mathfrak{G} \subseteq X$ for all $i\in I$, and write $\tilde{U}$
    for $X-U$ and take any intersection of the form $\bigcap _{i\in \varnothing}B_i$ 
    to be $X$. Then:
    \begin{enumerate}
        \item (de Morgan's law) $\widetilde{\bigcup _{i\in I}\mathfrak{G}_i }=\bigcap_{i\in I}\tilde{\mathfrak{G}_i}$and $\widetilde{\bigcap _{i\in I}\mathfrak{G}_i }=\bigcup_{i\in I}\tilde{\mathfrak{G}_i}$
        \item (distributive laws) $B\cap \bigcup _{i\in I }\mathfrak{G}_i=\bigcup_{i\in I}(B\cap \mathfrak{G}_i)$ and its dual
    \end{enumerate}
\end{theorem}
\begin{remark}
    The varible i in the expression $\bigcap_{i\in I}\mathfrak{G}_i$ is called 
    bound variable, and this kind of notation with bound variable is called 
    bound variable notation. For any family $K$ of sets, we write $\bigcup (K)$ for $\bigcup_{A\in K}A$,
    which is free of bound variables. Although the notations without bound variables seem
    to be more elegent, but sometimes when we have to point out various property of one 
    element, using bound varialbles may be more convenient.
\end{remark}
\subsection{Ordered couples and Cartesian products}
We read $\{a\}$ the singleton of a.

\begin{definition}
    $(a,b)=\{\{a\},\{a,b\}\}$
\end{definition}
\begin{theorem}
    If $(a,b)=(c,d)$,then $a=c$ and $ b=d$.
\end{theorem}

The set $A\times B=\{(a,b):a\in A \text{and} b\in B\}$ is called the Cartesian product of
the arbitrary sets $A$ and $B$.

\begin{theorem}
    \begin{enumerate}
        \item $A\times \bigcup_{i\in I}\mathfrak{B}_i=\bigcup_{i\in I}(A\times \mathfrak{B}_i)$
        \item $\bigcup_{i\in I}\mathfrak{B}_i\times A=\bigcup_{i\in I}(\mathfrak{B}_i\times A)$
        \item $A\times \bigcap_{i\in I}\mathfrak{B}_i=\bigcap_{i\in I}(A\times \mathfrak{B}_i)$
        \item $A\times (B-C)=A\times B-A\times C$
        \item $(A\cap B)\times (C\cap D)=(A\times C)\cap(B\times D)$
    \end{enumerate}
\end{theorem}
\subsection{Relations and functions}
We say $R$ is a relation if $R$ is a set of ordered couples. We write $Rxy$
(or $xRy$), and say $R$ holds betweem x and y, if $(x,y)\in R$.

The domain of a relation $R$,Dom $R$, is defined to be $\{x:for some y, xRy\}$ and
the range, Rng $R$,is $\{y:for some x, xRy\}$.

We say $f$ is a function if $f$ is a relation and for any $x,y,z$ if
$(x,y),(x,z)\in f$ then $y=z$.

`$f$ is on $A$' means Dom $f=A$;`$f$is to,or into $B$' means Rng $f\subseteq B$;
`$f$is onto $B$' means Rng $f=B$;`$f:A\rightarrow B$'means $f$ is a function 
on $A$ to $B$.

Taking $f=((x,\mathfrak{a}x):x\in A):$
\begin{theorem}
    There exists exactly one function $f$ on $A$ such that for each $x\in A,f(x)\in \mathfrak{a}_x.$
\end{theorem}
For the unique function $f$ on $X$ such that, for all $x\in X$, $f(x)\in \mathfrak{a}_x$,
all of the following are in use:
\[(\mathfrak{a}_x:x\in X)\phantom{111}(\lambda x\in X)\mathfrak{a}_x\phantom{111}x \in X \mapsto \mathfrak{a}_x.\]

The function $f=(\mathfrak{a}_x:x\in X)$ is sometimes called an indexed family (of sets).
The (plain) family $\{\mathfrak{a}_x:x\in X\}$has `less information'.

If $C\subseteq $Dom $f$, we write $f\upharpoonright C$(called the restriction of f to C)
for the function $g$ on $C$ such that, for any $x\in C,g(x)=f(x)$.

We take $\Breve{R}$($R$ converse) to be the set of all $(y,x)$ such that $(x,y)\in R$.
The converse of a function which is not one-to-one is a relation but not a function.
$f$ is one-to-one if and onlly if $\Breve{f}$ is a function. In this case,
the function $\Breve{f}$is called the inverse function (of $f$) and written $f^{-1}$.

If $R$ and $S$ are relations, the relative product $R/S$ is defined to
be $\{(x,y):\text{for some }z,(x,z)\in R \text{and }(z,y)\in S\}$.We also define
$R\circ S$to be $S/R$
\begin{theorem}
    \[R/(S/T)=(R/S)/T\]
\end{theorem}
\begin{theorem}
    \[\Breve{R/S}=\Breve{S}/\Breve{R}\]
\end{theorem}
We take $R[U]$ to be the set of all $y$ such that for some $x\in U,(x,y)\in R$.
A special case is the $f-$image of $U$,$f[U]$. Given a function $f$ (not necessery one-to-one),
we can form  the relation $\Breve{f}$ and the image set $\Breve{f}[V]$; this set is called the $f-$
inverse image of $V$, and is written $f^{-1}[V]$.
\begin{theorem}
    Suppose for all $i\in I,\mathfrak{a}_i\subseteq X$.Then
    \begin{enumerate}
        \item $R[\bigcup_{i\in I}\mathfrak{a}_i]=\bigcup_{i\in I}R[\mathfrak{a}_i]$
        \item $f^{-1}(A-B)=f^{-1}A-f^{-1}B.$
        \item $f^{-1}(\bigcap_{i\in I}\mathfrak{a}_i)=\bigcap_{i\in I}f^{-1}\mathfrak{a}_i,$if $I\neq \varnothing$.
    \end{enumerate}
\end{theorem}
\begin{theorem}
    \begin{enumerate}
        \item $A\underset{Id_A}{\sim}A$
        \item if $A\underset{f}{\sim}B,$then $B\underset{f^{-1}}{\sim}A$
        \item if $A\underset{f}{\sim}B$ and $B\underset{g}{\sim}C$,then $A\underset{g\circ f}{\sim}C$.
    \end{enumerate}
\end{theorem}
We write $A\sim B$,if, for some $f$, $A\underset{f}{\sim}B$.
\begin{theorem}
\begin{enumerate}
    \item $A\sim A$
    \item if $A\sim B$, then $B\sim A$
    \item if $A\sim B$ and $B\sim C$ then $A\sim C$.
\end{enumerate}
\end{theorem}
A notion like $\sim $for which satisfies previous three equations is called an `equivalence'.
\subsection{Sets of sets,power set, arbitrary Cartesian product}
We denote by $B^C$ the set of all function on $C$ to $B$. We will prove that,
if $C$ has n elements and B has m, then $B^C$ has $m^n$.

Cartesian product $\prod_{i\in I}\mathfrak{a}_i$ is defined to be the set of
all $f$ on $I$ such that, for each $i\in I,f(i)\in \mathfrak{a}_i$.If $\mathfrak{a}_i=B$,
then
\[\prod_{i\in I}\mathfrak{a}_i=\prod_{i\in I}B=B^I\]

The set $P(A)$ is the set of all subsets of the set $A$. It is also called the 
power set of $A$.

We form the `characteristic function ' $c_B$ of $B$ as follows:
$c_B$ is the function on $A$ such that, for any $x\in A,c_B(x)=u$ if $x\in B$,
and $c_B(x)=v$ if $x\neq B$. Then define $H(B)=c_B$.
\begin{theorem}
    \[P(A)\underset{H}{\sim}\{u,v\}^A\]
\end{theorem}
\begin{theorem}
    Consider sets $\mathfrak{a}_{ij}\subseteq X$ and take any $\bigcap_{t\in\varnothing}B_t=X$
    \[\bigcap_{i\in I}\bigcup_{j\in  J_i}\mathfrak{a}_{ij}=\bigcup_{f\in\prod_{i\in I} J_i}\bigcap_{i\in I}\mathfrak{a}_{if(i)}.\]
    (The dual follows by applying de Morgan's law.)
\end{theorem}
\begin{axiom}[The Axiom of Choice]
    Let $F$ be a function on $I$ such that for each $i\in I$,$F(i)$ is
    a non-empty set. Then there is a function $f$ on $I$ such that for each
    $i\in I$, $f(i)\in F_i$.
\end{axiom}
\begin{remark}
    Such an $f$ is called a choice function (or selector) for the indexed family
    $(F(i):i\in I).$
\end{remark}
\begin{remark}
    The Axiom of Choice is identical to the statement: If for each $i\in I,F(i)\neq\varnothing,$
    then $\prod_{i\in I}\neq\varnothing$.
\end{remark}
When we are proving, we should try to avoid using the Axiom of Choice directly.
\begin{theorem}
    Assume all $\mathfrak{a}_{ij}\subseteq x$ and take $\bigcap_{t\in \varnothing}\mathfrak{B}_t=X$
    \[\prod_{i\in I}\bigcup_{j\in J_i}\mathfrak{a}_{ij}=\bigcup_{f\in \prod_{i\in I} J_i}\prod_{i\in I}\mathfrak{a}_{if(i)}\]
\end{theorem}
\subsection{Structures}
We say $\circ$ is a binary operation over $A$ if $\circ:A\times A\rightarrow A$.
(As usual, $\circ((x,y))$ is written $x\circ y$). $R$ is called a binary relation over $A$ if $R\subseteq A\times A.$
A typical (algebraic) structure is something of the form $\underbar{A}=(A,R,\circ)$ where $\circ$ 
and $R$ are over $A$. The set $A$ is called the universe of $\underbar{A}$. A structure can have any number of $R$'s and
$\circ$'s and these can have any number of places, and there cam also be distinguished elements.

Let $\underbar{A}=(A,R),\underbar{A}'=(A',R')$. We say that $f$ is an $(\underbar{A},\underbar{A}')-$
isomorphism (or isomorphism between $\underbar{A}$ and $\underbar{A}'$), provided that $A\underset{f}{\sim}A'$ for any 
$x,y\in A,xRy$ if and only if $f(x)R'f(y)$(If instead $\underbar{A}=(A,\circ),\underbar{A}'=(A',\circ'),$one has instead the 
familiar condition: $f(x\circ y)=f(x)\circ'f(y)$).

We write $\underbar{A}\cong\underbar{A}'$ and say that $\underbar{A}$ is isomorphic to 
$\underbar{A}'$ if for some $f$,$\underbar{A}\underset{f}{\cong}\underbar{A}'$.
\begin{theorem}
    The theorem 1.11 and 1.12 about $A\underset{f}{\sim}B$ and $A\sim B$
    extend at once to $\underbar{A}\underset{f}{\cong}\underbar{B}$and $\underbar{A}\cong \underbar{B}$.
\end{theorem}

If $B\subseteq A$, we call $(B,R\cap(B\times B))$ the substructures of $\underbar{A}$,
written $\underbar{A}|B$.

We say that $f$ is an isomorphism of $\underbar{A}=(A,R)$ into $\underbar{A}'=(\underbar{A}',\underbar{R}')$
if $f$ is one-to-one on $A$ into $A'$ and for any $x,y\in A$, $xRy$ if and only if $f(x)R'f(y)$.
\begin{theorem}
    $f$ is an isomorphism of $\underbar{A}$ into $\underbar{A}'$ if and only if $f$
    is an isomorphism between $\underbar{A}$ and some substructures of $\underbar{A}'$.
\end{theorem}
\subsection{Partial orders and orders}
Let $\underbar{A}=(A,R)$ be a structure. $\underbar{A}$ is called reflective if
for any $x\in A,xRx$; otherwise it is called irreflective. $\underbar{A}$ is transitive if $xRz$
whenever $xRy$ and $yRz$;symmetric if $yRx$ whenever $xRy$; antisymmetric if $x=y$
whenever $xRy$ and $yRx$ and asymmetric if $y\slashed{R}x$ whenever $xRy.$Finally, $\underbar{A}$
is connected if for any $x,y\in A,xRy$ or $yRx$ or $x=y$.

Relation $R$ is said to be equivalence if $(A,R)$ is reflective,symmetric,and transitive.

A structure $\underbar{A}=(A,R)$ is called a partial order if it is reflective, transitive,and antisymmetric;
$\underbar{A}$ is called a strict partial order(a variant) if $\underbar{A}$ is irreflective and transitive.
$\underbar{A}$ is called an order if it is a partial order and is connected.
As a variant, $\underbar{A}$ is called a strict order if it is a strict partial order and is connected.

In a familiar way, we often use $\leq$ as a variable, instead of $R$, when $(A,R)$ is reflective,
and $<$ when $(A,R)$ is irreflective. Moreover, if we start with $(A,<)$, then we write $x\leq y$
to mean : $x<y$ or $x=y$; and if we start with $(A,\leq)$ , then we write $x<y$
to mean $x\leq y$ and $x\neq y$.

If $\underbar{A}=(A,\leq)$ is a partial order ( or order), then $(A,<)$ is 
a strict partial order (resp.,strict order). On the other hand, if $(A,<)$ is a strict partial 
order (or strict order), then $(A,\leq)$ is a partial order(resp., order).

 For any family  $Q$ of sets, consider the inclusion relation cut down to $Q$, i.e., the 
 relation $\subseteq_Q=\{(A,B):A,B\in Q \text{and} A \subseteq B\}.$

 \begin{theorem}
    $(Q,\subseteq_Q)$ is a partial order.
 \end{theorem}

 A family $Q$ of sets which is ordered by $\subseteq_Q$ is called a chain.

\begin{theorem}
    Suppose $(A,R)$ and  $(A',R')$ are strict orders, $f$ is on $A$ 
    onto $A$' and `order preserving', i.e., $f(x)R'f(y)$ whenever $xRy$. 
    Then $(A,R)\underset{f}{\cong}(A',R')$
\end{theorem}

If $\underbar{B}$ is a substructure of $\underbar{A}$ and $\underbar{A}$
is a partial order, strict partial order, order, ore strict order, then so is $\underbar{B}$.
If $(A,R)$ is a partial order, etc., as above, then so is $(A,\Breve{R})$.

If $B\subseteq A$ and $x\in A$, we say that $x$ is a minimal element of $B$ if $x\in B$
and for all $y\in B,y\slashed{<}x$. On the other hand, we say $x$ is a minimum
element if $x\in B$ and for all $y\in B,x\leq y$. Also, $x$ is called a lower bound for $B$
if for all $y\in B,x\leq y$. The notions maximal, maximum, upper bounde are obtained by passing 
$\leq$ to $\geq$.
\begin{theorem}
    Let $\underbar{A}=(A,\leq)$ be a partial order, and $B\subseteq A$.
    \begin{enumerate}
        \item `minimum' implies `minimal'(i.e., if $x$ is a minimum element of $B$, then $x$ is a minimal element of $B$)
        \item $\underbar{B}$ has at most one minimum element.
        \item If $\underbar{A}$ is an order, then `minimum' = `minimal'.(i.e., $x$ is a minimum element of $B$ if and only if $x$ is a minimal element of $B$).
    \end{enumerate}
\end{theorem}
\begin{remark}
    There are incredible number of synonyms for `minimum' in common usage - 
    for example, least, smallest, first, leftmost...
\end{remark}
In a partial order $(A,\leq)$, one says that $x$ precedes $y$(or $y$ succeed $x$)if $x<y$.
In an order $(A<\leq)$, to say that $y$ is an immediate successor of $x$ means of course that $y$ is a 
minimum (or equivalently, minimal) element of the set $B$ of all successors of $x$.
If it exists, then it is unique and $x$ is then the immediate predecessor of $y$.
$x$ is a least upper bound (also called a `supremum' or  `sup' for $B$ )means that 
$x$ is a least member of the set $C$ of all bounds for $B$. Such an x is unique if it exists. One also write
$x=\sup B=\sup_{u\in B}u$. Similarly $B$ may have a greatest lower bound, or infimum, called $\inf B.$

Let $\underbar{A}=(A,\leq)$ be an order. By an initial segment of $\underbar{A}$ we mean any subset
$B$ of $A$ such that for any $x,y$, if $x<y$ and $x\in B$ then $y\in B$. Each element of a of $A$
determines two initial segments of $\underbar{A}$, namely Pred$a$(in full, Pred $^{<}_{\underbar{A}}a)=\{x:x< a\}$,and Pred$^{\leq}_{\underbar{A}}a=\{x:x\leq a\}$.
The substructures of $\underbar{A}$ whose universe is Pred $a$ will be called $\underbar{A}_a$. An inital segment is 
called proper if $B\neq A$. 

Given $U,V\subseteq A$ we say that $U$ and $V$ are cofinal( or confinal or coterminal,or that
$U$ is cofinal with $V$ or that $V$ is with $U$,) all in the sense of the order $\underbar{A}$,
if for any $x\in U$ there is a $y\in V$ such that $x\leq y$ and if $x\in V$
then for some $y\in U$, $x\leq y$.

$\underbar{A}$ is called dense if it has at least two elements and for any
$x$ an $y$, if $x<y$, then for some $z$: $x<z<y$. $\underbar{A}$ is called discrete
if every element not a last element has an immediate successor and every element not a first
has an immediate predecessor. $\underbar{A}$ is called a well-order if every non-empty subset of 
$A$ has a first element. Any substructures of a well-order is a well-order.
\begin{theorem}
    Let $\underbar{A}=(A,\leq)$ be a well-order 
    \begin{enumerate}
        \item If $A$ is not empty, there is a first element.
        \item Every element not a last has an immediate successor
        \item Every proper initaial segment is of the form Pred a(for some a)
    \end{enumerate}
\end{theorem}
\section{CARDINAL NUMBERS AND FINITE SETS}
\subsection{Cardinal numbers,+,and $\leq$}
Cardinals or cardinal numbers are denoted by $\kappa,\lambda$ and $\mu$.
It is common to say sometimes `the power of $A$' instead of `the cardinal number of $A$'.
We can now define $0,1,2$ by putting $0=\bar{\bar{\varnothing}}=0;1=\overline{\overline{\{x\}}},2=\overline{\overline{\{x,y\}}}$ if
$x\neq y$.
\begin{theorem}
    $\bar{\bar{A}}=\bar{\bar{B}}$ if and only if $A\sim B$.
\end{theorem}
\begin{theorem}
    \begin{enumerate}
    \item If $A\sim A',B\sim B',A)(B,and A')(B',$then $A\cup B\sim A'\cup B'$
    \item For any $X,Y$ there exists $X',Y',$ disjoint, such that $X\sim X'$ and $Y\sim Y'$.
    \end{enumerate}
\end{theorem}
\begin{definition}
    $\kappa+\lambda=$the unique $\mu$ such that for some $A,B,\bar{\bar{A}}=\kappa,\bar{\bar{B}}=\lambda,A)(B,$
    and $\bar{\bar{A\cup B}}=\mu$
\end{definition}
\begin{theorem}
    \begin{enumerate}
        \item $\kappa+\lambda=\lambda+\kappa$
        \item $\kappa+(\lambda+\mu)=(\kappa+\lambda)+\mu$
        \item $\kappa+0=\kappa$
        \item If $\kappa\neq 0$ then for some $\lambda,\kappa=\lambda+1$
        \item If $\kappa+1=\lambda+1$, then $\kappa=\lambda$
    \end{enumerate}
\end{theorem}
\begin{definition}
    We say that $\kappa\leq \lambda$ if for some $\mu,\lambda=\kappa+\mu$
\end{definition}
\begin{theorem}
    The following are equivalent:
    \begin{enumerate}
        \item $\kappa\leq\lambda$
        \item For some $A$,$B$, $\bar{\bar{A}}=\kappa,\bar{\bar{B}}=\lambda$,and $A\subseteq B$
        \item For any set $B$ of power $\lambda$ there exists $A\subseteq B$ of power $\kappa$
        \item For any $A$,$B$ of respective powers $\kappa,\lambda$, there exists $f$ one-to-one on $A$ to $B$
        \item For any set $A$ of power $\kappa$ there exists $B\supseteq A$ of power $\lambda$ 
    \end{enumerate}
\end{theorem}
\begin{theorem}[Exchange Principle]
    For any $X,Y$ there exists \verb+Z)(X+ such that $Z\sim Y$.
\end{theorem}
\begin{theorem}
    \begin{enumerate}
        \item $\kappa\leq \kappa$
        \item If $\kappa<\lambda$ and $\lambda\leq \mu$ then $\kappa\leq \mu$
        \item If $\kappa\leq \lambda$ then $\kappa+\mu\leq\lambda+\mu$
        \item $0\leq \kappa$
        \item If $\kappa<\lambda$, then $\kappa+1\leq\lambda$
    \end{enumerate}
\end{theorem}
\begin{theorem}
    For any $\kappa,\{\lambda:\lambda\leq \kappa\}$ exists.
\end{theorem}
We write $\kappa\leq^*\lambda$ if, for some $A$, $B$,$\bar{\bar{A}}=\kappa,\bar{\bar{B}}=\lambda$,
and there is a funciton $f$ on $B$ onto $A$ ($A$ is `an image of $B$' )or else $\kappa=\varnothing$.
\begin{theorem}
    \begin{enumerate}
    \item If $\kappa<\lambda$ then $\kappa\leq^*\lambda$
        \item If $\kappa\leq^*\lambda$ then $\kappa\leq\lambda$
        \item Suppose $\bar{\bar{B}}=\lambda$ and some $R$ well-orders $A$. Then $\kappa\leq^*\lambda$ implies $\kappa\leq \lambda$.
    \end{enumerate}
\end{theorem}
\subsection{Natural numbers and finite sets}
\begin{definition}
    $\kappa$ is a natural number if $\kappa$ belongs to every set $X$ such that 
    $0\in X$ and, for any $\lambda$, if $\lambda\in X$ then $\lambda+1\in X$.
\end{definition}
We say $A$ is finite if $\bar{\bar{A}}$ is a natural number; $A$ is infinite otherwise.
\begin{theorem}
    \begin{enumerate}
        \item 0 is a natural number.
        \item If $\kappa$ is a natural number so is $\kappa+1$
        \item (Induction) Suppose that $\mathfrak{p}$0 (`$\mathfrak{p} holds for 0$');
        and that, for any $n$, if $\mathfrak{p}n$ then $\mathfrak{p}(n+1)$. Then for every $n,\mathfrak{p}n$.
    \end{enumerate}
\end{theorem}
\begin{theorem}
    \begin{enumerate}
        \item $m+n$ is a natural number
        \item If $\kappa<n$ then $\kappa$ is a natural number
        \item If $\kappa+n=\lambda+n$ then $\kappa=\lambda$
        \item Corollaries:\begin{itemize}
            \item If $m\leq n$ then there is exactly one $k$ such that $m+k=n$
            \item A finite set is not equivalent to a proper subset of itself
            \item $n<n+1$
            \item If $m\leq n$ and $n\leq m$ then $m=n$
        \end{itemize}
        \item $\kappa\leq n$ or $n\leq \kappa$
        \item Corollaries:\begin{itemize}
            \item $\leq$ `orders' the natural numbers. In this `order':
            \item 0 is the first element
            \item for each $n,n+1$ is the immediate successor of $n$ , and
            \item each $n\neq 0$ has an immediate predecessor.
        \end{itemize}
        \item Put $W_n=\{m:m<n\}.$ Then $\bar{\bar{W_n}}=n.$ So every finite set $A$ can be ordered.
        \item If $(A,R)$ is any order, every finite subset of $A$ has a first element\begin{remark}
            So also a last. Hence every finite order is a well-order.
        \end{remark}
        \item Any two orders $(A,\leq),(A',\leq')$ each on an n-element set are isomorphic.
        \item If $\kappa\leq^*n$ then $\kappa\leq n$
    \end{enumerate}
\end{theorem}
\begin{remark}
    When we use greek letter $\kappa,\lambda$ etc., we don't guarantee that they are 
    finite, but when we use letters like n,j,k etc., we assume that they are finite.
\end{remark}
\begin{theorem}
    \begin{enumerate}
        \item (The least element principle). If for some $n$, $\mathfrak{p}n$, then there is 
        ia a minimal( which is here the same as minimum) n such that $\mathfrak{p}$n.
        \item (Course-of-values induction) If, for any n, if $\mathfrak{Q}$m holds for all $m<n$,
        then $\mathfrak{Q}n$; then, for all $n,\mathfrak{Q}n$. 
    \end{enumerate}
\end{theorem}
  By an n-termed sequence or synonymously, an ordered n-tuple of elements of $A$
  we mean a function $s$ on $W_n$ to $A$. We may sometimes write $s_i$ instead of $s(i)$.
  The length of $s$ is $n$. We say that $R$ is an $n-$ary relation for $A$ if $R\subseteq A^{W_n}$.
  $o$ is called an $n-$ary operation for $A$ if $o:A^{W_n}\rightarrow A$
\subsection{Multiplication and exponentiation}
\begin{theorem}
    If $A\sim A'$ and $B\sim B'$ then $A\times B\sim A'\times B'$
\end{theorem}
\begin{definition}
    $\kappa\cdot\lambda$(or just $\kappa\lambda$)= the unique $\mu$ such that,
    for some $A$, $B$, $\bar{\bar{A}}=\kappa,\bar{\bar{B}}=\lambda$, and $\overline{\overline{A\times B}}=\mu.$
\end{definition}
\begin{theorem}
    \begin{enumerate}
        \item $\kappa\lambda=\lambda\kappa$
        \item $\kappa(\lambda\mu)=(\kappa\lambda)\mu$
        \item $\kappa(\lambda+\mu)=\kappa\lambda+\kappa\mu$
        \item $\kappa\cdot 0=0;\kappa\cdot 1=\kappa;\kappa\cdot 2=\kappa+\kappa$
        \item If $\kappa\lambda=0$ then $\kappa=0$ or $\lambda=0$.
        \item If $\kappa\leq\lambda$ then $\kappa\mu\leq\lambda\mu$.
        \item If $\kappa,\kappa\geq 2$ then $\kappa+\lambda\leq\kappa\cdot\lambda$
        \item $m\cdot n$ is a natural number
        \item If $m\neq 0$ and $n<p$ then $mn<mp$(so, if $m\neq 0 $ and $mn=mp$ then $n=p$)
    \end{enumerate}
\end{theorem}
\begin{theorem}
    If $A\sim A'$ and $B\sim B'$, then $A^B\sim A'^{B'}$
\end{theorem}
\begin{definition}
    $\kappa^{\lambda}$= the unique $\mu$ such that for some $A,B,\bar{\bar{A}}=\kappa,\bar{\bar{B}}=\lambda,$
    and $\mu=\bar{\bar{A^B}}.$
\end{definition}
\begin{theorem}
    \begin{enumerate}
        \item $(\kappa^{\lambda})^{\mu}=\kappa^{\lambda\mu}$
        \item $\kappa^{\lambda+\mu}=\kappa^{\lambda}\kappa^{\mu}$
        \item $(\kappa\lambda)^{\mu}=\kappa^{\mu}\lambda^{\mu}$
        \item $\kappa^0=1;\kappa^1=\kappa;\kappa^2=\kappa\kappa;$ and if $\kappa\neq 0$ then $0^{\kappa}=0$
        \item If $\kappa\leq\lambda$ then $\kappa^{\mu}\leq\lambda^{\mu}$
        \item If $\kappa\leq\lambda$ then $\mu^{\kappa}\leq \mu^{\lambda}$ unless $\kappa=\mu=0<\lambda$
        \item If $\bar{\bar{A}}=\kappa$, the $\overline{\overline{P(A)}}=2^{\kappa}$
        \item $\kappa\leq 2^{\kappa}$
        \item $m^n$ is a natural number.
    \end{enumerate}
\end{theorem}
\subsection{Definition by induction}
\begin{theorem}
    \begin{enumerate}
        \item If $f$ works for $n$ and $k<n$, then $f\upharpoonright W_{k+1}$ works for $k$.
        \begin{remark}
            `work for' means that it satisfy the condition in theorem 2.17.1
        \end{remark}
        \item If $\mathfrak{B}_0=a$ and for all $n,\mathfrak{n+1}=\mathfrak{a}_{\mathfrak{B}_n}$, then $(\mathfrak{B}_i:i\leq k)$ works for $k$.
    \end{enumerate}
\end{theorem}
\begin{definition}
    $D_x(\mathfrak{a}_x,a,n)=$ the unique $z$ such that for some $f$ on $W_{n+1},f(0)=a,f(i+1)=\mathfrak{a}_{f(i)}$
    for all $i<n$, and $f(n)=z$.
\end{definition}
\begin{theorem}[Definition by induction]
   \begin{enumerate}
    \item There is exactly one $f$ such that $f$ is on $W_{q+1},f(0)=a,$ and $f(n+1)=\mathfrak{a}_{f(n)}$ for all $n<q$.
    \item $\mathfrak{B}_0=a$ and for all $n,\mathfrak{B}_{n+1}=\mathfrak{a}_{\mathfrak{B}_n}$, and it also hold for $\mathfrak{C}$, then, for all $n,\mathfrak{B}_n=\mathfrak{C}_n$.
    \item $D_x(\mathfrak{a},a,0)=a.D_x(\mathfrak{a},a, n+1)=\mathfrak{a}_{D_x(\mathfrak{a}_x,a,n)}$
   \end{enumerate} 
\end{theorem}
\begin{theorem}
    We can justibly introduce by recursion a notion $\mathfrak{B}$ by requiring that 
    \[\mathfrak{B}_n=\mathfrak{C}((\mathfrak{B}:j<n)) \text{for all }n\]
    \begin{remark}
        It means that, we use some transformation $\mathfrak{C}$ on $(\mathfrak{B}:j<n)$ to give out $\mathfrak{B}_n$,
        which is a course-of-values form of recursion.
    \end{remark}
\end{theorem}
\begin{theorem}
    If $N$ exists, then in fact there is exactly one function $F$ on $N$ such that $F(0)=c$.
    and for all $n$, $F(n+1)=\mathfrak{a}_{f(n)}$.
\end{theorem}
\subsection{Axiom of infinity, Peano axioms, Dedekind infinite sets}
An algebraic structure $(A,S,z)$ where $z\in A,S:A\rightarrow A$ is called a
Peano structure if (P1) for any $a\in A, Sa\neq z$;(P2) S is one-to-one; and (P3) for any subset 
$B$ of $A$, if $z\in B$ and $Sa\in B$, whenever $a\in B$, then $B=A$.(These are the famous Peano axioms)
\begin{axiom}[Axiom of Infinity]
    There exists a Peano structure.
\end{axiom}
\begin{theorem}
    The following are equivalent( forms of the Axiom of Infinity)
    \begin{enumerate}
        \item There exists a Peano structure;
        \item There exists an infinite set;
        \item $N$ exists;
    \end{enumerate}
\end{theorem}
We put $\aleph_0=\bar{\bar{N}}$.

\begin{theorem}
    If $\kappa$ is infinite, then $\aleph_0\leq\kappa$
\end{theorem}
\begin{theorem}
    (A,S,z) is a Peano structure if and only if $(A,S,z)\cong(N,Sc,0)$
\end{theorem}
A cardinal $\kappa$ is called Dedekind infinite if $\kappa\geq\aleph_0,$
\section{THE NUMBER SYSTEMS}
\subsection{Introductory remarks}
We call structure $(N,+,\cdot,0,1,<)$ which we call $\underbar{N}$. Let us 
say that $\underbar{A}=(A,+,\cdot,0,1,<)$ is a Peano semiring if: $(A,S,0)$ is
a Peano system, where $Sa=a+1$ for any $a\in A$; for any $a,b\in A, a<b$ if and only 
if for some $c\neq 0, b=a+c,$ and also 
\[\begin{cases}
    a+0=a\\
    a+(b+1)=(a+b)+1\\
\end{cases} \text{and}\begin{cases}
    a\cdot 0=0\\
    a\cdot(b+1)=a\cdot b+a\\
\end{cases}\]
\begin{theorem}
    $\underbar{A}$ is a Peano semiring if and only if $\underbar{A}$ is 
    isomorphic to $\underbar{N}=(N,+,\cdot,0,1,<)$
\end{theorem}
It follows that any Peano semiring obeys the following rich list of laws (which 
should be taken as restricted to the natural numbers): 2.3.,2.4.1.,2.4.3.,2.10.3.-2.10.9. and
2.13.
\subsection{Construction and characerization up to isomorphism of the integers, rationals, and reals}
We define the notion $n\circ a$ by recursion, specifying that $0\circ a=0$ and $(n+1)\circ a=n\circ a+a, $ for any $a\in A$. 
\begin{theorem}
    There exists one and up to isomorphism only one ordered intergral domain $\underbar{Z}=(Z,+,\cdot,0,1,<)$
    such that every element $x$ of $Z$ if of the form $n\circ 1$ or $-(n\circ 1)$ (where $n\in N$).
\end{theorem}
If $(A,+,\cdot,0,1,<)$ is an ordered integral domain, $a\in A$, and $i=n$ or $i=-n$ we put $i\circ a=n\circ a$
or $-(n\circ a),$respectively.
\begin{theorem}
    There is one and up to isomorphism only one ordered field $\underbar{Q}=(Q,+,\cdot,0,1,<)$ such that 
    every element of $Q$ is of the form $(i\circ 1)/(j\circ 1)$(where $i,j\in Z$ and $j\neq 0.$) 
\end{theorem}
\begin{theorem}
    There is one and up to isomorphism only one Complete ordered field $\underbar{R}=(R,+,\cdot, 0,1,<)$
\end{theorem}
\section{MORE ON CARDINAL NUMBERS}
\subsection{The Cantor-Bernstein Theorem}
\begin{theorem}[Cantor-Bernstein]
   If $\kappa\leq\lambda$ and $\lambda\leq\kappa$ then $\kappa=\lambda$ 
\end{theorem}
\subsection{Infinite sums and products of cardinal}
\begin{axiom}
    Suppose for each $i\in I$, there is an $x$ such that $\mathcal{P}(i,x)$.
    Then there exists $f$ on $I$ such that, for each $i\in I,\mathcal{P}(i,f(i))$.
\end{axiom}
\begin{theorem}
    Given $(\kappa_i:i\in I),$ there exists $F$ on $I$ such that $\overline{\overline{F(i)}}=\kappa_i$ for each $i\in I$.
\end{theorem}
\begin{theorem}
    \begin{enumerate}
        \item If $F$ and $G$ are both disjointed and on $I$ and for each $i\in I, F(i)\sim G(i)$,
        then $\bigcup_{i\in I}F(i)\sim \bigcup_{i\in I}G(i)$
        \item For any $F$ on $I$ there exists a disjointed $G$ on $I$ such that for each $i\in I$,$F(i)\sim G(i)$.
    \end{enumerate}
\end{theorem}
\begin{definition}
    $\sum_{i\in I}\kappa_i$ is the unique $\mu$ such that there exists a disjointed $F$ on $I$ 
    such that, for each $i\in I,\overline{\overline{F(i)}}=\kappa_i$, and $\mu=\bigcup_{i\in I}F(i)$
\end{definition}
\begin{theorem}
    If $F$ and $G$ are on $I $ and for each $i\in I, F(i)\sim G(i), $ then $\prod_{i\in I}F(i)\sim \prod_{i\in I}G(i)$
\end{theorem}
\begin{definition}
    $\prod_{i\in I}\kappa_i$ is the unique $\mu$ such that there exists $F$ on $I$ such that for each $i\in I, \overline{\overline{F(i)}}=\kappa_i$,
    and $\overline{\overline{\prod_{i\in I}F(i)}}=\mu$.
\end{definition}
\begin{theorem}
    \begin{enumerate}
   \item (General communicative law) Let $I\underset{f}{\sim}J$. Then $\sum_{j\in J}\kappa_j=\sum_{i\in I}\kappa_{f(i)}$(Likewise for $\prod$)
   \item (General associative law)  \[\sum_{i\in I}\sum_{j\in J_i}\kappa _{ij}=\sum_{(i,j)\in W}\kappa _{ij},\]where $W=\{(i,j):i\in I \text{and} j\in J_i\}$.(Likewise for $\prod$)
   \item (General distributive law) $\kappa\cdot\sum_{i\in I}\lambda_i=\sum_{i\in I}(\kappa\lambda_i);$
   \[\prod_{i\in I}\sum_{j\in J_i}\kappa _{ij}=\sum_{h\in \prod_{i\in I}J_i}\prod_{i\in I}\kappa_{ih(i)}\]
   \item $\sum_{i\in \{0,1\}}\kappa_i=\kappa_0+\kappa_1$ and $\prod_{i\in \{0,1\}}\kappa_i=\kappa_0\cdot\kappa_1;\sum_{i\in I}\kappa=\kappa\cdot \bar{\bar{I}}$ and $\prod_{i\in I}\kappa=\kappa^{\bar{\bar{I}}}.$
\begin{remark}
    Thus $1+1+1+\cdots=\sum_{n\in N}1=\aleph_0$ and $2\cdot 2\cdot2\cdot \cdots =2^{\aleph_0}$.
\end{remark}
\item If $J\subseteq I$ and $\kappa_i=0$ for all $i\in I-J$, then $\sum_{i\in I}\kappa_i=\sum_{i\in J}\kappa_i;$ and likewise for 1 and $\prod$ 
    \end{enumerate}
\end{theorem}
\begin{theorem}
    \begin{enumerate}
        \item If, for all $i\in I,\kappa_i\leq\lambda_i, $ then $\sum_{i\in I}\lambda_i$ and $\prod_{i\in I}\kappa\leq\prod_{i\in I}\lambda_i$
        \item If $\lambda\in Q$ then $\lambda\leq\sum_{\kappa\in Q}\kappa$\begin{remark}
            Hence every set of cardinals is bounded above.
        \end{remark}
        \item $\kappa^{\sum_{i\in I}\lambda_i}=\prod_{i\in I}\kappa^{\lambda_i}$
        \item $(\prod_{i\in I}\kappa_i)^{\lambda}=\prod_{i\in I}\kappa_i^{\lambda}$
    \end{enumerate}
\end{theorem}
\begin{theorem}
    If $\kappa_i\geq 2$ for all $i\in I$, then $\sum_{i\in I}\kappa_i\leq\prod_{i\in I}\kappa_i$.
\end{theorem}
\subsection{Different kinds of infinity}
\begin{theorem}[Cantor's Theorem]
   $\kappa<2^{\kappa}$ 
\end{theorem}
\begin{theorem}[The Konig-Zermole Theorem]
   If, for each $i\in I,\kappa_i<\lambda_i,$ then $\sum_{i\in I}\kappa_i<\prod_{i\in I}\lambda_i$ 
\end{theorem}
\subsection{$\aleph_0,2^{\aleph_0},$ and $2^{2^{\aleph_0}}$- the simplest infinite cardinals}
A set $A$ is called denumerable if $\bar{\bar{A}}=\aleph_0,$ and countable if $\bar{\bar{A}}\leq \aleph_0$.
\begin{theorem}
    $A$ is countable if and only if $A$ is denumerable or finite.
\end{theorem}
\begin{theorem}
    If $\kappa\leq^*\aleph_0$, then $\kappa$ ia countable.
\end{theorem}
\begin{theorem}
    The sets $Z$ (of integers) and $Q$ (of rationals) are denumerable.
    Also
    \[\aleph_0=\aleph_0+\aleph_0=\aleph_0\cdot n=\aleph_0\cdot\aleph_0=\aleph_0^n=\overline{\overline{Sq}},\]
    where $Sq=\bigcup_{n\in N}N^{W_n}=$ the set of all finite sequences of natural numbers.
\end{theorem}
A real number is called algebraic if it is a root of some polynomial with rational
coefficients not all zero.
\begin{theorem}
    The set $Alg$ of all real algebraic numbers is countable.
\end{theorem}
\begin{theorem}
    A countable union of countable sets is countable.
\end{theorem}
The set $R$ is sometimes called the continum, and its cardinal number is 
denoted by $\mathfrak{c}$.
\begin{theorem}
    $2^{\aleph_0}=\mathfrak{c}$
\end{theorem}
\begin{theorem}
    There exists transcendental real numbers.(i.e., reals which are not algebraic.)
\end{theorem}
\begin{theorem}
    $\mathfrak{c}=\mathfrak{c}+\mathfrak{c}=\mathfrak{c}\cdot\mathfrak{c}=\mathfrak{c}^n=\mathfrak{c}^{\aleph_0}$
\end{theorem}
\begin{theorem}
    $\mathfrak{c}^{\mathfrak{c}}=2^{\mathfrak{c}}$
\end{theorem}
\section{ORDERS AND ORDER TYPES}
\subsection{Ordered sums and products}
The ordered sum of two orders can be defined in a natural way. If 
the orders $\underbar{A}=(A,\leq)$ and $\underbar{A}'=(A',\leq')$ are disjoint,
then, their ordered sum $\underbar{A}+\underbar{A}'$ is the structure $(A\cup A',\leq'')$,
where, 
\[x\leq''y \text{ if and only if } x\leq y \text{ or } x\leq'y \text{ or }(x\in A \text{ and } y\in A')\]
Let $\underbar{I}=(I,\leq)$ 
be an order and suppose 
$(\underbar{$A_i$}:i\in I)$
is a disjointed indexed family of orders (where each 
$\underbar{$A_i$}=(A_i,\leq_i)$
).
Then, we defined the ordered sum 
$\sum_{i\in \underbar{I}}\underbar{$A_i$}$
 to be 
$(\bigcup_{i\in I}A_i,\leq)$,
where, 
$x\leq y$ if and only if for the determined $i,j $such that
$x\in A_i $ and $y\in A_j,$either$ i<j $ or else $(i=j $ and x$\leq_i y)$

If $\underbar{A}$ and $\underbar{B}$ are orders, the ordered product $\underbar{A}\cdot\underbar{B}$
can be defined to be $\sum_{b\in B}\underbar{A}$

It is convenient to allow the case where the given $(\underbar{A}_i:i\in I)$is not disjointed.
In this case, we put $\sum_{i\in \underbar{I}}A_i=\sum_{i\in \underbar{I}}A_i',$ where $\underbar{A}'=(A_i',\leq_i'),A_i'=\{i\}\times A_i,$
and of course, $(i,a)\leq_i'(i,b)$ if and only if $a\leq_i b$. Thus, $\underbar{A}\cdot\underbar{B}$ is just the
the set $A\times B$ of ``two letter words''(a,b) under the antilexograhic order.

$^*$ take $A$ `turn around', i.e., given $\underbar{A}=(A,\leq)$, we have $A^*=(A,\geq)$
\begin{theorem}
    \begin{enumerate}
        \item If $\underbar{I}=(I,\leq)$ is a well order and for each $i\in I,\underbar{A}_i$ is 
        a well order, then $\sum_{i\in \underbar{I}\underbar{A}_i}$ is a well-order.
        \item If $\underbar{A}$ and $\underbar{B}$ are well-orders so are $\underbar{A}+\underbar{B}$ and $\underbar{A}\cdot\underbar{B}$.
    \end{enumerate}
\end{theorem}
\subsection{Order types}
Each order $\underbar{A}$ corresponds something called Tp $\underbar{A}$(the order-type of $\underbar{A}$).
Tp $\underbar{A}$= Tp $\underbar{B}$ if and only if $\underbar{A}\cong\underbar{B}$. Moreover, 
for convenience, we assume also that if the order $\underbar{A}$ has finite cardinal $n$, then, Tp $\underbar{A}=n$.

Order types are denoted by $\rho,\sigma,\tau.$ Order types of well-orders $\underbar{A}$
are called ordinals and we write $\underbar{A}=$Ord $\underbar{A}$. We let $\omega=$Tp $(N,\leq);\eta=$Tp $(Q,\leq);$
and $\lambda=$Tp $(R,\leq)$.
\begin{theorem}
    If $\underbar{A},\underbar{A}',\underbar{B},\underbar{B}'$ are orders, $\underbar{A}\cong\underbar{A}'$
    and $\underbar{B}\cong\underbar{B}'$, then $\underbar{A}+\underbar{B}\cong\underbar{A}'+\underbar{B}',\underbar{A}\cdot\underbar{B}\cong\underbar{A}'\cdot\underbar{B}',$
    and $\underbar{A}^*\cong\underbar{B}^*.$(For +, assume $\underbar{A})(\underbar{B}$ and $\underbar{A}')(\underbar{B}'$)
\end{theorem}
\begin{definition}
    $\sigma+\tau$( or $\sigma\cdot\tau$) is the unique $\rho$ such that for some
    (disjoint for +) orders $\underbar{A},\underbar{B},$Tp $\underbar{A}=\sigma,$Tp $\underbar{B}=\tau$ and 
    $\rho=$Tp$(\underbar{A}+\underbar{B})$(respectively, $\underbar{A}\cdot\underbar{B}$).$\sigma^*$ is defined similarly.
\end{definition}
For any ordinal $\alpha$ and $\beta,\alpha+\beta$ and $\alpha\cdot\beta$ are ordinals.
\begin{theorem}
    Assume AC,
    \begin{enumerate}
        \item For any $(\sigma_i:i\in I)$ there exist orders $(\underbar{A}_i:i\in I)$ such that 
        Tp $\underbar{A}_i=\sigma_i$ for each $i\in I$.
        \item If the orders $\underbar{A}_i$ and $\underbar{A}_i'$ are isomorphic for 
        each $i\in I$ and $\underbar{I}=(I,\leq)$ is an order, then $\sum_{i\in \underbar{I}}\underbar{A}_i\cong\sum_{i\in \underbar{I}}\underbar{A}_i'$
    \end{enumerate}
\end{theorem}
\begin{definition}
    If $\underbar{I}$ is an order, $\sum_{i\in \underbar{I}}\sigma_i$ is the unique $\rho$
    such that for some orders $(\underbar{A}_i:i\in I)$,Tp $\underbar{A}_i=\sigma_i$ for each $i\in I$
    and $\rho=$Tp($\sum_{i\in I}\underbar{A}_i$).
\end{definition}
Here are some distinct ordinals:
\[0,1,2,\cdots,\omega,\omega+1,\cdots,\omega\cdot 2(=\omega+\omega),\cdots,\omega\cdot 3,\cdots,\omega\cdot\omega,\cdots\]
\begin{theorem}
    For $+,\cdot,$ and $\sum$ over order types, 
    \begin{enumerate}
        \item Associative laws for + and $\cdot$
        \item communicative laws fail for + and $\cdot$
        \item $\sigma(\tau+\tau')=\sigma\tau+\sigma\tau'$
        \item But the law $(\alpha+\beta)\gamma=\alpha\gamma+\beta\gamma$ fails, even for ordinals.
        \item (1) and (3) extend suitably to $\sum$, assuming AC.
        \item $(\sigma+\tau)^*=\tau^*+\sigma^*$ and $(\sigma\cdot\tau)^*=\sigma^*\cdot\tau^*$
    \end{enumerate}
\end{theorem}
If the order $\underbar{A}=(A,\leq)$ and $\underbar{A}'=(A',\leq')$ have the same
type $\sigma$, then $A$ and $A'$ have the same cardinal; we define $\bar{\bar{\sigma}}$ to be this cardinal.
Clearly $\overline{\overline{\sigma+\tau}}=\overline{\overline{\sigma}}+\overline{\overline{\tau}}$ and $\overline{\overline{\sigma\cdot\tau}}=\overline{\overline{\sigma}}\cdot \overline{\overline{\tau}}$
\begin{theorem}
    \begin{enumerate}
        \item Any two denumerable dense orders without first or last elements are isomorphic (and hence of type $\eta$)
        \item Let $\underbar{A}'$ have type $\eta$. Then, every countable order $(A,\leq)$ is isomorphic to a suborder (i.e., substructure) of $\underbar{A}'$
    \end{enumerate}
\end{theorem}
\section{AXIOMATIC SET THEORY}
\subsection{A formalized language}
\begin{axiom}[Empty Set Axiom]
    There is exactly one $x$ such that, for any $y,y\notin x$.
\end{axiom}
We take the following as convention
\[y=\iota x\mathcal{P}x\leftrightarrow [\exists!x\mathcal{P}x\land\mathcal{P}y]\lor[\sim\exists!x\mathcal{P}x\land\forall z(z\notin y)]. \]
\subsection{The axioms of set theory} 
We list the axioms of ZFC below.
\begin{enumerate}
    \item (Empty Set Axiom) There is exactly one $x$ such that, for any $y,y\notin x$ 
    \item (Extensionality Axiom) If, for any $x$,$x\in A$ if and only if $x\in B$, then $A=B$.
    \item (Separation Axiom) $\{x:x\in B\land\mathcal{P}x\}$exists
    \item (Doubleton) $\{t:t=x \text{ or }t=y\}$exists.
    \item (Union Axiom) $\{x: \text{for some }A,x\in A \text{ and }A\in W\}$exists.(That is, $\bigcup_{A\in W}A$exists)
    \item (Power Set Axiom) $\{B:\text{for every }x\in B, x\in A\}$ exists.(That is, $P(A)$ exists)
    \begin{remark}
        The axioms listed so far constitue the theory of axiom system denoted by $Z_0$.
    \end{remark}
    \item (Replacement Axiom) $\{\mathfrak{a}_x:x\in A\}$ exists.
    \item (Axiom of Infinity)
    \item (Axiom of Choice)
    \item (Axiom of Regularity)
\end{enumerate}
Thus the theory ZFC has been describled. The theory Z consists of $Z_0$ 
plus Infiity and Regularity. ZF is Z plus F, meaning the Axiom of Replacement.
ZFC is ZF plus Choice, `C' standing for Choice.

\begin{theorem}
    One can drop from $Z_0F$ both Separation and Doubleton and still derive 
    them.
\end{theorem}
\begin{theorem}
    $\{x:\mathcal{P}x\}$ exists if there is a  `big enough B',i.e., a B such 
    that $x\in B$ whenever $\mathcal{P}$x.
\end{theorem}
\begin{theorem}
    We establish the existence of the following sets: First, directly, from 
    axioms: $\varnothing;\{x\in A:\mathcal{P}x\};\{\mathfrak{a}_x:x\in A\};\bigcup(W);\mathfrak{P}(A);\{x,y\}.$
    Hence easily $\{x\},\bigcup_{i\in I}\mathfrak{a}_i,(x,y)$ and $(\mathfrak{a}_x:x\in X)$.
    Next: $A\cap B$ and $A-B;A\cup B=\bigcup(\{A,B\});A\circleddash B=(A-B)\bigcup(B-A);\bigcap_{i\in I}\mathfrak{a}_i$
    if $I\neq\varnothing$. Now $A\times B$ and Dom $R$. Next, easily, Range $R,R[A].$ Then $R/S$, and
    $f\upharpoonright A.$ Finally, $A^B$ and $\prod_{i\in I}\mathfrak{a}_i$.
\end{theorem}
\section{WELL-ORDERINGS,ORDINALS AND CARDINALS}
\subsection{Well-orders}
In a well-order $\underbar{A}$, every element $x$ is clearly of just one of 
these three kinds: $x$ is the first element; $x$ is a successor element - i.e., $x$
has an immediate predecessor. 

Let $\underbar{A}=(A,\leq)$ be any well-order and $u\notin A$.
Let $B$ be the ordered sum of $\underbar{A}$ and the order with universe $\{u\}$.
Then $\underbar{B}$ is a well-order, and clearly $\underbar{A}=\underbar{B}_u$.
We have the fact that every well-order $\underbar{A}$ is of the form $\underbar{B}_u$
for some well-order $\underbar{B}$ and some $u$.
\begin{theorem}
    Suppose $\underbar{A}=(A,\leq)$ is a well-order and $f:A\rightarrow A$ is
    increasing or, what is the same, order preserving(i.e., $f(x)<f(y)$ whenever $x<y$).
    Then for every $x\in A,f(x)\geq x$.
\end{theorem}
\begin{theorem}
    No well-order is isomorphic to (even a substructure of) a proper initial segment of itself.
\end{theorem}
An order (or indeed any structure) $\underbar{A}=(A,\cdots)$ is called rigid
if it has no automorphisms except $Id_A$.(An automorphism of $\underbar{A}$ is an 
$(\underbar{A},\underbar{A})-$isomorphism)
\begin{theorem}
    Any well-order $\underbar{A}=(A,\leq)$ is rigid.
\end{theorem}
\begin{theorem}
    For any well-orders $\underbar{A}$ and $\underbar{B}$, at most 
    one of the following holds:
    \begin{enumerate}
        \item $\underbar{A}$ is isomorphic to a proper initial segment of $\underbar{B}$
        \item $\underbar{B}$ is isomorphic to a proper initial segment of $\underbar{A}$
        \item $\underbar{A}\cong\underbar{B}$.
    \end{enumerate}
    Moreover, in each of (1)=(3) the corresponding map is unique.
\end{theorem}
\begin{theorem}[Comparison Theorem for Well-orders]
   Of any well-orders $\underbar{A}$ and $\underbar{B}$, one is isomorphic to 
   an initial segment of the other. 
\end{theorem}
\begin{theorem}
    Suppose $\underbar{B}$ is any substructure of the well-order $\underbar{A}$.
    Then $\underbar{B}$ is isomorphic to an initial segment of $\underbar{A}$.
\end{theorem}
Consider two assertion about an order $\underbar{A}=(A,\leq)$
\begin{enumerate}
    \item If $B$ is any non-empty subset of $A$, then it has a minimal element.
    \item If $C$ is any subset of $A$, and, for each $a\in A$, if $x\in C$ for all $x<a$, then $a\in C$, then for all $a\in A,a\in C.$(Induction Principle)
\end{enumerate}
According to these two assertion, we can prove that 
\begin{theorem}
    Every well-order satisfies the induction principle(by the very definition of `well-order').
\end{theorem}
\begin{proof}
    Prove by contradiction.
    Suppose $\underbar{A}$ is well-ordered and for each $a$ in $A$,
    if $x\in C$, for all $x<a$ then $a\in C$. By the axiom of seperation,
    the set $B=\{x \in A:x \notin C\}$ exists, which is also a subset of $A$.
    Thus, $B$ is also well-ordered and has a minimal element $b$. For any $a\in A,a<b$,
    $a\notin B$. Thus, $a\in C$ and $b\in C$, which is contrary to $b\in B$.
    Therefore $B=\varnothing$ and each $a\in A,a\in C$.
\end{proof}
\begin{theorem}[Recursion over a fixed well-order]
   ($\mathfrak{a}$is given.)Let $\underbar{A}=(A,\leq)$  be a well-order. Then, there is exactly one
   $f$ such that $f$ is a function on $A$ and for each $a\in A,f(a)=\mathfrak{a}_{f\upharpoonright \text{Pred }a}$ 
\end{theorem}
\begin{theorem}
    We can introduce a notion $\mathfrak{O}(\underbar{A})$(defined for all 
    well-orders $\underbar{A}$) by saying:``$\mathfrak{O}$ is defined recursively by requiring that: $\mathfrak{O}(\underbar{A})=\mathfrak{a}_{\{\mathfrak{O}(\underbar{A}_a):a\in A\}}$for each well-order $\underbar{A}=(A,\leq)$''
    
    Let $\mathfrak{O}$ be defined recursively by previous way. If $\underbar{A}\cong\underbar{A}'$(both being well-orders)
    then $\mathfrak{O}(\underbar{A})=\mathfrak{O}(\underbar{A}')$
\end{theorem}
\subsection{Von Neumann ordinals}
\begin{definition}
    We define Ord $\underbar{A}$, for all well-order $\underbar{A}$, by recursion,by 
    requiring for any well-order $\underbar{A}=(A,\leq)$:
    \[\text{Ord }\underbar{A}=\{\text{Ord }\underbar{A}_a:a\in A\}.\]
\end{definition}
\begin{theorem}
    If $\underbar{A}$ and $\underbar{B}$ are well-orders and $\underbar{A}\cong\underbar{B}$,
    then Ord $\underbar{A}\cong=$ Ord $\underbar{B}$.
\end{theorem}
$\alpha,\beta,\gamma,\epsilon$ range over ordinals.
\begin{theorem}
    $\alpha\notin\alpha$
\end{theorem}
\begin{theorem}
    If Ord $\underbar{A}=$ Ord $\underbar{B}$ ($\underbar{A},\underbar{B}$ well-orders) then $\underbar{A}\cong\underbar{B}$.
\end{theorem}
\begin{definition}
    We say that $\alpha<\beta$ if for some well-orders $\underbar{A}$ and $\underbar{B}$, Ord $\underbar{A}=\alpha$, Ord $\underbar{B}=\beta$
    and $\underbar{A}$ is a proper initial segment of $\underbar{B}$.
\end{definition}
\begin{theorem}
    <`strictly orders' the ordinals. That is:
    \begin{enumerate}
        \item $\alpha\slashed{<}\alpha$;
        \item If $\alpha<\beta$ and $\beta<\gamma$ then $\alpha<\gamma$;
        \item $\alpha\leq\beta$ or $\beta\leq\alpha$
    \end{enumerate}
\end{theorem}
For any $\alpha$, put $W(\alpha)=\{\beta:\beta<\alpha\}$. If $\alpha=$ Ord $\underbar{A}$ 
then $W(\alpha)\{\text{Ord }\underbar{A}_a:a\in A\}.$Thus, by the Replacement Axiom:
\begin{theorem}
    $W(\alpha)$ exists.
\end{theorem}
Write $\underbar{W}(\alpha)=(W(\alpha),\leq_{W(\alpha)})$.
\begin{theorem}
    $\underbar{W}(\alpha)$ is a well-order and Ord $\underbar{W}(\alpha)=\alpha$.
    Indeed, if Ord $\underbar{A}=\alpha$, then $\underbar{A}\underset{f}{\cong}\underbar{W}(\alpha),$
    where $f(a)=$ Ord $\underbar{A}_a$ for each $a$ in $\underbar{A}$.
\end{theorem}
\begin{theorem}
    If for some $\alpha,\mathcal{P}\alpha$, then, there is a least $\alpha$ such that $\mathcal{P}\alpha$. 
\end{theorem}
\begin{theorem}[Induction over all ordinals]
   Suppose that, for all $\alpha$, if $\mathcal{P}\beta$ for all $\beta<\alpha$,
   then $\mathcal{P}\alpha$. Then, for all $\alpha,\mathcal{P}\alpha$. 
\end{theorem}
\begin{theorem}[Recursion over all ordinals]
   We can (justifiably) introduce a notion $\mathfrak{O}(\alpha)$ (over all $\alpha$)
   by saying that: $\mathfrak{O}(\alpha)$ is defined recursively by requiring that for each $\alpha,\mathfrak{O}(\alpha)=\mathfrak{a}_{\mathfrak{O}\beta:\beta<\alpha}$ 
\end{theorem}
\begin{theorem}
    \begin{enumerate}
        \item (Burali=Forti) The set of all ordinals does not exist.
        \item Every set $K$ of ordinals is bounded above.
    \end{enumerate}
\end{theorem}
\begin{theorem}
    $\alpha=\{\beta:\beta<\alpha\}$
\end{theorem}
\begin{theorem}
    \begin{enumerate}
        \item $0=\varnothing$
        \item $\beta<\alpha$ if and only if $\beta\in\alpha$
        \item Every member of an ordinal is an ordinal.
        \item $\alpha$ is a transitive set - that is, for any $x,y$, if $x\in y\in \alpha$ then $x\in \alpha$)
    \end{enumerate}
\end{theorem}
\begin{theorem}
    $x$ is an ordinal if and only if $x$ is a transitive set and $(x,\in_x)$ is a well-order.
\end{theorem}
\subsection{The well-ordering theorem}
\begin{theorem}[Well-ordering Principle]
   Every set $A$ can be well-ordered (that is, for some $\leq,(A,\leq)$ is a well-order).
   Obviously equivalent is: For any $A$ there exists $\alpha$ such that $A\sim\alpha$. 
\end{theorem}
\begin{theorem}
    The Well-ordering Principle implies AC.
\end{theorem}
\begin{theorem}[Zorn's Lemma]
    Ler $\underbar{A}=(A,\leq)$ be a partial order, and suppose that every 
    subset $B$ of $A$ which is ordered by $\leq$ has an upper bound. Then, $\underbar{A}$ 
    has a maximal member.
\end{theorem}
\begin{theorem}
    Zorn's Lemma is equivalent to AC.
\end{theorem}
\subsection{Defining $\overline{\overline{A}}$ and Tp$\underbar{A}$}
\begin{definition}
    We put $\overline{\overline{A}}=$ the least ordinal $\alpha$ such that 
    $A\sim\alpha$.
\end{definition}
\begin{theorem}
    $\overline{\overline{A}}=\overline{\overline{B}}$ if and only if $A\sim B$.
\end{theorem}
An ordinal $\alpha$ is called an initial ordinal if $\alpha$ is not set-theoretically
equivalent to any smaller ordinal. Obviously, $\alpha$ is a cardinal if and 
only if $\alpha$ is an initial ordinal. 

It is clear that 
\[\overline{\overline{\kappa}}=\kappa\]
\begin{theorem}
    Given $(\kappa_i:i\in I)$, there exists $F$ on $I$ such that for each $i\in I,\overline{\overline{F(i)}}=\kappa_i.$
    (take $f(i)=\kappa_i$)
\end{theorem}
\begin{definition}
    Let $\underbar{A}$ be an order of power $\kappa$. If $\underbar{A}$
    is not a well-order, put Tp$\underbar{A}=$ the set of all order $(\kappa,R)$
    isomorphic to $\underbar{A}$; and if $\underbar{A}$ is a well-order, put Tp $\underbar{A}=$Ord $\underbar{A}$
\end{definition}
\begin{theorem}
    For any orders $\underbar{A},\underbar{B}$, Tp $\underbar{A}$=Tp $\underbar{B}$ 
    if and only if $\underbar{A}\cong\underbar{B}$.
\end{theorem}
\begin{theorem}
    If $\underbar{A}=(A,\leq)$ is an order and $\overline{\overline{A}}=n$, then Tp $\underbar{A}=n$
\end{theorem}
\subsection{Easy consequences for cardinals of the Well-ordering Principle}
\begin{theorem}
    $\kappa\leq\lambda$(card) if and only if $\kappa\leq\lambda$(ord).
\end{theorem}
By theorem $7.31.$, the cardinals with their natural order are a `suborder' of the usual 
ordering of the ordinals.

What's more, we see at once that: Every non-empty set (or even `class')
of cardinals has a least member. In particular, for any $\kappa$,
since there is a larger cardinal, namely $2^{\kappa}$, there is an immediate 
successor of $\kappa$ (among cardinals); it is denoted by $\kappa^+$.
\begin{remark}
    Because every cardinal is a initial ordinal, by theorem 7.16., there 
    is a least one. What's more, the initial ordinals that is greater than 
    $\kappa$ also has a least one, which is denoted by $\kappa^+$.
\end{remark}
\begin{theorem}[Cantor-Bernstein]
    If $\kappa\leq\lambda$ and $\lambda\leq\kappa$ then $\kappa=\lambda$.
\end{theorem}
\begin{theorem}[Comparability of cardinals]
    $\kappa\leq\lambda$ or $\lambda\leq\kappa$
\end{theorem}
We define  the cardinals $\aleph_{\alpha}$ for all $\alpha$ by recursion, putting 
\[\aleph_{\alpha}=\text{the first infinite cardinal greater than }\aleph_{\beta} \text{ for all } \beta<\alpha\]
\begin{theorem}
    Every infinite $\kappa$ is an `aleph'(i.e., is of the form $\aleph_{\alpha}$).
\end{theorem}
\begin{remark}
    To solve the ambiguity of cardinal + and ordinal +(and likewise for $\cdot$), 
    we let $\aleph_{\alpha}+\aleph_{\beta}$ mean the cardinal sum, but $\omega_{\alpha}+\omega_{\beta}$ means the ordinal.
\end{remark}
\subsection{A harder consequences and its corollaries}
\begin{theorem}
    If $\kappa$ is infinite, then $\kappa\cdot\kappa=\kappa$
\end{theorem}
\begin{theorem}
    \begin{enumerate}
        \item If $\kappa$ is infinite then $\kappa+\kappa=\kappa$
        \item If at least one of $\kappa$ and $\lambda$ is infinite, then
        \[\kappa+\lambda=\kappa\cdot\lambda=\max(\kappa,\lambda)\]
    \end{enumerate}
\end{theorem}
\begin{theorem}
    If $\kappa$ is infinite, then 
    \begin{enumerate}
        \item $(2^{\kappa})^{\kappa}=2^{\kappa}$
        \item $\kappa^{\kappa}=2^{\kappa}$
    \end{enumerate}
\end{theorem}
\begin{theorem}
    If $\kappa$ is infinite, $\overline{\overline{I}}\leq\kappa,$ and $\lambda_i\leq\kappa$
    for each $i\in I$, then, $\sum_{i\in I}\lambda_i\leq\kappa$.
\end{theorem}
\begin{theorem}
    Let $\kappa$ be infinite.
    \begin{enumerate}
        \item Suppose $\kappa=\sup_{\alpha\in Q}\alpha$ where $Q\subseteq\kappa$, (in other words, $Q$ is a cofinal subset of $\kappa$)then $\kappa=\sum_{\alpha\in Q}\overline{\overline{\alpha}}$
        \item Suppose $\sum_{i\in I}\lambda_i=\kappa$ where $\overline{\overline{I}}<\kappa$
        and, for each $i\in I,\lambda_i<\kappa.$ Put $Q=\{\lambda_i:i\in I\}$. Then $\kappa=\sup_{i\in I}\lambda_i$
    \end{enumerate}
\end{theorem}
An infinite cardinal $\lambda$ is called regular if $\lambda$ has no cofinal subset 
of power less than $\lambda$ (otherwise, $\lambda$ is singular).
By theorem 7.39. we can also say: $\lambda$ is regular if and only if $\lambda$
cannot be expressed as a sum of fewer than $\lambda$ cardinals each less than $\lambda$.

Then $\aleph_0$ is regular, and 
\begin{theorem}
    If $\kappa$ is infinite, then $\kappa^+$ is regular.
\end{theorem}
The remaining cardinals are the limit cardinals greater than $\aleph_0$.
The first one $\aleph_{\omega}$ is singular, since $\aleph_{\omega}=\sum_{n\in\omega}\aleph_n$.
In general, if $\alpha$ is a countable limit ordinal, then $\aleph_{\alpha}$ is singular.

Regular limit cardianls are also called weakly inaccessible. An infinite 
cardinal $\kappa$ is called inaccessible if $\kappa$ is regular and $2^{\lambda}<\kappa$
whenever $\lambda<\kappa.$
\section{THE AXIOM OF REGULARITY}
\subsection{Partial universe and the axiom of regularity}
By recursion, put
\[ 
\begin{cases}
    V_0=\varnothing\\
    V_{\alpha+1}=P(V_{\alpha})\text{ for any ordinal }\alpha\\
    V_{\delta}=\cup_{\alpha<\delta}V_{\alpha} \text{ for any limit ordinal }\delta\\
\end{cases}
\]
Set of the form $V_{\delta}$ will be called partial universe.
We write $Vx$ to mean that for some $\alpha,x\in V_{\alpha}$.
\begin{theorem}
    \begin{enumerate}
        \item $V_{\alpha}$ is a transitive set (i.e., if $x\in y\in V_{\alpha}$ then $x\in V_{\alpha}$).
        \item If $\alpha<\beta$ then $V_{\alpha}\subseteq V_{\beta}$.
        \item $V_{\alpha}$ is supertransitive (i.e., transitive and if $x\subseteq y\in V_{\alpha}$ then $x\in V_{\alpha}$)
        \item The predicate $V$ is `supertransitive'; i.e, if $x\in y$ or $x\subseteq y$, and $Vy$ then $Vx$.
    \end{enumerate}
\end{theorem}
If $Vx$ then we put Rk $x$, the rank of $x$, equal to the least $\alpha$
such that $x\in V_{\alpha+1}$(or, what is the same, $x\subseteq V_{\alpha}$).
\begin{theorem}
    \begin{enumerate}
        \item If $Vx,Vy$, and $x\in y$, then Rk $x<$ Rk $y$.
        \item $V_{\alpha}=\{x:Vx \text{ and Rk}x<\alpha \}$.
        \item  For any ordinal $\alpha,V\alpha$ and Rk $\alpha=\alpha$
    \end{enumerate}
\end{theorem}
\begin{theorem}
    If, for any $y\in x,Vy$, then $Vx$.
\end{theorem}
\end{document}