\documentclass[a4paper,11pt]{article}%必须以此为开头,可以在[]内设置栏数,单双面,横竖向
\usepackage{latexsym}%符号字体
\usepackage{makeidx}%制作索引
\makeindex
\usepackage{ifthen}%提供分支语句
\usepackage{graphicx}%用于插入图片
\usepackage{amsmath}%用于数学公式
\usepackage{IEEEtrantools}%用于使用IEEE数学公式排版工具
\usepackage{amsfonts}%用于其他字体的数学符号
\usepackage{amsthm}%提供证明,定理等环境
\usepackage{amssymb}%用于提供各种数学符号
\usepackage{mathrsfs}%用于提供花体字母
\usepackage{verbatim}%使用\verbatiminput{filename}来直接导入文件中的文本内容
\usepackage{layouts}%用于设置页面布局
\usepackage{calc}%允许一些常量参量用算术表达式代替
\usepackage{indentfirst}
\usepackage{hyperref}
\usepackage{makecell}%允许表格的单元格内换行
\usepackage{bm}%使用bm来对希腊字母加粗
\usepackage{longtable}
\usepackage{slashed}
\theoremstyle{remark}
\newtheorem*{remark}{remark}
\theoremstyle{definition}
\newtheorem{problem}{Problem}[subsection]
\newcommand*{\abs}[1]{\lvert #1 \rvert}
\author{Fan}
\title{Set Theory Exercises}
\date{2023.7.21}
\begin{document}
\maketitle
\pagestyle{plain}
\tableofcontents
\printindex
\section{Sets and Relations and Operations Among Them}
\subsection{Set algebra and the set-builder}
\begin{problem}
State and prove the dual law of \[A\cap (B\cup C)=(A\cap B)\cup(A\cap C)\]

dual law:
\[A\cup(B\cap C)=(A\cup B)\cap(A\cup C)\]
\begin{proof}
If $x$ is in $A\cup (B\cap C)$, then $x$ must be either in A or $B\cap C$.
If $x$ is in $A$, then it must be in $A\cup B$ and $A\cup C$, so it is in 
$(A\cup B)\cap(A\cup C)$. Otherwise, $x$ is in $B\cap C$, which means that 
$x$ is both in $B$ and $C$, consequently being both in $A\cup B$ and $A\cup C$.
So in every possible cases, $x$ is in $(A\cup B)\cap(A\cup C)$.

If $x$ is in $(A\cup B)\cap(A\cup C)$, then $x$ must both in $A\cup B$ and 
$A\cup C$. If $x$ is in $A$, then $x$ is in $A\cup (B\cap C)$, otherwise, $x$
must be both in $B$ and $C$, consequently in $B\cap C$ and $A \cup (B \cap C)$.
So in every possible cases. $x$ is in $A\cup (B\cap C)$.

\end{proof}
\end{problem}
\begin{problem}
   Prove 
\[\widetilde{A\cup B}=\tilde{A}\cap\tilde{B} \]
\begin{proof}
    If $x$ is in $\tilde{A}\cap \tilde{B}$, then $x$ is neither in $A$ or $B$,
    which means that $x$ isn't in the union of $A$ and $B$. In other word, 
    $x$ is in $\widetilde{A\cup B}$.

    If $x$ is  in  $\widetilde{A\cup B}$, then $x$ isn't in the union of
    $A$ and $B$. In other word, $x$ is neither in $A$ and $B$, so $x$ is in 
    $\tilde{A}\cap \tilde{B}$.
\end{proof} 
\end{problem}
\begin{problem}
    Prove \[A\cap(B-C)=(A\cap B)-(A\cap C)\]
    \begin{proof}
        If $x$ is in LHS, then $x$ is in both $A$ and $B-C$, which means
        that $x$ is in $A$ and $B$ but $x$ isn't in $C$.
        So $x$ is in $A\cap B$ but not in $A\cap C$. In other word, $x$ is in
        RHS.

        If $x$ is in RHS, then $x$ is in $A\cap B$,but not in $A\cap C$. So,
        first of all, $x$ must be in $A$ and $B$. Then, in order not to be in
        $A\cap C$, $x$ isn't in $C$. So $x$ is in $A$ and $B$, but not in $C$.
        In other word, $x$ is both in $A$ and $B-C$, so $x$ is in LHS.
    \end{proof}
\end{problem}
\begin{problem}
    Show that $\circleddash$ is associative,i.e $A\circleddash(B\circleddash C)=(A\circleddash B)\circleddash C$.
    \begin{proof}
%    \begin{align*}
%        \phantom{\Leftrightarrow}&x\in A\circleddash (B\circleddash C)\\
        %\Leftrightarrow &x\in(A-(B\circleddash C))\cup((B\circleddash C)-A)\\
        %\Leftrightarrow &x\in(A-((B-C)\cup(C-B)))\cup(((B-C)\cup(C-B))-A)\\
        %\Leftrightarrow &x\in(A\cap\widetilde{(B-C)\cup(C-B)})\cup((B-C)\cup(C-B)\cap\tilde{A})\\
        %\Leftrightarrow &x\in(A\cap\widetilde{(B\cap\tilde{C})\cup(C\cap\tilde{B})})\cup((B\cap\tilde{C})\cup(C\cap\tilde{B})\cap\tilde{A})\\
        %\Leftrightarrow &x\in(A\cap\widetilde{(B\cap\tilde{C})}\cap\widetilde{(C\cap\tilde{B})})\cup((B\cap\tilde{C})\cup(C\cap\tilde{B})\cap\tilde{A})\\
        %\Leftrightarrow &x\in(A\cap(\tilde{B}\cup C)\cap(\tilde{C}\cup B))\cup((B\cap\tilde{C})\cup(C\cap\tilde{B})\cap\tilde{A})\\
        %\Leftrightarrow &x\in((\tilde{B}\cup C)\cap(\tilde{C}\cup B)\cap A)\cup((B\cap\tilde{C})\cup(C\cap\tilde{B})\cap\tilde{A})\\
    %\end{align*}
Now we are going to prove
\[A\circleddash (B\circleddash C)=\{x:x \text{is in A or B or C, but isn't in any of their intersection}\}\]
    If $x$ is in $A\circleddash(B\circleddash C)$, $x$ is either in $A$ or $B\circleddash C$.
    In the first case, $x$ isn't in $B\circleddash C$. Due to the symmtry, we assume that 
    $x$ is both in the union of $A$ and $B$ and the union of $A$ and $C$, so that $x$ is in the union
    of $B$ of $C$, which is contrary to previous argument.
    
    In the second case, $x$ isn't in $A$ and the union of $B$ and $C$. So that $x$ isn't in neither the
    union of $A$ and $B$ nor the union of $A$ and $C$.

    So that $x$ is the RHS.

    If $x$ is in the RHS.
    Obviously, $x$ is in $B\circleddash C$.
    If $x$ isn't in the LHS, then $x$ must be in the union of $A$ and $B\circleddash C$,
    so that $x$ must be either in the union of $A$ and $B$ or the union of $A$ and $C$, which is
    contrary to the assumption.

    So that
\[A\circleddash (B\circleddash C)=\{x:x \text{is in A or B or C, but isn't in any of their intersection}\}\]

Due to RHS's symmetry, we can replace A with C.

Meanwhile,
\[A\circleddash(B\circleddash C)=(B\circleddash C)\circleddash A=(C\circleddash B)\circleddash A\]
So we only need to prove thet the $A$ in the LHS can be replaced by $C$, which  is true because of the symmtry of 
the RHS of the equation we proved earilier.
    \end{proof}
%    \begin{remark}
%        Because $\widetilde{A\cup B}=\tilde{A}\cap\tilde{B}$,so
        %\[\widetilde{A\cap B}=\tilde{A}\cup\tilde{B}\]
    %\end{remark}
\end{problem}
\begin{problem}
    Prove that 
    \[\text{if}A\cap C=B \text{and} A\cup C =B \cup C \text{then}A=B.\]
    \begin{proof}
        Suppose $x$ is in $A$. If $x$ is also in $C$, then $x$ is in $A\cap C=B\cap C$,
        so that $x$ is also in $B$.
        If $x$ isn't in $C$, then $x$ is still in $A\cup C=B\cup C$, so that $x$ is
        either in $B$ or $C$. However $x$ isn't in $C$, so that $x$ is also in $B$.

        So in all possible cases, $x$ is in $B$.

        Due to the symmetry, we can guarantee that, supposing $x$ is in $B$, then it must be $A$ also.
    \end{proof}
\end{problem}
\begin{problem}
    (1),(2),(3) stands for an asserting expression, a naming expression, and neither of them respectively.
   \[ \begin{array}{cc}
        expression & type \\
        \hline 
        \int_{a}^{b}x^2d&(2)\\
       \text{ the unique x such that} & (3)\\
        \exists x(x>y)& (1)\\
        (A\cup B)\cup C& (3)\\
        A\cup B\in C& (3)\\
        \{x:x<2\}&(1)\\
    \end{array}
\]
\end{problem}
\begin{remark}
    The correct answer is (3)(3)(1)(2)(1)(2).
\end{remark}
\subsection{Russell's paradox}
\index{unsolved problem 1.2.1}
\begin{problem}
    Obtain a set $t$ such that given any $C$, $t\notin C$ and also (a)
    $t$ is a subset of $C$, and (b) $t$ is a specific ('definable') set.
\end{problem}
\subsection{Infinite unions and intersections}
\begin{problem}
    Prove de Morgan's law's first part.
    \begin{proof}
       We can use deduction.
       It is known that
       \[\widetilde{A\cup B}=\tilde{A}\cap \tilde{B}\] 
       Suppose that when there is k elements in set I, the law establish.
        When there is k+1 elements in set I,
        \[\widetilde{\bigcup_{i\in\{1,2,\cdots,k\}}\mathfrak{G}_i}=\bigcap_{i\in\{1,2,\cdots,k\}}\tilde{\mathfrak{G}}_i\]
        \[\widetilde{\bigcup_{i\in\{1,2,\cdots,k+1\}}\mathfrak{G}_i}=\widetilde{\bigcup_{i\in\{1,2,\cdots,k\}}\mathfrak{G}_i\cup \mathfrak{G}_{k+1}}
        =\widetilde{\bigcup_{i\in\{1,2,\cdots,k\}}\mathfrak{G}_i}\cap \tilde{\mathfrak{G}}_{k+1}=\bigcap_{i\in\{1,2,\cdots,k+1\}}\tilde{\mathfrak{G}}_i\]
    \end{proof}
\end{problem}
\begin{problem}
    prove the first part of the distributive laws.
    \begin{proof}
       It is known that
       \[A\cap(B\cup C)=(A\cap B)\cup (A\cap C)\] 
       If when there is k elements in the set I, the law establishs.
       
       Then, when there is k+1 elements in the set I:
       \[B\cap\bigcup_{i\in \{1,2,\cdots,k\}}A_i=\bigcup_{i\in\{1,2,\cdots,k\}}(B\cap A_i)\]
       \[
        \begin{array}{rl}
            B\cap\bigcup_{i\in\{1,2,\cdots,k+1\}}A_i&=(B\cap\bigcup_{i\in\{1,2,\cdots,k\}}A_i)\cup (B\cap A_{k+1})\\
                                                   &= (B\cup \bigcup_{i\in\{1,2,\cdots,k\}})\cap(B\cup A_{k+1})\\
                                                   &=\bigcup_{i\in\{1,2,\cdots,k\}}(B\cap A_i)\cap (B\cup A_{k+1})\\
                                                   &=\bigcup_{i\in\{1,2,\cdots,k+1\}}(B\cap A_i)\\
        \end{array}
       \]
    \end{proof}
\end{problem}
\begin{problem}
    In the plane, let $C_r=\{P:d(\mathfrak{O} ,\mathfrak{P} <r )\}$ and $C_r^*=\{P:d(\mathfrak{O}, \mathfrak{P} )\leq r\}.$

    What is $\bigcap^{\infty}_{n=1}{C_1+\frac{1}{n}}$?
    
    The answer is $C_i^*.$
\end{problem}
\begin{problem}
    In each expression classify each occurrence of any variable as free or bound.
\begin{enumerate}
    \item For all y, y<x.
    \item $y=\int_{1}^{u}(x^2+y)dy$
    \item $\sum_{i=1}^{n}(i^2+1)$
    \item For any positive real number y there is exactly one positive real number x such that $x^2=y$.
\end{enumerate}

For (1), y is bounded, x is free.

For (2), u is free, x is free, y is bounded.

For (3), i is bounded, n is free.

For (4), y is free, x is bounded.
\end{problem}
\subsection{Ordered couples and Cartesian products}
\begin{problem}
    Prove 
    \[\text{If} (a,b)=(c,d)\text{then} a=c \text{and}b=d\]
\begin{proof}
    According to the definition:
    \[(a,b)=\{\{a\},\{a,b\}\},(c,d)=\{\{c\},\{c,d\}\}\]
    Since this two ordered couples are equal, then the two sets on the RHS 
    are equal to. If two sets are equal, they must have same number  of element.
    So $\{a\}=\{c\},\{a,b\}=\{c,d\}$,the former equation shows that $a=c$.
    If $a=d$,then $a=c=d$. Due to the latter equation, a,b,c,d are the same, which of
    cause lead to $a=c$ and $b=d$.
    If $a\neq d$,then we must have $b=d$, so that we have $a=c$ and $b=d$.
\end{proof}
\end{problem}
\begin{problem}
    Prove
    \[A\times \bigcup_{i\in I}\mathfrak{B}_i=\bigcup_{i\in I}(A\times \mathfrak{B}_i)\]
\begin{proof}
   % \[\begin{array}{rl}
        LHS=$\{(a,b):a\in A,b\in \bigcup_{i\in I}\mathfrak{B}_i\}$
      %     &=\bigcup_{i\in I}\{(a,b):a\in A,b\in \mathfrak{B}_i\}\\
      %     &=\bigcup_{i\in I}(A\times \mathfrak{B}_i)
   % \end{array}\]
   Suppose ordered couple (a,b) is in LHS, then a is in $A$ and b is in the 
   union of each $\mathfrak{B}_i$, which means that it must be in one of them,
   so that $x$ must be in one of the set $(A\times \mathfrak{B}_i)$. Therefore,
   the ordered couple (a,b) is also in RHS. 
   
   Suppose ordered couple (a,b) is in RHS, then it is in one of the sets $A\times \mathfrak{B}_i$.
   so that a is in $A$ and b is in one of the sets $\mathfrak{B}_i$. Therefore, 
   b is in the union of $\mathfrak{B}_i$ and the ordered couple (a,b) is in
   the LHS.
\end{proof}
\end{problem}
\begin{problem}
    Prove 
    \[A\times(B-C)=A\times B-A\times C\]
    \begin{proof}
        If ordered couple $(a,b)$ is in the LHS, then a is in $A$ and 
        b is in $B-C$, which means that b is in $B$ but no in $C$, so
        that the ordered couple is in $A\times B$ but not in $A\times C$,
        which means that it is in the RHS.

        If ordered couple (a,b) is in the RHS, then it is in $A\times B$,
        which means that a ia in $A$ and b is in $B$. However, it is not 
        in $A\times C$, so b isn't in $C$. In other word, b is in $B-C$, and
        the ordered couple is in the LHS.
    \end{proof}
\end{problem}
\index{unsolved problem 1.4.4}
\begin{problem}
    Use the previous two equation to infer 
    \[A\times \bigcap_{i\in I}\mathfrak{B}_i=\bigcap_{i\in I}(A\times \mathfrak{B}_i)\]
\end{problem}
\begin{problem}
   Prove 
   \[(A\cap B)\times (C\cap D)=(A\times C)\cap(B\times D)\]
\begin{proof}
   Suppose ordered couple (x,y) is in the LHS, then x is in $A\cap B$, and
   y is in $C\cap D$, which means that x is both in A and B and y is both
   in C and D. Therefore, the ordered couple (x,y) is both in $A\times C$ and
   $B\times D$. 

    Suppose ordered couple (x,y) is in the RHS, then (x,y) is both in $(A\times C)$
    and $(B\times D)$, which means that x is both in $A$ ,$B$, therefore, $A\cap B$, while y is both in $C$ and $D$ 
    therefore, in $C\cap D$.So that the ordered couple is also in the LHS.
\end{proof}
\end{problem}
\subsection{Relations and functions}
\begin{problem}
    Prove
    \[ \Breve{R/S}=\Breve{S}/\Breve{R}\]
\begin{proof}
    Suppose $(x,y)$ is in the LHS, then $(y,x)$ is in $R/S$,
    so that there exist z such that $(y,z)$ is in $R$ and $(z,x)$ is in $R$.
    Therefore, $(z,y)$ and $(x,z)$ is in $\Breve{R}$ and $\Breve{S}$, respectively.
    Hence, $(x,y)$ is also in $\Breve{S}/\Breve{R}$.

    Suppose $(x,y)$ is in the RHS, then there exists z such that $(x,z)$ and
    $(z,y)$ is in $\Breve{S}$ and $\Breve{R}$ respectively. Hence, $(z,x)$ and
    $(y,z)$ is in $S$ and $R$ respectively. Therefore, $(y,x)$ is in $R/S$, which
    means that $(x,y)$ is in $\Breve{R/S}$.
\end{proof}
\end{problem}
\begin{problem}
    Prove
        \[R[\bigcup_{i\in I}\mathfrak{a}_i]=\bigcup_{i\in I}R[\mathfrak{a}_i]\]
\begin{proof}
    Suppose $x$ is in the LHS, then it is in the image of the union of $\mathfrak{a}_i$.
    Therefore, $x$ must be in the image of one $\mathfrak{a}_i$. In other word,
    $x$ is in one of the image of $\mathfrak{a}_i$. Hence, $x$ is in the RHS.

    Suppose $x$ is in the RHS, then $x$ is in one of the image of $\mathfrak{a}_i$.
    Therefore, $x$ is definitely in the image of the union of $\mathfrak{a}_i$. Hence, $x$
    is also in the LHS.
\end{proof}
\end{problem}
\begin{problem}
    Prove
    \[f^{-1}(A-B)=f^{-1}A-f^{-1}B.\]
    \begin{proof}
        If $x$ is in the LHS, then $f(x)\in A-B$, which means that $f(x)$ is
        in $A$ but not in $B$. Thereforem, $x$ is in $f^{-1}A$ but not in $f^{-1}B$.
        Hence, $x$ is in the RHS.

        Suppose $x$ is in the RHS, then $x$ is in $f^{-1}A$, but not in $f^{-1}B$.
        Therefore, $f(x)$ is in $A$ but not in $B$. Hence $f(x)$ is in $A-B$, and 
        $x$ is in the LHS.
    \end{proof}
\end{problem}
\begin{problem}
    Prove: If, for any $A$, $B$,$R[A\cap B]=R[A]\cap R[B],$ then $\Breve{R}$ is a function.
    \begin{proof}
        If $\Breve{R}$ isn't a funciton. Then, there exists $x,a,b$ such that 
        $\Breve{R}[\{x\}]=\{a\}=\{b\}$. Therefore, $\{x\}=R[\{a\}]=R[\{b\}]$.
        Since $a\neq b$, 
        \[R[\{a\}\cap\{b\}]=R[\varnothing]=\varnothing\]
        However,
        \[R[\{a\}\cap\{b\}]=R[\{a\}]\cap R[\{b\}]=\{x\}\cap\{x\}=\{x\}\neq\varnothing,\]
        whichi is a contradiction.
    \end{proof}
\end{problem}
\begin{problem}
    Prove
    \[R/\bigcup_{i\in I}S_i=\bigcup_{i\in I}R/S_i\]
    \begin{proof}
        Suppose $(x,y)$ is in the LHS, then there exist z and n such that
        $(x,z)$ is in $R$, and $(z,y)$ is in $S_n$. Therefore $(x,y)$ is in 
        $R/S_n$, hence $(x,y)$ is in the RHS.

        Suppose $(x,y)$ is in the RHS, then there exists n such that $(x,y)\in R/S_n$.
        Therefore, there exists z such that $(x,z)$ and $(z,y)$ is in $R$ and 
        $S_n$ respectively. Hence $(z,y)$ is in $\bigcup_{i\in I}S_i$ and $(x,y)$
        is also in LHS.
    \end{proof}
\end{problem}
\begin{problem}
    If $R$ is a function then $R/\bigcap_{i\in I}S_i=\bigcap_{i\in I}R/S_i$
    \begin{proof}
       This is a special case of the last question. 
    \end{proof}
\end{problem}
\begin{problem}
    Prove
    \[(R/S)[A]=S[R[A]]\]
    \begin{proof}
        Suppose b is in the LHS, then there exists a such that $(a,b)\in (R/S)$.
        Therefore, there exist z such that $(a,z)$ and $(z,b)$ is in R and S respectively.
        Hence z is in $R[A]$, and consequently b is in the RHS.

        Suppos b is in the RHS, then there exist z such that $z\in R[A]$ and 
        $(z,b)\in S$. Therefore, there exists a such that $a\in A$ and $(a,z)$ is in
        R. Hence, $(a,z)$ and $(z,b)$ is in R and S respectively. In other word, $(a,b)$
        is in $(R/S)$, and b is in the LHS.
    \end{proof}
\end{problem}
\begin{problem}
    Since the laws for $R-$image and \verb+R/_+ seem to be parallel, try to reduce the 
    first to the second. Specially, fill in the blank and prove :$\{z\}\times R[A]=$\verb+_/R+.
    (Here z is any fixed thing.(Aside: $U$ and $\{z\}\times U$ are practically the same thing, 
    so this does `reuce ' image to /.))

    Suppose $(z,x)$ is in the LHS, then $x$ is in $R[A]$. Therefore, there exists 
    a in $A$ such that $(a,x)$ is in $R$. Now we have $(z,x)$ in the LHS, $(a,x)$ in $R$, so
    we can replace \verb+_+ with $\{z\}\times A$.

    Now we are going to prove
    \[\{z\}\times R[A] =(\{z\}\times A)/R\]
    \begin{proof}
       Suppose $(z,b)$ is in the LHS, then $b$ is in $R[A]$.
       Therefore, there exists a in $A$ such that $(a,b)$ is in $R$.
       Since $(z,a)$ and $(a,b)$ is in $\{z\}\times A$ and $R$ respectively,
       $(z,b)$ is also in the RHS. 

       Suppose $(x,b)$ is in the RHS, then there exists a such that $(x,a)$ and $(a,b)$
       is in $\{z\}\times A$ and $R$ respectively. Obviously, $x=z$, we just replaced x with z
       in the following proof. Since $(a,b)$ is in $R$, b is in $R[A]$. Hence, $(x,b)$ is in the LHS
    \end{proof}
\end{problem}
\subsection{Sets of sets, power set, arbitrary Cartesian product}
\begin{problem}
    Prove
    \[P(A)\underset{H}{\sim}\{u,v\}^A\]
    where $H(B)=c_B$
    \begin{proof}
        We can choose a set $B$ from $P(A)$, so that $B$ is the subset of $A$.
        $H(B)=c_B$ is then a function that is on A to  $\{u,v\}$.
        If $c_B$=$c_D$, obviously, $B=D$ and for any function in RHS, we can find a 
        corresponding subset in A. So it is a one-to-one function.
    \end{proof}
\end{problem}
\begin{problem}
    Prove that if $x$ is in $\bigcup_{f\in\prod J_i,i\in I}\bigcap_{i\in I}\mathfrak{a}_{if(i)}$,
    then $x$ is also in $\bigcap_{i\in I}\bigcup_{j\in J_i}\mathfrak{a}_{ij}$.
    \begin{proof}
       For any $x$ in  $\bigcup_{f\in\prod J_i,i\in I}\bigcap_{i\in I}\mathfrak{a}_{if(i)}$
       For every i in I, there exists $f(i)$ in $J_i$ such that $x$ is in $\mathfrak{a}_{ij(i)}$.
       Hence for every i in I, $x$ is in $\bigcup_{j\in J_i}\mathfrak{a}_{ij}$, and in their union.
    \end{proof}
\end{problem}
\begin{problem}
    Prove
    \[\prod_{i\in I}\bigcup_{j\in J_i}\mathfrak{a}_{ij}=\bigcup_{f\in \prod J_i,i\in I}\prod_{i\in I}\mathfrak{a}_{if(i)}\]
    \begin{proof}
    Suppose $f$ is in the LHS, then, for every i in I, we can select an element from
    $\bigcup_{j\in J_i}\mathfrak{a}_{ij}$, which is corrospondent to a j in $J_i$, forming a function $g$.
    Hence $g$ is in $\prod J_i$, and therefore f is in $\prod _{i\in I}\mathfrak{a}_{ig(i)}$.
    So $f$ is in the RHS.
    
    Suppose $f$ is in the RHS, then there exists a function $g$ in $\prod_{i\in I}J_i$,
    that for every i in I, g(i) is in $J_i$, and then $\prod_{i\in I}\mathfrak{a}_{ig(i)}$
    corrospondent to a element in $\mathfrak{a}_{ig(i)}$. In  other word, 
    for every i in I, $f$ has corrospondent to a element in $\mathfrak{a}_{ig(i)}$, therefore
    in the union of $\mathfrak{a}_{ij}$. Hence, for every i in I, $f$ actually corrospondent 
    to a element in $\bigcup_{j \in J_i}\mathfrak{a}_{ij}$, and $f$ is in LHS too. 
    \end{proof}
\end{problem}
\begin{problem}
    Does the previous equation hold if $\bigcup$ is replaced everywhere by $\bigcap$?

    The answer is no. It is very easy to construct $\mathfrak{a}_{ij}$ such that 
    $\bigcap_{j\in J_i}\mathfrak{a}_{ij}=\varnothing$,and $\bigcup_{j\in J_i}\mathfrak{a}_{ij}\neq\varnothing$.
    In this case, the LHS is equal to $\varnothing$, but the RHS equals doesn't equal to $\varnothing$.
    (We can let $\prod_{i\in I}J_i $ only have one element.)
\end{problem}
\begin{problem}
    Let $f$ be on $A$ onto $B$. Show there exists $g:B\rightarrow A$ such that for
    each $b\in B$,$f(g(b))=b$.Show also that $g$ is one-to-one.
    \begin{proof}
        Since $f$ is a function on $A$ onto $B$, for each $b\in B$, there at
        least exists a $a\in A$ such that $f(a)=b$, so we can let $g(b)=a$.

        If $b,c\in B , g(b)=g(c)$ , then $b=f(g(b))=f(g(c))=c$, so it is one-to-one.
    \end{proof}
\end{problem}
\subsection{Structures}
\begin{problem}
    Complete theorem 1.16.

    \begin{enumerate}
        \item $\underbar{A}\underset{Id_{\underbar{A}}}{\cong}\underbar{A}$
        \item if $\underbar{A}\underset{f}{\cong}\underbar{B},$then $\underbar{B}\underset{f^{-1}}{\cong}\underbar{A}$
        \item if $\underbar{A}\underset{f}{\cong}\underbar{B}$ and $\underbar{B}\underset{g}{\cong}\underbar{C}$,then $\underbar{A}\underset{g\circ f}{\cong}\underbar{C}$.
    \item $\underbar{A}\cong \underbar{A}$
    \item if $\underbar{A}\cong \underbar{B}$, then $\underbar{B}\cong \underbar{A}$
    \item if $\underbar{A}\cong \underbar{B}$ and $\underbar{B}\cong \underbar{C}$ then $\underbar{A}\cong \underbar{C}$.
\end{enumerate}
\end{problem}
\begin{problem}
    Prove theorem 1.17.
    \begin{proof}
       If f is an isomorphism of $\underbar{A}$ into $\underbar{A}'$,
       we can regard $\underbar{A}'$ itself as $\underbar{A}'$'s substructures.
       
       If f is an isomorphism of $\underbar{A}$ into $\underbar{B}$, which is a
       substructure of $\underbar{A}'$, then, $A\underset{f}{\sim}B$,and for any
       $x,y\in A,xRy$ if and onlly if $f(x)R'f(y)$.\index{unsolved problem 1.7.2}
    \end{proof}
    \begin{remark}
        Maybe `an isomorphism into' means $A$ isn't necesserily one-to-one 
        to $A$'.
    \end{remark}
\end{problem}
\begin{problem}
    Prove that if $\underbar{A}=(A,R)$ and $A\underset{f}{\sim}A'$(A' is just a set),
    then there is exactly one $R'$ such that $(A,R)\underset{f}{\cong}(A',R')$.
    \begin{proof}
        when $(A,R)\underset{f}{\cong}$,for $x,y\in A$,$xRy$ if and only if
        $f(x)R'f(y)$.
        Since $R$ is a set of order couple, we can let $R'={(f(a),f(b)):(a,b)\in R}$.
        If there is another $R''$, we can easily prove that $R'$ and $R''$ share same 
        elements.
    \end{proof}
\end{problem}
\subsection{Partial order and orders}
\begin{problem}
    Prove that every partial order $(A,\leq)$ is isomorphic to $(Q,\subseteq_Q)$
    for some $Q$.

    We can construct $Q=\{\{b:b\leq a\}:a\in A\}$.
    It is obviously that the mapping $a\in A \rightarrow \{b:b\leq a\}$ is 
    one-to-one, and $x\leq y$ iff $\{b:b\leq x\}\subseteq\{b:b\leq y\}$.(because of the transitive)
\end{problem}
\begin{problem}
    Prove theorem 1.19. Then, looking at your proof, improve it by weakening its
    hypotheses and then prove it agian.
\begin{proof}
    We need to prove that f is one-to-one and $xRy$ whenver $f(x)R'f(y)$.
    Suppose $f(x)R'f(y)$, if $x\Breve{R}y$, then $f(x)\Breve{R}'f(y)$, thus $f(y)R'f(x)$.
    Due to the transitive, $f(x)R'f(x)$, which is contrary to the irreflective.
    If $x=y$, then $f(x)=f(y)$, however, this is also contrary to the irreflective.
    Due to the connect, $xRy$.

    Now we have to prove that f is one-to-one. If f(x)=f(y), then due to the 
    irreflective $f(x)\slashed{R'}f(y)$ and $f(x)\slashed{\Breve{R}'}f(y)$, consequently
    $x\slashed{R}y$ and $x\Breve{\slashed{R}}y$. Due to the connected, x=y. Since f is onto A',
    now f is one-to-one.
\end{proof}
 
During the proof, we only need  $(A',R')$ to be  irreflective and transitive
and $(A,R)$ to be connected. Meanwhile, f should still preserve order.

The proof remain the same.
\begin{remark}
    $(A',R')$ just need to be asymmetric.
\end{remark}
\end{problem}
\begin{problem}
    Prove theorem 1.20
    \begin{proof}
        If x is a minimum, then, for all $y\in B,x\leq y$, which means that
        $x<y$ or  $x=y$. If $y<x$, then $x<x$ which is contrary to irreflective,
        or $y<y$, which is also contrary to irreflective. So, for all $y\in B$,
        $y\slashed{<}x$.

        If a and b are both $\underbar{B}$'s minimum element, then $a\leq b,b\leq a$,
        due to the antisymmetric, $a=b$.

        If $\underbar{A}$ is an order, then $\underbar{B}$ is also an order.
        If $a$ is a minimum of B, then, for any $b\in B$, $a\leq b$, if for some 
        $c\in B$ such that $c<a$, then $a\leq c$, which means that $a<c$ or$a= c$.
        If $a<c$, it is contrary to the irreflective. If $a=c$, then $c<c$, which is 
        also contrary to irreflective. So a is also a minimal. 
        
        If $a$ is a minimal of $B$, then for any $b\in B$, $b\slashed{<}a$.
        Since $\underbar{B}$ is connected, $b=a$ or $b>a$, which means that $a\leq b$.
        Thus $a$ is also a minimum.
    \end{proof}
\end{problem}
\begin{problem}
    Prove theorem 1.21.3
    \begin{proof}
        For every proper initial segment $B\subset A$,
        we are going to prove that $B=$ Prod the first element of $A-B$.
        Suppose $x$ is in the RHS, then $x<\text{the first element of} A-B$,
        so that $x$ isn't in $A-B$, thus in B.

        Suppose $x$ is in the LHS, then if $x$ isn't in the RHS, then $x\geq $
        the first element of $A-B$. Since $x$ is in $B$, so $x$ counldn't in $A-B$,
        so that $x\geq $the first element of $A-B$. Since $B$ is a initial segment,
        the first element of $A-B$ should be in $B$, which is a contradiction.
    \end{proof}
\end{problem}
\begin{problem}
    Prove that for any order $\underbar{A}=(A,\leq)$, the following are equivalent:
    \begin{enumerate}
        \item $\underbar{A}$ is a well-order.
        \item Every non-empty final segment of $\underbar{A}$ has a least element.
        \item Every proper initial segment of $\underbar{A}$ is of the form Pred $a$(for some $a\in A$)
    \end{enumerate}
    \begin{proof}
        We have already proof $(1)\rightarrow(3)$.

        Since every non-empty subset of $A$ has a first element, we have $(1)\rightarrow(2)$.

        For every non-empty subset $B$, $\underbar{B}$ is also an order, and we can 
        regard B itself as a final segment. According to (2), it has a least element.
        Hence, we have $(2)\rightarrow(1)$
    \end{proof}
\end{problem}
\begin{problem}
    Let $\underbar{A}=(A,\leq)$ be any order. Show the following are equivalent:
    \begin{enumerate}
        \item Every non-empty subset of $A$ bounded above has a least upper bound.
        \item Every non-empty subset of $A$ bounded below has a greatest lower bound.
        \item Given any non-empty initial segment $U$ having a non-empty complement $V$,
        either $U$ has a last element or $V$  has a first.
    \end{enumerate}
    \begin{proof}
    \index{unsolved problem 1.8.6}
    \begin{remark}
        After refering to the hint, we give the following proof. We 
        first prove $(1)\rightarrow(2)$

        Given B is a non-empty subset of A, and $x$ is in A,
        then, $x$ is obviously a upper bound of the set of B's lowerbound.
        According to (1), we have U as the least upper bound of the 
        set of B's lowerbound. Then, for every lowerbound l, $l \leq U$.
        If U is in B, then, obviously, U is the greatest lower bound of B.
        Otherwise, U is not in $B$, then, for every $x$ in $B$, $U<x$,
        so that U is a lower bound, so that U is also the greatest lower bound of B.
        (If not so, then there will be a upper bound of the lower bound of B that is less than
        U, which is contrary to the fact that U is the least lower bound.)

        Due to the symmetry, $(2)\rightarrow(1)$.
    \end{remark}
    \end{proof}
\end{problem}
\begin{problem}
    Assume knowledge of the usual orders $\underbar{R}=(R,\leq)$ of the real
    numbers and $\underbar{Q}=(Q,\leq)$ of the rationals.
    Show true or false:
    \begin{enumerate}
        \item Every non-empty proper initial segment of $\underbar{R}$ is Pred a, for some $a\in R$
        \item Every non-empty proper initial segment of $\underbar{R}$ is Pred $^{\leq}$ a , for some $a\in R$
        \item Every non-empty proper initial segment of $\underbar{R}$ is Pred  a or Pred $^{\leq}$, for some $a\in R$
        \item Every non-empty proper initial segment of $\underbar{Q}$ is Pred a or Pred $^{\leq}$ a (in $\underbar{Q}$ for some $a\in Q$)
    \end{enumerate}
\index{problem 1.8.7 uncorrect}
    The answer is FFFF
    \begin{remark}
        The answer should be FFTF.
    \end{remark}
\end{problem}
\begin{problem}
    Which of $\underbar{N},\underbar{Z},\underbar{Q},\underbar{R}$are dense? discrete? well-ordered? continuous?

    $\underbar{N}$ is discrete, well-ordered,continuous.

    $\underbar{Z}$ is discrete, continuous.

    $\underbar{Q}$ is dense

    $\underbar{R}$ is dense, continuous.
\end{problem}
\section{CARDINAL NUMBERS AND FINITE SETS}
\subsection{Crdinal numbers,+,and $\leq$}
\begin{problem}
    Prove the Exchange Principle
    \begin{proof}
        First, we let $Z=\{t\}\times Y$.
        Then we try to let $Z)(X$.
        So that for any y in Y,$(t,y)\notin X$
    \end{proof}
\end{problem}
\begin{problem}
    Show that in Theorem 2.5, (1) is equivalent to (5).
\begin{proof}
   If  $\kappa\leq\lambda$, then there exists a $\mu$ such that $\lambda=\kappa+\mu$.
    So that for any set $A$ of power $\kappa$, there exists some C such that $A$
    and $C$ are disjoint and $\overline{\overline{A\cup C}}=\lambda$. We can
    We can let $B=A\cup C$

    If for any set $A$ of power $\kappa$, there exists $B\supseteq A$ of power $\lambda$.
    Let $C=B-A$, then $C)(A$,and for $\mu=\bar{C}$, $\lambda=\kappa+\mu$
\end{proof}
\end{problem}
\begin{problem}
    Prove theorem 2.8.1.
    \begin{proof}
        If $\kappa\leq\lambda$, then there is $\mu$ such that $\kappa+\mu=\lambda$.
        Then there is some $A$,$B$ such that $\bar{\bar{A}}=\kappa,\bar{\bar{B}}=\mu,A)(B,\overline{\overline{A\cup B}}=\lambda$
        Since $A\cup B\supseteq B$, it is easy to construct a function on $A\cup B$ onto $B$.
    \end{proof}
\end{problem}
\begin{problem}
    Prove theorem 2.8.3.
    \begin{proof}
        Since $\kappa\leq^*\lambda$, we have a function $f$ on $B$ onto $A$.
        We have $\{x:x \text{is the R-least x such that }f(x)=a, \text{for every a in }B \}$
        \index{unsolved problem 2.1.4.}
    \end{proof}
\end{problem}
\subsection{Natural numbers and finite sets}
\begin{problem}
    Prove theorem 2.9.1 and 2.9.2
    \begin{proof}
        According to the definition. Every $X$ has a 0 in it.
        So 0 is a natural number.

        If $\kappa$ is a natural number, then, there is a $\kappa$ in 
        every $X$. Hence, there is also a $\kappa+1$ in every $X$.
        Therefore, $\kappa+1$ is also a natural number.
    \end{proof}
\end{problem}
\begin{problem}
    Suppose we have a notion $\mathfrak{n}(\kappa+1)$ for which theorem 2.9
    can be proved, i.e., such that we can prove the following:
    \begin{enumerate}
        \item $\mathfrak{n}0$
        \item If $\mathfrak{n}\kappa$ then $\mathfrak{n}(\kappa+1)$
        \item (In general) If $\mathfrak{p}0$ and for any $\lambda$ such
        that $\mathfrak{n}\lambda$ if and only if $\lambda$ is a natural number.
    \end{enumerate}
    \begin{proof}
    Since 0 is a natural number, for any $\lambda$ such that $\mathfrak{n}\lambda,$
    if $\lambda$ is a natual number, then $\lambda+1$ is also a natural number,
    then for all $\lambda$, $\mathfrak{n}\lambda$ and $\lambda$ is a natural number.

    According to the normal induction, we can prove that when $\lambda$ is a 
    natural number, $\mathfrak{\lambda}$.
    \end{proof}
\end{problem}
\begin{problem}
    Prove theorem 2.10 respectively.
\begin{proof}
    We show that for all n, m+n is a natual number, by induction on 
    `n'. Since $m+0=m$, so it is a natural number. Suppose $m+n$ is a 
    natural number, then, because $m+(n+1)=(m+n)+1$, so it is also a natural number by the
    inductive hypothesis.
\end{proof}
\begin{proof}
    Since $\kappa\geq 0$, when $n=0,$ let $\bar{\bar{A}}=\kappa$, then there is a 
    one-to-one function on $A$ to $\varnothing$. Thus $A=\varnothing$
   % then we have $0\leq \kappa\leq n=0$.

    Suppose when $\kappa\leq n$, $\kappa$ is a natural number.
    Then, when $\kappa\leq (n+1)$, if $\kappa\leq n$, by induction hypothesis, 
    it is a natural number. Otherwise, $n<\kappa\leq(n+1)$. If $\kappa<(n+1)$,
    Since $n<\kappa$, $\kappa\neq 0$, so there exists $\lambda$, such that $\kappa=\lambda+1$.
    So $\lambda+1<n+1$ or $\lambda<n<\kappa$. So we have $\lambda+1=\kappa\leq n<\kappa$, which means that
   there is a one-to-one function on $A$ to $B$, and then there is a one-to-one function on $B$ to $A$, 
   where $\bar{\bar{A}}=\kappa,\bar{\bar{B}}=n$,so that there have to be a one-to-one function on $A$ to $B$,
   which means that $n=\kappa$,which is a contradiction. So $\kappa=n+1$, which is a natural number.
\end{proof}
   \begin{remark}
    During the proof, we also prove that if $n<\kappa\leq(n+1)$, then $\kappa=(n+1)$.
   \end{remark}
\begin{proof}
    When $n=0$, we have $\kappa+0=\kappa,\lambda+0=\lambda$, so $\lambda=\kappa$.
    Suppose when $\kappa+n=\lambda+n$, we have $\kappa=\lambda$.
    Then, when $\kappa+(n+1)=\lambda+(n+1)$, we have $(\kappa+n)+1=(\lambda+n)+1$,
    by induction hypothesis and theorem 2.3.5., we have $\kappa+n=\lambda+n$ and $\kappa=\lambda$.
\end{proof}
\begin{proof}
 %  If $m\neq n$, then $m<n$, so that $m+1\leq n$. If $m+k\neq n,k=1,2,\cdots,n$,
  % then $m+k<n,k=1,2,\cdots,n$. Thus $m+n<n$, which is contrary to $m\geq 0$ or $m+n\geq n$.
 If $m\leq n$, then, for some $k$, $n=m+k$ or $m+k=n.$
 If there is a and b such that $m+a=n=m+b$.
 Let $\bar{\bar{M_1}}=\bar{\bar{M_2}}=m,\bar{\bar{A}}=a,\bar{\bar{B}}=b,M_1)(A,M_2)(B$,
 then $M_1\cup A\sim M_2\cup B$. Since $M_1\sim M_2,$ then $A\sim B$, thus $a=b.$

    If $A$ has n element, then $\bar{\bar{A}}=n$, any subset $B$ of $A$ has less than n
    element, so $\bar{\bar{B}}<\bar{\bar{A}}$. Thus, $A$ is not equivalent to a proper subset of itself.

Since $n<n+1$, then n is finite, let $A$ is a set with at least one element and $\bar{\bar{A}}=n+1$,
then we can remove one element and get a subset with cardinal number n. It is obvious that there is 
a one-to-one function on the subset to $A$, and the subset is not equal to $A$. Hence $n<n+1.$

n,m are all natural number, and there is a one-to-one function on $A$ to $B$,
and another one-to-one function on $B$ to $A$, so that the number of elements of $A$ and $B$
should be equal, thus $m=n$.
\end{proof}
\begin{proof}
 When $n=0$, it is obvious that $\kappa\geq 0$.
 Suppose that $n\leq \kappa$ or $\kappa\leq n$.
On one hand, if $\kappa \geq n$, suppose $\kappa=n$, then $\kappa\leq n+1$;
else, $\kappa >n$, then $\kappa\geq n+1.$
On the other hand, if $\kappa\leq n$, then due to $n<n+1$, thus $n\leq n+1$ and $\kappa\leq n+1$.
\end{proof}
\begin{proof}
   We need to proof the connected, reflective, transitive and antisymmetric property.
   Theorem 2.6 show its reflective and transitive property. Theorem 2.10.4 show its antisymmetric property.
   Theorem 2.10.5 show its connected property.

   Since $\kappa\geq 0$, so that 0 is the first element.

   If $m> n$, then there exist one $k$ such that $n+k=m$ and $k\neq 0$.
   So that $k\geq 1$, and $m=n+k\geq n+1$. So $n+1$ is the immediate successor of $n$.

   If n is a natural number, then $n+1$ is also a natural number and also the immediate successor of $n$.
\end{proof}
\begin{proof}
   For every finite set $A$, $A\sim W_{\bar{\bar{A}}}$. Then, according to problem 1.7.3.
   it can be ordered.
\end{proof}
\begin{proof}
    Any finite order subset of an order is also a order, and thus can isomorphic to $W_n$,
    since $W_n$ has the first element, so is this finite subset.
\end{proof}
\begin{proof}
    They are both isomorphic to $W_n$, due to the transitive, they are also isomorphic.
\end{proof}
\begin{proof}
   If $\kappa\leq n$, then there exist a set $A$ such that $\bar{\bar{A}}=\kappa$,
   and there is a one-to-one function on $A$ to $W_n$. Then $A$ is a finite set, 
   and $A\sim W_{\bar{\bar{\kappa}}}\subseteq W_n$. Then $\kappa\leq n$. 
\end{proof}
\end{problem}
\begin{problem}
    Prove theorem 2.11.
    \begin{proof}
        When $n=0$, $\mathfrak{Q}m$ holds for all $m<n$ since m doesn't exist at all.
        Suppose when n, for every $m<n$, we have $\mathfrak{Q}m$, then we also have $\mathfrak{Q}n$,
        which means that, for every $m<n+1$, we have $\mathfrak{Q}m$. By induction hypothesis, we have that,
        for all m, we have $\mathfrak{Q}n$ for every $n<m$, thus we have $\mathfrak{Q}n$ for all n. 
    \end{proof}
\end{problem}
\begin{problem}
    Prove without AC the `axiom of choice for finitely many sets': If $I$ is 
    finite, F is on I, and for each $i\in I,F(i)\neq \varnothing$ then there is 
    a choice function for $F$.
    \begin{proof}
       Since for each $i\in I, F(i)\neq \varnothing,$ then there is at least one element in 
       $F(i)$, and our choice function will pick out this element. 
    \end{proof}
\end{problem}
\begin{problem}
    Prove: If $A$ if finite and $f:A\rightarrow A$ then $f$ is one-to-one 
    if and only if $f$ is onto.
    \begin{proof}
        If $f$ is one-to-one, then the image of $f$ also have cardinal of $A$,
        however, it is also a subset of $A$, so that f is onto $A$.

        If $f$ is onto f, then, The cardinal of the image of $f$ is same as the 
        cardinal of $A$, so that $f$ has to be one-to-one.
        \begin{remark}
            To be more rigorous, we should use previous results to finish the proof.
            Use contradiction to prove the second direction.
        \end{remark}
    \end{proof}
\end{problem}
\subsection{Multiplication and exponentiation}
\begin{problem}
    Prove theorem 2.13.7
    \begin{proof}
        \[\begin{array}{rl}
            \kappa\cdot\lambda&=((\kappa-1)+1)\cdot((\lambda-1)+1)\\
                              &=(\kappa-1)+(\lambda-1)+(\kappa-1)\cdot(\lambda-1)+1\\
                              &=\lambda+\kappa+(\kappa-1)(\lambda-1)-1\\
                              &\geq \lambda+\kappa\\
        \end{array}\]
    \end{proof}
\end{problem}
\begin{problem}
    Prove theorem 2.14.
    \begin{proof}
        We let $A\underset{f}{\sim}A',B\underset{g}{\sim}B'$,
        \[F(h)(g(b))=f(h(b)),\phantom{111}h\in A^B\]
        We only need to prove that this is a one-to-one function onto $A'^{B'}$

        If $F(u)=F(v)$, then, for every b in $B$, we have
        \[f(u(b))=f(v(b))\]
        Since f is a one-to-one function, so that $u(b)=v(b)$,
        which means that $u=v$ and F is a one-to-one function.
        
        For any function u in $A'^{B'}$ such that $u(b)=a$, we have $v$ in $A^B$,
        such that $v(g^{-1}(b))=f^{-1}(u(b))$, and \[F(v)(b)=F(v)(g(g^{-1}(b)))=f(v(g^{-1}(b)))=f(f^{-1}(u(b)))=u(b)\]
    \end{proof}
\end{problem}
\begin{problem}
   Prove  theorem 2.15.1
   \begin{proof}
    We take $F(f)(b,c)=f(c)(b)$, and we are going to prove that 
    \[(A^B)^C\underset{F}{\sim}A^{B\times C}\]
    We need to prove that $F$ is one-to-one and is onto $A^{B\times C}$

    If for $u,v\in (A^B)^C, F(u)=F(v),$ then, for every ordered couple $(b,c) \in B\times C$
    $u(c)(b)=v(c)(b)$, which means that function $u(c)=v(c)$, thus $u=v$, so $F$ is one-to-one.

    For any function $u$ in $A^{B\times C}$, we need to prove that there exists a
    function $v$ in $(A^B)^C$ that $F(v)=u$.

    Suppose $u(b,c)=a$, then we let $v(c)(b)=a$.
\begin{remark}
    We should let $h_c(b)=g((b,c))$ and $f(c)=h_c$ for $g\in A^{B\times C}$
\end{remark}
   \end{proof}
\end{problem}
\begin{problem}
    Give the function F needed for the proof of theorem 2.15.2.

    We want 
    \[A^{B\cup C}\underset{F}{\sim}A^BA^C\]
    where B and C are disjoint.

    Let 
    \[F(f)(a,b)=(f(a),f(b))\]
    \begin{remark}
        The correct answer is $F(f)=(f\upharpoonright B,f\upharpoonright C)$
    \end{remark}
\end{problem}
\begin{problem}
    Give the function F needed for the proof of theorem 2.15.3.

    We want
    \[(AB)^C\underset{F}{\sim}A^CB^C\]

    Let
\[F(f)(c_1,c_2)=(f(c_1),f(c_2))\]
\end{problem}
\begin{problem}
    Prove theorem 2.15.6.
    \begin{proof}
        If $\kappa=\lambda$ or $\mu=0$, then, we have $\mu^{\kappa}=\mu^{\lambda}$.
        Else, we have $\alpha\neq 0$ such that $\lambda=\alpha+\kappa$
        \[\begin{array}{rl}
           \mu^{\lambda}&=\mu^{\kappa+\alpha}\\
                        &=\mu^{\kappa}\mu^{\alpha}\\
                        &\geq\mu^{\kappa}\mu^{0}\\
                        &=\mu^{\kappa}\cdot 1\\
                        &=\mu^{\kappa}\\
        \end{array}\]
    \end{proof}
\end{problem}
\begin{problem}
    Prove $2\kappa\leq 2^{\kappa}$.
    \begin{proof}
    If $\kappa=0$, then $2\kappa=0\leq 2^{\kappa}=1$;
    
    \index{unsolved problem 2.3.7.}
    \begin{remark}
        Consider $2\times A$ and $2^A$, if $\kappa\geq 3$.( The cases $\kappa=0,1,2$ are trival seperately.)
    \end{remark}
    \end{proof}
\end{problem}
\subsection{Definition by induction}
\begin{problem}
    Prove theorem 2.17.1.
    \begin{proof}
        We only need to prove that there is only one function that 
        satisfies the condition.

        Suppose $f$ and $g$ all satisfy the condition, then, 
        $f(0)=g(0)$. We now prove $f=g$ by induction on i;
        Suppose $f(i)=g(i)$, then $f(i+1)=\mathfrak{f(i)}=\mathfrak{g(i)}=g(i+1)$,
        thus, $f(n)=g(n)$ for all $n<q$. Since Dom f = Dom g, we have $f=g$
    \end{proof}
\end{problem}
\begin{problem}
    Prove that 
    \[\begin{cases}
        m\cdot n \text{is the unique z such that for some f on} W_{n+1},\\
        f(0)=0,f(i+1)-f(i)+m \text{for all } i<n, and f(n)=z\\
    \end{cases}\]
    \begin{proof}
       Now we have $f(i)+m$ as $\mathfrak{a}_{f(i)}$, by theorem 2.17.1., we have
       exactly one funciton that satisfy the condition. What's more, according to theorem 2.16.1,
       f also works for $i<n$. Thus, f(n) is unique, and the recursion process is unique too.
    \end{proof}
\end{problem}
\begin{problem}
    Obtain theorem 2.18. as a corollary of theorem 2.17.
    \begin{proof}
       Let $a=\mathfrak{B}_1,\mathfrak{O}_n=(\mathfrak{B}:i<n)$, then,$\mathfrak{O}_1=a$.
       Regarding $\mathfrak{C}$ as $\mathfrak{O}$, by theorem 2.17.2., we have a unique $\mathfrak{O}_n$,
       therefore, a unique $\mathfrak{B}_i$. 
    \end{proof}
\end{problem}
\subsection{Axiom of infinity, Peano axioms, Dedekind infinite sets}
\begin{problem}
    Assume $(A,S,z)$ is a Peano structure. Take $f$ to be the function on $N$ such that 
    $f(0)=z,$ and $f(n+1)=S(f(n))$ for all n.

    Prove that $f$ is onto A and $f$ is one-to-one.

    \begin{proof}
        Let $B$ be the image of f, then, for any $x$ in $B$, $f(x)$ is also in $B$. What's more $z\in B$.
        According to P3, we have $B=A$, which means thath f is onto $A$.

        Suppose $f(m)=f(n)$. If $m=0$, then, $f(m)=f(n)=z$. According to P1, for all $n$, 
        $f(n+1)=S(f(n))\neq z$, so that $n=0=m$. If $m,n\neq 0$, then $n-1,m-1\in N$.
        $f(n)=S(f(n-1))=f(m)=S(f(m-1))$. Since $S$ is one-to-one, suppose $m\geq n$, then we have $f(m-n)=f(0)$.
        Thus $m-n=0$, or $m=n$.
    \end{proof}
\end{problem}
\begin{problem}
    Prove the following are equivalent:
    \begin{enumerate}
        \item $kappa$ is Dedekind infinite, i.e., $\bm{\alpha}_0\leq\kappa$
        \item $\kappa=\kappa+1$
        \item for some $A,\kappa=\bar{\bar{A}}$ and $A$ is equivalent to a proper subset of itself.
    \end{enumerate}
    \begin{proof}
        Since $N$ is a infinite set, let $A=N-{0}$,which is a subset of $N$.
        We have $A\sim N$,so that $\aleph_0=\aleph_0 -1$ or $\aleph_0+1=\aleph$.

        Suppose $\kappa\geq\aleph_0$, then, there exist $\lambda$, such that $\kappa=\aleph_0+\lambda=\aleph_0+1+\lambda=\kappa+1$.

        For some $A$, if $\bar{\bar{A}}=\kappa,$ such that $\kappa=\kappa+1$.
        Then, for any $x$ that is not in $A$, we have $\overline{\overline{A\cup\{x\}}}=\kappa+1=\kappa$.
        Hence, $A\cup\{x\}\sim A$, while $A\cup\{x\}\supset A$.

        \index{unsolved problem 2.5.2.}
    \end{proof}
    \begin{remark}
        ($(3)\rightarrow(1)$)

        According to (3), there exists $a$ such that $A\underset{f}{\sim } B\subseteq A-\{a\}$,
        where $B$ is a subset of $A$. By recursion, put $h(0)=a$ and $h(n+1)=f(h(n))$
        for all $n$. We show by induction that for all n, $\mathcal{P}(n)$ holds, where 
        \[\mathcal{P}(n) \text{is : for all } k\neq n, h(n)\neq h(k)\]
        $\mathcal{P}(0)$ holds, as $a\in $ Rng $f$ wile $k\in $ Rng f if $k\neq 0$. 
        Assume $\mathcal{P}(n).$ Let $h(n+1)=h(k), $ where $k\neq n+1$. Then $k\neq 0$ as above.
        Say $k=l+1$. Thus $f(h(n))=f(h(l))$ . Since f is one-to-one, so that $(n)=h(l)$, where $n\neq l$,
        which is contrary to the induction hypothesis. So $\mathcal{P}(n+1)$ holds.
        Now put $C=$Rng $h\cup\{a\}$. We showed $h$ is one-to-one on $N$ onto $C$, so $\bar{\bar{C}}=\aleph_0.$
        Thus $\bar{\bar{C}}<\bar{\bar{A}}=\kappa,$ so $\aleph_0\leq\kappa$, as desired.
    \end{remark}
\end{problem}
\section{MORE ON CARDINAL NUMBERS}
\subsection{Infinite sums and products of cardinals}
\begin{problem}
    Prove theorem 4.3.1.
    \begin{proof}
        Suppose $F(i)\underset{f_i}{\sim}G(i)$, we are going to prove that 
        \[\bigcup_{i\in I}F(i)\underset{\bigcup_{i\in I}f_i}{\sim}\bigcup_{i\in I} G(i)\]
        For any  $x$ in the RHS, it must be in one of $G(i)$. Thus, for 
        that very $i$, there exists $a$ in $F(i)$ that $f_i(a)\sim x$, since $F(i)\underset{f}{\sim}G(i)$

        Let $H=\bigcup_{i\in I}f_i$, then, if $H(a)=H(b)$, there exists an i such that they are both in $G(i)$.
        Since $f_i$ is also one-to-one, we have $a=b$.
    \end{proof}
\end{problem}
\begin{problem}
    Prove that the $\mu=\overline{\overline{\bigcup_{i\in I}F(i)}}$ is unique, where for each $i\in I$,$\bar{\bar{F(i)}}=\kappa_i$ and $F$ is disjointed.
    \begin{proof}
        Suppose we have f and g that both satisfy the condition of $F$,
        then, $\bar{\bar{f(i)}}=\kappa_i=\bar{\bar{g(i)}}$. Thus, $f(i)\sim g(i)$. By theorem 4.3.1.,
        we have $\bigcup_{i\in I}F(i)\sim\bigcup_{i\in I}G(i)$, thus $\overline{\overline{p\bigcup_{i\in I}F(i)}}=\overline{\overline{p\bigcup_{i\in I}G(i)}}=\mu$,
        which means that $\mu$ is unique.
    \end{proof}
\end{problem}
\begin{problem}
    Prove theorem 4.4.
    \begin{proof}
        Let $F(\prod_{i\in I}f(i))(i)=h_i(f(i))$, where $F(i)\underset{h_i}{\sim}G(i)$

        For every $\prod_{i\in I}g(i)$ in the RHS, we can let $f(i)=h_i^{-1}(g(i))$, then 
        \[F(\prod_{i\in I}f(i))(i)=h_i(f(i))=g(i)\]
        Thus, H is onto RHS.

        If $F(\prod_{i\in I}g(i))=F(\prod_{i\in I}f(i))$, then,$h_i(g(i))=h_i(f(i))$. However, $h_i$ is one-to-one.
        Hence $g(i)=f(i)$, which means that $H$ is also one-to-one.
    \end{proof}
\end{problem}
\begin{problem}
    Prove theorem 4.7.
    \begin{proof}
    Suppose $A_i$ are disjointed and $\bar{\bar{A_i}}=\kappa_i$,let
    \[F\upharpoonright A_i(a)=(i,a)\]
    Then, we have construct a funciton on LHS into RHS.
    \end{proof}
\end{problem}
\subsection{$\aleph_0,2^{\aleph_0},$ and $2^{2^{\aleph_0}}$- the simplest infinite cardinals}
\begin{problem}
    Prove without AC: A finite union of countable set is countable.
    \begin{proof}
        Use diagonal method.
    \end{proof}
\end{problem}
\begin{problem}
    Prove theorem 4.15.
    \begin{proof}
        Since every element in $R$ can be written as $\overline{a_0a_1\cdots a_n.a_{n+1}\cdots}$,
        which is in $10^{\aleph_0}$. So that $\mathfrak{c}\leq 10^{\aleph_0}$

        What's more, every $x$ in $2^{\aleph_0}$, is in $R$, and also can be represented in tenary.
        So that $2^{\aleph_0}\leq \mathfrak{c}$ and $2^{\aleph_0}\leq 10^{\aleph_0}$. However, every number in tenary can also be represented in 
        binary, thus $2^{\aleph_0}\geq 10^{\aleph_0}$. Therefore, $\mathfrak{c}=2^{\aleph_0}=10^{\aleph_0}$
    \end{proof}
\end{problem}
\begin{problem}
    Prove theorem 4.18.
    \begin{proof}
       \[ \begin{array}{rl}
           \mathfrak{c}^{\mathfrak{c}}&=(2^{\aleph_0})^{\mathfrak{c}}\\
                                      &=2^{\aleph_0\mathfrak{c}}\\
        \end{array}\]
        However, $\mathfrak{c}\leq\aleph_0\mathfrak{c}\leq \mathfrak{c}\cdot\mathfrak{c}=\mathfrak{c}$
        Thus,
        \[\mathfrak{c}^{\mathfrak{c}}=2^{\mathfrak{c}}\]
    \end{proof}
\end{problem}
In problems 4-6 below, show that $\bar{\bar{A}}=\aleph_0$, or that $\bar{\bar{A}}=2^{\aleph_0}$, or else that $\bar{\bar{A}}=2^{\mathfrak{c}}$
\begin{problem}
   $A$ is the set of all continuous functions on $R$ to $R$. 
   
   Since the continuous funciton can be determined only by its value on 
   rational points, so that $\bar{\bar{A}}\leq \overline{\overline{R^Q}}$.
   What's more, we have constant function that is absolutely continuous funciton.
   Thus, $\bar{\bar{R}}\leq\mathfrak{c}\leq\overline{\overline{R^Q}}=\overline{\overline{\mathfrak{c}^{\aleph_0}}}=\mathfrak{c}$.
   Therefore, $\bar{\bar{A}}=\mathfrak{c}$.
\end{problem}
\begin{problem}
    $A$ ia the set of all permutations of R.

    Since function like $ax+b,$ where $a,b\in R$, can be regarded as a
    permutation. Thus, $\bar{\bar{A}}\geq \mathfrak{c}^{\mathfrak{c}}$.
    Meanwhile, permutation is just a function that is on $R$ onto $R$.
    Thus, $\bar{\bar{A}}\leq \mathfrak{c}^{\mathfrak{c}}$.
    So, $\bar{\bar{A}}=\mathfrak{c}^{\mathfrak{c}}$.
\end{problem}
\begin{problem}
    $A$ is the family of all open sets of $R$.
    \begin{proof}
        Let $F(a,b)=(\frac{a+b}{2},\frac{b-1}{2}),b\geq a.$
        Then $A\underset{F}{\sim}R\times R$.
        Therefore, $\bar{\bar{A}}=\mathfrak{c}^{\mathfrak{c}}$. 
    \end{proof}
\end{problem}
\begin{problem}
    Let $A\subseteq R$ and $\bar{\bar{A}}=\aleph_0$.
    Prove now, without using AC,, that $\overline{\overline{R-A}}=\mathfrak{c}$.
    \begin{proof}
        Since $\overline{\overline{A}}=\aleph_0,$ $A$ is isomorphic to $N$.
        We can using cantor's proof method, starting from the 1st digit, 2ed digit, and so 
        on, to get countable numbers that is in $R-A$. Thus, $\overline{\overline{R-A}}\geq \aleph_0$.

        Let $\overline{\overline{R-A}}=\aleph_0+\lambda,$ then $\mathfrak{c}=\aleph_0+\aleph_0+\lambda=\aleph_0+\lambda=\overline{\overline{R-A}}.$
    \end{proof}
\end{problem}
\section{ORDERS AND ORDER TYPES}
\subsection{Ordered sums and products}
\begin{problem}
    Show $\sum_{i\in \underbar{I}}\underbar{A}_i$ is an order if $\underbar{I}$ is and each $\underbar{A}_i$ is.
    \begin{proof}
        We need to show that the ordered sum is reflective, antisymmetric, transitive and connected.

        Since each $\underbar{A}_i$ is an order, the ordered sum is obviously reflective and antisymmetric.
        Suppose $a\leq b,b\leq c$, then, if $a,b,c$ are all in one $A_i$, since $A_i$ is 
        transitive, we have $a\leq b\leq c$. Otherwise, we let $a,b,c$ be in $A_i,A_j,A_k$ 
        respectively. If $i,j,k$ are all different, then, $a\leq b\leq c$.
        If $i=j<k$ or $i<j=k$, we also have $a\leq b\leq c$. Thus, it is transitive.
        Given $a$ in $A_i$, and $b$ in $A_j$. If $i<j$, then $a\leq b$.
        If $i=j$, since $A_i$ is connected, so that $a\leq b$ or $a=b$ or $a\geq b$.
        If $i\geq j$, then, $a\geq b$. Thus, it is also connected.
    \end{proof}
\end{problem}
\begin{problem}
    Prove theorem 5.1.1.
    \begin{proof}
        Given any non-empty subset of the ordered sum, the i of the element in that subset
        form a non-empty subset of $I$. Thus, it has a least element j. Likewise, in the union of $A_j$ and
        the subset, we also have a least element $(x,j)$. Thus, the ordered sum is a well-order.
    \end{proof}
\end{problem}
\subsection{Order types}
\begin{problem}
    Prove theorem 5.3.1.
    \begin{proof}
        Using AC.
    \end{proof}
\end{problem}
\begin{problem}
    Prove $\omega^*,\omega,$ and $\omega,\omega^*$ are all different.
    \begin{proof}
        $\overline{\overline{\omega}}=\overline{\overline{\omega^*}}=\overline{\overline{\omega+\omega^*}}=\overline{\overline{\omega^*+\omega}}$,
        So, there is a least element and greatest element in $\omega$ and $\omega^*$ respectively,
        which is also the least element and the greatest element of $\omega+\omega^*$.
        However, $\omega^*+\omega$ has neither the least element nor the greatest element.
        Thus, these four structure are all different.
    \end{proof}
\end{problem}
\begin{problem}
    Fill in and prove: $1+\omega=?$
    
    $1+\omega=\omega$
    \begin{proof}
        We can add a new element into $N$, letting it to be the new least element x.
        Let $f(0)=x,f(1)=0,f(2)=1,\cdots$
    \end{proof}
\end{problem}
\begin{problem}
    Prove 5.4.2.
    \begin{proof}
        The counterexample for + is that 
        \[\omega+\omega^*\neq\omega^*+\omega\]
        The counterexample for $\cdot$ is that 
    \index{unsolved problem 4.2.4.}
    \begin{remark}
        $2\cdot\omega\neq\omega\cdot 2$
    \end{remark}
    \end{proof}
\end{problem}
\begin{problem}
    Prove theorem 5.4.3.
    \begin{proof}
        Let $B,C$ be disjointed, and $\overline{\overline{A}}=\overline{\overline{\sigma}},
        \overline{\overline{B}}=\overline{\overline{\tau}},\overline{\overline{C}}=\overline{\overline{\tau}}'$.
        Suppose $(x,y),(m,n)$ is in LHS, and $(x,y)\leq(m,n)$.
        For the first case, $n,y$ is both in B or C, then, it is also in the RHS.
        For the second case, $n,y$ is in C,B seperately. Then, it is also in the RHS.
        Suppose they are in the RHS. There is also two cases. In all cases, they are also in the LHS.
    \end{proof}
\end{problem}
\begin{problem}
    Prove theorem 5.4.4.
    \index{unsolved problem 4.5.6.}
    \begin{remark}
        take $\alpha=\beta=1,\gamma=\omega.2\cdot\omega=\omega,\omega\cdot \omega\neq\omega$
    \end{remark}
\end{problem}
\begin{problem}
    Show that $\eta+\eta=\eta$ but $\lambda+\lambda\neq\lambda$.
\end{problem}
\begin{problem}
    $B\subseteq A$ is said to be dense in the order $\underbar{A}=(A,\leq)$
    if whenever $x<y$ there exists $b\in B$ such that $x<b<y$. Prove that an order $\underbar{A}$
    is isomorphic to the real order if $\underbar{A}$ has no firsr or last element,
    there is a countable set $B$ dense in $\underbar{A}$, and in $\underbar{A}$,
    every non-empty subset bounded above has a least upper bound.
\end{problem}
\section{AXIOMATIC SET THEORY}
\begin{problem}
    Following the order in theorem 6.3., prove in $Z_0F$ the existence of:
    \begin{enumerate}
        \item $(\mathfrak{a}_x:x\in X)$
        \item $A\times B$.
        \item Dom $R$
        \item $R/S$
        \item $A^B$ and then $\prod_{i\in I}\mathfrak{a}_i$.
    \end{enumerate}
    \begin{proof}
        Since function is a relation, $(\mathfrak{a}_x:x\in X)=\{(x,\mathfrak{a}_x):x\in X\}$.
        
        $A\times B=\{(a,b):a\in A,b\in B\}$

        Dom $R=\{x:(x,\mathfrak{a}_x)\in (\mathfrak{a}_x:x\in X)\}$

        $R/S=\{(x,y):\text{for some z in R,(x,z) in R, and (z,y)}\in  S \}$

        $A^B=\{f:\text{Dom }f =B \text{ and is into }A\}$

        $\prod_{i\in I}\mathfrak{a}_i=\{f:\text{Dom }f= I \text{ and }f(i) \text{ is in }\{\mathfrak{a}_i:x\in X\}\}$
    \end{proof}
\end{problem}
\section{WELL-ORDERINGS,ORDINALS AND CARDINAL}
\subsection{Well-orders}
\begin{problem}
    Prove theorem 7.1.
    \begin{proof}
        If there exist $a$ in $A$ such that $f(a)<a$, then, let $f^n(x)=ff\cdots ff(x)$.
        The subset $\{x:x=f^n,n\in N\}$ has no least element, which is contrary to the fact that $A$ is a well-order.
    \end{proof}
\end{problem}
\begin{problem}
    Prove theorem 7.3.
    \begin{proof}
        Suppose $\underbar{A}\underset{f}{\cong}\underbar{A}$, then,
        $f(x)\geq x, f^{-1}(x)\geq x$, thus, $x\geq f(x)\geq x$ or $f(x)=x$. 
        Thus, $f=Id$
    \end{proof}
\end{problem}
\begin{problem}
    Let $H$ be the set of all $f$ such that $f$ is an isomorphism between 
    an initial segment of $\underbar{A}$ and an initial segment of $\underbar{B}$.
    Prove that $H$ is a chain(i.e., if $f,g\in H$ then $f\subseteq g$ or $g\subseteq f$).
    \begin{proof}
        Let f be a function on initial segment $a$ in $A$ onto initial segment 
        $b$ in $B$. Let g be a function on intial segment $c\subseteq a$ in $A$
        onto initial segment $d$ in $B$. If $g\slashed{\subseteq}f$, then,
        \index{unsolved problem 6.1.3.}
    \end{proof}
\end{problem}
\end{document}