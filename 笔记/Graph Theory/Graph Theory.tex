\documentclass[a4paper,11pt]{article}%必须以此为开头,可以在[]内设置栏数,单双面,横竖向
\usepackage{latexsym}%符号字体
\usepackage{makeidx}%制作索引
\makeindex
\usepackage{ifthen}%提供分支语句
\usepackage{graphicx}%用于插入图片
\usepackage{amsmath}%用于数学公式
\usepackage{IEEEtrantools}%用于使用IEEE数学公式排版工具
\usepackage{amsfonts}%用于其他字体的数学符号
\usepackage{amsthm}%提供证明,定理等环境
\usepackage{amssymb}%用于提供各种数学符号
\usepackage{mathrsfs}%用于提供花体字母
\usepackage{verbatim}%使用\verbatiminput{filename}来直接导入文件中的文本内容
\usepackage{layouts}%用于设置页面布局
\usepackage{calc}%允许一些常量参量用算术表达式代替
\usepackage{indentfirst}
\usepackage{hyperref}
\usepackage{makecell}%允许表格的单元格内换行
\usepackage{bm}%使用bm来对希腊字母加粗
\usepackage{longtable}
\usepackage{slashed}%支持字母上加斜线
\theoremstyle{remark}
\newtheorem*{remark}{remark}
\theoremstyle{definition}
\newtheorem{theorem}{THEOREM}[section]
\theoremstyle{definition}
\newtheorem{corollary}{COROLLARY}[section]
\theoremstyle{definition}
\newtheorem{lemma}{LEMMA}[section]
\theoremstyle{definition}
\newtheorem*{definition}{definition}
\theoremstyle{plain}
\newtheorem*{property}{性质}
\newcommand*{\abs}[1]{\lvert #1 \rvert}
\theoremstyle{definition}
\newtheorem{axiom}{Axiom}
\author{Fan}
\title{Graph Theory}
\date{2023.8.13}
\begin{document}
\maketitle
\tableofcontents
\pagestyle{plain}%页面风格,plain为中下方有页码.heading是页眉中间有页数,同时有章节名,empty是空页眉页尾
%\thispagestyle{pagestyle}%本页页面风格
\begin{longtable}{cc}
        \caption{various notation} \\
       \multicolumn{1}{c}{notations}&\multicolumn{1}{c}{meanings}\\
       \hline
       \endfirsthead
       \multicolumn{1}{c}{notations}&\multicolumn{1}{c}{meanings}\\
       \hline\endhead
       $G$ & Graph\\
       $\bar{G}$&complement of $G$\\
       $V(G)$& Vertex-set\\
       $E(G)$ & Edge-set(family)\\
       $N_n$& Null graph on $n$ vertices \\
       $K_n$& Complete graph on $n$ vertices \\
       $C_n$& Cycled graph on $n$ vertices \\
       $P_n$& Path graph on $n$ vertices \\
       $W_n$& Wheel graph on $n$ vertices \\
       $K_{r,s}$&Complete bipartite graph\\ 
       $Q_k$& $k-$cube\\
       $D$& Digraph\\
       $A(D)$& arc-family\\
       $D'$& Converse of $D$\\
       $L(G)$&Line graph of $G$\\
       $\lambda(G)$&Edge-connectivity of $G$\\
       $\kappa(G)$& Vertex-connectivity of $G$\\
       $d(v,w)$&Distance between a vertex $v$ and a vertex $w$\\
       $\gamma(G)$&Cycle rank of $G$\\
       $\zeta(G)$&Cutset rank of $G$\\
       cr$(G)$&Crossing number of $G$\\
       $G^*$&Dual of $G$\\
       $g(G)$& Genus of $G$\\
       $t(G)$&Thickness of $G$\\
       $\chi (G)$&Chromatic number of $G$\\
       $P_G(k)$&Chromatic polynomial of $G$\\
       $\chi'(G)$&Chromatic index of $G$\\
\end{longtable}
\newpage 
\section{Definitions and examples}
\subsection{Definitions}
A simple graph $G$ consist of a non-empty finite set $V(G)$ of elements called 
vertices (or nodes or points) and a finite set $E(G)$ of distinct unordered 
pairs of distinct elements of $V(G)$ called edges (or lines). We call $V(G)$
the vertex-set and $E(G)$ the edge-set of $G$. An edge $\{v,w\}$ is 
said to join the vertices $v$ and $w$, and is usually abbreviated to $vw$.

In any simple graph there is at most one edge joining a given pair of vertices.
However, many results for simple graphs also hold for  more general objects
in which two vertices may have several edges joining them; such edges are called multiple
edges. In addition , we may remove the restriction that an edge must join 
two distinct vertices, and allow loops- edges joining a vertex to itself.
The resulting object, with loops and multipls edges allowed, is called a general 
graph- or , simply, a graph.
\begin{remark}
    Every simple graph is a graph, but not every graph is a simple graph.
\end{remark}
We call $V(G)$ the vertex-set and $E(G)$ the edge-family of G.
\begin{remark}
    The use of word `family' permits the existence of multiple edges.
\end{remark}
\begin{remark}
    In this note, all graphs are finite and undirected, with loops 
    and multiple edges allowed unless specifically excluded.
\end{remark}

Two graph $G_1$ and $G_2$ are isomorphic if there is a one-one 
correspondence between the vertices of $G_1$ and those of $G_2$ 
such taht the number of edges joining any two vertices of $G_1$ equals
the number of edges joining the corresponding vertices of $G_2$.

We say that two `unlabelled graphs' are isomorphic if we can assign lables to 
their vertices so that the resulting `lablled graphs' are isomorphic.

If the two graphs are $G_1$ and $G_2$ and their vertex-sets $V(G_1)$ and $V(G_2)$
are disjoint, then their union $G_1\cup G_2$ is the graph with vertex-set $V(G_1)\cup V(G_2)$
and edge-family $E(G_1)\cup E(G_2)$.

A graph is connected if it cannot be expressed as a union of graphs, and 
disconnected otherwise. Clearly, any disconnected graph $G$ can be expressed 
as the union of connected graphs. each of which is called a component of $G$.

We say that two vertices $v$ and $w$ of a graph are adjacent if there is an edge 
$vw$ joining them, and the vertices $v$ and $w$ are then incident with such 
an edge. We also say that two distinct edges $e$ and $f$ are adjacent if they 
have a vertex in common.

The degree of a vertex $v$ is the number of edges incident with $v$,
and is written $\deg(v)$; when calculating the degree of $v$, we usually make 
the convention that a loop at $v$ contributes 2 (rather than 1). A vertex of degree 
0 is an isolated vertex and a vertex of degree 1 is an end-vertex.

The degree sequence of a graph consists of the degrees written in increasing 
order, with repeats where necessary.

\begin{theorem}[Handshaking theorem]
   In any graph, the sum of all the vertex-degrees is an even number. 
\end{theorem}
\begin{remark}
    Maybe that's why we let a loop contributes 2.
\end{remark}
\begin{corollary}
    In any graph, the number of vertices of odd degree is even.
\end{corollary}
A graph $H$ is a subgraph of a graph $G$ if each of its vertices belongs to $V(G)$ and 
each of its edges belongs to $E(G)$.

We can obtain subgraph by deleting edges and vertices. If $F$ is 
any set of edges in $G$, we denote by $G-F$ the graph obtained by deleting the 
edges in $F$. Similarly, if $S$ is any set of vertices in $G$, we denote 
by $G-S$ the graph obtained by deleting the vertices in $S$ and all 
edges incident with any of them.

We also denote by $G\verb|\|e$ the graph obtained by taking an edge $e$ and `contracting'
it- that is, removing it and identifying its end $v$ and $w$, and let them 
overlap without change other parts of the graph.
\begin{remark}
    You can imaging that the edge $e$ shrink to disappear.
\end{remark}
If $G$ is a simple graph with vertex-set $V(G)$, its complement $\bar{G}$
is the simple graph with vertex-set $V(\bar{G})$ in which two vertices are
adjacent if and only if they are not adjacent in $G$.

If $G$ is a graph without loops, with vertices labelled $\{1,2,\cdots,n\}$,
its adjacent matrix $A$ is the $n\times n$ matrix whose ijth entry is the 
number of edges joining vertex $i$ and vertex $j$. If, in addition, the edges 
are labelled $\{1,2,\cdots,m\}$, its incidence matrix $M$ is the $n\times m$
matrix whose ijth entry is 1 if vertex $i$ is incident to edge $j$, and is 0 otherwise.
\subsection{Examples}
\begin{definition}
    A graph whose edge-set is empty is a null graph. We denote the null graph on $n$ vertices by $N_n$.
\end{definition}
\begin{remark}
    Each vertex of a null graph is isolated.
\end{remark}
\begin{definition}
    A simple graph in which each pair of distinct vertices are adjacent is a 
    complete graph. We denote the complete graph on $n$ vertices by $K_n$.
\end{definition}
\begin{remark}
    $K_n$ has $n(n-1)/2$ edges.
\end{remark}
\begin{definition}
    A connected graph in which each vertex has degree 2 is a cycle graph.
    We denote the cycle graph on $n$ vertices by $C_n$.
\end{definition}
\begin{definition}
    The graph obtained from $C_n$ by removing an edge is the path graph on $n$ 
    vertices, denoted by $P_n$.
\end{definition}
\begin{definition}
    The graph obtained from $C_{n-1}$ by joining each vertex to a new vertex 
    $v$ is the wheel on  $n$ vertices, denoted by $W_n$.
\end{definition}
\begin{definition}
    A graph in which each vertex has the same degree is a regular 
    graph. If each vertex has degree $r$, the graph is regular of 
    degree $r$ or $r-$regular.
\end{definition}
\begin{remark}
    The null graph $N_n$ is regular of degree 0; the cycle graph $C_n$
    is regular of degree 2; the complete graph $K_n$ is regular of 
    degree $n-1$.
\end{remark}
\begin{definition}
    Cubic graphs are graphs that are regular of degree 3.
\end{definition}
\begin{definition}
    If the vertex-set of a graph $G$ can be split 
    into two disjoint sets $A$ and $B$ so that each edge of $G$ joins a
    vertex of $A$ and a vertex of $B$, then $G$ is a bipartite graph.

    Alternatively, a bipartite graph is one whose vertices can be colored 
    black and white in such a way that each edge joins a black vertex (in $A$)
    and a white vertex (in $B$). 

    We sometimes write $G=G(A,B)$ when we wish to specify the sets $A$ and $B$.
\end{definition}
\begin{definition}
    A complete bipartite graph is a bipartite graph in which each vertex in $A$
    is joined to each vertex in $B$ by just one edge. We denote the complete bipartite
    graph with $r$ black vertice and $s$ white vertices by $K_{r,s}$
\end{definition}
\begin{definition}
    The $k-$cube $Q_k$ is the graph whose vertices correspond to 
    the sequence $(a_1,a_2,\cdots,a_k)$, where each $a_i=0$ or $1$,
    and whose edges join the sequences that differ in just one place.
\end{definition}
\begin{remark}
    $Q_k$ has $2^k$ vertices and is regular of degree $k$.
\end{remark}
\begin{definition}
    The line graph $L(G)$ of a simple graph $G$ is the graph whose vertices 
    are in one-one correspondence with the edges of $G$, with two vertices of
    $L(G)$ being adjacent if and only if the corresponding edges of $G$ are adjacent.
\end{definition}
\subsection{Digraphs and infinite graphs}
A directed graph, or digraph, $D$ consists of non-empty finite set $V(D)$
of elements called vertices and a finite family $A(D)$ of ordered pairs of elements 
of $V(D)$ called arcs( or directed edges). We called $V(D)$ the vertex-set 
and $A(D)$ the arc-family of $D$. An arc $(v,w)$ is usually 
abbreviated to $vw$.

If $D$ is a digraph, the graph obtained from $D$ by `removing the arrows'
is the underlying graph of $D$.

$D$ is a simple digraph if the arcs of $D$ are all distinct, and if there 
are no `loops'(arcs of the form $vv$).
\begin{remark}
    The underlying graph of a simple digraph need not be a simple graph.
\end{remark}
Two digraphs are isomorphic if there is an isomorphism between their underlying graphs 
that preserve the ordering of the vertices in each arc.

A digraph $D$ is (weakly) connected if it cannot be expressed as the union 
of two digraphs, defined in the obvious way. This is equivalent to saying that the 
underlying graph of $D$ is a connected graph.

Two vertices $v$ and $w$ of a digraph $D$ are adjacent if there is an arc 
in $A(D)$ of the form $vw$ or $wv$, and the vertices $v$ and $w$ are incident
with such an arc.

The our-degree of a vertex $v$ of $D$ is the number of arcs of the form 
$vw$, and is denoted by outdeg $(v)$. Similarly, the in-degree of $v$ is the 
number of arcs of $D$ of the form $wv$, and is denoted by indeg $(v)$.

\begin{theorem}[Handshaking dilemma]
   In any digraph, the sum of all the out-degrees is equal to the sum of
   all the in-degrees. 
\end{theorem}
If $D$ is a digraph without loops, with vertices labelled $\{1,2,\cdots,n\}$,
its adjacency matrix $A$ is the $n\times n$ matrix whose $ij$th entry is the number 
of arcs from vertex $i$ to vertex $j$.
\begin{definition}
    A diagraph in which any two vertices are joined by exactly one arc is called
    a tournament.
\end{definition}
\begin{remark}
    The underlying graph of a tournament is a complete graph.
\end{remark}
\begin{definition}
    The converse $D'$ if a digraph $D$ is obtained from $D$ by reversing the 
    direction of each arc.
\end{definition}
An infinite graph $G$ consists of an infinite set $V(G)$ of elements called 
vertices and an infinite family $E(G)$ of unordered pairs of elements of 
$V(G)$ called edges. If $V(G)$ and $E(G)$ are both countably infinite, 
then $G$ is a countable graph. For convenience, we only 
consider the situation where $V(G)$ and $E(G)$ are both infinite sets.

The degree of a vertex $v$ of an infinite graph is the cardinality of the set 
of edges incident with $v$, and may be finite or infinite. An infinite graph 
is locally finite if each of its vertices has finite degree. We similarly define 
a locally countable infinite garph to be one in which each vertex has countable 
degree.
\begin{theorem}
    Every connected locally countable infinite graph is a countable graph.
\end{theorem}
\begin{proof}
    Using the theorem `the union of countable set is still countable'.
    Start the proof by choosing a vertice from the $V(G)$, then let $A_1$
    be the set of vertices adjacent to $v$, $A_2$ be the set of all vertices adjacent
    to a vertex of $A_1$, and so on. $\{v\},A_1,A_2$ is a sequence of sets whose
    union is countable and contains every vertex of the finite garph, by connectness.
\end{proof}
\begin{remark}
    Actually, the direct definition of connectness doesn't tell us that, 
    in a connected graph, there is a path connec each pair of vertices.
    However, we can prove it from the definition.
\begin{proof}
    Suppose $G$ is a (infinite) connected graph. If there exists vertices $v$
    and $w$ such that there isn't exists a path between them. Then, 
    $G$ can be express as $G_1\cup G_2$, where $G_1$ consists of the 
    vertices that have a path to $v$ and $G_2$ to $w$.The $V(G_1)$ and $V(G_2)$
    are disjointed. Otherwise, there is a path between $v$ and $w$.
    Of course, if $G$ is infinite, then, $G$ can be expressed as 
    several (maybe infinitely many) graphs' union. Anyway, it is contrary to 
    the definition of connectness.
\end{proof}
Perhaps the proof can begin with `randomly choosing a vertice $v$ from $V(G)$',
and get a set consists of the vertices that cannot be connected with $v$
by a path and $v$ itself.
\end{remark}
\begin{corollary}
    Every connected locally finite infinite graph is countable graph.
\end{corollary}
\section{Paths and cycles}
\subsection{Connectivity}
Given a graph $G$, a walk in $G$ is a finite sequence of edges of the form 
\[v_0v_1,v_1v_2,\cdots,v_{m-1}v_m \text{ also denoted by }v_0\rightarrow v_1\rightarrow v_2\rightarrow\cdots\rightarrow v_m,\]
in which any two consecutive edges are adjacent or identical. 
Such a walk determines a sequence of vertices $v_0,v_1,\cdots,v_m$. We call
$v_0$ the initial vertex and $v_m$ the final vertex of the walk, and speak of 
a walk from $v_0$ to $v_m$. The number of edges in a walk is called its 
length.

A walk in which all the edges are distinct is a trail. If, in addition, the 
vertices $v_0,v_1,\cdots,v_m$ are distinct(except, possibly, $v_0=v_m$), 
then the trail is a path. A walk, path or trail is closed if $v_0=v_m$,
and a closed path with at least one edge is a cycle.

The girth of a graph is the length of its shortest cycle.

In a connected graph, the distance $d(v,w)$ between a vertex $v$ and a vertex $w$
is the length of the shortest path from $v$ to $w.$
\begin{remark}
    A loop is a cycle of length 1, and a pair of multiple edges is a cycle of 
    length 2. A cycle of length 3 is called a triangle.
\end{remark}
\begin{theorem}
    A graph $G$ is bipartite if and only if every cycle of $G$ has even length.
\end{theorem}
\begin{proof}
    $\Rightarrow)$ Let the vertices in the bipartite colored in 
    white or black, then the number of white vertices and black vertices 
    on each cycle should be the same. Thus, there is evne numbers of edges 
    on a cycle.

    $\Leftarrow)$ Choose a vertex $v$ randomly from $G$.
    Then, let $A$ be the set of vertices for which the shortest path 
    from $v$ to them has even length and vertices in $B$ has odd length.
    Suppose there are two vertices that is both in $A$ or $B$ and 
    are adjacent. Then, the shortest path from $v$ to these two vertices and
    the edge incident with them would include a cycle of odd length. 
    (Find that cycle by erasing the overlap part of the two path, 
    to satisfy the condition to be a cycle, the remain part 
    has same length. Thus, adding the edge incident with them, the total
    length is always odd.) This is contrary to the assumption.
    Thus, every edges in $G$ is incident with a vertex in $A$ and a 
    vertex in $B$.
\end{proof}
\begin{theorem}
    Let $G$ be a simple graph on $n$ vertices. If $G$ has $k$ components,
    then the number $m$ of edges of $G$ satisfies
    \[n-k\leq m\leq (n-k)(n-k+1)/2\]
\end{theorem}
\begin{proof}
    Prove the lower bound by induction on the number of edges in $G$.
    The induction hypothesis should be `when there is $m$ edges in $G$, $n-k\leq m$,
    where $k$ should be the component of $G$ and n the number of vertices in $G$'.
    When there is $m+1$ edges in $G$, suppose there is the number of the 
    component and the vertices are $k,n$ respectively. Then,
    if $m$ is the lower bound, by removing one edge from it, we 
    can get graph $G'$ with $k+1$ component and $n$ vertices. By 
    hypothesis, we have $n-(k+1)\leq m-1$ or $n-k\leq m$. Thus, we finish the 
    proof of lower bound by induction.

For the upper bound, it is obvious that, to achieve the upper bound, 
each component of $G$ should be a complete graph. Suppose there are 
two component that both have at least two vertices. We can show that 
if we move one vertex from one of them to another, the sum of 
edges of them will increase. Thus, to achieve upper bound, $k-1$ component
are just one vertex, and the last component is $K_{n-k}$. Therefore we prove 
the upper bound.
\end{proof}
\begin{corollary}
    Any simple graph with $n$ vertices and more than $(n-1)(n-1)/2$ edges 
    is connected.
\end{corollary}
A disconnecting set in a connected graph $G$ is a set of edges whose deletion 
disconnects $G$. A cutset is a minimal disconnecting set - that is 
a diconnecting set, no proper subset of which is a disconnecting set.
\begin{remark}
    The definition of cutset doesn't guarantee that there is only one 
    cutset for $G$, and each cut set may contain different numbers of 
    edges.
\end{remark}
\begin{remark}
    The deletion of the edges in a cutset always leaves a graph with exactly 
    two components.
\end{remark}
If a cutset has only one edge $e$, we call $e$ a bridge.

These definitions can easily be extended to disconnected graphs.
If $G$ is any such graph, a disconnecting set of $G$ is a set of 
edges whose removal increases the number of component of $G$, and 
a cutset of $G$ is a minimal disconnecting set.

If $G$ is connected, its edge-connectivity $\lambda(G)$
is the size of the smallest cutset in $G$. We also say that $G$
is $k-$edge-connected if $\lambda(G)\geq k$.
\begin{theorem}[Menger]
   A graph $G$ is $k-$edge-connected if and only if any 
   two distinct vertices of $G$ are joined by at least $k$ paths, 
   no two of which have any edges in common. 
\end{theorem}
A separating set in a connected graph $G$ is a set of vertices 
whose deletion disconnects $G$. If a separating set containsonly one vertex 
$v$, we call $v$ a cut-vertex. These definitons extend immediately to disconnected 
graphs, as above.

If $G$ is connected and not a complete graph,
its (vertex) connectivity $\kappa(G)$ is the size of the smallest separating 
set in $G$. We also say that $G$ is $k-$connected if $G(\kappa)\geq k$.
\begin{theorem}[Menger]
   A graph $G$ with at least $k+1$ vertices is $k-$connected if and only if 
   any two vertices of $G$ are joined by at least $k$ paths, no two 
   of which have any other vertices in common. 
\end{theorem}
\begin{theorem}
    If $G$ is any connected graph, then 
    \[\kappa(G)\leq\lambda(G)\leq\delta(G),\]
    where $\delta(G)$ is the smallest vertex-degree in $G$.
\end{theorem}
A walk in a diagraph $D$ is a finite sequence of arcs of the form 
\[v_0v_1,v_1v_2,\cdots,v_{m-1}v_m.\]
We sometimes write this sequence as 
\[v_0\rightarrow v_1\rightarrow\cdots\rightarrow v_m,\]
and speak of a walk from $v+0$ to $v_m$. In an analogous way, we can define 
directed trails, directed paths and ditected cycles (or simply trails, 
paths and cycles, when there is no possibility of confusion).
\begin{remark}
    Although a taril cannot contain a given arc $vw$ more than once, it cna 
    contain both $vw$ and $wv$.
\end{remark}
We ay that $D$ is strongly connected if, for any two 
vertices $v$ and $w$ of $D$, there is a directed path from $v$ to $w$.

For convenience, we define a graph $G$ to be orientable if each edge of $G$
can be directed so that the resulting digraph is strongly connected; such a
digraph is an orientation of $G$.
\begin{theorem}
    A connected graph $G$ is orientable if and only if each edge of $G$ lies 
    in at least one cycle.
\end{theorem}
\begin{proof}
    $\Rightarrow)$ This direction is obvious.

    $\Leftarrow)$ First, we can choose two adjacent vertex $v_1$ and $v_2$ from $G$.
    Suppose $v_1v_2$ is in circle $C:v_1v_2\cdots v_1$.
    We direct this circle clockwisely.
    If this circle is exactly the graph itself, then the proof is complete.
    Otherwise, in the remain part of $G$, there must exist at least one vertex
    that is adjacent with one of the vertex of $C$. We denote that vertex by $w_1$.
    By the condition, $w_1v_1$ should be in another circle $C'$. For the part 
    that is overlapped with $C$, we keep the original direction. 
    For the other part of $C'$, we also directed them clockwisely.
    Hiterto, the graph $C\cup C'$ is orientable, since we can regard 
    the overlapped part as an interchange. The rest can be done in the same 
    manner. $G=C\cup C'\cup C''\cup\cdots$ and for each step, the graph is orientable.
\end{proof}
In an infinite graph $G$, there are essentially three different types of walk in $G$:
\begin{enumerate}
    \item a finite walk is defined exactly as above
    \item a one-way infinite walk with initial vertex $v_0$ is an infinite sequ of edges of the form \[v_0\rightarrow v_1\rightarrow v_2\rightarrow\cdots\]
    \item a two-way infinite walk is an infinite sequence of edges of the form \[\cdots\rightarrow v_{-2}\rightarrow v_{-1}\rightarrow v_0\rightarrow v_1\rightarrow v_2\rightarrow\cdots\]
\end{enumerate}
One-way and two-way infinite trails and paths are defined analogously.
\begin{theorem}
   Let $G$ be a connected locally finite infinite graph. Then, for any vertex $v$ of $G$,
   there exists a one-way infinite path with initial vertex $v$. 
\end{theorem}
\begin{remark}
    Trival as it is, it still need to prove, since it involves infinity.
\end{remark}
\begin{proof}
    For each vertex $z$ other than $v$, there is a non-trival path from $v$
    to $z$. It follows that there are infinitely many path in $G$ with initial vertex $v$.
    Since the degree of $v$ is finite, infinitely many paths must start with the 
    same edge. If $vv_1$ is such an edge, then we repeat this procedure and get 
    \[v\rightarrow v_1\rightarrow v_2\rightarrow\cdots.\]
\end{proof}
\subsection{Eulerian graph and digraphs}
A connected graph $G$ is Eulerian if there exists a closed trail that includes
every edge of $G$; such trail is an Eulerian trail. A non-Eulerian graph $G$
is semi-Eulerian if there exists a (non-closed) trail that includes every edge
of $G$.
\begin{remark}
    Every Eulerian graph is orientable.
\end{remark}
\begin{lemma}
    If $G$ is a graph in which the degree of each vertex is at least 2, then 
    $G$ contains a cycle.
\end{lemma}
\begin{proof}
    Choose the next vertex on the path not to be previous one, until you cannot 
    do it.
\end{proof}
\begin{theorem}
    A connected graph $G$ is Eulerian if and only if the degree of each vertex
    of $G$ is even.
\end{theorem}
\begin{proof}
    $\Rightarrow)$Since a trail cannot contain one edge twice, and a Eulerian
    trail contains every edge, so that every vertices' degree is even. Thus we 
    finished the proof.

    $\Leftarrow)$Prove by induction on the edges of a graph whose 
    vertices all have even degrees.
    By lemma 2.1., there exists a circle in $G$. If that circle equals $G$,
    then we finished the proof. Otherwise, we delete that circle from $G$ 
    and get a new graph $H$. Since every vertice both in $G$ and $H$ have 
    even degree in $G$ and still have even degree in $H$, for 
    the fact that $C$ is a circle, $H$ satisfies the induction hypothesis.
    Then the Eulerian trail can be constructed by starting from a vertex on 
    $C$ and move clockwisely. Whenever we meet a vertex that is in $H$,
    then we follow the Eulerian trail of that part of $H$. When we back to 
    that vertex again, we keep moving clockwisely along $C$, until we reach the 
    starting vertex.
    Thus we finished the proof.(When there is only one edge with degree, we actually have a loop.)
\end{proof}
\begin{corollary}
    A connected graph is Eulerian if and only if its set of edges can be split 
    up into edge-disjoint cycles.
\end{corollary}
\begin{proof}
$\Rightarrow)$ Inmitate the proof in the $\Leftarrow$ direction of the proof of 
theorem 2.8. and use the $\Rightarrow$ direction of it and lemma 2.1..

$\Leftarrow)$ Use the $\Leftarrow$ direction of theorem 2.8., since the connecting
vertex of multiple circle still has an even degree.
\end{proof}
\begin{corollary}
    A connected graph is semi-Eulerian if and only if it has exactly two vertices of odd
    degree.
\end{corollary}
\begin{proof}
    The key is that suppose $H$ is a semi-Eulerian graph, and 
    $P$ is its semi-Eulerian trail, then, by connecting the head and the end 
    of $P$ we can get a Eulerian graph, whose vertices all have even degrees.
\end{proof}
\begin{theorem}[Fleury's algorithm]
   Let $G$ be an Eulerian graph. Then the following construction is always 
   possible, and produces an Eulerian trail of $G$.
   
   Start at any vertex $u$ and traverse the edges in an arbitrary manner,
   subjected only to the following rules:
   \begin{enumerate}
    \item erase the edges as they are traversed, and if any isolated vertices result, erase them too;
    \item at each stage, use a bridge only if there is no alternative.
   \end{enumerate}
\end{theorem}
A connected digraph $D$ is Eulerian if there exists a closed directed trail 
that includes every arc of $D$; such a trail is an Eulerian trail.
\begin{remark}
    The diagraph must be strongly connected for an Eulerian trail to exist.
\end{remark}
\begin{theorem}
    A strongly connected diagraph is Eulerian if and only if, for each vertex 
    $v$ of $D$,
    \[ \text{outdeg}(v)=\text{indeg}(v).\]
\end{theorem}
\begin{proof}
    $\Rightarrow)$ This direction is obvious.

    $\Leftarrow)$ We still prove it by induction.
    When there is only two vertices $v,w$, it is obviously that 
    when there is n multiple arcs of $wv$, there is also multiple arc of $vw$,
    and $n\geq 1$. Mealwhile, there maybe some loops. We construct the Eulerian 
    trail by travelling between $v$ and $w$ and traverse all the loops as 
    soon as we can.

    Suppose there is a digraph with n vertices that satisfies the condition.
    It can be easily proof that, there at least exist one circle in 
    the diagraph. The digraph we get by deleting the circle from the 
    original digraph exists a Eulerian trail for each component, which share 
    at least one vertex with the circle. Thus we can construct the 
    Eulerian trail similarly as the $\Leftarrow)$ direction of the
    proof of theorem 2.9.
\end{proof}
A non-Eulerian digraph $D$ is semi-Eulerian if there exists a 
(non-closed) directed trail that includes every edge of $G$.
\begin{corollary}
    A strongly connected digraph is Eulerian if and only if its set of 
    edges cna be split up into edge-disjoint directed cycles.
\end{corollary}
\begin{corollary}
    A strongly connected digraph is semi-Eulerian if and only if it has exact
    two vertices $v,w$ such that outdeg $(v)=$ indeg $(V)+1$, indeg $(w)=$ outdeg $(w)+1$,
    and for other vertices $z$, outdeg $(z)=$ indeg $(z)$.
\end{corollary}
A connected infinite graph $G$ is Eulerian if there exists a two-way 
infinite trail that includes every edges of $G$; such an 
infinite trail is a two-way Eulerian trail.
\begin{remark}
    The definition require $G$ to be countable.
\end{remark}
\begin{theorem}
    Let $G$ be a countable connected graph which is Eulerian.
    Then
    \begin{enumerate}
        \item $G$ has no vertices of odd degree.
        \item for each finite subgraph $H$ of $G$, the infinite graph $K$
        obtained by deleting from $G$ the edges of $H$ has at most two infinite 
        component;
        \item if, in addition, each vertex of $H$ has even degree, then $K$
        has exactly one infinite component.
    \end{enumerate}
\end{theorem}
\begin{proof}
   $(1)$ is obvious.

    The key of the proof for $(2),(3)$ is that,
    let the Eulerian trail for $G$ to be $P$, and separate it into 
    3 parts: $P^-,P_0,P^+$. $P_0$ is finite and covers all $H$ and thus 
    $P^-,P^+$ is both infinite. Then, (2) is obvious.

    Let the starting vertex and the end vertex of $P_0$ to be $v,w$.
    If $v=w$, then we finish the proof of $(3)$. Otherwise, for the other 
    vertices of $K$, they all have even degree. Meanwhile, $v,w$ is the 
    only two vertices that have odd degree, by the $\Leftarrow$direcrtion of corollary 2.3., there exist a
    semi-Eulerian trail. Thus, $K$ has exactly one infinite component.
\end{proof}
\begin{remark}
   The $\Leftarrow$ direcrtion of corollary 2.3. mainly build on the $\Leftarrow$ direction of 
   theorem  2.8., which is depend on the induction on the number of edge.
   Since we assume that $G$ is a countable graph, the induction on 
   the number of edge still valid.
\end{remark}
\begin{theorem}
    If $G$ is a connected countable graph, then $G$ is Eulerian if and only if 
    the three conditions in theorem 2.11. are satisfied.
\end{theorem}
\subsection{Hamiltonian graphs and digraphs}
A closed trail passing exactly once through each vertex of $G$ must be a cycle,
and is called a Hamiltonian cycle. A graph with a Hamiltonian cycle is a Hamiltonian graph.
A non-Hamiltonian graph is semi-Hamiltonian if there exists a path through 
every vertex.
\begin{theorem}
   If $G$ is a simple graph with $n(\geq 3)$ vertices, and if 
   \[\text{deg}(v)+\text{deg}(w)\geq n\]
   for each pair of non-adjacent vertices $v$ and $w$, then $G$ is Hamiltonian. 
\end{theorem}
\begin{proof}
    We will prove it by contradiction.

    Suppose there is a graph satisfies the conditions but is 
    non-Hamiltonian. Then, by adding enough edges, we can change it 
    into a semi-Hamiltonian graph with a semi-Hamiltonian cycle 
    \[v_1\rightarrow v_2\rightarrow\cdots v_n,\]
    where $v_1\neq v_n$. By the condition, there exists an $i$ such that 
    edge $vv_i$ and $vv_{i-1}$ exists. But this gives us the required 
    contradiction since 
    \[v_1\rightarrow v_2\rightarrow\cdots v_{i-1}\rightarrow v_n\rightarrow v_{n-1}\rightarrow v_{i+1}\rightarrow v_i\rightarrow v_1\]
    is then a Hamiltonian cycle.
\end{proof}
\begin{corollary}
   If $G$ is a simple graph with $n(\geq 3)$ vertices, and if 
   deg $(v)\geq n/2$ for each vertex $v$, then $G$ is Hamiltonian.
\end{corollary}
A digraph $D$ is Hamiltonian if there is a directed cycle that includes every 
vertex of $D$. A non-Hamiltonian digraph that contains a directed path through 
every vertex is semi-Hamiltonian.
\begin{theorem}
    Let $D$ be a strongly connected digraph with $n$ vertices. 
    If outdeg $(v)\geq n/2$ and indeg $(v)\geq n/2$ for each vertex $v$,
    then $D$ is Hamiltonian.
\end{theorem}
\begin{theorem}
    \begin{enumerate}
        \item Every non-Hamiltonian tournament is semi-Hamiltonian.
        \item Every strongly connected tournament is Hamiltonian.
    \end{enumerate}
\end{theorem}
\begin{proof}
    We are going to prove (1) by induction on the number of vertices.
    The statement is clearly true if the tournament has fewer than four vertices.
    Assume that every non-Hamiltonian on $n$ vertices is semi-Hamiltonian.
    Then, for the non-Hamiltonian on $n+1$ vertices. First, by hypothesis,
    we can construct a semi-Hamiltonian cycle on $n$ vertices and 
    denote the $n+1$th vertex by $v$.

    For the case that the arc in form $vv_i$ and $v_iv$ both exists, 
    we can find an i such that $v_iv$ and $vv_{i+1}$ both exists and 
    add them into the n-vertices semi-Hamiltonian cycle.
    Otherwise, we can let $v$ connected to the 
    head or the end of the n-vertices semi-Hamiltonian cycle. 

    We also prove $(2)$ by induction on the number of the vertices of 
    the cycle that we can construct in a strongly connected tournament.

    When $n=3$. Chose a vertex $v$ in $G$, then, the other vertices in $G$
    can be classified into two group $A$ and $B$. For every vertex in $A$,
    there is an arc from that vertex to $v$. For every vertex in $B$, 
    there is an arc from $v$ to that vertex.
    Since the graph is strongly conncted there exist an arc $ba$ such that 
    $a\in A,b\in B$ such that there is a path from any vertex in  $B$ to 
    any vertex in $A$. Thus, we have a cycle $a\rightarrow v \rightarrow b\rightarrow a$.

    Suppose there is a cycle of length $k$, where $k<n$.
    Let  
    \[v_1\rightarrow v_2\cdots\rightarrow v_k\rightarrow v_1\]
    be such cycle.
    The remainder of the proof for (2) is similar to the second part of the 
    proof for (1) and the former part of this proof for (2).
\end{proof}
\section{Trees}
\subsection{Properties of trees}
A connected graph that has no cycles is a tree.
\begin{theorem}
    Let $T$ be a graph with $n$ vertices. Then the following statements are equivalent:
    \begin{enumerate}
        \item $T$ is a tree;
        \item $T$ contains no cycles, and has $n-1$ edges;
        \item $T$ is connected, and has $n-1$ edges;
        \item $T$ is connected, and each edge is a bridge;
        \item any two vertices of $T$ are connected by exactly one path;
        \item $T$ contains no cycles, but the addition of any new edge creates exactly one cycle.
    \end{enumerate}
\end{theorem}
\begin{proof}
    $(1)\rightarrow(2)$Since $T$ contains no cycles, if we remove any graph, we 
    will get two trees or one tree with one edge less, then, we can prove by induction on the number of vertices.

    $(2)\rightarrow(3)$ can be proved by contradiction easily.

    $(3)\rightarrow(4)$ remove one edge and prove by contradiction.

    $(4)\rightarrow(5)$ obvious.

    $(5)\rightarrow(6)$ the former part is ovbious. If an edge $e$ is 
    added into the graph, since the vertices that $e$ are incident with 
    are already adjacent with each other, there will be a circle.
    If there are two cicle, then, by removing the edge $e$, we can 
    get a cicle in the original graph, which is a contradiction.

    $(6)\rightarrow(1)$ if $T$ is disconnected, then, by adding exactly 
    one edge between the two components, there will be no cycle.
\end{proof}
\begin{corollary}
    If $G$ is a forest with $n$ vertices an $k$ components, then $G$ has $n-k$ edges.
\end{corollary}
By Handshaking lemma, the sum of the degrees of the $n$ vertices of a 
tree is equal to twice the number of edges $(=2n-2)$.  It follows that if $n>2$,
any tree on $n$ vertices has at least two end-vertices.

Given any connected graph $G$, we can choose a cycle and remove any one of its edges
andd the resulting graph remains connected. We repeat this procedure with one of 
the remaining cycles, continuing until there are no cycles left. The graph that remains 
is a tree that connects all the vertices of $G$. It is called a spanning tree of $G$.

We can turn each components of a disconnected graph into spanning tree, then we get a 
spanning forest, and the total number of edges removed in this process 
is the cycle rank of $G$, denoted by $\gamma(G)$.
Note that $\gamma(G)=m-n+k$, where $n,m,k$ are the number of vertices, edges and components
of the original graph.

The cutset rank of $G$ is defined similarly, denoted by $\zeta (G)$. Note that $\zeta(G)=n-k$.
In the following theorem, the complement of a spanning forest $T$ of a (not necessarily simple)
graph $G$ is the graph obtained from $G$ by removing the edges of $T$.
\begin{theorem}
    If $T$ is any spanning forest of a graph $G$, then 
    \begin{enumerate}
        \item each cutset of $G$ has an edge in common with $T$;
        \item each cycle of $G$ has an edge in common with the complement of $T$.
    \end{enumerate}
\end{theorem} 
\begin{proof}
    $(1)$ there must be an edge incident with two vertices that are in different complement
    after the deletion of the cutset.

    If not, then there is a cycle contained in $T$.
\end{proof}
The cycles obtained by adding separately each edge of $G$ not contained 
in $T$, is the fundamental set of cycles associated with $T$, simply referred as 
a fundamental set of cycles of $G$. Note that the number of cycles in any fundamental 
set must equal the cycle rank of $G$.

By removing any edges of $T$, we can divides the vertex oset of $T$ into two disjoint
sets $V_1$ and $V_2$. The set of all edges of $G$ joining a vertex of $V_1$
to one of $V_2$ is a cutset of $G$, and the set of all cutsets obtained 
in this way, by removing separately each edge of $T$ is the fundamental set of cutsets 
associated with $T$.
\subsection{Counting trees}
\begin{theorem}
    There are $n^{n-2}$ distinct labelled trees on $n$ vertices.
\end{theorem}
\begin{corollary}
    The number of spanning trees of $K_n$ is $n^{n-2}$.
\end{corollary}
\begin{theorem}
    Let $G$ be a connected simple graph with vertex set $\{v_1,v_2,\cdots,v_n\}$,
    and let $M=(m _{ij})$ be the $n\times n$ matrix in which $m_{ii}=\deg(v_i),m _{ij}=-1$
    if $v_i$ and $v_j$ are adjacent, and $m _{ij}=0$ otherwise. Then 
    the number of spanning trees of $G$ is equal to the cofactor of any element
    of $M$.
\end{theorem}
\section{Planarity}
\subsection{Planar graphs}
A planar graph is a graph that can be drawn in the plane without crossing. Any such drawing is a plane drawing. We use the abbreviation 
plan graph for a plane drawing of a planar graph. 
\begin{remark}
    It is proved that every simple planar graph can be drawn with straight lines.
\end{remark}
The crossing number cr $(G)$ of a graph $G$ is the smallest number of crossings that can occur when $G$ is drawn in the plane.
We have cr $(K_5)$=cr($K_{3,3}$)=1.
It is clear that every subgraph of a planar graph is planar, and that every graph with a non-planar subgraph must be non-planar.
\begin{theorem}
    $K_{3,3}$ and $K_5$ are non-planar.
\end{theorem}
We define two graphs to be homeomorphic if both can be obtained from the same graph by inserting new vertices of degree 2 into its edges.
\begin{theorem}
    A graph is planar if and only if it contains no subgraph homeomorphic to $K_5$ or $K_{3,3}$.
\end{theorem}
We define a graph $H$ to be contractible to $K_5$ or $K_{3,3}$ if we can obtain $K_5$ or $K_{3,3}$ by successively contracting edges of $H$.
\begin{theorem}
    A graph is planar if and only if it contains no subgraph contractible to $K_5$ or $K_{3,3}$.
\end{theorem}
\begin{theorem}
    If $G$ is a countable graph, every finite subgraph of which is planar, then $G$ is planar.
\end{theorem}
\subsection{Euler's formula}
If $G$ is a planar graph, then any plane drawing of $G$ divides the set of points of the plane not lying on $G$ into regions, called faces.
The face with no bound is called the infinite face.

Through sterographic projection, given any face, we can map the original graph to a graph where the given face is the finite face.
\begin{theorem}
    Let $G$ be a plane drawing of a connected planar graph, and let $n,m$ and $f$ denote respectively the number of vertices, edges and faces of $G$. Then
    \[n-m+f=2.\]
    \begin{proof}
        If $G$ is a tree, then the formula is trival. Otherwise, we can prove it by induction.
        When there are $n$ vertices in $G$, since it is not a tree, there is at least one cycle in $G$. Thus we can remove one edge from the cycle 
        without change its connectness to get a connected graph with $n-1$ vertices. We then finish the proof by the induction hypothesis.
    \end{proof}
\end{theorem}
\begin{remark}
    Project the polyhedron out onto its circumsphere, and then use sterographic projection to project it down onto the plane.
    The resulting graph is a 3-connected plane graph in which each face is bounded by a polygon- such graph is called a polyhedral graph.
\end{remark}
\begin{corollary}
    Let $G$ be a polyhedral graph. Then, with the above notation,
    \[n-m+f=2.\]
\end{corollary}
\begin{corollary}
    Let $G$ be a plane graph with $n$ vertices, $m$ edges, $f$ faces and $k$ components.
    Then 
    \[n-m+f=k+1.\]
\end{corollary}
\begin{corollary}
    \begin{enumerate}
        \item If $G$ is a simple connected planar graph with $(n\geq 3)$ vertices and $m$ edges, then 
        \[m\leq 3n-6\]
        \item If, in addition, $G$ has no triangle, then $m\leq 2n-4$.
    \end{enumerate}
\end{corollary}
\begin{corollary}
    Every simple planar graph $G$ contains a vertex of degree at most 5.
\end{corollary}
We define the thickness $t(G)$ of a graph $G$ to be the smallest number of planar graphs that can be superimposed to form $G$.
\begin{remark}
    In each planar graph, the vertices keep same as those in the original graph.
\end{remark}
\begin{theorem}
    Let $G$ be a simple graph with $n(\geq 3)$ vertices and $m$ edges. Then the thickness $t(G)$ of $G$ satisfies the inequility
    \[t(G)\geq\left\lceil m/(3n-6)\right\rceil \text{  and  }t(G)\geq\left\lfloor (m+3n-7)/(3n-6)\right\rfloor .\]
\end{theorem}
\begin{remark}
    $\left\lceil a/b\right\rceil =\left\lfloor (a+b-1)/b\right\rfloor $ where $a,b$ are all positive integer.
\end{remark}
\begin{remark}
    The thickness of a graph is obviously less than or equal its crossing number.
\end{remark}
\subsection{Dual graphs}
Given a plane drawing of a planar graph $G$, we construct another graph $G^*$, called the (geometric) dual of $G$.
The construction is in two stages:
\begin{enumerate}
    \item inside each face $f$ of $G$ we choose a point $v^*$-these points are the vertices of $G^*$.
    \item corresponding to each edge $e$ of $G$ we draw a line $e^*$ that cross $e$(but no other edge of $G$)
    and joins the vertices $v^*$ in the faces $f$ adjoining $e$ - these lines are the edges of $G^*$.
\end{enumerate}
\begin{remark}
An end-vertex or a bridge give rise to a loop of $G^*$, and if two faces have more than one edge in common, then $G^*$ has multiple edges.
\end{remark}
\begin{remark}
    If $G$ is isomorphic to $G$, it does not necessarily follow that $G^*$ is isomorphic to $H^*$.
\end{remark}
\begin{lemma}
    Let $G$ be a connected plane graph with $n$ vertices, $m$ edges and $f$ faces, and let its geometric dual $G^*$ have $n^*$
    vertices, $m^*$ edges  and $f^*$ faces. Then 
    \[n^*=f,m^*=m \text{ and }f^*=n.\]
\end{lemma}
\begin{theorem}
    If $G$ is a connected plane graph, then $G**$ is isomorphic to $G$.
\end{theorem}
\begin{theorem}
    If $G$ is a connected plane graph, then $G^{**}$ is isomorphic to $G.$
\end{theorem}
\begin{theorem}
    Let $G$ be a planar graph and let $G^*$ be a geometric dual of $G$. Then a set of edges in $G$ forms a cycle in $G$ if 
    and only if the corresponding set of edges of $G^*$ forms a cutset of in $G^*$.
\end{theorem}
\begin{corollary}
    A set of edges of $G$ forms a cutset in $G$ if and only if the corresponding set of edges of $G^*$ forms a cycle in $G^*$.
\end{corollary}
We say that a graph $G^*$ is an abstract dual of a graph $G$ is there is a one-one correspondence bewteen the edges of $G$
and those of $G^*$, with the property that a set of edges of $G$ forms a cycle in $G$ if and only if the corresponding set of edges of $G^*$
forms a cutset in $G^*$.
\begin{theorem}
    If $G^*$ is an abstract dual of $G$, then $G$ is an abstract dual of $G^*$.
\end{theorem}
\begin{theorem}
    A graph is planar if and only if it has an abstract dual.
\end{theorem}
\begin{remark}
    Two of the key points of the proof is that:
    \begin{enumerate}
        \item An edge $e$ is removed from $G$, then the abstract dual of the remaining graph is obtained from $G^*$ by contracting the corresponding edge $e^*$.
        \item Consequently, if $G$ has an abstract dual, then so does any subgraph of $G$.
        \item The insertion or removal in $G$ of a vertex of degree 2 results in the addition or deletion of a `multiple edge' in $G^*$.
        \item It follows that if $G$ has an abstract dual, and if $G'$ is a graph that is homeomorphic to $G$, then $G'$ also has an abstract dual.  
    \end{enumerate}
    For the first key point, it is easy to verify it using the numbers of vertices, edges and faces.
    What's more, if $e$ is in a cycle of the original graph, then, the deletion of it should also break the cutset of the original dual graph,
    as a result, the two point which are supposed to be separated now become one.
\end{remark}
\subsection{Graphs on other surfaces}
A surface is of genus $g$ if it is topologically homeomorphic to a sphere with $g$ handles.
The genus of a sphere is 0, and that if a torus is 1.

A graph that can be drawn without crossings on a surface of genus $g$, but not on one of genus $g-1$, is a graph of genus $g$.
Thus, $K_5$ and $K_{3,3}$ are graphs of genus 1, also called toroidal graphs.
\begin{theorem}
    The genus of a graph does not exceed the crossing number.
\end{theorem}
\begin{remark}
    Every unordered pair of line that are crossing. Thus, if three lines are crossing with each other, but they are overlapped at the 
    same point, they still contribute 3 crossings.
\end{remark}
\begin{theorem}
    Let $G$ be a connected graph of genus $g$ with $n$ vertices, $m$ edges and $f$ faces. Then 
    \[n-m+f=2-2g.\]
\end{theorem}
\begin{remark}
    In this generalization, a face of a graph of genus $g$ is defined in the obvious way.
\end{remark}
\begin{corollary}
    The genus $g(G)$ of a simple graph $G$ with $n(\geq 4)$ vertices and $m$ edges satisfies the inequility
    \[g(G)\geq\left\lceil \frac{1}{6}(m-3n)+1\right\rceil \]
\end{corollary}
\begin{theorem}
    $g(K_n)=\left\lceil \frac{1}{12}(n-3)(n-4)\right\rceil  $
\end{theorem}
\section{Colouring graphs}
\subsection{Colouring vertices}
If $G$ is a graph without loops, then $G$ is $k-$colourable if we can assign one of $k$ colours to each vertex so that adjacent vertices 
have different colours. If $G$ is $k-$colourable but is not $(k-1)$-colourable, we say that $G$ is $k-$chromatic number of $G$ is $k$,
and write $\chi(G)=k$.
\begin{remark}
    We assume that all graphs here are simple, as multiple edges are irrelevant to out discussion. We also assume, when necessary, that they 
    are connected.
\end{remark}
\begin{remark}
    $\chi(G)=1$ if and only if $G$ is a null graph. $\chi(G)=2$ if and only if $G$ is a non-null bipartite graph.
    Note that every tree is $2-$colourable, as is any cycle graph with an even number of vertices.
\end{remark}
\begin{theorem}
    If $G$ is  simple graph with largest vertex-degree $\Delta,$ then $G$ is $(\Delta+1)-$colourable.
\end{theorem}
\begin{proof}
    We can colour one vertex at a time. In other words, we can prove it easily by induciton.
\end{proof}
\begin{theorem}
    If $G$ is a simple connected graph which is not a complete graph, and if the largest vertex-degree of $G$ is $\Delta(\geq 3)$, then 
    $G$ is $\Delta-$colourable.
\end{theorem}
\begin{theorem}
    Every simple planar graph is $6-$colourable.
\end{theorem}
\begin{proof}
    Use the fact that there are at least one vertex in the planar graph that is of degree at most 5.
\end{proof}
\begin{theorem}
    Every simple planar graph is $5-$colourable.
\end{theorem}
\begin{theorem}
    Every simple planar graph is $4-$colourable.
\end{theorem}
\subsection{Chromatic polynomials}
Let $G$ be a simple graph, and let $P_G(k)$ be the number of ways of colouring the vertices of $G$ with $k$ colours 
so that no two adjacent vertices have the same colour. $P_G$ is called the chromatic function of $G$.
\begin{remark}
    If $G$ is any tree with $n$ vertices, then $P_G(k)=k(k-1)^{n-1}$.
    If $G$ is the complete graph $K_n$, then $P_G(k)=k(k-1)\cdots(k-n+1)$
\end{remark}
\begin{theorem}
    Let $G$ be a simple graph, then 
    \[P_G(k)=P_{G-e}(k)-P_{G/e}(k).\]
\end{theorem}
\begin{proof}
    Let $e$ incident with $v$ and $w$. Then, discussion the possibility that $v$ and $w$ is in the same color or not.
\end{proof}
\begin{corollary}
    The chromatic function of a simple graph is polynomial.
\end{corollary}
\begin{proof}
    Repeatedly using the last theorem until the chromatic function we are looking forward is decomposited into several 
    chromatic function of null graph $N_n$, which is $k^n$.
\end{proof}
In the light of this corollary, we can now call $P_G(k)$ the chromatic polynomial of $G$.
\begin{remark}
    If $G$ has $n$ vertices, then $P_G(k)$ is of degree $n$, since no new vertices are introduced at any stage
    and the constuction yield only one null graph on $n$ vertices, which also means that the coefficients of $k^n$ is 1.
\end{remark}
We can prove by induction that the coefficients alternate in sign, and that the coefficients of $k^{n-1}$ is $-m$, where
$m$ is the number of edges of $G$. Since we cannot colour a graph if no colours are available, the constant term of any 
chromatic polynomial is 0.
\subsection{Colouring maps}
We define a map to be a $3-$connected plane graph and define a map to be $k-$colourable-(f) if its faces can be coloured 
with $k$ colours so that no two faces with a boundary edge in common have the same colour. 
To avoid confusion, we use $k-$colourable-(v) to mean $k-colourable$ in the usual sense.
\begin{remark}
    The infinite face is also need to be coloured.
\end{remark}
\begin{theorem}
    A map $G$ is $2-$colourable-(f) if and only if $G$ is an Eulerian graph.
\end{theorem}
\begin{theorem}
    Let $G$ be a plane graph without loops, and let $G^*$ be a geometric dual of $G$. Then $G$ is $k-$colourable-(v)
    if and only if $G^*$ is $k-$colourable-(f).
\end{theorem}
It follows that we can dualize any theorem on the colouring of the vertices of a planar graph to give a theorem on the colouring of the 
faces of a map, and conversely.
\begin{remark}
    In the Euler's formula, $n,f$ are a pair of dual.
\end{remark}
\begin{corollary}
The four-colour theorem for maps is equivalent to the four-colour theorem for planar graphs.
\end{corollary}
\begin{theorem}
    Let $G$ be a cubic map. Then $G$ is $3-$colourable-(f) if and only if each face is bounded by an even number of edges.
\end{theorem}
\begin{theorem}
    In order to prove the four-colour theorem, it is sufficient to prove that every cubic map is 4-colourable-(f).
\end{theorem}
\subsection{The four-colour theorem}
\begin{theorem}
    Every cubic map must contain at least one of the following part:
    \begin{enumerate}
        \item a triangle 
        \item a quadrilateral
        \item two adjacent pentagon
        \item a pentagon adjacent to a hexagon
    \end{enumerate}
    This collection of faces is called an unavoidable set of configurations.
\end{theorem}
The four-colour theorem is proved by showing that there exists a unavoidable set of configurations each of which 
can fit into a 4-coloured map of any kinds (maybe after some necessary switch of colours).
\subsection{Colouring edges}
A graph $G$ is $k-$colourable-(e) if its edge can be coloured with $k$ colours so that no two adjacent edges have the same colour.
If $G$ is $k-$colourable-(e) but not $(k-1)-$colourable-(e), we say that the chromatic index of $G$ is $k$, and write $\chi'(G)=k$.
\begin{theorem}
If $G$ is a graph with largest vertex-degree $\Delta$, then 
\[\Delta\leq\chi'(G)\leq\Delta+1.\]    
\end{theorem}
\begin{theorem}
    $\chi'(K_n)=n$ if $n$ is odd(n $\geq 3$), and $\chi'(K_n)=n-1$ if $n$ is even.
\end{theorem}
\begin{theorem}
    The four-colour theorem is equivalent to the statement that $\chi'(G)=3$ for each cubic map $G$.
\end{theorem}
\begin{theorem}
    If $G$ is a bipartite graph with largest vertex-degree $\Delta$, then $\chi'(G)=\Delta$.
\end{theorem}
\begin{corollary}
    $\chi'(K_{r,s})=\max(r,s)$
\end{corollary}
\section{Matching, marriage and Menger's theorem}
\subsection{Hall's `marriage' theorem}
A complete matching from $V_1$ to $V_2$ in a bipartite graph $G(V_1,V_2)$
is a one-one correspondence between the vertices in $V_1$ and some of the vertices in $V_2$, such that corresponding vertices are joined.
\begin{theorem}
    Let $G=G(V_1,V_2)$ be a bipartite graph, and for each subset $A$ of $V_1$,
    let $\varphi(A)$ be the set of vertices of $V_2$ that are adjacent to at least one vertex of $A$. Then a complete matching from $V_1$ 
    to $V_2$ exists if and only if $\abs{A}\leq\abs{\varphi(A)}$, for each subset $A$ of $V_1.$
\end{theorem}
In general, if $E$ is a non-empty finite set, and if $\mathcal{F} =(S_1,S_2,\cdots,S_m)$ is a family of (not necessarily distinct)
non-empty subsets of $E$, then a transveral of $\mathcal{F}$ is a set of $m$ distinct elements of $E$, one chosen from each set $S_i$.
We call a transversal of a subfamily of $\mathcal{F}$ a partial transveral of $\mathcal{F}$.
\begin{remark}
    Any subset of a partial transveral is a partial transveral. Especially, $\varnothing$ is always a partial transveral.
\end{remark}
\begin{theorem}
    Let $E$ be a non-empty finite set, and let $\mathcal{F}=(S_1,S_2,\cdots,S_m)$ be a family of non-empty subsets of $E$.
    Then $\mathcal{F}$ has a transveral if and only if the union of any $k$ of the subsets $S_i$ contains at least $k$ elements, for $1\leq k\leq m$.
\end{theorem}
\begin{corollary}
    If $E$ and $\mathcal{F}$ are as before, then $\mathcal{F}$ has a partial transveral of size $t$ if and only if the union of any $k$
    of the subset $S_i$ contains at least $k+t-m$ elements.
\end{corollary}
\subsection{Menger's theorem}
The paths from $v$ to $w$, no two of which have an edge in common, are called edge-disjoint paths. 
The paths from $v$ to $w$, no two of which have a vertex in common, are called vertex-disjoint paths.
A $vw-$disconnecting set of $G$ is a set $E$ of edges of $G$ such that each path from $v$ to $w$ inculdes an edge of $E$.
A vw-separating set of $G$ is a set $S$ of vertices, other than $v$ or $w$, such that each path from $v$ ro $w$ passes through a vertex of $S$.
\begin{theorem}
    The maximum number of edge-disjoint paths connecting two distinct vertices $v$ and $w$ of a connected graph is equal to the 
    minimum number of edges in a $vw-$disconnecting set.
\end{theorem}
\begin{theorem}
    The maximum number of vertex-disjoint paths connecting two distinct non-adjacent vertices $v$ and $w$ of a graph is equal 
    to the minimum number of vertices in a $vw-$separating set.
\end{theorem}
\begin{corollary}
    A graph $G$ is $k-$edge-connected if and only if any two distinct vertices of $G$ are connected by at least $k$ edge-disjoint paths.
\end{corollary}
\begin{corollary}
    A graph $G$ with at least $k+1$ vertices is $k-$connected if and only if any two distinct vertices of $G$ are connected by at least 
    $k$ vertex-disjoint paths.
\end{corollary}
\begin{theorem}
    The maximum number of arc-disjoint paths from a vertex $v$ to a vertex $w$ in a digraph is equal to the minimum number of arcs in a 
    $vw$-disconnecting set.
\end{theorem}
\begin{theorem}
    Menger's theorem implies Hall's theorem.
\end{theorem}
\subsection{Network flows}
We define a network flow $N$ to be a weighted digraph - that is, a digraph to each arc $a$ of which is assigned a non-negative 
real number $c(a)$ called its capacity. The out-degree outdeg $(x)$ of a vertex is the sum of the capacities of the arcs of the 
form $xz$, and the in-degree indeg $(x)$ is similarly defined. 
\begin{remark}
    The sum of the out-degrees of the vertices of a network is equal to the sum of the in-degrees.
\end{remark}
A vertex with in-degree 0 is a source, and one with out-deg 0 is a sink. Usually we assume that any network has exactly one source $v$ 
and one sink $w$.

A flow in a network is a function $\varphi$ that assigns to each arc $a$ a non-negative real number $\varphi$(a), called the flow in $a$,
in such a way that 
\begin{enumerate}
    \item for each arc $a$, $\varphi(a)\leq c(a)$
    \item the out-degree and in-degree of each vertex, other than $v$ or $w$, are equal.
\end{enumerate}
The flow in which every arc is 0 is called the zero flow; any other flow is a non-zero flow. An arc $a$ for which 
$\varphi(a)=c(a)$ is called saturated and the remaining arcs are unsaturated.

It follows from the handshaking dilemma that the sum of the flows in the arcs out of $v$ is equal to the sum of the flow in the arcs 
into $w$; this sum is called the value of the flow.

A cut in a network is a $vw-$disconnecting set in the corresponding digraph $D$. The capacity of a cut is the sum of the capacities of the 
arcs in the cut. 

The largest flows of $G$ is are called maximum flows. The cut with smallest capacity is called the minimum cuts.
\begin{theorem}[Max-flow min-cut theorem]
   In any network, the value of any maximum flow is equal to the capacity of any minimum cut. 
\end{theorem}
\end{document}