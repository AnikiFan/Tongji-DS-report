\section{数域}
\begin{mydef}
    复数集的一个子集 $K$如果满足:

(1)$0,1\in K$;

(2)$a,b\in K\Rightarrow a\pm b,ab\in K$;

(3)$a,b\in K,b\neq 0\Rightarrow \frac{a}{b}\in K$,

则称 $K$为一个数域.
\end{mydef}
\begin{myrmk}
    即,数域是对于加减乘除封闭的数集。
\end{myrmk}
有理数集,实数集,复数集都是数域;但整数集不是。

显然,任一数域都包含有理数域,即有理数域是最小的数域;复数域则是最大的数域。
\section{环}
\subsection{环的定义}
\begin{mydef}
    设 $R$是一个非空集合,如果它有两个代数运算:加法和乘法,分别记作 $a+b$和 $ab$ 。
    且这两个代数运算满足一下6条运算法则:
    \begin{enumerate}
        \item 加法结合律
        \item 加法交换律
        \item 加法具有零元
        \item 加法具有负元
        \item 乘法结合律
        \item 乘法对于加法有左右分配律
    \end{enumerate}
    则称 $R$是一个环。


\end{mydef}
\begin{myrmk}
    所谓 $R$上的一个代数运算,是指 $R\times R$到 $R$的一个映射。

    
\end{myrmk}
    零元记作0.
    元素 $a$的负元记作 $-a$。
显然,环的零元和对于元素 $a$的负元都是唯一的。环中还可以定义减法为
\[a-b:=a+(-b).\]

\begin{mydef}
    若环 $R$中的乘法还满足交换律,则称 $R$为交换环。
\end{mydef}
\begin{mydef}
     若环 $R$中有一个元素 $e$具有性质:
    \[ea=ae=a,\forall a \in R.\]
    则称 $e$是 $R$的单位元,此时称 $R$是有单位元的环。
\end{mydef}
容易证明,在有单位元的环 $R$中,单位元是唯一的。
    通常把单位元记成1.
\begin{mydef}
    环 $R$中的元素 $a$称为一个左零因子(右零因子),如果 $R$中有元素 $b\neq 0$,则使得 $ab=0(ba=0)$。
    左零因子和右零因子都简称为零因子。特别地,称0为平凡的零因子;其余的零因子称为非平凡的零因子。
\end{mydef}
\begin{mydef}
    如果环 $R$没有非平凡的零因子,那么称 $R$是无零因子环。有单位元 $1(\neq 0)$的无零因子的交换环称为整环。
\end{mydef}
    $\mathbf{Z},K,K[x]$都是整环,$M_n(K)$不是整环,因为它不满足乘法交换律,且它有非平凡的零因子.
\begin{mydef}
    设 $R$是一个有单位元 $1(\neq 0)$的环。对于 $a\in R$,如果存在 $b\in R$,使得
    \[ab=ba=1,\]
    那么称 $a$是可逆元(或单位),称 $b$是 $a$的逆元,记作 $a^{-1}$。
\end{mydef}
可以证明,如果 $a$是可逆元,则它的逆元唯一。
\subsection{子环与扩环}
\begin{mydef}[子环]
    如果环 $R$的一个非空子集 $R_1$对于 $R$的加法和乘法也成为一个环,那么称 $R_1$是 $R$的
    一个子环.
\end{mydef}
\begin{mythm}[子环判定定理]
    环 $R$的一个非空子集 $R_1$为一个子环的充分必要条件是 $R_1$对于 $R$的减法与乘法都封闭.
\end{mythm}
\begin{mydef}[扩环]
    设 $R$是有单位元 $1'$的交换环,如果 $R$有一个子环 $R_1$满足下列条件:
    \begin{enumerate}
        \item $1'\in R_1$
        \item  数域$K$到 $R_1$有一个双射 $\tau$,且 $\tau$保持加法和乘法运算,
    \end{enumerate}
    那么 称 $R$可看成是 $K$的一个扩环.
\end{mydef}
\begin{myrmk}
    可以证明, $\tau(1)=1'.$
\end{myrmk}
\section{常用符号}
\begin{longtable}{p{2in}|l}
    \caption{符号表}\\
    \hline\hline
    \sffamily\bfseries{含义}&\sffamily\bfseries{符号}\\
    \hline
    \endfirsthead
    \caption{(续)}\\
    \hline\\
    \sffamily\bfseries{含义}&\sffamily\bfseries{符号}\\
    \hline\\
    \endhead
    \hline\hline
    \endfoot
    数域&$K$\\
    数域$K$中的所有非零数&$K^*$\\
    数域$K$上的一元多项式&$K[x]$\\
    数域$K$上的$n$级矩阵&$M_n(K)$\\
    环& $R$\\
    实数环&$\mathbf{R}$\\
    整数环&$\mathbf{Z}$\\
    不定元&$x$\\
    多项式 $f$的次数& $\deg f$\\
    多项式 $f$整除多项式 $g$& $f\mid g$\\
    多项式 $f$不整除多项式 $g$& $f\nmid g$\\
    多项式 $f$相伴多项式 $g$& $f\sim g$\\
    多项式 $f(x),g(x)$的最大公因式& $(f(x),g(x))$\\
    多项式 $f(x),g(x)$的最小共倍式& $[f(x),g(x)]$\\
\end{longtable}